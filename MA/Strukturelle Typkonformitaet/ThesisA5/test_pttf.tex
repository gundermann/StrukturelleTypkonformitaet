\section{Ergebnisse für die Heuristik PTTF}\label{sec_evalPTTF}
Für die \Gls{Heuristik} \emph{PTTF} gilt es zu evaluieren, ob die Suche nach einem \emph{Proxy}, der die vordefinierten Tests besteht, beschleunigt werden kann. Hierzu wird der \emph{Explorationsprozess} für alle in Tabelle \ref{tab:eIShort} genannten \emph{required Typen} unter der Verwendung der in Abschnitt \ref{sec_pttf} beschriebenen \Gls{Heuristik} durchgeführt.
\\\\
Die folgenden Vier-Felder-Tafeln zeigen die Ergebnisse für die \emph{required Typen} \emph{TEI1}-\emph{TEI7} auf.
\begin{multicols}{3}
\vft{1}{29}{$p_1(44)-30$}{1}{0}{Ergebnisse \emph{PTTF} für TEI1 1.~\mbox{Durchlauf}}{pttf_TEI1_1}
\vft{1}{5544}{$p_1(55)-5545$}{1}{0}{Ergebnisse \emph{PTTF} für TEI2 1.~\mbox{Durchlauf}}{pttf_TEI2_1}
\vft{1}{4761}{$p_1(50)-4762$}{1}{0}{Ergebnisse \emph{PTTF} für TEI3 1.~\mbox{Durchlauf}}{pttf_TEI3_1}
\end{multicols}

\begin{multicols}{2}
\vft{1}{$1174$}{0}{0}{0}{Ergebnisse \emph{PTTF} für TEI4 1.~\mbox{Durchlauf}}{pttf_TEI4_1}
\vft{2}{466}{$p_2(2247)-467$}{1}{0}{Ergebnisse \emph{PTTF} für TEI4 2.~\mbox{Durchlauf}}{pttf_TEI4_2}
\end{multicols}
\pagebreak
\begin{multicols}{2}
\vft{1}{$4984$}{0}{0}{0}{Ergebnisse \emph{PTTF} für TEI5 1.~\mbox{Durchlauf}}{pttf_TEI5_1}
\vft{2}{2172}{$p_2(2775)-2173$}{1}{0}{Ergebnisse \emph{PTTF} für TEI5 2.~\mbox{Durchlauf}}{pttf_TEI5_2}
\end{multicols}

\begin{multicols}{2}
\vft{1}{$1051$}{0}{0}{0}{Ergebnisse \emph{PTTF} für TEI6 1.~\mbox{Durchlauf}}{pttf_TEI6_1}
\vft{2}{13122}{$p_2(1323)-13123$}{1}{0}{Ergebnisse \emph{PTTF} für TEI6 2.~\mbox{Durchlauf}}{pttf_TEI6_2}
\end{multicols}

\begin{multicols}{2}
\vft{1}{$161294$}{0}{0}{0}{Ergebnisse \emph{PTTF} für TEI7 1.~\mbox{Durchlauf}}{pttf_TEI7_1}
\vft{2}{149961}{$p_2(52150)-149962$}{1}{0}{Ergebnisse \emph{PTTF} für TEI7 2.~\mbox{Durchlauf}}{pttf_TEI7_2}
\end{multicols}
\newpage
\noindent
Folgendes kann aus diesen Ergebnissen abgeleitet werden:
\begin{enumerate}
\item Die \Gls{Heuristik} \emph{PTTF} erzielt im Vergleich zum Ausgangspunkt (Abschnitt \ref{sec_ausgangspunkt}) für jeden \emph{required Typ} eine weitere Reduktion der zu prüfenden \emph{Proxies}.

\item Die \Gls{Heuristik} \emph{PTTF} hat keine Auswirkung auf einen Durchlauf, in dem kein \emph{Proxy} erzeugt wird, mit dem die vordefinierten Tests erfolgreich durchgeführt werden können. Dies kann durch einen Vergleich des ersten Durchlaufs für den \emph{required Typ} \emph{TEI4}-\emph{TEI7} im Ausgangspunkt (Tabelle \ref{tab:tmr_start_tei4_1}, \ref{tab:tmr_start_tei5_1}, \ref{tab:tmr_start_tei6_1} und \ref{tab:tmr_start_tei6_1}) mit dem ersten Durchlauf unter Anwendung der \Gls{Heuristik} (Tabellen \ref{tab:pttf_TEI4_1}, \ref{tab:pttf_TEI5_1}, \ref{tab:pttf_TEI6_1} und \ref{tab:pttf_TEI7_1}) festgestellt werden. Aus diesem Grund kommt die in Punkt 1 beschriebene Reduktion erst im jeweils letzten Durchlauf zum Tragen.
\end{enumerate}