\documentclass[a4paper,12pt]{article}
\usepackage[utf8]{inputenc}
\usepackage[ngerman]{babel}
\usepackage{amsmath}
\usepackage{mathtools}
\usepackage{float}
\usepackage{listings}
\usepackage{color}

\definecolor{pblue}{rgb}{0.13,0.13,1}
\definecolor{pgreen}{rgb}{0,0.5,0}
\definecolor{pred}{rgb}{0.9,0,0}
\definecolor{pgrey}{rgb}{0.46,0.45,0.48}

\lstdefinestyle{dsl}{
	 morekeywords={with, subproxy, structproxy, for, simpleproxy, contentproxy, containerproxy, nominalproxy, provided, required, interface, extends},
  showspaces=false,
  showtabs=false,
  breaklines=true,
   literate=
               {-->}{$\rightarrow{}$}{1}
               ,
  showstringspaces=false,
  breakatwhitespace=true,
  commentstyle=\color{pgreen},
  keywordstyle=\color{pblue},
  stringstyle=\color{pred},
  basicstyle=\fontsize{9}{10}\selectfont\ttfamily,
  moredelim=[il][\textcolor{pgrey}]{$ $},
  moredelim=[is][\textcolor{pgrey}]{\%\%}{\%\%}
}




\newcommand{\matchTyp}[3]
{
#1  
\Rightarrow_{#2}
 #3
}

\begin{document}

\section{Struktur für die Definition von Typen}
Die Typen seien in einer Bibliothek $\text{L}$ in folgender Form zusammengefasst:
\begin{table}[H]
\centering
\begin{tabular}{|p{5.5cm}|p{8.5cm}|}
\hline
\hline
\centering\textbf{Regel} & \textbf{Erläuterung} \\
\hline
\hline
$\mathit{L} ::= \mathit{TD}\text{*}$ & Eine Bibliothek \emph{L} besteht aus einer Menge von Typdefinitionen.\\
\hline
$\mathit{TD} ::= \mathit{PD} | \mathit{RD}$ & Eine Typdefinition kann entweder die Definition eines provided Typen (PD) oder eines required Typen (RD) sein.\\
\hline
$\mathit{PD} ::= \newline\texttt{provided }T \texttt{ extends } T' \newline  \texttt{\{} \mathit{FD}\text{*} \mathit{MD}\text{*}\texttt{\}}$& Die Definition eines provided Typen besteht aus dem Namen des Typen \emph{T}, dem Namen des Super-Typs \emph{T'} von \emph{T} sowie mehreren Feld- und Methodendeklarationen.\\
\hline
$\mathit{RD} ::= \texttt{required } T \texttt{ \{}\mathit{MD}\text{*}\texttt{\}}$ & Die Definition eines required Typen besteht aus dem Namen des Typen \emph{T} sowie mehreren Methodendeklarationen.\\
\hline
$\mathit{FD} ::= \mathit{f} \texttt{:} T$ & Eine Felddeklaration besteht aus dem Namen des Feldes \emph{f} und dem Namen seines Typs \emph{T}.\\
\hline
$\mathit{MD} ::= \mathit{m(T)}\texttt{:}\mathit{T}$' & Eine Methodendeklaration besteht aus dem Namen der Methode \emph{m}, dem Namen des Parameter-Typs \emph{T} und dem Namen des Rückgabe-Typs \emph{T'}.\\
\hline
\hline
\end{tabular}
\caption{Struktur für die Definition einer Bibliothek von Typen}
 \label{tab:eIShort}
\end{table}
\noindent
Weiterhin sei die Relation $<$ auf Typen durch folgende Regeln definiert:
\begin{gather*}
\frac{\texttt{provided }T \texttt{ extends } T' \in L}{T < T'}
\end{gather*}
\begin{gather*}
\frac{\texttt{provided } T \texttt{ extends } T'' \in L \wedge T'' < T'}{T < T'}
\end{gather*}

Darüber hinaus seien folgende Funktionen definiert:
\begin{gather*}
\mathit{felder(T)} :=  \left\{ 
				\begin{array}{l|l}
					f : T' & \text{ \emph{f : T'} ist Felddeklaration von \emph{T}}
				\end{array}
              \right\}\\
\mathit{methoden(T)} := \left\{ 
				\begin{array}{l|l}
					m(T'):T'' & \text{ \emph{m(T'):T''} ist Methodendeklaration von 										\emph{T}}
				\end{array}
              \right\}\\
\mathit{feldTyp(f,T)} := 
				\begin{array}{l|l}
					T' & \text{ \emph{f:T'} ist Felddeklatation von 										\emph{T}}
				\end{array}              
\end{gather*}
\noindent
Das Matching eines Typs $A$ zu einem Typ $B$ wird durch die asymmetrische Relation $\matchTyp{A}{}{B}$ beschrieben. Dabei wird $A$ auch als \emph{Source-Typ} und $B$ als \emph{Target-Typ} bezeichnet.

\section{Struktur für die Definition von Proxies}\label{sec:proxygram}
Ein Proxy wird auf der Basis einer Matchingrelation erzeugt. In Abhängigkeit von der zugrundeliegenden Matchingrelation zwischen dem \emph{Source-} und dem \emph{Target-Typen} werden unterschiedliche Arten von Proxies erzeugt:
\begin{itemize}
\item \emph{Struktureller Proxy}
\item \emph{Simple-Proxy}
\item \emph{Sub-Proxy}
\item \emph{Container-Proxy}
\item \emph{Content-Proxy}
\end{itemize}
Der Typ des Proxies entspricht immer dem \emph{Source-Typ} der zugrundeliegenden Matchingrelation. Die unterschiedlichen Proxies werden dabei durch die Regeln der Tabellen \ref{tab:grProxies1} und \ref{tab:grProxies2} beschrieben:
\begin{table}[H]
\centering
\begin{tabular}{|p{6cm}|p{8cm}|}
\hline
\hline
\centering\textbf{Regel} & \textbf{Erläuterung} \\
\hline
\hline
$\mathit{STRUCTPROXY} ::=$\newline $\texttt{structproxy } \texttt{for } R$\newline$ \texttt{\{}TARGET\text{*}\texttt{\}}$ & Ein \emph{struktureller Proxy} wird für ein \emph{required Interface R} mit einer Mengen von Targets erzeugt.  \\
\hline
$\mathit{TARGET} ::=$\newline $P \texttt{ \{}MDEL\text{*}\texttt{\}}$ & Ein Target besteht aus dem Typ \emph{P} des Targets (ein \emph{provided Typ}) und einer Mengen von Methodendelegationen.\\
\hline
$\mathit{MDEL} ::= CALLM \rightarrow DELM $  & Eine Methodendelegation besteht aus einer aufgerufenen Methode und aus einem Delegationsziel.\\
\hline
$\mathit{CALLM} ::=$\newline $m(SP):STPROXY $  & Eine aufgerufene Methode besteht aus dem Namen der Methode \emph{m}, dem Parametertyp \emph{SP} und einem Single-Target-Proxy zur Konvertierung des Rückgabetyps des Delegationsziels.\\
\hline
$\mathit{DELM} ::=$\newline $n(STPROXY):R $  & Ein Delegationsziel besteht aus demdem Namen der Methode \emph{n}, dem Rückgabetyp \emph{TR} und einem Single-Target-Proxy zur Konvertierung des Parametertyps der aufgerufenen Methode.\\
\hline
$\mathit{STPROXY} ::= NPX$ & Ein Nominal-Proxy ist ein Single-Target-Proxy.\\


\hline
\end{tabular}
\caption{Grammatikregeln für die Definition eines Proxies}
 \label{tab:grProxies1}
\end{table}
\newpage
\begin{table}[H]
\centering
\begin{tabular}{|p{6cm}|p{8cm}|}
\hline
\centering\textbf{Regel} & \textbf{Erläuterung} \\
\hline
\hline
$\mathit{STPROXY} ::=$\newline $\texttt{contentproxy }  \texttt{for } P$\newline $\texttt{with } P' \texttt{ \{}CEMDEL\text{*}\texttt{\}}$ & Ein \emph{Content-Proxy} ist ein Single-Target-Proxy, der für ein \emph{provided Typ P} mit einem \emph{provided Typ P'} als Target-Typ sowie einer Mengen von Content-Proxy-Methodendelegationen erzeugt wird.\\
\hline
$\mathit{STPROXY} ::=$\newline $\texttt{containerproxy} \texttt{ for } P$\newline
$ \texttt{with } P' \texttt{ \{} f = \mathit{NPX} \texttt{\}}$ &  Ein \emph{Container-Proxy} ist ein Single-Target-Proxy, der für ein \emph{provided Typ P} mit einem \emph{provided Typ P'} als Target-Typ sowie der Zuweisung eines Nominal-Proxies für den Target-Typ zu einem Feld \emph{f} erzeugt wird.\\
\hline
$\mathit{NPX} ::=$ \newline $\texttt{subproxy } \texttt{for } P$\newline $\texttt{with } P' \texttt{ \{}\mathit{NOMMDEL}\text{*\}}$ & Ein Sub-Proxy ist ein Nominal-Proxy, derfür ein \emph{provided Typ P} mit einem \emph{provided Typ P'} als Target-Typ sowie einer Mengen von Nominal-Proxy-Methodendelegationen erzeugt wird. Dabei gilt $P < P'$.\\
\hline
$\mathit{NPX} ::=$ \newline $\texttt{simpleproxy } \texttt{for } P$ & Ein \emph{Simple-Proxy} ist ein Nominal-Proxy, der aus einem Typen \emph{P}, für den der Proxy erzeugt wird, besteht. Der Target-Typ ist in diesem Fall ebenfalls \emph{P}. Alle Methoden werden in diesem Fall an den Target-Typ delegiert.\\
\hline
$\mathit{NOMMDEL} ::=$\newline
$\mathit{m(SP):SR} \rightarrow \mathit{m(TP):TR} $  & Eine Nominal-Proxy-Methodendelegation besteht aus zwei Methoden mit demselben Namen \emph{m} und den jeweiligen Parameter- und Rückgabetypen \emph{SP} und \emph{SR} bzw. \emph{TP} und \emph{TR}. \\
\hline
$\mathit{CEMDEL} ::= \mathit{m(SP):NPX} \rightarrow$\newline
$\mathit{f.m(NPX):TR} $  & Eine Content-Proxy-Methodendelegation besteht aus zwei Methoden mit demselben Namen \emph{m}, wobei die delegierte Methode (rechte Seite) auf einem Feld \emph{f} des Target-Typs aufgerufen wird. Dabei besteht die aufgerufene Methode aus dem Parametertyp \emph{SP} und einem Nominal-Proxy für den Rückgabetyp. Ferner besteht die delegierte Methode aus dem jeweiligen Rückgabetyp \emph{TR} und einem Nominal-Proxy für den Parametertyp.\\
\hline
\hline
\end{tabular}
\caption{Grammatikregeln für die Definition eines Proxies (Fortsetzung)}
 \label{tab:grProxies2}
\end{table}

\newpage
\section{Beispiel-Bibliothek}



\begin{lstlisting}[style = dsl]
provided Fire extends Object{}
\end{lstlisting}

\begin{lstlisting}[style = dsl]
provided ExtFire extends Fire{}
\end{lstlisting}


\begin{lstlisting}[style = dsl]
provided FireState extends Object{
	isActive : boolean
}
\end{lstlisting}

\begin{lstlisting}[style = dsl]
provided Medicine extends Object{
	String getDescription()
}
\end{lstlisting}

\begin{lstlisting}[style = dsl]
provided Injured extends Object{
	void heal(Medicine med)	
}
\end{lstlisting}


\begin{lstlisting}[style = dsl]
provided Patient extends Injured{}
\end{lstlisting}
\begin{lstlisting}[style = dsl]
provided FireFighter extends Object{
	FireState extinguishFire(Fire fire)
}
\end{lstlisting}

\begin{lstlisting}[style = dsl]
provided Doctor extends Object{	
	void heal( Patient pat, Medicine med )
}
\end{lstlisting}



\begin{lstlisting}[style = dsl]
provided MedCabinet extends Object{
	med : Medicine
}
\end{lstlisting}

\begin{lstlisting}[caption={Bibliothek \emph{ExampLe} von Typen},captionpos=b, style = dsl]
required MedicalFireFighter {
	void heal( Injured injured, MedCabinet med )
	boolean extinguishFire( ExtFire fire )	
}
\end{lstlisting}\label{lst:libEx}

\newpage
\section{Beispiel-Proxy für MedicalFireFighter}\label{xmpl_prxy}
\begin{lstlisting}[caption={Proxy für MedicalFireFighter},captionpos=b,style = dsl]
structproxy for MedicalFireFither{
	FireFighter {
	 extinguishFire(ExtFire): 
	 	containerproxy for FireState with boolean {
		 isActive = simpleproxy for boolean		
		} 
		--> extinguishFire(simpleproxy for Fire):boolean
	}
		
	Doctor {
	 heal(Injured, MedCabinet):simpleproxy for void
		--> heal(subproxy for Patient with Injured{
			heal(Medicine):	void
				--> heal(Medicine):void					
		   }, contentproxy for Medicine with MedCabinet{
			getDescription(): simpleproxy for String
				--> med.getDescription():String			
		      }):void		
	}
}
\end{lstlisting}

\newpage
\section{Generierung der Proxies auf Basis von Matchern}
Die Matcher beinhalten zum Einen die Definition der jeweiligen Matchingrelation ($\Rightarrow$) sowie die Regeln zur Erzeugung eines Proxies, die auf jener Matchingrelation basieren. Alle Arten von Proxies, die durch die folgenden Matcher erzeugt werden, können am Beispiel aus Abschnitt \ref{xmpl_prxy} nachvollzogen werden.
\subsection{Matcher}
\subsubsection{ExactTypeMatcher}\label{sec:exacttypematcher}
Der \emph{ExactTypeMatcher} stellt ein Matching von einem Typ $T$ zu demselben Typ $T$ her. Die dazugehörige Matchingrelation $\Rightarrow_{exact}$ wird durch folgende Regel beschrieben:
\begin{gather*}
\frac{}{T \Rightarrow_{exact} T}
\end{gather*}
\subsubsection{GenTypeMatcher}\label{sec:gentypematcher}
Der \emph{GenTypeMatcher} stellt ein Matching von einem Typ $T$ zu einem Typ $T'$ mit $T > T'$ her. Die dazugehörige Matchingrelation $\Rightarrow_{gen}$ wird durch folgende Regel beschrieben:
\begin{gather*}
\frac{T > T'}{T \Rightarrow_{gen} T'}
\end{gather*}
\subsubsection{SpecTypeMatcher}
Der \emph{SpecTypeMatcher} stellt im Verhältnis zum \emph{GenTypeMatcher} das Matching in der entgegengesetzten Richtung dar. Die dazugehörige Matchingrelation $\Rightarrow_{spec}$ wird durch folgende Regel beschrieben: 
\begin{gather*}
\frac{T < T'}{T \Rightarrow_{spec} T'}
\end{gather*}
\\\\
Die oben genannten Matchingrelationen werden für die Definition weiterer Matcher zusammengefasst, wodurch sich die Matchingrelation $\Rightarrow_{internCont}$ ergibt:
\begin{gather*}
\frac{T \Rightarrow_{exact} T' \vee T \Rightarrow_{gen} T' \vee
T \Rightarrow_{spec} T'  }{T \Rightarrow_{internCont} T'}
\end{gather*}
\subsubsection{ContentTypeMatcher}
Der \emph{ContentTypeMatcher} matcht einen Typ $T$ auf einen Typ $T'$, wobei $T'$ ein Feld enthält, auf dessen Typ $T''$ der Typ $T$ über die Matchingrelation $\Rightarrow_{internCont}$ gematcht werden kann. So kann bspw. der Typ $\texttt{FireState}$ aus Listing \ref{lst:libEx} auf den Typ $\texttt{boolean}$ gematcht werden.
\\\\
Die dazugehörige Matchingrelation $\Rightarrow_{content}$ wird durch folgende Regel beschrieben:
\begin{gather*}
\frac{\exists f:T''\in felder(T'). T \Rightarrow_{internCont} T''}{T \Rightarrow_{content} T'}
\end{gather*}
\subsubsection{ContainerTypeMatcher}
Der \emph{ContainerTypeMatcher} stellt im Verhältnis zum \emph{ContentTypeMatcher} das Matching in der entgegengesetzten Richtung dar. So kann bspw. auch der Typ $\texttt{boolean}$ auf den Typ $\texttt{FireState}$ aus Listing \ref{lst:libEx} gematcht werden.
\\\\
Die dazugehörige Matchingrelation $\Rightarrow_{container}$ wird durch folgende Regel beschrieben:
\begin{gather*}
\frac{\exists f:T''\in felder(T). T'' \Rightarrow_{internCont} T'}{T \Rightarrow_{container} T'}
\end{gather*}
\\\\
Zur Definition des letzten Matchers werden die Matchingrelationen der oben genannten Matcher noch einmal zusammengefasst. Dabei entsteht die Matchingrelation $\Rightarrow_{internStruct}$, welche durch folgende Regel beschrieben wird:
\begin{gather*}
\frac{\splitfrac{T \Rightarrow_{exact}T' \vee T \Rightarrow_{spec}T' 
\vee T \Rightarrow_{gen}T'}{\vee T \Rightarrow_{container} T' \vee T \Rightarrow_{content} T'}}{T \Rightarrow_{internStruct}T'}
\end{gather*}
\subsubsection{StructuralTypeMatcher} 
Der \emph{StructuralTypeMatcher} matcht einen \emph{required Typ} $R$ auf einen \emph{provided Typ} $P$ auf der Basis struktureller Eigenschaften der Methoden, die in den Typen deklariert sind. 
\\\\
Somit soll bspw. der Typ $\texttt{MedicalFireFighter}$ auf den Typ $\texttt{FireFighter}$ (siehe Listing \ref{lst:libEx}) gematcht werden. Ein weiteres Beispiel bezogen auf die Typen aus Listing \ref{lst:libEx} ist das Matching des Typs $\texttt{MedicalFireFighter}$ auf den Typ $\texttt{Doctor}$ angebracht werden.
\\\\
Damit ein Typ $R$ auf einen Typ $P$ über den \emph{StrukturalTypeMatcher} gematcht werden kann, muss mindestens eine Methode aus $R$ zu einer Methode aus $P$ gematcht werden. Die Menge der gematchten Methoden aus $R$ in $P$ wird wie folgt beschrieben:
\begin{gather*}
structM(R,P) := \left\{ 
				\begin{array}{l|l}
						& \exists n(S):S' \in methoden(P) .\\													m(T):T' \in methoden(R) &  S\Rightarrow_{internStruct}T \wedge \\
										& T' \Rightarrow_{internStruct}S'
				\end{array}
              \right\}
\end{gather*}
Da die Notation es nicht hergibt, ist zusätzlich zu erwähnen, dass, sofern in $m$ und $n$ mehrere Parameter verwendet werden, deren Reihenfolge irrelevant ist.\\\\
Die Matchingrelation für die \emph{StructuralTypeMatcher} wird durch folgende Regel beschrieben:
\begin{gather*}
\frac{structM(R,P) \neq \emptyset}{R \Rightarrow_{struct}P}
\end{gather*}
\subsection{Generierung der Proxies}
Zum besseren Verständnis werden die Regeln zur Generierung der Proxies auf das Beispiel aus Abschnitt \ref{xmpl_prxy} angewendet. Da die Proxies verschachtelt sind, erfolgt die beispielhafte Generierung in den folgenden Abschnitten schrittweise parallel zur Beschreibung der Generatoren.

\subsubsection{Struktureller Proxy}
Der \emph{strukturelle Proxy} bildet immer den Einstiegspunkt für die Generierung eines Proxies für einen \emph{required Typ} über das Nichterminal $\mathit{STRUCTPROXY}$ aus der Grammatik für die Definition von Proxies (Abschnitt \ref{sec:proxygram}). Als Basis des Generators für einen \emph{strukturellen Proxy} fungiert der \emph{StructuralTypeMatcher}.
Daher wird durch diesen Generator ein \emph{struktureller Proxy} für einen \emph{required Typ} $R$ generiert.\\\\
Für die Generierung eines solchen Proxies vom Typ $R$ muss sichergestellt werden, dass alle in $R$ enthaltenen
Methoden durch ein oder mehrere \emph{provided Typen} innerhalb der gesamten Bibliothek $L$ gematcht werden. Die folgende Funktion $\mathit{cover}$ beschreibt daher eine Menge von Mengen von \emph{provided Typen}, die für die Erzeugung eines \emph{strukturellen Proxies} für $R$ verwendet werden können.
\begin{gather*}
cover(R,L) := 
\left\{\begin{array}{l|l}
					& P_1 \in L \wedge \text{...} \wedge P_n \in L \wedge \\
					& methoden(R) = structM(R,P_1) \cup \\
	\{P_1,...,P_n\}	& \texttt{...} \cup structM(R, P_n) \wedge \\
					& \mathit{structM(R,P_1)} \neq \emptyset \wedge \\
					& \texttt{...}\wedge \mathit{structM(R,P_n)} \neq \emptyset 
\end{array}\right\}
\end{gather*}
Ausgehend von einer Bibliothek $L$ kann ein \emph{strukureller Proxy} zum \emph{require Typ} $R$ aus einer Menge von \emph{provided Typen} $P$ mit $P \in \mathit{cover(R,L)}$ generiert werden. Dazu werden die Grammatikregeln aus Abschnitt \ref{sec:proxygram}, die für den \emph{strukturellen Proxy} relevant sind, um Nebenbedingungen erweitert, die Tabelle \ref{tab:structproxy} zu entnehmen sind.
\begin{table}[H]
\centering
\begin{tabular}{|p{4cm}|p{10cm}|}
\hline
\hline
\centering\textbf{Regel} & \textbf{Nebenbedingungen} \\
\hline
\hline
$\mathit{STRUCTPROXY} ::=$\newline $\texttt{structproxy } \texttt{for } R$\newline$ \texttt{\{}TARGET_1 \texttt{ ... }$ \newline
$ TARGET_n\texttt{\}}$ & $typ(\mathit{STRUCTPROXY})=R$\newline
$methoden(\mathit{STRUCTPROXY}) = $\newline
 $\mathit{cmethoden(TARGET_1)} \cup \texttt{...} \cup \mathit{cmethoden(TARGET_n)} $\newline
$methoden(R) = methoden(\mathit{STRUCTPROXY})$\\
\hline
$\mathit{TARGET} ::=$\newline $P \texttt{ \{}MDEL_1 \texttt{ ...}$\newline
$ MDEL_n\texttt{\}}$ &
$\mathit{typ(TARGET)} = P$ \newline
$\mathit{cmethoden(TARGET)} = $\newline
 $\mathit{cmethode(MDEL_1)} \cup \texttt{...} \cup \mathit{cmethode(MDEL_n)}$ \newline
$\mathit{dmethoden(TARGET)} =  $\newline 
 $\mathit{dmethode(MDEL_1)} \cup \texttt{...} \cup \mathit{dmethode(MDEL_n)}$ \newline
 $\mathit{dmethoden(TARGET)} \subseteq \mathit{methoden(P)} $ \\
\hline
$\mathit{MDEL} ::= $\newline
$ CALLM \rightarrow DELM $  & 
$\mathit{cmethode(MDEL}) = \mathit{methode(CALLM)}$\newline
$\mathit{dmethode(MDEL}) = \mathit{methode(DELM)}$\newline
$\mathit{paramTargetTyp(DELM)} = \mathit{paramTyp(CALLM)}$\newline
$\mathit{returnTargetTyp(CALLM)} = \mathit{returnTyp(DELM)}$
\\
\hline
$\mathit{CALLM} ::=$\newline $m(SP):STPROXY $  &
$SR = \mathit{typ(STPROXY)} $\newline
$\mathit{methode(CALLM)} = \mathit{m(SP):SR} $\newline
$\mathit{paramTyp(CALLM) = SP}$\newline
$\mathit{targetTyp(STPROXY) = returnTargetTyp(CALLM)}$ \\
\hline
$\mathit{DELM} ::=$\newline $n(STPROXY):R $  & 
$DP = \mathit{typ(STPROXY)} $\newline
$\mathit{methode(DELM)} = \mathit{n(DP):R} $\newline
$\mathit{returnTyp(DELM) = R}$\newline
$\mathit{targetTyp(STPROXY) = paramTargetTyp(DELM)}$ \\
\hline
\hline
\end{tabular}
\caption{Grammatikregeln und Nebenbedingungen für die Definition eines strukturellen Proxies}
 \label{tab:structproxy}
\end{table}

\paragraph{Beispiel:}
Bezogen auf das Beispiel aus Abschnitt \ref{xmpl_prxy} soll ein \emph{struktureller Proxy} für den \emph{required Typ} $\texttt{MedicalFireFighter}$ aus den \emph{provided Typen} $\texttt{FireFighter}$ und $\texttt{Doctor}$ generiert werden. Dazu gelte folgende Bedingung:
\begin{gather*}
	\text{\{}\texttt{FireFighter},\texttt{Doctor} \text{\}} \in \mathit{cover(\texttt{MedicalFireFighter}, ExampLe)}
\end{gather*}
Die Generierung der \emph{strukturellen Proxies} erfolgt dann in folgenden Schritten, beginnend mit dem Nichtterminal $\mathit{STRUCTPROXY}$. Bei jedem Schritt werden ein oder mehrere Nichtterminal ersetzt und die Attribute aus den Nebenbedingungen soweit wie möglich zugewiesen.
\begin{gather*}
\begin{array}{l|l}
	\mathit{STRUCTPROXY} & \mathit{typ(STRUCTPROXY)} = \texttt{MedicalFireFighter}
\end{array}
\end{gather*}
\begin{gather*}
\begin{array}{l|l}
	\texttt{structproxy for MedicalFireFighter} & \mathit{typ(TARGET_1)} = \texttt{FireFighter}\\
	\texttt{\{} \mathit{TARGET_1} & \mathit{typ(TARGET_2)} = \texttt{Doctor} \\
	 \mathit{TARGET_2} \text{\}} &
\end{array}
\end{gather*}

\begin{gather*}
\begin{array}{l|l}
\texttt{structproxy for MedicalFireFighter} & \\
\texttt{\{} \texttt{FireFighter} \texttt{\{} & \\
\mathit{MDEL_1}\texttt{\}} & \\
\texttt{Doctor} \texttt{\{} & \\
\mathit{MDEL_2}\texttt{\}} \text{\}} &
\end{array}
\end{gather*}
\begin{gather*}
\begin{array}{l|l}
\texttt{structproxy for MedicalFireFighter} & \\
\texttt{\{} \texttt{FireFighter} \texttt{\{} & \\
\mathit{CALLM_1} \rightarrow \mathit{DELM_1}\texttt{\}} & \\
\texttt{Doctor} \texttt{\{} & \\
\mathit{CALLM_2} \rightarrow \mathit{DELM_2}\texttt{\}} \text{\}} &
\end{array}
\end{gather*}

\subsubsection{Container-Proxy}
\subsubsection{Content-Proxy}
\subsubsection{Sub-Proxy}
\subsubsection{Simple-Proxy}
Ein \emph{Simple-Proxy} kann sowohl auf der Basis des \emph{ExactTypeMatchers} als auch auf Basis des \emph{GenTypMatchers} generiert werden. Die für den \emph{Simple-Proxy} relevanten Grammatikregeln aus Abschnitt \ref{sec:proxygram} werden für die Generierung noch um Nebenbedingungen erweitert. Die vollständige Liste der Regeln und Nebenbedingungen für den \emph{Simple-Proxy} ist Tabelle \ref{tab:simpleproxy} zu entnehmen.

\begin{table}[H]
\centering
\begin{tabular}{|p{4cm}|p{10cm}|}
\hline
\hline
\centering\textbf{Regel} & \textbf{Nebenbedingungen} \\
\hline
\hline
$\mathit{STPROXY} ::= NPX$ &
$\mathit{typ(STPROXY)}=\mathit{typ(NPX)}$\newline
$\mathit{targetTyp(STPROXY)} = \mathit{targetTyp(NPX)}$\\
\hline
$\mathit{NPX} ::=$ \newline $\texttt{simpleproxy } \texttt{for } P$ & 
$\mathit{targetTyp(NPX)} \Rightarrow_{exact} P $\newline
$\mathit{typ(NPX)}=P$\newline
$\mathit{methoden(NPX)} = \mathit{methoden(P)}$\\
\hline
$\mathit{NPX} ::=$ \newline $\texttt{simpleproxy } \texttt{for } P$ & 
$\mathit{targetTyp(NPX)} \Rightarrow_{gen} P $\newline
$\mathit{typ(NPX)}=P$\newline
$\mathit{methoden(NPX)} = \mathit{methoden(P)}$
\\
\hline
\hline
\end{tabular}
\caption{Regeln und Nebenbedingungen für Simple-Proxies}
 \label{tab:simpleproxy}
\end{table}


Als Beispiel wird von dem Matching ausgegangen, welches auch bei der Beschreibung des \emph{ExactTypeMatchers} (Abschnitt \ref{sec:exacttypematcher}) angeführt wurde:
\begin{gather*}
\texttt{Fire} \Rightarrow_{exact} \texttt{Fire}
\end{gather*}

 
Der \emph{Simple-Proxy} basiert auf zwei Matchern, von denen 
Die Basis für den \emph{Simple-Proxy} sind sowohl der \emph{•}

\section{alter kram}
\subsection{StructuralTypeMatcher}

Ein struktureller Proxy für ein \emph{required Interface} $R$ aus einer Menge von  \emph{provided Typen} $P$ wird durch folgende Regeln und Nebenbedingungen beschrieben:
\begin{table}[H]
\centering
\begin{tabular}{|p{4cm}|p{10cm}|}
\hline
\hline
\centering\textbf{Regel} & \textbf{Nebenbedingungen} \\
\hline
\hline
$\mathit{STRUCTPROXY} ::=$\newline $\texttt{structproxy } \texttt{for } R$\newline$ \texttt{\{}TARGET_1 \texttt{ ... }$ \newline
$ TARGET_n\texttt{\}}$ & $typ(\mathit{STRUCTPROXY})=R$\newline
$methoden(\mathit{STRUCTPROXY}) = $\newline
 $\mathit{cmethoden(TARGET_1)} \cup \texttt{...} \cup \mathit{cmethoden(TARGET_n)} $\newline
$methoden(R) = methoden(\mathit{STRUCTPROXY})$\\
\hline
$\mathit{TARGET} ::=$\newline $P \texttt{ \{}MDEL_1 \texttt{ ...}$\newline
$ MDEL_n\texttt{\}}$ &
$\mathit{typ(TARGET)} = P$ \newline
$\mathit{cmethoden(TARGET)} = $\newline
 $\mathit{cmethode(MDEL_1)} \cup \texttt{...} \cup \mathit{cmethode(MDEL_n)}$ \newline
$\mathit{dmethoden(TARGET)} =  $\newline 
 $\mathit{dmethode(MDEL_1)} \cup \texttt{...} \cup \mathit{dmethode(MDEL_n)}$ \newline
 $\mathit{dmethoden(TARGET)} \subseteq \mathit{methoden(P)} $ \\
\hline
$\mathit{MDEL} ::= $\newline
$ CALLM \rightarrow DELM $  & 
$\mathit{cmethode(MDEL}) = \mathit{methode(CALLM)}$\newline
$\mathit{dmethode(MDEL}) = \mathit{methode(DELM)}$\newline
$\mathit{paramTargetTyp(DELM)} = \mathit{paramTyp(CALLM)}$\newline
$\mathit{returnTargetTyp(CALLM)} = \mathit{returnTyp(DELM)}$
\\
\hline
$\mathit{CALLM} ::=$\newline $m(SP):STPROXY $  &
$SR = \mathit{typ(STPROXY)} $\newline
$\mathit{methode(CALLM)} = \mathit{m(SP):SR} $\newline
$\mathit{paramTyp(CALLM) = SP}$\newline
$\mathit{targetTyp(STPROXY) = returnTargetTyp(CALLM)}$ \\
\hline
$\mathit{DELM} ::=$\newline $n(STPROXY):R $  & 
$DP = \mathit{typ(STPROXY)} $\newline
$\mathit{methode(DELM)} = \mathit{n(DP):R} $\newline
$\mathit{returnTyp(DELM) = R}$\newline
$\mathit{targetTyp(STPROXY) = paramTargetTyp(DELM)}$ \\
\hline
\hline
\end{tabular}
\caption{Grammatik für die Definition eines Proxies}
 \label{tab:eIShort}
\end{table}
\noindent
Regeln für das Nonterminal \emph{STPROXY} unterliegen Nebenbedingungen, die teilweise erst unter Zuhilfenahme der folgenden Matcher erfüllt werden können.


\subsection{SpecTypeMatcher}
Ein Proxy für einen Typ $T$, der mit einem Target-Typ $T'$ mit $T \Rightarrow_{spec} T'$ erzeugt werden soll, ist ein \emph{Sub-Proxy} und wird durch die folgenden Regeln und Nebenbedingungen beschrieben: 
\begin{table}[H]
\centering
\begin{tabular}{|p{5cm}|p{9cm}|}
\hline
\hline
\centering\textbf{Regel} & \textbf{Nebenbedingungen} \\
\hline
\hline
$\mathit{NPX} ::=$ \newline $\texttt{subproxy } \texttt{for } P$\newline $\texttt{with } P' \texttt{ \{}\mathit{NOMMDEL_1}$\newline
$ \texttt{... }\mathit{NOMMDEL_n}\texttt{\}}$ & 
$\mathit{targetTyp(NPX)} = P'$\newline
$\mathit{typ(NPX)} = P$\newline
$P \Rightarrow_{spec} P'$\newline
$\mathit{methoden(NPX)} = \mathit{cmethode(NOMMDEL_1)} \cup $\newline
$\texttt{...} \cup \mathit{cmethode(NOMMDEL_n)}$ \newline
$\mathit{methoden(NPX)} \subseteq \mathit{methoden(P)}  $\newline 
$\mathit{methoden(P')} \supseteq \mathit{dmethode(NOMMDEL_1)} \cup  $\newline 
$\texttt{...} \cup \mathit{dmethode(NOMMDEL_n)}$ 
\\
\hline
$\mathit{NOMMDEL} ::=$\newline
$\mathit{m(SP):SR} \rightarrow$\newline
$ \mathit{m(TP):TR} $  &
$SP >= TP$\newline
$SR <= TR$\newline
$\mathit{cmethode(MOMMDEL)} = \mathit{m(SP):SR}$\newline
$\mathit{dmethode(MOMMDEL)} = \mathit{m(TP):TR}$ \\
\hline
\hline
\end{tabular}
\caption{Regeln und Nebenbedingungen für Sub-Proxies}
 \label{tab:subAttr}
\end{table}
\subsection{ContentTypeMatcher}
Die Matchingrelation für diesen Matcher wird durch folgende Regel beschrieben:
\begin{gather*}
\frac{\exists f:T''\in felder(T'). T \Rightarrow_{internCont} T''}{T \Rightarrow_{content} T'}
\end{gather*}
Für die Relation $\Rightarrow_{internCont}$ gilt dabei:
\begin{gather*}
\frac{T \Rightarrow_{exact} T' \vee T \Rightarrow_{gen} T' \vee
T \Rightarrow_{spec} T'  }{T \Rightarrow_{internCont} T'}
\end{gather*}
Ein Proxy für einen Typ $P$, der mit einem Target-Typ $P'$ mit $P \Rightarrow_{content} P'$ erzeugt werden soll, ist ein \emph{Content-Proxy} und wird durch die folgenden Regeln und Nebenbedingungen beschrieben:
\begin{table}[H]
\centering
\begin{tabular}{|p{4cm}|p{10cm}|}
\hline
\hline
\centering\textbf{Regel} & \textbf{Nebenbedingungen} \\
\hline
\hline
$\mathit{STPROXY} ::=$\newline $\texttt{contentproxy }  \texttt{for } P$\newline $\texttt{with } P' \texttt{ \{}CEMDEL_1$\newline
$\texttt{... } CEMDEL_n\texttt{\}}$ &
$\mathit{typ(STPROXY)}= P$\newline
$\mathit{targetTyp(STPROXY)}= P'$\newline
$P \Rightarrow_{content} P'$\newline
$\mathit{methoden(STPROXY)} = \mathit{cmethode(CEMDEL_1)} \cup $\newline
$\texttt{...} \cup \mathit{cmethode(CEMDEL_n)}$ \newline
$\mathit{methoden(STPROXY)} \subseteq \mathit{methoden(P)}  $\newline
$\mathit{containerType(CEMDEL_1)} = P'$ \newline
$ \texttt{... } \mathit{containerType(CEMDEL_n)} = P' $ \\
\hline
$\mathit{CEMDEL} ::=$\newline
$\mathit{CECALLM} \rightarrow$\newline
$\mathit{f.CEDELM} $  &
$\mathit{f:FT} \in \mathit{felder(containerType(CEMDEL))}$ \newline
$\mathit{methode(CEDELM)} \in \mathit{methoden(FT)} $\newline
$\mathit{paramTargetTyp(CEDELM)} = \mathit{paramTyp(CECALLM)}$\newline
$\mathit{returnTargetTyp(CECALLM)} = \mathit{returnTyp(CEDELM)}$\\
\hline
$\mathit{CECALLM} ::=$\newline
$\mathit{m(SP):NPX} $  &
$\mathit{paramTyp(CECALLM)} = \mathit{SP}$\newline
$SR = \mathit{typ(NPX)}$\newline
$\mathit{targetTyp(NPX)} = \mathit{returnTargetTyp(CECALLM)}$\newline
$\mathit{methode(CECALLM)} = \mathit{m(SP):SR}$\\
\hline
$\mathit{CEDELM} ::=$\newline
$\mathit{m(NPX):TR} $  &
$\mathit{returnTyp(CEDELM)} = \mathit{TR} $\newline
$\mathit{TP} = \mathit{typ(NPX)} $\newline
$\mathit{targetTyp(NPX)} = \mathit{paramTargetTyp(CEDELM)} $\newline
$\mathit{methode(CEDELM)} =\mathit{m(TP):TR} $\\
\hline
\hline
\end{tabular}
\caption{Regeln und Nebenbedingungen für Contentproxies}
 \label{tab:contentAttr}
\end{table}

\subsection{ContainerTypeMatcher}
Die Matchingrelation für diesen Matcher wird durch folgende Regel beschrieben:
\begin{gather*}
\frac{\exists f:T''\in felder(T). T'' \Rightarrow_{internCont} T'}{T \Rightarrow_{container} T'}
\end{gather*}
Ein Proxy für einen Typ $P$, der mit einem Target-Typ $P'$ mit $P \Rightarrow_{container} P'$ erzeugt werden soll, ist ein \emph{Container-Proxy} und wird durch die folgenden Regeln und Nebenbedingungen beschrieben:
\begin{table}[H]
\centering
\begin{tabular}{|p{6cm}|p{8cm}|}
\hline
\hline
\centering\textbf{Regel} & \textbf{Nebenbedingungen} \\
\hline
\hline
$\mathit{STPROXY} ::=$\newline $\texttt{containerproxy} \texttt{ for } P$\newline
$ \texttt{with } P' \texttt{ \{} f = \mathit{NPX} \texttt{\}}$ & 
$\mathit{targetTyp(STPROXY)} = P'$\newline
$\mathit{typ(STPROXY)} = P$\newline
$P \Rightarrow_{container} P'$\newline
$\mathit{f:FT} \in \mathit{felder(P)}$\newline
$\mathit{targetTyp(NPX)} = P'$\newline
$\mathit{typ(NPX)} = FT$\\
\hline
\hline
\end{tabular}
\caption{Regeln und Nebenbedingungen für Container-Proxies}
 \label{tab:containerAttr}
\end{table}
\section{Erweiterung um einen DVMatcher}
Die o.g. Struktur für die Definition von Typen wird die Definition von \emph{provided Typen} erweitert.
\begin{table}[H]
\centering
\begin{tabular}{|p{5cm}|p{9cm}|}
\hline
\hline
\centering\textbf{Regel} & \textbf{Erläuterung} \\
\hline
\hline
$\mathit{PD} ::= \newline \texttt{provided } T \texttt{ extends }T' \newline \texttt{\{} \mathit{FD}\text{*} \mathit{MD}\text{*} \mathit{FCD}\text{?}\texttt{\}}$& Die Definition eines provided Typen besteht aus dem Namen des Typen \emph{T}, dem Namen des Super-Typs \emph{T'} von \emph{T} sowie mehreren Feld- und Methodendeklarationen und einer optionalen Definition eines factory Typen.\\
\hline
$\mathit{FCD} ::= \texttt{factory }T \texttt{ \{}\newline \mathit{FD}\text{*} \mathit{MD}\text{*} \texttt{\}} $& Die Definition eines factory Typen besteht aus dem Namen des Typen \emph{T} sowie mehreren Feld- und Methodendeklarationen.\\
\hline
\hline
\end{tabular}
\caption{Erweiterte Struktur für die Definition einer Bibliothek von Typen}
 \label{tab:extTDv}
\end{table}
\noindent
Darüber hinaus wird folgende Funktion definiert:
\begin{gather*}
\mathit{fabriken(T)} :=  \left\{ 
				\begin{array}{l|l}
					F & F \text{ ist ein \emph{factory Typ}, der in \emph{T} definiert wurde}
				\end{array}
              \right\}
\end{gather*}
Weiterhin muss die Struktur für die Definition von Proxies um eine weitere Definition für einen \emph{Single-Target-Proxy} erweitert werden.
\begin{table}[H]
\centering
\begin{tabular}{|p{4.5cm}|p{9.5cm}|}
\hline
\centering\textbf{Regel} & \textbf{Erläuterung} \\
\hline
\hline
$\mathit{STPROXY} ::=$\newline 
$\texttt{dvproxy }  \texttt{for } P$\newline
$\texttt{with } F \texttt{ on } \texttt{m(}\mathit{P'}\texttt{):}\mathit{P}$ & Ein \emph{DV-Proxy} ist ein Single-Target-Proxy, der für ein \emph{provided Typ P} erzeugt wird. Die Methodenaufrufe auf diesem Proxy werden an das Objekt delegiert, welches über die Methode \emph{m} des Factory-Typen \emph{F} aus dem Target-Typen \emph{P'} erzeugt wird.\\
\hline
\hline
\end{tabular}
\caption{Erweiterung der Struktur für die Definition eines Proxies}
 \label{tab:extPDv}
\end{table}
\noindent
Die Matchingrelation $\Rightarrow_{dv}$ wird über folgende Regel beschrieben:
\begin{gather*}
\frac{\exists F \in fabriken(T). \exists m(T'):T}{T \Rightarrow_{dv} T'}
\end{gather*}
Darüber hinaus müssen einige der oben beschriebenen Regeln angepasst werden, damit der \emph{ContainerTypeMatcher}, der \emph{ContentTypeMatcher} und der \emph{StructuralTypeMatcher} den \emph{DVMatcher} verwenden:
\begin{gather*}
\frac{T \Rightarrow_{exact} T' \vee T \Rightarrow_{gen} T' \vee
T \Rightarrow_{spec} T' \vee T \Rightarrow_{dv} T'  }{T \Rightarrow_{internCont} T'}
\end{gather*}
\begin{gather*}
\frac{T \Rightarrow_{internCont} T' \vee T \Rightarrow_{content} T' \vee
T \Rightarrow_{container} T'  }{T \Rightarrow_{internStruct} T'}
\end{gather*}
Ein Proxy für einen Typ $P$, der mit einem \emph{Target-Typ} $P'$ mit $P \Rightarrow_{dv} P'$ erzeugt werden soll, ist ein DV-Proxy und wird durch die folgenden Regeln und Nebenbedingungen beschrieben:
\begin{table}[H]
\centering
\begin{tabular}{|p{6cm}|p{8cm}|}
\hline
\centering\textbf{Regel} & \textbf{Nebenbedingungen} \\
\hline
\hline
$\mathit{STPROXY} ::=$\newline 
$\texttt{dvproxy }  \texttt{for } P$\newline
$\texttt{with } F \texttt{ on } \texttt{m(}\mathit{P'}\texttt{):}\mathit{P}$ 
& 
$\mathit{targetTyp(STPROXY) = P'}$\newline
$\mathit{typ(STPROXY) = P}$\newline
$\mathit{P} \Rightarrow_{dv} \mathit{P'}$\newline
$\mathit{F} \in \mathit{fabriken(P)}$\\
\hline
\hline
\end{tabular}
\caption{Regeln und Nebenbedingungen für DV-Proxies}
 \label{tab:dvAttr}
\end{table}

\end{document}