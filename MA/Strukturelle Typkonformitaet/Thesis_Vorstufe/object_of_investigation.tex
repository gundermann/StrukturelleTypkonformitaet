\section{Gegenstand dieser Arbeit}
In dieser Arbeit soll jedoch nicht das gesamte Internet als Quelle oder Repository f�r die Codesuche dienen. Vielmehr wird der Suchbereich weiter eingeschr�nkt. \\\\
Es wird von einem System ausgegangen, in dem ein EJB-Container zur Verf�gung steht. Die Suche soll sich auf die Menge der angemeldeten Bean-Implementierungen beschr�nken. Die angemeldeten Bean-Implementierungen stellen damit die Menge der angebotenen Komponenten dar. Dabei wird eine angebotenen Komponente als Kombination eines Interfaces, welches die Schnittstelle f�r die Aufrufer definiert, und einer Implementierung des Interfaces. Das Interfaces einer angebotenen Komponenten wird im Folgenden auch als angebotenes Interfaces bezeichnet. Die Beans werden bspw. als Provider f�r Informationen oder im weitesten Sinne  auch als Services verwenden, die von unterschiedlichen Komponenten des Systems verwendet werden. Bei der Entwicklung bzw. Weiterentwicklung einer Komponente kann es zu folgenden Szenario kommen, welches durch die Erf�llung der unten aufgef�hrten Annahmen charakterisiert wird:
\begin{itemize}
\item Es werden Informationen und Services ben�tigt, bei denen der Entwickler davon ausgehen kann, dass es innerhalb des Systems angebotene Komponenten gibt, die diese Informationen liefern k�nnen bzw. die Services erf�llen.
\item Der Entwickler wei� nicht, �ber welche konkreten angebotenen Komponenten er die Informationen abfragen bzw. die Services in Anspruch nehmen kann.
\end{itemize}
\subsection{Funktionale Anforderungen}
In dieser Arbeit soll ein Konzept entwickelt werden, welches dem entwickler erm�glicht die Erwartungen an die angebotenen Komponenten zu spezifizieren. Darauf aufbauend soll ein Algorithmus vorgeschlagen werden, welcher die angebotenen Komponenten zur Laufzeit hinsichtlich der spezifizierten Erwartungen des Entwicklers evaluiert und eine Auswahl derer trifft, die diese Erwartungen erf�llen. Da die Evaluation zur Laufzeit durchgef�hrt wird, kann der Entwickler anders als bei den oben genannten Arbeiten nicht aus einer Liste von Vorschl�gen ausw�hlen, welche der evaluierten Komponenten letztendlich verwendet werden soll. Diese Entscheidung ist durch den Algorithmus zu treffen.
\subsection{Nichtfunktionale Anforderungen}
Aufgrund bestimmter Konfigurationen des Gesamtsystem gibt es folgende weitere nichtfunktionale Anforderungen:
\begin{itemize}
\item Die Suche muss innerhalb des Transaktionstimeouts von 5 Minuten zu einem Ergebnis f�hren.
\item Die Suche soll hinsichtlich der Besonderheiten des System, in dem sie verwendet wird, angepasst werden k�nnen. (Bspw. bei der Verwendung bestimmter Typen, deren Fachlogik bei der Suche nicht untergraben werden darf.)
\item Bei einem Fehlschlag der Suche, sollen dem Entwickler Informationen zur Verf�gung gestellt werden, die eine zielgerichtete Anpassung seiner spezifizierten Erwartungen erlauben.
\end{itemize}