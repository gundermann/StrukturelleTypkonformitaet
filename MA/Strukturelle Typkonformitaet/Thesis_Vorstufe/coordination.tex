\subsection{Koordination}
Der Einsatz der Heuristiken muss bei der Suche koordiniert werden. Der Explorationsalgorithmus ist so aufgebaut, dass im 2. Schritt der 2. Stufe (siehe \ref{sem_eval_step2}) die Heuristiken zum Einsatz kommen.\\\\
Begonnen wird mit der Heuristik TMR\_Quant, sodass zuerst alle möglichen Kombinationen von Methoden-Konvertierungsvarianten ermittelt werden, die aus einer einzelnen Typ-Konvertierungsvariante stammen. Sind die möglichen Kombinationen von Methoden-Konvertierungsvarianten ausgeschöpft, wird der Prozess mit der Ermittlung aller möglichen Kombinationen von Methoden-Konvertierungsvarianten, die aus 2 Typ-Konvertierungsvarianten stammen, wiederholt. Die Anzahl der zu kombinierenden Typ-Konvertierungsvarianten wird somit jedes mal erhört, wenn die bereits ermittelten Kombinationen von Methoden-Konvertierungsvarianten ausgeschöpft sind.\\\\
Innerhalb eines der eben beschriebenen Iterationsschritte werden die anderen Heuristiken eingesetzt.\\\\
Die Heuristik TMR\_Qual sortiert die Typ-Konvertierungsvarianten, die bei der Ermittlung der Kombinationen von Methoden-Konvertierungsvarianten verwendet werden. Diese werden dann ihrer Reihenfolge entsprechend verwendet um Methoden-Konvertierungsvarianten zu ermittelt. Diesen Methoden-Konvertierungsvarianten wurde ebenfalls ein Type-Matcher Rating mitgegeben, nach welchem jene nun sortiert werden und dann entsprechend dieser Reihenfolge sequentiell getestet werden.\\\\
Die Heuristik PREV\_PASSED wird, wie alle weiteren Heuristiken, erst nach der ersten Iterationsstufe eingesetzt. Das liegt daran, dass für die Anwendung dieser Heuristiken bereist Testergebnisse vorliegen müssen. PREV\_PASSED sorgt nochmals für eine Umsortierung der Typ-Konvertierungsvarianten, die bei der Ermittlung der Kombinationen von Methoden-Konvertierungsvarianten verwendet werden, sodass die bevorzugten Typ-Konvertierungsvarianten zuerst verwendet werden.\\\\
Die Heuristiken BL\_PM und BL\_SM sorgen dafür, dass bei der Kombination von Methoden-Konvertierungsvarianten diejenigen übersprungen werden, die laut der jeweiligen Heuristik nicht mehr in Betracht gezogen werden sollen.
