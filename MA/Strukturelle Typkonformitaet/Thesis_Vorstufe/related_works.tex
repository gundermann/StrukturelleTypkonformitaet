\section{Verwandte Arbeiten}
Ein solcher Ansatz wurde bereits in \cite{sourcerer} von Bajaracharya et al.  verfolgt. Diese Gruppe entwickelte eine Search Engine namens Sourcerer, welche Suche von Open Source Code im Internet erm�glichte. Darauf aufbauend wurde von derselben Gruppe in \cite{Lemos} ein Tool namens CodeGenie entwickelt, welches einem Softwareentwickler die Code Suche �ber ein Eclipse-Plugin erm�glicht. In diesem Zusammenhang wurde erstmals der Begriff der Test-Driven Code Search (TDCS) etabliert. Parallel dazu wurde in Verbindung mit der Dissertation Oliver Hummel \cite{hummel08} ebenfalls eine Weiterentwicklung von Sourcerer ver�ffentlicht, welche unter dem Namen Merobase bekannt ist, welches ebenfalls das Konzept der TDCS verfolgt. TDCS beruht grundlegend darauf, dass der Entwickler Testf�lle spezifiziert, die im Anschluss verwendet werden, um relevanten Source Code aus einem Repository hinsichtlich dieser Testf�lle zu evaluieren. Damit kann das jeweilige Tool dem Entwickler Vorschl�ge f�r die Wiederverwendung bestehenden Codes unterbreiten.\\\\
Bezogen auf die am Ende des vorherigen Abschnitts formulierte �berlegung erm�glichen die genannten Search Engines, das Internet nach bestehendem Source Code zu durchsuchen und damit bereits bestehende Implementierungen f�r eine nachfragende Komponente zu ermitteln. 