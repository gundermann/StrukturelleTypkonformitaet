\chapter{Motivation}
In gr��eren Software-Systemen ist es �blich, dass mehrere Komponenten miteinander �ber Schnittstellen kommunizieren. In der Regel werden diese Schnittstellen so konzipiert, dass sie Informationen oder Services anbieten, die von anderen Komponenten abgefragt bzw. benutzt werden k�nnen. Dabei wird zwischen der Komponente, welche die Schnittstelle implementiert, als angebotene Komponente und der Komponente, welche die Schnittstelle nutzen soll, als nachfragende Komponente unterschieden (siehe \abbref{motiv}). 
\myBigFigure{motiv}{Abh�ngigkeiten von nachfragenden und angebotenen Komponenten}{motiv}
\noindent
Wird von einer nachfragenden Komponente eine Information ben�tigt, die in dieser Form noch nicht angeboten wird, so wird h�ufig ein neues Interface f�r diese ben�tigte Information erstellt, welches dann passend dazu implementiert wird. Dabei muss neben der Anpassung der nachfragenden Komponente auch eine Anpassung oder Erzeugung der anbietenden Komponente erfolgen und zus�tzlich das neue Interface deklariert werden. Zudem bedingt eine nachtr�gliche �nderung der neuen Schnittstelle ebenfalls eine Anpassung der drei genannten Artefakte.\\\\
In einem gro�en Software-System mit einer Vielzahl von bestehenden Schnittstellen ist eine gewisse Wahrscheinlichkeit gegeben, dass die Informationen oder Services, die von einer neuen nachfragenden Komponente ben�tigt werden, in einer �hnlichen Form bereits existieren. Das Problem ist jedoch, dass die manuelle Evaluation der Schnittstellen mitunter sehr aufwendig bis, aufgrund von unzureichender Dokumentation und Kenntnis �ber die bestehenden Schnittstellen, unm�glich ist.\\\\
Weiterhin ist es denkbar, dass ein Software-System auf unterschiedlichen Maschinen verteilt wurde und dadurch Teile des Systems ausfallen k�nnen. Das hat zur Folge, dass die Implementierung bestimmter Schnittstellen nicht erreichbar ist. Dadurch, dass eine Schnittstelle durch eine nachfragende Komponente explizit referenziert wird, kann eine solche Komponente nicht korrekt arbeiten, wenn die Implementierung der Schnittstelle nicht erreichbar ist, obwohl die ben�tigten Informationen und Services vielleicht durch andere Schnittstellen, deren Implementierung durchaus zur Verf�gung stehen, bereitgestellt werden k�nnten.\\\\
Dies f�hrt zu der �berlegung, ob es nicht m�glich ist, dass eine nachfragende Komponente einfach selbst spezifizieren kann, welche Informationen oder Services sie erwartet, wodurch auf der Basis dieser Spezifikation eine passende anbietende Komponente gefunden werden kann.