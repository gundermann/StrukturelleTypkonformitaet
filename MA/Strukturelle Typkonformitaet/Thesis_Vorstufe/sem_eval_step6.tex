\subsubsection{Auswertung des Testergebnisses}\label{sem_eval_step6}
Sofern alle Tests erfolgreich durchgelaufen sind, wird die aktuell selektierte ben�tigte Komponente als passend bewertet und als Ergebnis des Explorationsalgorithmus zur�ckgegeben.\\\\
Sollte einer der Tests nicht erfolgreich sein, wird die semantische Evaluation ab Schritt 3 (siehe \ref{sem_eval_step3}) wiederholt. Sofern keine ben�tigten Komponenten mehr erzeugt werden k�nnen, ist die Suche nach einer passenden Komponente gescheitert.\\\\
Da die Suche zur Laufzeit ausgef�hrt wird, reicht es, wenn eine passende Komponente gefunden wird. Selbst wenn es mehrere passende Komponenten geben sollte, g�be es in dem beschriebenen Verfahren keine M�glichkeit festzustellen, welche die semantischen Anforderungen besser erf�llt. Zwar w�ren k�nnte man die passenden Komponenten hinsichtlich der ben�tigten Systemressourcen untersuchen. Aufgrund der Vielzahl von m�glichen Kombinationen (siehe \ref{sem_eval_step2} und \ref{sem_eval_step3}) rechtfertigt der daf�r notwendige Aufwand den daraus resultierenden Performancegewinn vermutlich nicht.
