\section{Testergebnis basierte Heuristiken}
Diese Heuristiken werden auf der Basis des TestResult-Objektes, welches bei der semantischen Evaluation (2. Stufe, siehe \ref{sem_eval}) bei der Durchf�hrung der Tests (Schritt 5, siehe \ref{sem_eval_step5}) erzeugt wird. Die daf�r notwendigen Informationen werden im TestResult-Objekt dementsprechend bei der Testausf�hrung vermerkt. Ausgehend von dieser Basis f�hren diese Heuristiken im Allgemeinen dazu, dass bestimmte Methoden-Konvertierungsvarianten bei der weiteren Suche nach Kombinationen solcher (Schritt 2, , siehe \ref{sem_eval_step2}) nicht mehr oder bevorzugt verwendet werden.
\subsection{PREV\_PASSED: Beachtung der teilweise bestandenen Tests}
Es wird davon ausgegangen, dass es innerhalb der Testklassen Testmethoden gibt, die einzelne erwartete Methoden testen. Eine ben�tigte Komponente, die einen Teil dieser Tests besteht, verwendet f�r bestimmte Methoden scheinbar eher passende Methoden-Konvertierungsvarianten, als solche ben�tigten Komponenten, die keine dieser Tests bestehen.\\\\
Sofern sichergestellt ist, dass die ben�tigte Komponente aus einer einzigen Typ-Konvertierungsvariante erzeugt wurde und einen Teil der Tests besteht, sollte sie bei der Erzeugung ben�tigter Komponenten aus mehreren Typ-Konvertierungsvarianten bevorzugt verwendet werden.
\subsection{BL\_PM: Beachtung aufgerufener Pivot-Methode}
Es wird davon ausgegangen, dass es beim Aufruf von Methoden und deren Delegation an angebotene Komponenten zu Fehlern/Exceptions kommen kann. Eine Methoden-Konvertierungsvariante, die zu solchen Fehlern f�hrt, ist offensichtlich unbrauchbar. Daher ist es sinnvoll, diese Methoden-Konvertierungsvariante bei der weiteren Suche zu ignorieren. Zu diesem Zweck wird beim Auftreten einer Exception bei der Delegation einer Methode eine spezielle Excpetion (SigMaGlueException) geworfen, die bei der Testdurchf�hrung entsprechend ausgewertet werden kann.\\\\
Da in einer Testmethode jedoch mehrere erwartet Methoden aufgerufen werden k�nnen, besteht die M�glichkeit, dass das Ergebnis der zuerst aufgerufenen erwarteten Methoden aufgrund einer passenden Methoden-Konvertierungsvariante nicht direkt bei deren Aufruf zu einer Exception f�hrt, sondern erst bei der Verwendung des Ergebnisses als Parameter des folgenden Aufrufs einer erwarteten Methode. In so einem Fall ist es nicht m�glich zu erkennen, f�r welche der beiden Methoden tats�chlich eine unpassende Methoden-Konvertierungsvariante verwendet wird. Beim Auftreten einer Exception w�hrend des Aufrufs der ersten erwarteten Methode, ist jedoch davon auszugehen, dass f�r diesen Aufruf eine unpassende Methoden-Konvertierungsvariante verwendet wurde, weshalb diese bei der weiteren Suche ignoriert werden sollte.\\\\
Eine Pivot-Methode beschreibt dabei die zuerst aufgerufene erwartete Methode innerhalb einer Testmethode. Um eine M�glichkeit zu schaffen, den Aufruf dieser Pivot-Methode nach au�en mitzuteilen, m�ssen die Testklassen erweitert werden. Hierzu steht das Interface PivotMethodTestInfo bereit.\\\\
Dieses Interface deklariert drei Methoden, die in der Testklasse spezifiziert werden m�ssen. Grundlegend ist der Mechanismus so angedacht, dass innerhalb der Testklasse ein Flag spezifiziert wird, welches durch die Methode reset() auf den Ausgangswert zur�ckgesetzt wird und durch die Methode markPivotMethodCallExecuted() auf einen anderen Wert umgesetzt wird. Der Aufruf der Methode pivotMethodCallExecuted() sollte true liefern, wenn dieses Flag nicht dem Ausgangswert �bereinstimmt.\\\\
Innerhalb der Testmethode sollte dann vor zu Beginn immer die Methode reset() aufgerufen werden, da andernfalls die Testergebnisse verf�lscht werden k�nnen. Zudem muss die Methode markPivotMethodCallExecuted() direkt nach dem Aufruf der Pivot-Method aufgerufen werden.\\\\
So kann bei der Testdurchf�hrung festgestellt werden, ob die m�gliche Exception vor oder nach dem Aufruf der Pivot-Methode erfolgte. Welche Methode beim Aufruf zu einer Exception gef�hrt hat, wird innerhalb der SigMaGlueException �berliefert.
\subsection{BL\_SM: Beachtung fehlgeschlagener Single-Method Tests}
Es wird davon ausgegangen, dass es innerhalb der Testklassen Testmethoden gibt, die auf eine ganz bestimmte erwartete Methode zugeschnitten sind (Single-Method Test). Das setzt unter anderem voraus, dass in dieser Testmethode von den erwarteten Methoden nur diese eine aufgerufen wird. \\\\
Weiterhin muss an der Testmethode eine Information zur Verf�gung stehen, die eine Auskunft dar�ber gibt, welche Methode dort getestet wird. Zu diesem Zweck kann an der @QueryTypeTest-Annotation ein Parameter mit der Bezeichnung testedSingleMethod spezifiziert werden. Dort soll dementsprechend der Name der getesteten Methode angegeben werden. So kann bei der Testausf�hrung evaluiert werden, ob eine bestimmte Methode von der getesteten ben�tigten Komponente semantisch nicht passt.\\\\
Da die konkrete Methode, deren Test fehlschl�gt, bekannt ist, kann auch die verwendete Methoden-Konvertierungsvariante ermittelt werden. Diese Methoden-Konvertierungsvariante sollte bei der weitere Suche nicht mehr beachtet werden, da mit diesem Test sichergestellt wurde, dass sie nicht Teil einer passenden ben�tigten Komponente sein kann.

