\subsubsection{Kombination von Typ-Konvertierungsvarianten}\label{sem_eval_step2}
In diesem Schritt werden die ermittelten Typ-Konvertierungsvarianten miteinander kombiniert, was einer Kombination der angebotenen Interfaces gleicht. Die Anzahl k der kombinierten Typ-Konvertierungsvarianten kann jedoch variieren. Wenn $|EM|$ die Anzahl der Methoden im erwarteten Interface ist, gilt f�r $k$:
\begin{align*}
1 \leq k \leq |EM|
\end{align*}
\noindent
Da $k$ variabel ist, wird dieser Schritt zusammen mit allen folgenden Schritten mitunter mehrfach durchlaufen. Die Nummer des jeweiligen Iterationsschrittes wird mit $k$ gleichgesetzt. \abbref{tkv_alv_alu_1} zeigt die Kombinationen von Typ-Konvertierungsvarianten, die sich bezogen auf die Beispiele aus \abbref{konvar_voll} und \abbref{konvar_unv} im ersten Durchlauf ergeben. Im zweiten Durchlauf w�rde sich nur eine Kombination von Typ-Konvertierungsvarianten ergeben, da die beiden Typ-Konvertierungsvarianten von AIv und AIu miteinander kombiniert werden (siehe \abbref{comb_tkv_alv_alu_1}).
\myFigure{tkv_alv_alu_1}{Kombinationen von Typ-Konvertierungsvarianten von AIu und AIv im ersten Durchlauf}{tkv_alv_alu_1}
\myFigure{comb_tkv_alv_alu_1}{Kombinationen von Typ-Konvertierungsvarianten von AIu und AIv im zweiten Durchlauf}{comb_tkv_alv_alu_1}
\noindent
So berechnet sich die Anzahl an ermittelten Kombinationen von Typ-Konvertierungsvarianten  ($|KombTKV|$) f�r jeden Durchlauf $k$ in Abh�ngigkeit von der Anzahl der in der 1. Stufe ermittelten Typ-Konvertierungsvarianten ($|TKV|$) wie folgt:
\begin{align*}
|KombTKV| = \frac{|TKV|! }{ ((|TKV|! - k!) * k!)}
\end{align*}

