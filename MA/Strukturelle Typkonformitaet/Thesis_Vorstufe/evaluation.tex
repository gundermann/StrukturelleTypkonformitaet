\chapter{Evaluierung}
Die Evaluierung erfolgt innerhalb von Systemen, in denen mindestens 889 angebotene Interfaces existieren. Es wird zwischen einem Test-System und einem Hei�-System unterschieden.\\\\
Das Test-System wurde vorrangig f�r die Evaluation der Type-Matcher Rating basierten Heuristiken verwendet, da f�r diese Heuristiken keine Implementierungen der angebotenen Interfaces vorliegen m�ssen.\\\\
Das Hei�-System wurde vorrangig f�r die Evaluation der testergebnis basierten Heuristiken verwendet, da hier zu jedem der 889 angebotenen Interfaces eine Implementierung existiert. Die angebotenen Komponenten wurden  im Hei�-System als Java Enterprise Beans umgesetzt.

\myparagraph{Darstellung der Evaluationsergebnisse}
Die Evaluationsergebnisse werden in der Form von Vier-Felder-Tafeln dargestellt (Beispiel siehe \tabref{vft:beispiel}). F�r jedes erwartete Interface wird eine Vier-Felder-Tafel f�r jeden Durchlauf des Explorationsalgorithmus aufgezeigt. Aus der jeweiligen Tafel geht hervor, wie viele Kombinationen von Methoden-Konvertierungsvarianten aus den Kombinationen der ermittelten Typ-Konvertierungsvarianten innerhalb des Durchlaufs erzeugt werden k�nnten. Die Nummer des Durchlaufs wird in der oberen rechten Ecke der Tafel abgebildet. In der Spalte positiv ist die Anzahl der Kombinationen von Methoden-Konvertierungsvarianten verzeichnet, die innerhalb des Durchlaufs tats�chlich erzeugt wurden. Die Zahl in der Spalte ``negativ'' dr�ckt hingegen aus, wie viele der Kombinationen aufgrund bestimmter Kriterien (bzw. Heuristiken) gar nicht erst erzeugt wurden.  Die Zeile ``falsch'' beschreibt die Anzahl der relevanten Kombinationen, aus denen ben�tigte Komponenten erzeugt werden, welche die semantischen Tests nicht bestehen. Dementsprechend stellt die Zeile ``richtig'' die Anzahl der Kombinationen dar, aus denen sich ben�tigte Komponenten erzeugen lassen, welche die semantischen Test bestehen. Der Fall, in dem eine Kombination nicht erzeugt wurde, aber dennoch f�r die Erstellung einer ben�tigten Komponente genutzt wurde und die semantischen Tests besteht, (negativ und richtig) kann nicht auftreten.\\\\
F�r die Anzahl der zu kombinierenden Methoden-Konvertierungsvarianten $MK$ wird der h�chste m�gliche Wert angenommen. Dieser ist von der Anzahl der angebotenen Methoden $am$ sowie der Anzahl der erwarteten Methoden $em$ abh�ngig und wird wie folgt berechnet: 
\begin{equation*}
MK = \frac{am!}{(am-em)!*em!}
\end{equation*}
\noindent
Die Anzahl der angebotenen Methoden $am$ ist wiederum abh�ngig von den angebotenen Interfaces deren Typ-Konvertierungsvarianten im jeweiligen Durchlauf miteinander kombiniert wurden. Die Anzahl der Kombinationen von Typ-Konvertierungsvarianten innerhalb des Durchlaufs sei mit $TK$ beschrieben. Der Wert f�r $TK$ berechnet sich in Abh�ngigkeit von der Nummer des Durchlaufs $d$ und der Anzahl der strukturell passenden angebotenen Interfaces $n$ (siehe auch Abschnitt Explorationskomponente, 2. Stufe, 2. Kombination von Typ-Konvertierungsvarianten).
\begin{equation*}
TK = \frac{n!}{(n-d)!*d!}
\end{equation*}
\noindent
Da die Anzahl der angebotenen Methoden von System zu System schwanken kann, sei die Funktion $am(TK)$ eine n�herungsweise Darstellung von $am$, in Abh�ngigkeit von der Anzahl der kombinierten Typ-Konvertierungsvarianten $TK$.\\\\
Da durch die Heuristiken letztendlich Methoden-Konvertierungsvarianten aus der Suche herausfallen, wird die Anzahl der entsprechenden Methoden-Konvertierungsvarianten in dem jeweiligen Feld der Vier-Felder-Tafeln als Funktion $mk(TK)$ dargestellt, die wie folgt definiert wird:
\begin{equation*}
mk(TK) = \frac{am(TK)!}{(am(TK)-em)!*em!}
\end{equation*}
\noindent
\tabref{vft:beispiel} zeigt ein Beispiel f�r eine solche Vier-Felder-Tafel, in der die Ergebnisse des 1. Durchlauf des Explorationsalgorithmus dargestellt sind. Dabei wurden Methoden-Konvertierungsvarianten aus 10 Kombinationen von Typ-Konvertierungsvarianten erzeugt. Den Methoden-Konvertierungsvarianten, die nicht beachtet wurden, lagen insgesamt 20 Typ-Konvertierungsvarianten  zugrunde. Weiterhin zeigt das Beispiel, dass es eine Kombination von Methoden-Konvertierungsvarianten gibt, aus der eine passende ben�tigte Komponente erzeugt werden konnte.
\vft{1}{$mk(10)$}{$mk(20)$}{1}{0}{Beispiel: Vier-Felder-Tafel}{vft:beispiel}


\section{Test-System}
Wie bereits erw�hnt werden im Test-System die 889 angebotenen Interfaces verwendet, die auch im Hei�-System verwendet werden. Dar�ber hinaus wurden noch 6 weitere angebotene Interfaces dem Test-System hinzugef�gt, um bestimmte Konstellationen gezielter zu evaluieren. Damit stehen in diesem System f�r die Evaluation der insgesamt wurde 6 erwartete Interfaces f�r die Evaluation verwendet. Die 6 erwarteten Interfaces wurden wie folgt deklariert (siehe Abbildung 16 - 21).
\subsection{Type-Matcher Rating basierte Heuristiken}
\subsubsection{Ausgangspunkt}
F�r ein erwarteten Interfaces konnten mehrere angebotene Interfaces gefunden werden, die eine strukturelle �bereinstimmung aufwiesen. Tabelle 1 zeigt die Anzahl der strukturell �bereinstimmenden angebotenen Interfaces je erwartetes Interface.
\begin{table}[H]
\centering
\small
\singlespacing
			\begin{tabular}[c]{|>{\centering\arraybackslash}p{2cm}|>{\centering\arraybackslash}p{5cm}|}
			\hline
			\hline
				 \textbf{erwartetes Interface} & \textbf{Anzahl strukturell �bereinstimmender angebotener Interfaces} \\
				\hline\hline
				TEI1 & 169 \\
				\hline
				TEI2 & 179\\
				\hline
				TEI3 & 187\\
				\hline
				TEI4 & 62\\
				\hline
				TEI5 & 60\\
				\hline
				TEI6 & 33\\
				\hline
				\hline
			\end{tabular} 
 \caption{Anzahl strukturell �bereinstimmender angebotener Interfaces je erwartetes Interfaces}
 \label{tab:amountMatchedInterfaces}
\onehalfspacing
\end{table}
\noindent
Die \tabsrefs{tmr_start_tei1}{tmr_start_tei6_2} zeigen Vier-Felder-Tafeln f�r die Durchl�ufe des Explorationsalgorithmus f�r die Suche nach einer jeweils passenden ben�tigten Komponente der erwarteten Interfaces aus \tabref{amountMatchedInterfaces} ohne die Verwendung von Heuristiken. Dies stellt somit den Ausgangspunkt f�r die weitere Evaluation dar.
\begin{multicols}{3}
\vft{1}{$mk(169)$}{0}{1}{0}{Ausgangspunkt Test-System TMR f�r TEI1}{tmr_start_tei1}\columnbreak
\vft{1}{$mk(179)$}{0}{1}{0}{Ausgangspunkt Test-System TMR f�r TEI2}{tmr_start_tei2}\columnbreak
\vft{1}{$mk(187)$}{0}{1}{0}{Ausgangspunkt Test-System TMR f�r TEI3}{tmr_start_tei3}
\end{multicols}
\begin{multicols}{3}
\vft{1}{$mk(62)$}{0}{0}{0}{Ausgangspunkt Test-System TMR f�r TEI4 1. Durchlauf}{tmr_start_tei4_1}\columnbreak
\vft{1}{$mk(60)$}{0}{0}{0}{Ausgangspunkt Test-System TMR f�r TEI5 1. Durchlauf}{tmr_start_tei5_1}\columnbreak
\vft{1}{$mk(33)$}{0}{0}{0}{Ausgangspunkt Test-System TMR f�r TEI6 1. Durchlauf}{tmr_start_tei6_1}
\end{multicols}
\begin{multicols}{3}
\vft{2}{$mk(1891)$}{0}{1}{0}{Ausgangspunkt Test-System TMR f�r TEI4 2. Durchlauf}{tmr_start_tei4_2}\columnbreak
\vft{2}{$mk(1770)$}{0}{1}{0}{Ausgangspunkt Test-System TMR f�r TEI5 2. Durchlauf}{tmr_start_tei5_2}\columnbreak
\vft{2}{$mk(528)$}{0}{1}{0}{Ausgangspunkt Test-System TMR f�r TEI6 2. Durchlauf}{tmr_start_tei6_2}
\end{multicols}
\noindent
F�r die Interfaces TEI4 - TEI6 werden zwei Durchl�ufe ben�tigt, da die semantischen Test nur von einer ben�tigten Komponente bestanden werden, die auch einer Kombination zweier Typ-Konvertierungsvarianten erzeugt wurde.\\\\
Die Typ-Matcher Rating basierten Heuristiken sollten zu einer Reduktion der Anzahl von erzeugten Kombinationen von Methoden-Konvertierungsvarianten, die die semantischen Tests nicht bestehen w�rden, f�hren (positiv und falsch).
\subsubsection{Ergebnisse TMR\_Quant}
Durch die Verwendung der Heuristik TMR\_Quant kann f�r die ersten 3 erwarteten Interfaces eine Besserung erzielt werden. Der Grund daf�r ist, dass die ben�tigte Komponente, die letztendlich alle semantischen Tests besteht auf der Basis genau einer Typ-Konvertierungsvariante erzeugt wurde. Damit ben�tigt der Explorationsalgorithmus lediglich einen Durchlauf. TMR\_Quant sorgt dennoch daf�r, dass die erzeugten Kombinationen von Typ-Konvertierungsvarianten im 2. Schritt reduziert werden, da solche, die ein quantitatives Type-Matcher Rating von < 100\% haben nicht in die Ergebnismenge des 2. Schrittes einflie�en. Die unten aufgef�hrten Tafeln zeigen die Auswirkung auf die ersten drei erwarteten Interfaces (TEI1 - TEI3).
\begin{multicols}{3}
\vft{1}{$mk(29)$}{$mk(140)$}{1}{0}{TMR\_Quant Test-System TMR f�r TEI1}{tmr_quant_tei1}\columnbreak
\vft{1}{$mk(22)$}{$mk(157)$}{1}{0}{TMR\_Quant Test-System TMR f�r TEI2}{tmr_quant_tei2}\columnbreak
\vft{1}{$mk(24)$}{$mk(163)$}{1}{0}{TMR\_Quant Test-System TMR f�r TEI3}{tmr_quant_tei3}
\end{multicols}
\noindent
F�r die anderen erwarteten Interfaces (TEI4 - TEI6) kann durch diese Heuristik h�chstens f�r den ersten Durchlauf eine eine Verbesserung erzielen. Die unteren Tafeln zeigen, dass sich diese Verbesserung signifikant nur auf das erwartete Interface TEI4 auswirkt.
\begin{multicols}{3}
\vft{1}{$mk(30)$}{$mk(32)$}{0}{0}{TMR\_Quant Test-System TMR f�r TEI4}{tmr_quant_tei4}\columnbreak
\vft{1}{$mk(30)$}{0}{0}{0}{TMR\_Quant Test-System TMR f�r TEI5}{tmr_quant_tei5}\columnbreak
\vft{1}{$mk(31)$}{$mk(2)$}{0}{0}{TMR\_Quant Test-System TMR f�r TEI6}{tmr_quant_tei6}
\end{multicols}
\subsubsection{Ergebnisse TMR\_Qual}
F�r die Heuristik TMR\_Qual gibt es drei Aspekte, deren Konfiguration zu unterschiedlichen Ergebnissen f�hren kann:
\begin{enumerate}
\item Die Wahl des Basiswertes der einzelnen Type-Matcher
\item Die Ermittlung des akkumulierten qualitativen Type-Matcher Ratings einer Typ-Konvertierungsvariante
\item Die Ermittlung des akkumulierten qualitativen Type-Matcher Ratings einer Methoden-Konvertierungsvariante
\end{enumerate}
\myparagraph{Auswahl der Basiswerte} 
Die Basiswerte wurden bei den Untersuchungen konstant gelassen. Die konkreten Basiswerte, die f�r die Untersuchungen verwendet wurden, sind der Tabelle 2 zu entnehmen.\\\\
Die Werte bilden meiner Meinung nach die Wertigkeit der einzelnen Type-Matcher in Hinblick auf die Typisierung innerhalb der Sprache Java ab. So ist der ExactTypeMatcher, der nur zwei identische Typen als �bereinstimmend bewertet, mit dem niedrigsten Wert und damit der h�chsten Qualit�t hinsichtlich TMR\_Qual zu konfigurieren. Gleich dahinter folgt der GenSpecTypeMatcher, der Typen als �bereinstimmend bewertet, wenn sie innerhalb der Sprache auch miteinander substituiert werden k�nnen. An dritter Stelle kommt meiner Meinung nach der WrappedTypeMatcher, da dieser immerhin eine vollst�ndige �bereinstimmung von Typen fordert (auch wenn ein Typen innerhalb eines anderes enthalten ist), w�hrend der StructuralTypeMatcher lediglich einen Teil der deklarierten Methoden f�r eine �bereinstimmung fordert.
\begin{table}[H]
\centering
\small
			\begin{tabular}[c]{|c|c|}
			\hline
			\hline
				 \textbf{Type-Matcher} & \textbf{Basiswert} \\
				\hline\hline
				ExactTypeMatcher & 100 \\
				\hline
				ExactTypeMatcher & 200\\
				\hline
				WrappedTypeMatcher & 300\\
				\hline
				StructuralTypeMatcher & 400\\
				\hline
				\hline
			\end{tabular} 
 \caption{Type-Matcher mit Basiswerten
}
 \label{tab_basevalues}
\end{table}
\noindent
\myparagraph{Auswahl des Akkumulationsverfahrens des Type-Matcher Ratings einer\\Typ-Konvertierungsvariante bzw. Methoden-Konvertierungsvariante}
Das Akkumulationsverfahren f�r das qualitative Type-Matcher Rating einer Typ-Konvertierungsvariante  $TMR_{TK}$ ist von dem Type-Matcher Rating der verwendeten Type-Matcher abh�ngig. Das Akkumulationsverfahren f�r das qualitative Type-Matcher Rating einer Methoden-Konvertierungsvariante  $TMR_{MK}$ ist von dem qualitativen Type-Matcher Rating der verwendeten Type-Matcher f�r den R�ckgabe- und den Parametertypen der Methode abh�ngig abh�ngig. Somit kann das qualitative Type-Matcher Rating als Funktion von einer Typ- bzw. Methoden-Konvertierungsvariante $tmr_{Qual}(v)$ beschrieben werden.
Das Type-Matcher Rating der verwendeten Type-Matcher wird als Funktion $tmr_{Base}(m)$ beschrieben. Dabei stellt m den jeweiligen Type-Matcher dar. Die Funktion $tmr_{Base}(m)$ ist durch die Tabelle 2 definiert.\\\\
F�r einen Menge von Type-Matcher $m_1, m_2, ..., m_i$, die zur Erzeugung einer Typ-Konvertierungsvariante bzw. Methoden-Konvertierungsvariante $v$ verwendet wurden, werden folgende Akkumulationsverfahren f�r das Type-Matcher Rating der Typ-Konvertierungsvariante bzw. Methoden-Konvertierungsvariante im weiteren Verlauf evaluiert:
\begin{enumerate}
\item Wahl des Durchschnitts
\begin{equation*}
tmr_{Qual}(v) = \frac{ \sum_{n=1}^{i} tmr_{Base}(m_n)}{i}
\end{equation*}
\item Wahl des Maximums
\begin{equation*}
tmr_{Qual}(v) = max(tmr_{Base}(m_1), ..., tmr_{Base}(m_i))
\end{equation*}
\item Wahl des Minimums
\begin{equation*}
tmr_{Qual}(v) = min(tmr_{Base}(m_1), ..., tmr_{Base}(m_i))
\end{equation*}
\item Wahl des Durchschnitts aus Minimum und Maximum
\begin{equation*}
tmr_{Qual}(v) = \frac{min(tmr_{Base}(m_1), ..., tmr_{Base}(m_i)) +  max(tmr_{Base}(m_1), ..., tmr_{Base}(m_i))}{2}
\end{equation*}

\end{enumerate}
\noindent
Die folgenden Abschnitte stellen eine Auswahl der Ergebnisse hinsichtlich der Kombinationen der oben genannten Akkumulationsverfahren dar. Die Ergebnisse von Kombinationen, deren Ergebnisse nicht dargestellt wurden, sind mit den Ergebnissen einer der dargestellten Kombinationen gleichzusetzen. An entsprechender Stelle wird darauf verwiesen.\\\\
An den �berschriften der folgenden Abschnitte ist abzulesen, welche Akkumulationsverfahren miteinander kombiniert wurden. Dabei haben die �berschriften die Form ``Typ: T Methoden: M'' wobei f�r ``T'' und ``M'' die Nummern der oben genannten Akkumulationsverfahren eingesetzt werden.
\myparagraph{Typ: 1 Methoden: 2}\label{tmrquant_1-2}
\begin{multicols}{3}
\vft{1}{$mk(48)$}{$mk(121)$}{1}{0}{TMR\_Qual Test-System TMR f�r TEI1 mit 1-2}{tmr_qual_2_2_tei1}\columnbreak

\vft{1}{$mk(47)$}{$mk(132)$}{1}{0}{TMR\_Qual Test-System TMR f�r TEI2 mit 1-2}{tmr_qual_2_2_tei2}\columnbreak
\vft{1}{$mk(46)$}{$mk(141)$}{1}{0}{TMR\_Qual Test-System TMR f�r TEI3 mit 1-2}{tmr_qual_2_2_tei3}
\end{multicols}

\begin{multicols}{3}
\vft{1}{$mk(62)$}{0}{0}{0}{TMR\_Qual Test-System TMR f�r TEI4 mit 1-2 1. Durchlauf}{tmr_qual_2_2_tei4_1}\columnbreak
\vft{1}{$mk(60)$}{0}{0}{0}{TMR\_Qual Test-System TMR f�r TEI5 mit 1-2 1. Durchlauf}{tmr_qual_2_2_tei5_1}\columnbreak
\vft{1}{$mk(33)$}{0}{1}{0}{TMR\_Qual Test-System TMR f�r TEI6 mit 1-2 1. Durchlauf}{tmr_qual_2_2_tei6_1}
\end{multicols}
\newpage
\begin{multicols}{3}
\vft{2}{$mk(1)$}{$mk(1890)$}{1}{0}{TMR\_Qual Test-System TMR f�r TEI4 mit 1-2 2. Durchlauf}{tmr_qual_2_2_tei4_2}\columnbreak
\vft{2}{$mk(1)$}{$mk(1769)$}{1}{0}{TMR\_Qual Test-System TMR f�r TEI5 mit 1-2 2. Durchlauf}{tmr_qual_2_2_tei5_2}\columnbreak
\vft{2}{$mk(1)$}{$mk(527)$}{1}{0}{TMR\_Qual Test-System TMR f�r TEI6 mit 1-2 2. Durchlauf}{tmr_qual_2_2_tei6_2}
\end{multicols}


\myparagraph{Typ: 3 Methoden: 2}\label{tmrquant_3-2}
\begin{multicols}{3}
\vft{1}{$mk(49)$}{$mk(120)$}{1}{0}{TMR\_Qual Test-System TMR f�r TEI1 mit 3-2}{tmr_qual_3_2_tei1}\columnbreak
\vft{1}{$mk(49)$}{$mk(130)$}{1}{0}{TMR\_Qual Test-System TMR f�r TEI2 mit 3-2}{tmr_qual_3_2_tei2}\columnbreak
\vft{1}{$mk(48)$}{$mk(139)$}{1}{0}{TMR\_Qual Test-System TMR f�r TEI3 mit 3-2}{tmr_qual_3_2_tei3}
\end{multicols}


\begin{multicols}{3}
\vft{1}{$mk(62)$}{0}{0}{0}{TMR\_Qual Test-System TMR f�r TEI4 mit 3-2 1. Durchlauf}{tmr_qual_3_2_tei4_1}\columnbreak
\vft{1}{$mk(60)$}{0}{0}{0}{TMR\_Qual Test-System TMR f�r TEI5 mit 3-2 1. Durchlauf}{tmr_qual_3_2_tei5_1}\columnbreak
\vft{1}{$mk(33)$}{0}{1}{0}{TMR\_Qual Test-System TMR f�r TEI6 mit 3-2 1. Durchlauf}{tmr_qual_3_2_tei6_1}
\end{multicols}

\newpage
\begin{multicols}{3}
\vft{2}{$mk(1)$}{$mk(1890)$}{1}{0}{TMR\_Qual Test-System TMR f�r TEI4 mit 3-2 2. Durchlauf}{tmr_qual_3_2_tei4_2}\columnbreak
\vft{2}{$mk(1)$}{$mk(1769)$}{1}{0}{TMR\_Qual Test-System TMR f�r TEI5 mit 3-2 2. Durchlauf}{tmr_qual_3_2_tei5_2}\columnbreak
\vft{2}{$mk(1)$}{$mk(527)$}{1}{0}{TMR\_Qual Test-System TMR f�r TEI6 mit 3-2 2. Durchlauf}{tmr_qual_3_2_tei6_2}
\end{multicols}

\myparagraph{Typ: 4 Methoden: 3}\label{tmrquant_4-3}
\begin{multicols}{3}
\vft{1}{$mk(52)$}{$mk(117)$}{1}{0}{TMR\_Qual Test-System TMR f�r TEI1 mit 4-3}{tmr_qual_4_3_tei1}\columnbreak
\vft{1}{$mk(62)$}{$mk(117)$}{1}{0}{TMR\_Qual Test-System TMR f�r TEI2 mit 4-3}{tmr_qual_4_3_tei2}\columnbreak
\vft{1}{$mk(62)$}{$mk(125)$}{1}{0}{TMR\_Qual Test-System TMR f�r TEI3 mit 4-3}{tmr_qual_4_3_tei3}
\end{multicols}

\begin{multicols}{3}
\vft{1}{$mk(62)$}{0}{0}{0}{TMR\_Qual Test-System TMR f�r TEI4 mit 4-3 1. Durchlauf}{tmr_qual_4_3_tei4_1}\columnbreak
\vft{1}{$mk(60)$}{0}{0}{0}{TMR\_Qual Test-System TMR f�r TEI5 mit 4-3 1. Durchlauf}{tmr_qual_4_3_tei5_1}\columnbreak
\vft{1}{$mk(33)$}{0}{1}{0}{TMR\_Qual Test-System TMR f�r TEI6 mit 4-3 1. Durchlauf}{tmr_qual_4_3_tei6_1}
\end{multicols}
\newpage
\begin{multicols}{3}
\vft{2}{$mk(1891)$}{0}{1}{0}{TMR\_Qual Test-System TMR f�r TEI4 mit 4-3 2. Durchlauf}{tmr_qual_4_3_tei4_2}\columnbreak
\vft{2}{$mk(1770)$}{0}{1}{0}{TMR\_Qual Test-System TMR f�r TEI5 mit 4-3 2. Durchlauf}{tmr_qual_4_3_tei5_2}\columnbreak
\vft{2}{$mk(528)$}{0}{1}{0}{TMR\_Qual Test-System TMR f�r TEI6 mit 4-3 2. Durchlauf}{tmr_qual_4_3_tei6_2}
\end{multicols}
\noindent
Die \tabref{akkuverfahren} zeigt durch die Markierung mit einem ``x'', welche Kombinationen der oben genannten Akkumulationsverfahren hinsichtlich der Testergebnisse mit denen gleichzusetzen sind, die oben ausf�hrlich aufgef�hrt wurden. Die Kombinationen werden in der Tabelle �hnlich wie in den vorherigen �berschriften beschrieben. Die Notation ``1-4'' beschreibt die Kombination des 1. Akkumulationsverfahrens f�r die Typ-Konvertierungsvarianten und den 4. Akkumulationsverfahrens f�r die Methoden-Konvertierungsvarianten.
\begin{table}[H]
\centering
\begin{tabular}[c]{|c|c|c|c|}
\hline\hline
\textbf{Kombination} & \textbf{1-2} & \textbf{3-2} & \textbf{4-3} \\
\hline
1-1 & x& & \\
\hline
1-3 & & & x\\
\hline
1-4 & x& & \\
\hline
2-1 & x& & \\
\hline
2-2 & x& & \\
\hline
2-3 & & &x \\
\hline
2-4 & x& & \\
\hline
3-1 & &x & \\
\hline
3-3 & & & x\\
\hline
3-4 & &x & \\
\hline
4-1 & x& & \\
\hline
4-2 & x& & \\
\hline
4-4 & x& & \\
\hline\hline
\end{tabular}
\caption{Kombinationen von Akkumulationsverfahren mit gleichen Ergebnissen}
\label{tab:akkuverfahren}
\end{table}

Aus diesen Ergebnissen l�sst sich folgendes ableiten:
\begin{enumerate}
\item Das Akkumulationsverfahren Nummer 3. (Minimum) f�hrt sowohl f�r die Typ- und Methoden-Konvertierungsvarianten zu schlechteren Ergebnissen als die anderen drei Akkumulationsverfahren. Es sollte daher f�r die Heuristik TMR\_Quant nicht verwendet werden.
\item Die Ergebnisse von 1-2 und 3-2 unterscheiden sich nur geringf�gig, obwohl bei 3-2 das Akkumulationsverfahren Nummer 3. zum Einsatz kam. Dies konnte auch bei anderen Kombinationen festgestellt werden, bei denen das 3. Akkumulationsverfahren f�r die Akkumulation des Type-Matcher Ratings der Typ-Konvertierungsvariante verwendet wurde. Das l�sst vermuten, dass die Beachtung des Type-Matcher Ratings einer ganzen Typ-Konvertierungsvariante weitgehend unerheblich f�r die Heuristik TMR\_Quant ist, wenn das Type-Matcher Rating je Methoden-Konvertierungsvarianten �ber ein entsprechend gutes Akkumulationsverfahren ermittelt wurde. Dies ist jedoch darauf zur�ckzuf�hren, dass das Type-Matcher Rating je Methoden-Konvertierungsvariante die Parameter f�r die Ermittlung des Type-Matcher Ratings einer Typ-Konvertierungsvariante darstellen.
\item An den Ergebnissen zu den erwarteten Interfaces TEI4-TEI6 ist zu erkennen, dass die Heuristik TMR\_Quant keinen Einfluss auf den 1. Durchlauf hat. Daraus kann geschlussfolgert werden, dass die Heuristik nur in dem Durchlauf einen Gewinn bringt, in dem auch eine passende ben�tigte Komponente gefunden werden kann. 
\end{enumerate}
Aufgrund der Ergebnisse stehen f�r die weitere Verwendung der Heuristik TMR\_Qual mehrere Kombinationen von Akkumulationsverfahren zur Auswahl. Die Entscheidung f�llt aufgrund der etwas geringeren Komplexit�t auf die Kombination 1-2. 

\myparagraph{TMR\_Quant und TMR\_Qual in Kombination}
Bei der Kombination der beiden Heuristiken TMR\_Quant und TMR\_Qual ist vor allem f�r die erwarteten Interfaces TEI4-TEI6 eine weitere Verbesserung zu erwarten. Der Grund daf�r ist, dass die Heuristik TMR\_Qual keinen Einfluss auf den ersten Durchlauf des Explorationsalgorithmus f�r diese erwarteten Interfaces hat, die Heuristik TMR\_Quant hingegen schon. Die \tabsrefs{tmr_quant+qual:tei1}{tmr_quant+qual:tei6_2} zeigen wiederum die bekannten Vier-Felder-Tafeln f�r den jeweiligen Durchlauf und dem jeweiligen erwarteten Interface.
\begin{multicols}{3}
\vft{1}{$mk(2)$}{$mk(167)$}{1}{0}{TMR\_Quant + TMR\_Qual Test-System TMR f�r TEI1}{tmr_quant+qual:tei1}\columnbreak
\vft{1}{$mk(2)$}{$mk(177)$}{1}{0}{TMR\_Quant + TMR\_Qual Test-System TMR f�r TEI2}{tmr_quant+qual:tei2}\columnbreak
\vft{1}{$mk(1)$}{$mk(186)$}{1}{0}{TMR\_Quant + TMR\_Qual Test-System TMR f�r TEI3}{tmr_quant+qual:tei3}
\end{multicols}
\newpage
\begin{multicols}{3}
\vft{1}{$mk(30)$}{$mk(32)$}{0}{0}{TMR\_Quant + TMR\_Qual Test-System TMR f�r TEI4 1. Durchlauf}{tmr_quant+qual:tei4_1}\columnbreak
\vft{1}{$mk(60)$}{0}{0}{0}{TMR\_Quant + TMR\_Qual Test-System TMR f�r TEI5 1. Durchlauf}{tmr_quant+qual:te5_1}\columnbreak
\vft{1}{$mk(31)$}{$mk(2)$}{0}{0}{TMR\_Quant + TMR\_Qual Test-System TMR f�r TEI6 1. Durchlauf}{tmr_quant+qual:tei6_1}
\end{multicols}

\begin{multicols}{3}
\vft{1}{$mk(1)$}{$mk(1890)$}{1}{0}{TMR\_Quant + TMR\_Qual Test-System TMR f�r TEI4 2. Durchlauf}{tmr_quant+qual:tei4_2}\columnbreak
\vft{1}{$mk(1)$}{$mk(1769)$}{1}{0}{TMR\_Quant + TMR\_Qual Test-System TMR f�r TEI5 2. Durchlauf}{tmr_quant+qual:te5_2}\columnbreak
\vft{1}{$mk(1)$}{$mk(527)$}{1}{0}{TMR\_Quant + TMR\_Qual Test-System TMR f�r TEI6 2. Durchlauf}{tmr_quant+qual:tei6_2}
\end{multicols}

\noindent
Wie man diesen Ergebnissen zu erkennen ist, wird der Explorationsalgorithmus f�r eine Suche nach einer passenden ben�tigten Komponente f�r TEI1 und TEI2 lediglich f�r zwei angebotene Interfaces bzw. Typ-Konvertierungsvarianten durchlaufen. In Bezug auf TEI3 ist es sogar nur noch eine Typ-Konvertierungsvariante. F�r diese erwarteten Interfaces erf�llen die Heuristiken die Erwartungen.\\\\
Bei der Betrachtung der Ergebnisse f�r die erwarteten Interfaces TEI4-TEI6 zeigt sich gut, wie sich die beiden Heuristiken gegenseitig erg�nzen. So wirkt die Heuritik TMR\_Quant grunds�tzlich nur auf den ersten Durchlauf des Explorationsalgorithmus aus. Die Heuristik TMR\_Qual hingegen erweist ihre St�rke erst in dem Durchlauf, in dem auch eine passende ben�tigte Komponente gefunden wird. Die Evaluationsergebnisse best�tigen also auch hier die Annahmen.\\\\
Im Allgemeinen kann festgehalten werden, dass die passenden ben�tigten Komponenten trotz der Kombination der beiden Heuristiken gefunden werden konnten. Die Reduktion der notwendigen Durchl�ufe des Explorationsalgorithmus ist jedoch haupts�chlich auf die Heuristik TMR\_Qual zur�ckzuf�hren. 



\section{Heiß-System}\label{sec_hotsystem}
Im Heiß-System werden, wie beschrieben, die 889 \emph{provided Typen} verwendet, für die jeweils mindestens eine Implementierung bereitgestellt wurde. Die \emph{provided Typen}, die im Test-System ergänzt wurde, befinden sich nicht im Heiß-System.
\\\\
Für die Evaluation im Test-System werden insgesamt 4 \emph{required Typen} verwendet. Drei dieser \emph{required Typen} wurden ebenfalls im zur Evaluation im Test-System verwendet. Die Deklaration der \emph{required Typen} ist im Anhang \ref{xyz} zu finden.  
\\\\
Die Java-Interfaces, die sich aus dieser Deklaration ableiten lassen, die dazugehörigen Implementierungen der \emph{provided Typen} und die vordefinierten Testfälle sind in Anhang \ref{abc} zu finden.
%\\\\
In \tabref{eIMainShort} sind die Namen der \emph{required Typen} zusammen mit jeweils einem Kürzel aufgeführt. Die Kürzel dienen im weiteren Verlauf der Identifizierung der \emph{required Typen}.
\begin{table}[h!]
\centering
\small
\begin{tabular}{|l|c|}
\hline
\hline
\centering\textbf{required Typ} & \textbf{Kürzel} \\
\hline
\hline
ElerFTFoerderprogrammeProvider & TEI1\\
\hline
FoerderprogrammeProvider & TEI2\\
\hline
MinimalFoerderprogrammeProvider & TEI3\\
\hline
KOFGPCProvider & TEI4\\
\hline
\hline
\end{tabular}
\caption{Kürzel der required Typen für die Evaluation im Heiß-System}
 \label{tab:eIMainShort}
\end{table}
\noindent
%Neben den in Abschnitt \ref{sec_matcher} und \ref{sec_proxies} beschriebenen Matcher und Proxy-Generatoren wird im Heiß-System ein zusätzlicher Matcher sowie ein darauf basierender Generator für Proxies benötigt.
%\\\\
%Der Grund dafür ist, dass im Heiß-System zur Laufzeit Objekte existieren, die über bestimmte Zeichenketten oder numerische Werte eindeutig identifiziert werden. Die Typen solcher Objekte werden im Heiß-System als Domain-Values bezeichnet.
%\\\\
%Die Semantik dieser Domain-Values soll bei der Suche nach einem passenden Proxy zu einem \emph{required Typ} nicht untergraben werden. 


\subsection{Ausgangspunkt}
Für ein \emph{reqiured Typ} können mehrere \emph{provided Typen} gefunden werden, die eine strukturelle Übereinstimmung aufwiesen. \tabref{amountMatchedInterfacesHot} zeigt die Anzahl der strukturell übereinstimmenden \emph{provided Typen} je \emph{reqiured Typ}. Diese kommen einzeln oder in Kombination für die semantische Evaluation in Frage.
\begin{table}[H]
\centering
\small
\singlespacing
			\begin{tabular}[c]{|>{\centering\arraybackslash}p{2cm}|>{\centering\arraybackslash}p{5cm}|}
			\hline
			\hline
				 \textbf{required Interface} & \textbf{Anzahl strukturell übereinstimmender provided Interfaces} \\
				\hline\hline
				TEI1 & 221 \\
				\hline
				TEI2 & 272\\
				\hline
				TEI3 & 268 \\
				\hline
				TEI4 & 348 \\
				\hline
				\hline
			\end{tabular} 
 \caption{Anzahl strukturell übereinstimmender provided Typen je required Typ im Heiß-System}
 \label{tab:amountMatchedInterfacesHot}
\onehalfspacing
\end{table}
\noindent
Die \tabsrefs{hs_start_tei1}{hs_start_tei4_2} zeigen die Vier-Felder-Tafeln, in denen die Ergebnisse der benötigten Iterationen innerhalb des Explorationsalgorithmus für jeden der \emph{required Typen} aus \tabref{amountMatchedInterfacesHot}. Dabei wurden keine Heuristiken verwendet. Somit stellt dies den Ausgangspunkt für die weitere Evaluation im Heiß-System dar.
\begin{multicols}{3}
\vft{1}{$p(44)-1$}{0}{1}{0}{Ausgangspunkt im Heiß-System für TEI1}{hs_start_tei1}\columnbreak
\vft{1}{$p(30)-1$}{0}{1}{0}{Ausgangspunkt im Heiß-System für TEI2}{hs_start_tei2}\columnbreak
\vft{1}{$p(30)-1$}{0}{1}{0}{Ausgangspunkt im Heiß-System für TEI3}{hs_start_tei3}
\end{multicols}
\begin{multicols}{2}
\vft{1}{$p(50)-1$}{0}{1}{0}{Ausgangspunkt im Heiß-System für TEI4 \\1. Durchlauf}{hs_start_tei4_1}\columnbreak
\vft{2}{$p(7714)-1$}{0}{1}{0}{Ausgangspunkt im Heiß-System für TEI4 \\2. Durchlauf}{hs_start_tei4_2}\columnbreak
\end{multicols}
Für den \emph{required Typen} \emph{TEI4} werden zwei Durchläufe benötigt, da die semantischen Test nur von einem Proxy bestanden werden, der aus einer Kombination zweier \emph{provided Typen} erzeugt wurde.

\section{Ergebnisse für die Heuristik PTTF}\label{sec_evalPTTF}
Für die \Gls{Heuristik} \emph{PTTF} gilt es zu evaluieren, ob die Suche nach einem \emph{Proxy}, der die vordefinierten Tests besteht, beschleunigt werden kann. Hierzu wird der \emph{Explorationsprozess} für alle in Tabelle \ref{tab:eIShort} genannten \emph{required Typen} unter der Verwendung der in Abschnitt \ref{sec_pttf} beschriebenen \Gls{Heuristik} durchgeführt.
\\\\
Die folgenden Vier-Felder-Tafeln zeigen die Ergebnisse für die \emph{required Typen} \emph{TEI1}-\emph{TEI7} auf.
\begin{multicols}{3}
\vft{1}{29}{$p_1(44)-30$}{1}{0}{Ergebnisse \emph{PTTF} für TEI1 1.~\mbox{Durchlauf}}{pttf_TEI1_1}
\vft{1}{5544}{$p_1(55)-5545$}{1}{0}{Ergebnisse \emph{PTTF} für TEI2 1.~\mbox{Durchlauf}}{pttf_TEI2_1}
\vft{1}{4761}{$p_1(50)-4762$}{1}{0}{Ergebnisse \emph{PTTF} für TEI3 1.~\mbox{Durchlauf}}{pttf_TEI3_1}
\end{multicols}

\begin{multicols}{2}
\vft{1}{$1174$}{0}{0}{0}{Ergebnisse \emph{PTTF} für TEI4 1.~\mbox{Durchlauf}}{pttf_TEI4_1}
\vft{2}{466}{$p_2(2247)-467$}{1}{0}{Ergebnisse \emph{PTTF} für TEI4 2.~\mbox{Durchlauf}}{pttf_TEI4_2}
\end{multicols}
\pagebreak
\begin{multicols}{2}
\vft{1}{$4984$}{0}{0}{0}{Ergebnisse \emph{PTTF} für TEI5 1.~\mbox{Durchlauf}}{pttf_TEI5_1}
\vft{2}{2172}{$p_2(2775)-2173$}{1}{0}{Ergebnisse \emph{PTTF} für TEI5 2.~\mbox{Durchlauf}}{pttf_TEI5_2}
\end{multicols}

\begin{multicols}{2}
\vft{1}{$1051$}{0}{0}{0}{Ergebnisse \emph{PTTF} für TEI6 1.~\mbox{Durchlauf}}{pttf_TEI6_1}
\vft{2}{13122}{$p_2(1323)-13123$}{1}{0}{Ergebnisse \emph{PTTF} für TEI6 2.~\mbox{Durchlauf}}{pttf_TEI6_2}
\end{multicols}

\begin{multicols}{2}
\vft{1}{$161294$}{0}{0}{0}{Ergebnisse \emph{PTTF} für TEI7 1.~\mbox{Durchlauf}}{pttf_TEI7_1}
\vft{2}{149961}{$p_2(52150)-149962$}{1}{0}{Ergebnisse \emph{PTTF} für TEI7 2.~\mbox{Durchlauf}}{pttf_TEI7_2}
\end{multicols}
\newpage
\noindent
Folgendes kann aus diesen Ergebnissen abgeleitet werden:
\begin{enumerate}
\item Die \Gls{Heuristik} \emph{PTTF} erzielt im Vergleich zum Ausgangspunkt (Abschnitt \ref{sec_ausgangspunkt}) für jeden \emph{required Typ} eine weitere Reduktion der zu prüfenden \emph{Proxies}.

\item Die Heuristik \emph{PTTF} hat keine Auswirkung auf einen Durchlauf, in dem kein \emph{Proxy} erzeugt wird, mit dem die vordefinierten Tests erfolgreich durchgeführt werden können. Dies kann durch einen Vergleich des ersten Durchlaufs für den \emph{required Typ} \emph{TEI4}-\emph{TEI7} im Ausgangspunkt (Tabelle \ref{tab:tmr_start_tei4_1}, \ref{tab:tmr_start_tei5_1}, \ref{tab:tmr_start_tei6_1} und \ref{tab:tmr_start_tei6_1}) mit dem ersten Durchlauf unter Anwendung der Heuristik (Tabellen \ref{tab:pttf_TEI4_1}, \ref{tab:pttf_TEI5_1}, \ref{tab:pttf_TEI6_1} und \ref{tab:pttf_TEI7_1}) festgestellt werden. Aus diesem Grund kommt die in Punkt 1 beschriebene Reduktion erst im jeweils letzten Durchlauf zum Tragen.
\end{enumerate}
\section{Ergebnisse für die Heuristik BL\_NMC}\label{sec_evalBLNMC}
Für die \Gls{Heuristik} \emph{BL\_NMC} gilt es zu evaluieren, ob die Suche nach einem \emph{Proxy}, der die vordefinierten Tests besteht, beschleunigt werden kann. Hierzu wird der \emph{Explorationsprozess} für alle in Tabelle \ref{tab:eIShort}genannten \emph{required Typen} unter der Verwendung der in Abschnitt \ref{sec_bl_nmc} beschriebenen \gls{Heuristik} durchgeführt.
\\\\
Die folgenden Vier-Felder-Tafeln zeigen die Ergebnisse für die \emph{required Typen} \emph{TEI1}-\emph{TEI7} auf.
\begin{multicols}{3}
\vft{1}{105}{$p_1(44)-106$}{1}{0}{Ergebnisse \emph{BL\_NMC} für TEI1 1.~\mbox{Durchlauf}}{blnmc_TEI1_1}
\vft{1}{342}{$p_1(55)-343$}{1}{0}{Ergebnisse \emph{BL\_NMC} für TEI2 1.~\mbox{Durchlauf}}{blnmc_TEI2_1}
\vft{1}{357}{$p_1(50)-358$}{1}{0}{Ergebnisse \emph{BL\_NMC} für TEI3 1.~\mbox{Durchlauf}}{blnmc_TEI3_1}
\end{multicols}

\begin{multicols}{2}
\vft{1}{120}{$1054$}{0}{0}{Ergebnisse \emph{BL\_NMC} für TEI4 1.~\mbox{Durchlauf}}{blnmc_TEI4_1}
\vft{2}{442}{$p_2(2247)-443$}{1}{0}{Ergebnisse \emph{BL\_NMC} für TEI4 2.~\mbox{Durchlauf}}{blnmc_TEI4_2}
\end{multicols}

\begin{multicols}{2}
\vft{1}{550}{$4434$}{0}{0}{Ergebnisse \emph{BL\_NMC} für TEI5 1.~\mbox{Durchlauf}}{blnmc_TEI5_1}
\vft{2}{1304}{$p_2(2775)-1305$}{1}{0}{Ergebnisse \emph{BL\_NMC} für TEI5 2.~\mbox{Durchlauf}}{blnmc_TEI5_2}
\end{multicols}
\pagebreak
\begin{multicols}{2}
\vft{1}{366}{$685$}{0}{0}{Ergebnisse \emph{BL\_NMC} für TEI6 1.~\mbox{Durchlauf}}{blnmc_TEI6_1}
\vft{2}{204}{$p_2(1323)-205$}{1}{0}{Ergebnisse \emph{BL\_NMC} für TEI6 2.~\mbox{Durchlauf}}{blnmc_TEI6_2}
\end{multicols}

\begin{multicols}{2}
\vft{1}{1051}{$160243$}{0}{0}{Ergebnisse \emph{BL\_NMC} für TEI7 1.~\mbox{Durchlauf}}{blnmc_TEI7_1}
\vft{2}{135089}{$p_2(52150)-135090$}{1}{0}{Ergebnisse \emph{BL\_NMC} für TEI7 2.~\mbox{Durchlauf}}{blnmc_TEI7_2}
\end{multicols}

Folgendes kann aus diesen Ergebnissen abgeleitet werden:
\begin{enumerate}
\item Die \Gls{Heuristik} \emph{BL\_NMC} erzielt im Vergleich zum Ausgangspunkt (Abschnitt \ref{sec_ausgangspunkt}) für jeden \emph{required Typ} eine weitere Reduktion der zu prüfenden \emph{Proxies}.

\item Die Heuristik \emph{BL\_NMC} hat das Potential jeden Durchlauf innerhalb der \emph{semantischen Evaluation} zu beschleunigen. Für den jeweils ersten Durchlauf kann dies durch einen Vergleich der Tabellen \ref{tab:tmr_start_tei1}, \ref{tab:tmr_start_tei2}, \ref{tab:tmr_start_tei3}, \ref{tab:tmr_start_tei4_1}, \ref{tab:tmr_start_tei5_1}, \ref{tab:tmr_start_tei6_1} und \ref{tab:tmr_start_tei7_1} zum Ausgangspunkt mit den Tabellen \ref{tab:blnmc_TEI1_1}, \ref{tab:blnmc_TEI2_1}, \ref{tab:blnmc_TEI3_1}, \ref{tab:blnmc_TEI4_1}, \ref{tab:blnmc_TEI5_1}, \ref{tab:blnmc_TEI6_1} und \ref{tab:blnmc_TEI7_1} festgestellt werden. Ein Vergleich der Tabelle \ref{tab:tmr_start_tei4_2}, \ref{tab:tmr_start_tei5_2}, \ref{tab:tmr_start_tei6_2} und \ref{tab:tmr_start_tei7_2} im Ausgangspunkt mit den Tabellen \ref{tab:blnmc_TEI4_2}, \ref{tab:blnmc_TEI5_2}, \ref{tab:blnmc_TEI6_2} und \ref{tab:blnmc_TEI7_2} belegt dies für den zweiten Durchlauf auf.
\end{enumerate}
