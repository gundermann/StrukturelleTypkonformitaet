\subsection{Type-Matcher Rating basierte Heuristiken}
Wie die �berschrift bereits andeutet, werden die Type-Matcher mit einem Rating versehen. Das Rating wird durch einen numerischen Wert dargestellt. Auf dieser Basis werden zwei Kategorien von Type-Matcher Ratings unterschieden:
\begin{enumerate}
\item Qualitatives Type-Matcher Rating
\item Quantitatives Type-Matcher Rating
\end{enumerate}
\subsubsection{Qualitatives Type-Matcher Rating}
Das qualitative Type-Matcher Rating beschreibt den Grad der strukturellen �bereinstimmung des Source- und des Target-Typen. Daf�r wird jeder Type-Matcher mit einem Basiswert versehen. Die konkreten Basiswerte sind im Abschnitt Evaluation beschrieben. Dieser Basiswert wird beim Erzeugen der Typ-Konvertierungsvarianten an die methodendelegationsrelevanten Informationen geh�ngt, sodass zu jeder Methode, zu der eine Methoden-Konvertierungsvariante existiert, auch ein Wert bzgl. des qualitatives Type-Matcher Rating zur Verf�gung steht.\\\\
Der konkrete Wert f�r das qualitative Type-Matcher-Rating ermittelt sich grundlegend, wie bereits erw�hnt, anhand eines Basiswertes. Sofern ein Type-Matcher jedoch einen internen Type-Matcher verwendet, um die konkreten methodendelegationsrelevanten Informationen zu erzeugen, ergibt sich der Wert des Type-Matcher-Ratings aus einer Akkumulation der Basiswerte des Type-Matchers und des internen Type-Matchers.\\\\
Damit ist das qualitative Type-Matcher Rating von folgenden Faktoren abh�ngig:
\begin{enumerate}
\item Die Wahl des Basiswertes der einzelnen Type-Matcher
\item Das Akkumulationsverfahren f�r das Type-Matcher Rating einer Typ-Konvertierungsvariante
\item Das Akkumulationsverfahren f�r das Type-Matcher Rating einer Methoden-Konvertierungsvariante
\end{enumerate}
Alle drei Punkt werden bei der Evaluierung dieser Heuristik betrachtet.
\subsubsection{Quantitatives Type-Matcher Rating}
Das quantitative Type-Matcher Rating beschreibt, zu wie vielen der erwarteten Methoden in der erzeugten Typ-Konvertierungsvariante methodendelegationsrelevante Informationen vorliegen. Hierzu wird der methodendelegationsrelevante Informationen innerhalb der Typ-konvertierungsvariante durch die Anzahl der erwarteten Methoden geteilt. So stellt das quantitative Type-Matcher Rating also einen Prozentsatz dar. \\\\
Darauf aufbauend werden im Folgenden zwei Heuristiken vorgestellt, durch die es erm�glicht wird die ermittelten Typ-Konvertierungsvarianten in eine Reihenfolge zu bringen. So kann die Selektion der Kombination von Methoden-Konvertierungsvarianten im 3. Schritt der 2. Stufe des Explorationsalgorithmus anhand dieser Sortierung erfolgen.
\subsection{Heuristik - TMR\_Quant: Beachtung des quantitativen Type-Matcher Ratings}
Es wird davon ausgegangen, dass eine angebotene Komponente, deren Schnittstelle strukturell mit der erwarteten Schnittstelle �bereinstimmt und dabei zu jeder erwarteten Methode eine passende Methode anbietet, eher die semantischen Erwartungen erf�llt, als eine Kombination aus mehreren angebotenen Komponenten.\\\\
Daher sollten bei der Selektion der Kombinationen von Methoden-Konvertierungsvarianten in Schritt 3 der 2. Stufe des Explorationsalgorithmus (siehe \ref{sem_eval_step3}) zuerst diejenigen Kombinationen gew�hlt werden, deren Elemente (Methoden-Konvertierungsvarianten) aus ein und derselben Typ-Konvertierungsvariante stammen, sprich deren quantitatives Type-Matcher Rating m�glichst hoch ist.
\subsection{Heuristik - TMR\_Qual: Beachtung des qualitativen Type-Matcher Ratings}
Es wird davon ausgegangen, dass eine von einer angebotene Komponente angebotene Methode, die strukturell mit einer erwarteten Methode zu einem h�heren Grad �bereinstimmt, auch eher die semantischen Erwartungen an diese Methode erf�llt.\\\\
Daher sollten bei der Selektion der Kombinationen von Methoden-Konvertierungsvarianten in Schritt 3 der 2. Stufe des Explorationsalgorithmus  (siehe \ref{sem_eval_step3}) zuerst diejenigen Kombinationen gew�hlt werden, deren Elemente (Methoden-Konvertierungsvarianten) aus den methodendelegationsrelevanten Informationen mit den niedrigsten qualitativen Type-Matcher Rating der Typ-Konvertierungsvarianten erzeugt wurden.

