\section{Kritik am Ansatz}\label{sec_discApproach}

\subsection{Seiteneffekte durch Testevaluation}\label{sec_sideeffects}
Der beschriebene \emph{Explorationsprozess} erfordert die Ausführung der vordefinierten Testfälle zur Laufzeit. Sofern diese Testfälle eine Änderung des Zustands bestimmter Objekte bewirken, kann dies auch Auswirkungen auf die Funktionsweise des Systems haben. 
\\\\
Um dieses Problem zu beheben könnte sichergestellt werden, dass die Generierung der \emph{Proxies} nur auf Basis von \emph{provided Typen} (\emph{EJBs}) erfolgt, die solche Seiteneffekte nicht aufweisen. Diese Eigenschaft kann jedoch nur durch die Entwickler*innen festgestellt. Solche \emph{EJBs} können dann bspw. über Annotationen markiert werden. Während des \emph{Explorationsprozesses} könnten solche \emph{EJBs} über solche Markierungen erkannt werden. Dieser Ansatz reduziert jedoch die Anzahl der \emph{provided Typen}, die für die Generierung eines \emph{Proxies} verwendet werden können. Dadurch sinkt auch die Wahrscheinlichkeit, dass ein passender \emph{Proxy} gefunden wird.
\\\\
Um die zu markierenden \emph{EJBs} zu identifizieren ist zu prüfen, wie sich die Ausführung der einzelnen Methoden der \emph{Bean} auf das System auswirken. Es kann festgehalten werden, dass alle Methoden, die den persistenten oder den transienten Zustand von Objekten verändern, das Potential für solche unerwünschten Seiteneffekte besitzen. 
\\\\
Aufbauend auf einer solchen Prüfung einzelner Methoden, kann auch die Markierung von Methoden in Betracht gezogen werden. So dürften markierte Methoden bei der Generierung eines \emph{Proxies} nicht als \emph{Delegationsmethode} verwendet werden.
\subsection{Auswirkung auf die Verfügbarkeit eines Systems}\label{sec_stabliliy}
Die Verfügbarkeit eines Systems bzw. von Systemkomponenten, bezeichnet die Wahrscheinlichkeit, ein System oder Systemkomponenten zu einem vorgegebenen Zeitpunkt in einem funktionsfähigen Zustand anzutreffen. \cite{it-admin}
Die Auswirkung des Ansatzes auf die Verfügbarkeit wurde in dieser Arbeit nicht systematisch untersucht. Da der Ansatz jedoch darauf abzielt, bestimmte Komponenten - in diesem Fall \emph{EJBs} - zur Laufzeit zu kombinieren, können Überlegungen bzgl. der Verfügbarkeit durchaus angestellt werden.
\\\\
Dabei muss allerdings bedacht werden, dass die Betrachtung der Verfügbarkeit einer \emph{EJB} in diesem Zusammenhang nicht ausreichend ist. Immerhin kann ein passender \emph{Proxy} auch auf einer Kombination von \emph{EJBs} erzeugt werden. Insofern bilden eher die Methoden, die von den EJBs angeboten werden, die Komponenten in Bezug auf die oben beschriebene betrachtete Verfügbarkeit.
\\\\
Ausgehend davon kann die These aufgestellt werden, dass mit diesem Ansatz eine höhere Verfügbarkeit erreicht wird, sofern die Methoden im System redundant vorliegen. Dabei ist das Vorliegen redundanter Methoden innerhalb einer Systems wahrscheinlicher, als das Vorliegen redundanter Komponenten, die diese Methoden enthalten\footnote{Dies ist bei \emph{EJBs} der Fall, denn eine \emph{EJB} enthält Methoden und nicht anders herum.}. 

\subsection{Auswirkung von Änderungen an bestehenden Komponenten}
Da die \emph{EJBs} bei dem vorgestellten Ansatz nicht explizit adressiert werden, weiß der Entwickler auch nicht, an welche \emph{EJBs} die Methodenaufrufe letztendlich delegiert werden. Somit sind die Auswirkungen von Änderungen an bestehenden Komponenten nicht direkt vorhersehbar, da sich die Menge der matchenden \emph{provided Typen} (\emph{EJBs}) und dementsprechend auch die generierten \emph{Proxies} ändern.
\\\\
Im Folgenden wird zum einen die Erweiterung um zusätzlichen \emph{provided Typen} und zum anderen die Entfernung von \emph{provided Typen} betrachtet. Dabei sei angenommen, dass die \emph{required Typen}, zu denen ein passender \emph{Proxy} gefunden werden soll, nicht verändert werden.
\subsubsection{Erweiterungen um neue Komponenten}
Die Erweiterung von Systemen geht in Bezug auf den beschriebenen Ansatz zur testgetriebenen Exploration zur Laufzeit damit einher, dass sich die Anzahl der \emph{provided Typen} verändert. Wie in Abschnitt \ref{sec_anzahlProxies} beschrieben, besteht damit auch die Gefahr, dass die Anzahl der möglichen \emph{Proxies} steigt. Dazu muss jedoch gelten, dass eine Methode aus einem \emph{required Typ} auf eine der Methode aus dem neuen \emph{provided Typ} gematcht werden kann.
\\\\
Mehrere mögliche \emph{Proxies} haben wiederum einen Einfluss auf die Laufzeit und das Ergebnis des \emph{Explorationsprozesses}. So kann nicht davon ausgegangen werden, dass ein passender \emph{Proxy} zu einem bestimmten \emph{required Typ} genauso schnell gefunden wird, nachdem ein \emph{provided Typen} im System ergänzt wurde.
\subsubsection{Entfernen von bestehenden Komponenten}
Ebenso wirkt sich das Entfernen eines \emph{provided Typs}, der während eines früheren \emph{Explorationsprozesses} für die Generierung eines \emph{Proxies} verwendet wurde, auf den \emph{Explorationsprozess} nach einer solchen Änderung aus. Dadurch, dass der früher verwendete \emph{provided Typ} nicht mehr vorhanden ist, muss ein anderer \emph{Proxy}, der auf andere \emph{provided Typen} basiert, erzeugt werden\footnote{sofern dies gelingt, unterstützt dies die These aus Abschnitt \ref{sec_stabliliy}}.
\\\\
Da der \emph{Explorationsprozess} beendet wird, sofern ein passender \emph{Proxy} gefunden wurde, kann es auch unter diesen Umständen dazu kommen, dass der \emph{Explorationsprozess} mitunter länger dauert als vorher. Zudem besteht in diesem Fall die Gefahr, dass während des \emph{Explorationsprozesses} kein passender \emph{Proxy} gefunden wird.

\subsection{Nutzen für den Entwickler}
Aus den vorherigen Absätzen ergibt sich, dass die Entwickler*innen bei der Verwendung dieses Ansatzes eine große Verantwortung tragen. Dieser Verantwortung können sie umso besser gerecht werden, je genauer sie das System, in dem der Ansatz verwendet werden soll, kennen. 
\\\\
So kann festgehalten werden, dass Entwickler*innen, die das System gut kennen und somit wissen, welche Komponenten innerhalb dessen verwendet werden, diesen Ansatz wohl kaum benötigen. Vielmehr ist es ihnen möglich die passenden Komponenten aufgrund ihres Wissens explizit zu benennen, wie es im \emph{EJB}-Framework grundlegend der Fall ist.
\\\\
Entwickler*innen, die das System hingegen weniger kennen, können von diesem Ansatz profitieren, da sie nicht selbst nach einer für ihren Anwendungsfall passenden \emph{EJB} (mitunter auch mehreren) suchen müssen. Diese können sie über die Deklaration eines \emph{required Typen} und der Spezifikation dazugehöriger Tests suchen lassen. Dabei ist jedoch zu erwähnen, dass der \emph{Explorationsprozess} insbesondere mit der vorgestellten \Gls{Heuristik} \emph{LMF} umso schneller ist, je genauer die in den Methoden des \emph{required Typs} verwendeten Typen mit den Typen, die in den Methoden der \emph{provided Typen} übereinstimmen (\emph{Matcherrating}).
\\\\
Ist den Entwickler*innen das System unbekannt, besteht die Gefahr, dass der \emph{required Typ} so deklariert wird, dass das \emph{Matcherrating} relativ hoch ausfällt und somit der \emph{Explorationsprozess} mehr Zeit in Anspruch nimmt.
\\\\
Zusammenfassend kann folgende These formuliert werden: Der Nutzen dieses Ansatzes für Entwickler*innen steht im umgekehrt proportionalen Verhältnis zum Wissen dieser Entwickler*innen über das System, in dem der Ansatz verwendet werden soll. 
