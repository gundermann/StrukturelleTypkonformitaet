\section{Vor- und Nachteile des Ansatzes}
%formale Ermittlung des Matcherratings
%Die Aussagekraft der Ergebnisse ist aufgrund der gewählten required Typen eher gering. Hier spielt auch der Abstraktionsgrad der Typen mit rein, die in den Methoden als Parameter- oder Rückgabetypen verwendet werden. Eine These ist, dass die Anzahl der zu evaluierenden Proxies steigt, je weiter der Abstraktionsgrad der in den required Typen und den provided Typen verwendeten Parameter- und Rückgabetypen auseinandergeht. Weiterhin ist der Entwickler als kreative Komponente nicht wegzudenken. Dieser hat bei der Gestaltung der required Typen einen großen Freiraum, was die Wahl der Parameter- und Rückgabetypen angeht. These: Je schlechter er das vorliegende System kennt, desto eher wird der o.g. Abstraktionsgrad abweichen. Dies ließe sich mitunter durch eine Umfrage mehrere Mitarbeiter unterschiedlicher Betriebsangehörigkeit in Erfahrung bringen.
%Tests zur Laufzeit können zu Problemen im System führen, da die Semantik der provided Typen ausgeführt wird und im Fehlerfall mitunter zu schiefen Datenständen führt
%Wenn die Anzahl der provided Typen größer wird, wächst die Anzahl der zu evaluierenden Proxies im schlimmsten Fall (fakultativ/exponentiell??)
