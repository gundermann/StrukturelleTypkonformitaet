\section{Strukturelle Evaluation}
\subsection{Struktur für die Definition von Typen}\label{sec:strukturTypen}
Die Typen seien in einer Bibliothek $\text{L}$ in folgender Form zusammengefasst:
\begin{table}[H]
\centering
\begin{tabular}{|p{5.5cm}|p{8.5cm}|}
\hline
\hline
\centering\textbf{Regel} & \textbf{Erläuterung} \\
\hline
\hline
$\mathit{L} ::= \mathit{TD}\text{*}$ & Eine Bibliothek \emph{L} besteht aus einer Menge von Typdefinitionen.\\
\hline
$\mathit{TD} ::= \mathit{PD} | \mathit{RD}$ & Eine Typdefinition kann entweder die Definition eines provided Typen (PD) oder eines required Typen (RD) sein.\\
\hline
$\mathit{PD} ::= \newline\texttt{provided }T \texttt{ extends } T' \newline  \texttt{\{} \mathit{FD}\text{*} \mathit{MD}\text{*}\texttt{\}}$& Die Definition eines provided Typen besteht aus dem Namen des Typen \emph{T}, dem Namen des Super-Typs \emph{T'} von \emph{T} sowie mehreren Feld- und Methodendeklarationen.\\
\hline
$\mathit{RD} ::= \texttt{required } T \texttt{ \{}\mathit{MD}\text{*}\texttt{\}}$ & Die Definition eines required Typen besteht aus dem Namen des Typen \emph{T} sowie mehreren Methodendeklarationen.\\
\hline
$\mathit{FD} ::= T \texttt{ }\mathit{f}$ & Eine Felddeklaration besteht aus dem Namen des Feldes \emph{f} und dem Namen seines Typs \emph{T}.\\
\hline
$\mathit{MD} ::= \mathit{T'}\texttt{ }\mathit{m(T_1,...,T_n)}$ & Eine Methodendeklaration besteht aus dem Namen der Methode \emph{m}, $n$ Namen der Parameter-Typen $T_1$ bis $T_n$ und dem Namen des Rückgabe-Typs \emph{T'}.\\
\hline
\hline
\end{tabular}
\caption{Struktur für die Definition einer Bibliothek von Typen}
 \label{tab_typeStruct}
\end{table}
\noindent
Weiterhin sei die Relation $<$ auf Typen durch folgende Regeln definiert:
\begin{gather*}
\frac{\texttt{provided }T \texttt{ extends } T' \in L}{T < T'}
\end{gather*}
\begin{gather*}
\frac{\texttt{provided } T \texttt{ extends } T'' \in L \wedge T'' < T'}{T < T'}
\end{gather*}

Darüber hinaus seien folgende Funktionen definiert:
\begin{gather*}
\mathit{felder(T)} :=  \left\{ 
				\begin{array}{l|l}
					T \texttt{ }\mathit{f} & T \texttt{ }\mathit{f}\text{ ist Felddeklaration von }T
				\end{array}
              \right\}
\\
\mathit{feldTyp(f,T)} := 
				\begin{array}{l|l}
					T' & T' \texttt{ }\mathit{f}\text{ ist Felddeklatation von }T
				\end{array}   
\\
\mathit{ret(T'\text{ }m(T''_1,...T''_n))} := T'
\\
\mathit{params(T''\text{ }m(T'_1,...T'_n))} := \{ T'_1,...,T'_n \}
\\   
\mathit{methoden(T)} := \left\{ 
				\begin{array}{l|l}
					T'' \text{ }m(T'_1,...,T'_n) & T'' \text{ }m(T'_1,...,T'_n) \text{ ist Methodendeklaration von }T
				\end{array}
              \right\}
\\        
\end{gather*}
\noindent
Listing \ref{lst:libEx} zeigt ein Beispiel für eine Bibilothek mit \emph{required} und \emph{provided Typen}.
\begin{lstlisting}[style = dsl]
provided Fire extends Object{}
\end{lstlisting}

\begin{lstlisting}[style = dsl]
provided ExtFire extends Fire{}
\end{lstlisting}


\begin{lstlisting}[style = dsl]
provided FireState extends Object{
	boolean isActive
}
\end{lstlisting}

\begin{lstlisting}[style = dsl]
provided Medicine extends Object{
	String getDescription()
}
\end{lstlisting}

\begin{lstlisting}[style = dsl]
provided Injured extends Object{
	void heal(Medicine med)	
}
\end{lstlisting}


\begin{lstlisting}[style = dsl]
provided Patient extends Injured{
	String getName()
}
\end{lstlisting}
\begin{lstlisting}[style = dsl]
provided FireFighter extends Object{
	FireState extinguishFire(Fire fire)
}
\end{lstlisting}

\begin{lstlisting}[style = dsl]
provided Doctor extends Object{	
	void heal( Patient pat, Medicine med )
}
\end{lstlisting}


\begin{lstlisting}[style = dsl]
provided InverseDoctor extends Object{	
	void heal( Medicine med, Patient pat )
}
\end{lstlisting}

\begin{lstlisting}[style = dsl]
provided MedCabinet extends Object{
	Medicine med
}
\end{lstlisting}

\begin{lstlisting}[style = dsl]
required PatientMedicalFireFighter {
	void heal( Patient patient, MedCabinet med )
	boolean extinguishFire( ExtFire fire )	
}
\end{lstlisting}

\begin{lstlisting}[caption={Bibliothek \emph{ExampLe} von Typen},captionpos=b, style = dsl, label=lst:libEx]
required MedicalFireFighter {
	void heal( Injured injured, MedCabinet med )
	boolean extinguishFire( ExtFire fire )	
}
\end{lstlisting}

\newpage

\subsection{Definition der Matchern}\label{sec_matcher}
Ein Matcher definiert das Matching eines Typs $T$ zu einem Typ $T'$ durch die asymmetrische Relation $T \Rightarrow T'$.
\subsubsection{ExactTypeMatcher}\label{sec:exacttypematcher}
Der \emph{ExactTypeMatcher} stellt ein Matching von einem Typ $T$ zu demselben Typ $T$ her. Die dazugehörige Matchingrelation $\Rightarrow_{exact}$ wird durch folgende Regel beschrieben:
\begin{gather*}
\frac{}{T \Rightarrow_{exact} T}
\end{gather*}
\subsubsection{GenTypeMatcher}\label{sec:gentypematcher}
Der \emph{GenTypeMatcher} stellt ein Matching von einem Typ $T$ zu einem Typ $T'$ mit $T > T'$ her. Die dazugehörige Matchingrelation $\Rightarrow_{gen}$ wird durch folgende Regel beschrieben:
\begin{gather*}
\frac{T > T'}{T \Rightarrow_{gen} T'}
\end{gather*}
\paragraph{SpecTypeMatcher}
Der \emph{SpecTypeMatcher} stellt im Verhältnis zum \emph{GenTypeMatcher} das Matching in die entgegengesetzte Richtung dar. Die dazugehörige Matchingrelation $\Rightarrow_{spec}$ wird durch folgende Regel beschrieben: 
\begin{gather*}
\frac{T < T'}{T \Rightarrow_{spec} T'}
\end{gather*}
\\\\
Die oben genannten Matchingrelationen werden für die Definition weiterer Matcher zusammengefasst, wodurch sich die Matchingrelation $\Rightarrow_{internCont}$ ergibt:
\begin{gather*}
\frac{T \Rightarrow_{exact} T' \vee T \Rightarrow_{gen} T' \vee
T \Rightarrow_{spec} T'  }{T \Rightarrow_{internCont} T'}
\end{gather*}
\subsubsection{ContentTypeMatcher}
Der \emph{ContentTypeMatcher} matcht einen Typ $T$ auf einen Typ $T'$, wobei $T'$ ein Feld enthält, auf dessen Typ $T''$ der Typ $T$ über die Matchingrelation $\Rightarrow_{internCont}$ gematcht werden kann. So kann bspw. der Typ $\texttt{boolean}$ aus Listing 1 auf den Typ $\texttt{FireState}$ gematcht werden.
\\\\
Die dazugehörige Matchingrelation $\Rightarrow_{content}$ wird durch folgende Regel beschrieben:
\begin{gather*}
\frac{\exists \mathit{T''\text{ }f}\in felder(T'): T \Rightarrow_{internCont} T''}{T \Rightarrow_{content} T'}
\end{gather*}
\noindent
So würde für die Typen $\texttt{boolean}$ und $\texttt{FireState}$ gelten: 
\begin{gather*}
\texttt{boolean} \Rightarrow_{content} \texttt{FireState}
\end{gather*}
\subsubsection{ContainerTypeMatcher}
Der \emph{ContainerTypeMatcher} stellt im Verhältnis zum \emph{ContentTypeMatcher} das Matching in die entgegengesetzte Richtung dar. So kann bspw. auch der Typ $\texttt{FireState}$ auf den Typ $\texttt{booealn}$ aus Listing 1 gematcht werden.
\\\\
Die dazugehörige Matchingrelation $\Rightarrow_{container}$ wird durch folgende Regel beschrieben:
\begin{gather*}
\frac{\exists \mathit{T''\text{ }f}\in felder(T): T'' \Rightarrow_{internCont} T'}{T \Rightarrow_{container} T'}
\end{gather*}
\noindent
So gilt für die Typen $\texttt{FireState}$ und $\texttt{boolean}$: 
\begin{gather*}
\texttt{FireState} \Rightarrow_{container} \texttt{boolean}
\end{gather*}
\\\\
Zur Definition des letzten Matchers werden die Matchingrelationen der oben genannten Matcher noch einmal zusammengefasst. Dabei entsteht die Matchingrelation $\Rightarrow_{internStruct}$, welche durch folgende Regel beschrieben wird:
\begin{gather*}
\frac{T \Rightarrow_{internCont}T' \vee T \Rightarrow_{container} T' \vee T \Rightarrow_{content} T'}{T \Rightarrow_{internStruct}T'}
\end{gather*}
\subsubsection{StructuralTypeMatcher} 
Der \emph{StructuralTypeMatcher} matcht einen \emph{required Typ} $R$ auf einen \emph{provided Typ} $P$ auf der Basis struktureller Eigenschaften der Methoden, die in den Typen deklariert sind. 
\\\\
Somit soll bspw. der Typ $\texttt{MedicalFireFighter}$ auf den Typ $\texttt{FireFighter}$ (siehe Listing 1) gematcht werden. Als ein weiteres Beispiel, bezogen auf die Typen aus Listing 1, kann das Matching des Typs $\texttt{MedicalFireFighter}$ auf den Typ $\texttt{Doctor}$ angebracht werden.
\\\\
Damit ein required Typ $R$ auf einen provided Typ $P$ über den \emph{StrukturalTypeMatcher} gematcht werden kann, muss mindestens eine Methode aus $R$ zu einer Methode aus $P$ gematcht werden. Die Reihenfolge, in der die Parameter in der jeweiligen Methode deklariert sind, soll dabei keine Rolle spielen. Von daher wird das Matching der Parameter zweier Methoden $m$ und $m'$ wie folgt beschrieben:
\begin{gather*}
\mathit{matchingParams(m, m')} :=
\left\{
\begin{array}{l|l}
	&
	\{\mathit{P_1,...,P_n}\} = \mathit{params(m)} \wedge \mathit{ }
	\\
	\{\mathit{mP_1,...,mP_n}\}
	&
	\forall i \in \{1,...,n\}: \mathit{mP_i} \in \mathit{params(m'}) \wedge \mathit{ }
	\\
	&
	\mathit{mP_i} \Rightarrow_{internStruct} \mathit{P_i}
\end{array}
\right\}
\end{gather*}
\noindent
Das strutkurelle Matching zweier Methoden $m$ und $m'$ wird durch folgende Regel beschrieben:
\begin{gather*}
\frac{\mathit{ret(m)} \Rightarrow_{internStruct} \mathit{ret(m')} \wedge \mathit{matchingParams(m,m')}}{m \Rightarrow_{method} m'}
\end{gather*}
\noindent
Die Menge der gematchten Methoden aus $R$ in $P$ wird darauf aufbauend durch folgende Funktion beschrieben:
\begin{gather*}
structM(R,P) := \left\{ 
				\begin{array}{l|l}
m	& \mathit{m} \in \mathit{methoden(R)} \wedge \mathit{ }
\\
	& \exists \mathit{m'} \in \mathit{methoden(P)} : m \Rightarrow_{method} m'
				\end{array}
              \right\}
\end{gather*}
\noindent

Die Matchingrelation für die \emph{StructuralTypeMatcher} wird durch folgende Regel beschrieben:
\begin{gather*}
\frac{structM(R,P) \neq \emptyset}{R \Rightarrow_{struct}P}
\end{gather*}


\subsection{Ergebnis der strukturellen Evaluation}\label{sec_ergStructEval}
Die gesamte Exploration wird für einen required Typ durchgeführt. Bei der strukturellen Evaluation sollen dabei Mengen von provided Typen ermittelt werden, deren Methoden in Kombination zu jeder Methode des required Typ ein Matching aufweisen. Die Mengen von provided Typen innerhalb einer Bibliothek $L$ für die dies in Bezug auf ein required Typ $R$ zutrifft, wird über die Funktion $cover$ beschrieben.

\begin{gather*}
cover(R,L) := 
\left\{\begin{array}{l|l}
					& T_1 \in L \wedge \text{...} \wedge T_n \in L 								\wedge \mathit{ }\\
\{T_1,...,T_n\}		& \mathit{methoden(R)} = \mathit{structM(R,T_1)}							\cup \mathit{ }\\
					& \texttt{...} \cup \mathit{structM(R, T_n)} 								\wedge \mathit{ }\\
					& \forall T \in \{T_1,...,T_n\}:											\mathit{structM(R,T)} \neq \emptyset
\end{array}\right\}
\end{gather*}

\begin{example}{bsp_cover}
Sei folgende Bibliothek $L$ gegeben.
\begin{lstlisting}[style = dsl]
provided Come extends Object{
	String hello()
	String goodMorning()
}

provided Leave extends Object{
	String bye()
}

required Greeting{
	String hello()
	String bye()
}
\end{lstlisting}
Über die Funktion $\mathit{cover}$ werden folgenden Mengen von Target-Typen für die Bildung von Proxies für den required Typ $\texttt{Greeting}$ ermittelt.
\begin{gather*}
\mathit{cover(\texttt{Greeting},L)} = \{
	\{\texttt{Come}\},\{\texttt{Leave}, \texttt{Come}\}
\}
\end{gather*}
\end{example}

