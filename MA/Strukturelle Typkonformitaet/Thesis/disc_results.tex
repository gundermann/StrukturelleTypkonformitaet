\section{Auswertung der Untersuchungsergebnisse}
\subsection{Einzelbetrachtung}\label{disc_einzel}
Die in Kapitel \ref{chap_eval} beschriebenen Untersuchungsergebnisse zeigen, dass die Heuristiken die Anzahl der zu generierenden und zu evaluierenden Proxies reduzieren. Dabei zeigt sich, dass sich die Heuristiken nicht auf alle Explorationsdurchläufe positiv auswirken. So kann für die Heuristiken LMF und PTTF festgehalten werden, dass diese nur in den Durchlauf eine positive Wirkung erzielt, in dem ein passender Proxy auch gefunden wird.
\\\\
Die Heuristiken BL\_NMC hingegen wirkt sich auf jeden der durchgeführten Durchläufe aus. Dies liegt zu einen daran, dass die Menge der Informationen, auf deren Basis sie arbeitet, während eines Durchlaufs anwächst. Bei der Heuristik LMF ist dies nicht der Fall. Allerdings weist die Heuristik PTTF ebenfalls dieses Merkmal auf.
\\\\
Ein weiterer Grund ist, dass die Heuristik BL\_NMC dafür sorgt, dass Proxies bei der Evaluierung mitunter übersprungen werden, oder diese gar nicht erst generiert werden. Die anderen Heuristiken hingegen sorgen lediglich für eine Umsortierung der zu generierenden bzw. zu evaluierenden Proxies. Somit müssen unter der Verwendung der Heuristiken LMF und PTTF im Zweifelsfall alle Proxies generiert und erzeugt werden, auch wenn kein passender Proxy ausgemacht werden kann.
\\\\
Weiterhin ist festzuhalten, dass mit der Heuristik BL\_NMC scheinbar die besten Ergebnisse erzielt werden. Eine Ausnahme könnte hier die Exploration zum required Typ $\texttt{KOFGPCProvider}$ (\emph{TEI7}) darstellen. Die Ergebnisse bzgl. dieser Exploration sehen für die Heuristik LMF auf den ersten Blick besser aus. Allerdings konnte festgestellt werden, dass die Anzahl der Proxies, die im ersten Durchlauf im Ausgangszustand für \emph{TEI7} generiert und evaluiert werden mussten insgesamt 161294 ($\mathit{p(174)}$) beträgt. So kann festgehalten werden, dass mit der Heuristik BL\_NMC insgesamt 136141 generiert und evaluiert wurden. Damit liefert diese Heuristik die besten Ergebnisse, da die Heuristik LMF auf den ersten Durchlauf keine positive Auswirkung hat und somit in diesem 161294 Proxies generiert und evaluiert wurden.
\\\\
Bei einer Betrachtung der finalen Durchläufe - also in denen ein passender Proxy gefunden werden konnte - ergibt sich jedoch ein anderes Bild. Hierbei erweist sich in den meisten Fällen die Heuristik LMF als diejenige, mit der die besten Ergebnisse erzielt wurde. Eine Ausnahme stellen die required Typen $\texttt{FoerderprogrammeProvider}$ (\emph{TEI2}) und $\texttt{MinimalFoerderprogrammeProvider}$ (\emph{TEI3}) dar. Bei diesen beiden Typen lieferte wiederum die Heuristik BL\_NMC die besten Ergebnisse.
\\\\
Dies kann dadurch begründet werden, dass die provided Typen, die in den Methoden dieser beiden required Typen ab stärksten von denen abwichen, die in den Delegationsmethoden des passenden Proxies verwendet wurden. Aus dieser Tatsache ergibt sich für das Matcherrating ein höherer Wert, wodurch die Evaluierung des Proxies durch die Heuristik LMF weiter nach hinten verschoben wird, als es bei den Explorationen zum den übrigen required Typen der Fall war.
\subsection{Synergien}
Neben der Einzelbetrachtung der Heuristiken wurden in Abschnitt \ref{sec_kombis} auch die Kombinationen der drei Heuristiken untersucht. Aus den Feststellungen in Abschnitt \ref{disc_einzel} lässt sich ableiten, dass eine Kombinationen mit der Heuristik BL\_NMC durchaus sinnvoll ist; egal ob sie mit der Heuristik LMF oder PTTF kombiniert wird. Der Grund dafür liegt wiederum in der Tatsache, dass 
\subsection{Erhöhte Komplexität}
Die vorliegende Untersuchung zweigt zwar, dass die Anzahl der zu evaluierungen Proxies in dem verwendeten System mit den vorgeschlagenen Heuristiken reduziert werden können. Allerdings wurden negative Auswirkungen wie bspw. Speichernutzung (Speicherkomplexität) oder die benötigte Zeit  (Zeitkomplexität) für die Evaluation nicht untersucht.
\\\\
Die Anwendung der Heuristiken hängt, wie in Abschnitt \ref{sec_heuristics} beschrieben, von Informationen ab, die teilweise aus den für die Proxies verwendeten \emph{provided Typen} ermittelt werden müssen (Matcherrating) bzw. nach der Ausführung der Tests über die gesamte restliche Laufzeit der Exploration verwaltet werden müssen. Von daher ist davon auszugehen, dass sich die Anwendung der Heuristiken durchaus auf den Speicherverbrauch auswirkt.
\\\\
Da die benötigte Zeit für die Verwaltung von Listen, wie sie bei den Heuristiken vorgenommen wird, mit der Anzahl der zu verwaltenden Elemente wächst, kann davon ausgegangen werden, dass die Anwendung der Heuristiken ebenfalls mehr Zeit in Anspruch nimmt, je weiter fortgeschritten die Exploration ist. Die gilt insbesondere für die Heuristiken \emph{PTTF} und \emph{BL\_NMC}. 
\\\\
Aufgrunddessen, dass in dieser Arbeit lediglich die Anzahl der zu evaluierenden Proxies während der Exploration untersucht wurden, ist es auch nicht auszuschließen, dass die verwendete Implementierung kein Optimierungspotential besitzt.


