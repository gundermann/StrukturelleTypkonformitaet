\section{Auswertung der Untersuchungsergebnisse}
\subsection{Einzelbetrachtung}\label{disc_einzel}
Die in Kapitel \ref{chap_evaluation} beschriebenen Untersuchungsergebnisse zeigen, dass die Heuristiken die Anzahl der zu generierenden und zu evaluierenden Proxies reduzieren. Dabei zeigt sich, dass sich die Heuristiken nicht auf alle Explorationsdurchläufe positiv auswirken. So kann für die Heuristiken \emph{LMF} und \emph{PTTF} festgehalten werden, dass diese nur in dem Durchlauf eine positive Wirkung erzielt, in dem ein passender Proxy auch gefunden wird.
\\\\
Die Heuristik \emph{BL\_NMC} hingegen wirkt sich auf jeden der durchgeführten Durchläufe aus. Dies liegt zum einen daran, dass die Menge der Informationen, auf deren Basis sie arbeitet, während eines Durchlaufs anwächst. Bei der Heuristik LMF ist dies nicht der Fall. Allerdings weist die Heuristik \emph{PTTF} ebenfalls dieses Merkmal auf.
\\\\
Ein weiterer Grund ist, dass die Heuristik \emph{BL\_NMC} dafür sorgt, dass Proxies bei der Evaluierung mitunter übersprungen werden, oder diese gar nicht erst generiert werden. Die anderen Heuristiken hingegen sorgen lediglich für eine Umsortierung der zu generierenden bzw. zu evaluierenden Proxies. Somit müssen unter der Verwendung der Heuristiken \emph{LMF} und \emph{PTTF} im Zweifelsfall alle Proxies generiert und erzeugt werden, auch wenn kein passender Proxy ausgemacht werden kann.
\\\\
Weiterhin ist festzuhalten, dass mit der Heuristik \emph{BL\_NMC} scheinbar die besten Ergebnisse erzielt werden. Eine Ausnahme bildet hier lediglich die Exploration zum required Typ $\texttt{ElerFTFoerderprogrammeProvider}$ (\emph{TEI1}). Die Ursache dafür liegt darin begründet, dass die in den Methoden von TEI1 verwendeten \emph{provided Typen} mit denen des erwarteten \emph{provided Typen}, auf dessen Basis ein passender Proxy erzeugt wird, genau übereinstimmen. Damit wird ein vergleichsweise geringes Matcherrating für das Matching dieser beiden Typen ermittelt, wodurch der Proxy sehr früh während der Exploration generiert und evaluiert wird.
\subsection{Synergien}\label{disc_synergien}
Neben der Einzelbetrachtung der Heuristiken wurden in Abschnitt \ref{sec_evalKombis} auch die Kombinationen der drei Heuristiken untersucht. Aus den Feststellungen in Abschnitt \ref{disc_einzel} lässt sich ableiten, dass eine Kombination mit der Heuristik \emph{BL\_NMC} durchaus sinnvoll ist, egal ob sie mit der Heuristik \emph{LMF} oder \emph{PTTF} kombiniert wird. Der Grund dafür liegt wiederum in der Tatsache, dass die Heuristiken \emph{LMF} und \emph{PTTF} lediglich auf einen der Explorationsdurchläufe einen positiven Effekt haben. Aus diesem Grund kann in Kombination mit der Heuristik \emph{BL\_NMC} wenigstens in den anderen Durchläufen eine positive Auswirkung festgestellt werden.
\\\\
Dementgegen liefert die Kombination der Heuristiken \emph{LMF} und \emph{PTTF} miteinander kaum bessere Ergebnisse als die Heuristik LMF alleine. Eine Ausnahme bildet der required Typ $\texttt{KOFGPCProvider}$ (\emph{TEI7}). Dazu ist jedoch zu sagen, dass gerade zu diesem \emph{required Typ} im Vergleich zu den anderen required Typen die meisten matchenden \emph{provided Typen} existieren. Insofern darf dieser scheinbare Ausreißer nicht unterschätzt werden, weshalb auch die Kombination der oben genannten Heuristiken sinnvoll ist.
\\\\
Ähnliches gilt für die Kombination aller vorgestellten Heuristiken (\emph{LMF + PTTF + BL\_NMC}). Dies ergibt sich ebenfalls aus den vorherigen Auswertungen bzgl. der Synergien in diesem Abschnitt. Bei der Betrachtung der Untersuchungsergebnisse zeigt sich hier ein ähnliches Muster wie zuvor: Die Kombination aller vorgestellten Heuristiken liefert nur für den \emph{required Typ} $\texttt{KOFGPCProvider}$ (\emph{TEI7}) bessere Ergebnisse, als die Kombination der Heuristiken \emph{BL\_NMC} und LMF. Aber auch hier darf dieses Ergebnis aufgrund der Eigenschaften von \emph{TEI7} nicht vernachlässigt werden.

\subsection{Erhöhte Komplexität}
Die vorliegende Untersuchung zeigt, dass die Anzahl der zu evaluierenden Proxies in dem verwendeten System mit den vorgeschlagenen Heuristiken reduziert werden können. Allerdings wurden negative Auswirkungen wie bspw. Speichernutzung (Speicherkomplexität) oder die benötigte Zeit (Zeitkomplexität) für die Exploration nicht untersucht.
\\\\
Die Anwendung der Heuristiken hängt, wie in Abschnitt \ref{sec_heuristics} beschrieben, von Informationen ab, die teilweise aus den für die Proxies verwendeten \emph{provided Typen} ermittelt werden müssen (Matcherrating) bzw. nach der Ausführung der Tests über die gesamte restliche Laufzeit der Exploration verwaltet werden müssen. Von daher ist davon auszugehen, dass sich die Anwendung der Heuristiken durchaus auf den Speicherverbrauch auswirkt.
\\\\
Da die benötigte Zeit für die Verwaltung von Listen, wie sie bei den Heuristiken vorgenommen wird, mit der Anzahl der zu verwaltenden Elemente wächst, kann davon ausgegangen werden, dass die Anwendung der Heuristiken ebenfalls mehr Zeit in Anspruch nimmt, je weiter fortgeschritten die Exploration ist. Dies gilt insbesondere für die Heuristiken PTTF und BL\_NMC. 
\\\\
Aufgrund dessen, dass in dieser Arbeit lediglich die Anzahl der zu evaluierenden Proxies während der Exploration untersucht wurden, ist es auch nicht auszuschließen, dass die verwendete Implementierung kein Optimierungspotential besitzt.

\subsection{Zusammenfassung}
Die Ausführungen der Abschnitt \ref{disc_einzel} und \ref{disc_synergien} lassen vermuten, dass lediglich die Heuristiken \emph{LMF} und\emph{ BL\_NMC} eine Daseinsberechtigung haben. Dies ist nicht korrekt. Die Heuristik \emph{PTTF} liefert zwar schlechtere Ergebnisse, dennoch hat sie die zu generierenden und zu prüfenden Proxies im Vergleich zum schlimmst Fall ohne Heuristiken stark reduziert. Allerdings hat der Entwickler keinen höheren Aufwand bei der Implementierung der Testfälle. Die Heuristik \emph{BL\_NMC}, welche sich in dieser Untersuchung häufig als diejenige mit den besten Ergebnissen herausgestellt hat, bedarf einer speziellen Implementierung der Testfälle.
\\\\
Dasselbe gilt für die Heuristik \emph{LMF}. Diese liefert zwar bessere Ergebnisse als die Heuristik PTTF, kann aber aufgrund dessen, dass sie sich lediglich auf den finalen Explorationsdurchlauf positiv auswirkt, nur in wenigen Fällen mit der Heuristik BL\_NMC mithalten. Allerdings gilt auch hier, dass keine weiteren Anforderungen an die Arbeit des Entwicklers gestellt werden. Dazu kommt noch, dass die Ermittlung der Matcherratings quasi bei dem Matching der Typen mit abfällt, wodurch die Verwendung dieser Heuristik kaum eine Auswirkung auf die Komplexität der Exploration hat.

