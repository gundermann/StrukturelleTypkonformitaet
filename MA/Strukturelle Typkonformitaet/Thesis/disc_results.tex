\section{Auswertung der Untersuchungsergebnisse}
\subsection{Einzelbetrachtung}\label{disc_einzel}
Die in Kapitel \ref{chap_evaluation} beschriebenen Untersuchungsergebnisse zeigen, dass die \Gls{Heuristik}en die Anzahl der zu generierenden und zu prüfenden \emph{Proxies} reduzieren. Dabei zeigt sich, dass sich die \Gls{Heuristik}en nicht auf alle Durchläufe der \emph{semantischen Evaluation} positiv auswirken. So kann für die \Gls{Heuristik}en \emph{LMF} und \emph{PTTF} festgehalten werden, dass diese nur in dem Durchlauf eine positive Wirkung erzielt, in dem auch ein passender \emph{Proxy} gefunden wird.
\\\\
Die \Gls{Heuristik} \emph{BL\_NMC} hingegen wirkt sich auf jeden der durchgeführten Durchläufe aus. Ein Grund dafür ist, dass die Menge der Informationen, auf deren Basis diese \Gls{Heuristik} arbeitet, während eines Durchlaufs anwächst. Bei der \Gls{Heuristik} \emph{LMF} ist dies nicht der Fall. Hier stehen die notwendigen Informationen bereits nach der \emph{strukturelle Evaluation} zur Verfügung und ändern sich nicht mehr. Anders ist es wiederum bei der Heuristik \emph{PTTF}, die ebenfalls mit Informationen arbeitet, die während dem Fortschreiten der \emph{semantischen Evaluation} anwachsen. Daher muss die oben genannte Wirkung der \Gls{Heuristik} \emph{BL\_NMC} auf etwas anderes zurückzuführen sein. 
\\\\
Ein weiterer Grund dafür ist, dass die Heuristik \emph{BL\_NMC} dafür sorgt, dass \emph{Proxies} bei der \emph{semantischen Evaluation} mitunter übersprungen werden, oder diese gar nicht erst generiert werden. Die anderen \Gls{Heuristik}en hingegen sorgen lediglich für eine Umsortierung der zu generierenden und zu prüfenden \emph{Proxies}. Somit müssen unter der Verwendung der \Gls{Heuristik}en  \emph{LMF} und \emph{PTTF} im Zweifelsfall alle \emph{Proxies} generiert und geprüft werden, auch wenn kein passender \emph{Proxy} während des aktuellen Durchlaufs ausgemacht werden kann.
\\\\
Weiterhin ist festzuhalten, dass mit der \Gls{Heuristik} \emph{BL\_NMC} scheinbar die besten Ergebnisse erzielt werden. Eine Ausnahme bildet hier lediglich die Exploration zum \emph{required Typ} \linebreak$\texttt{ElerFTFoerderprogrammeProvider}$ (\emph{TEI1}). Für diesen \emph{required Typ} wurden die besten Ergebnisse mit der \Gls{Heuristik} \emph{LMF} erzielt. Die Ursache dafür liegt darin begründet, dass die in den Methoden von \emph{TEI1} verwendeten Typ mit denen, die innerhalb des erwarteten \emph{provided Typen}, auf dessen Basis ein passender \emph{Proxy} erzeugt wird, exakt übereinstimmen. Damit wird ein vergleichsweise geringes \emph{Matcherrating} für das Matching dieser beiden Typen ermittelt, wodurch der \emph{Proxy} sehr früh während der \emph{semantischen Evaluation} generiert und geprüft wird.
\subsection{Synergien}\label{disc_synergien}
Neben der Einzelbetrachtung der \Gls{Heuristik}en wurden in Abschnitt \ref{sec_evalKombis} auch die Kombinationen der drei \Gls{Heuristik}en untersucht. Aus den Feststellungen in Abschnitt \ref{disc_einzel} lässt sich ableiten, dass eine Kombination mit der \Gls{Heuristik} \emph{BL\_NMC} durchaus sinnvoll ist, egal ob sie mit der \Gls{Heuristik} \emph{LMF} oder \emph{PTTF} kombiniert wird. Der Grund dafür liegt wiederum in der Tatsache, dass die \Gls{Heuristik}en \emph{LMF} und \emph{PTTF} lediglich auf einen der Durchläufe einen positiven Effekt haben. Aus diesem Grund kann in Kombination mit der Heuristik \emph{BL\_NMC} wenigstens in den anderen Durchläufen eine positive Auswirkung festgestellt werden.
\\\\
Dem entgegen liefert die Kombination der \Gls{Heuristik}en \emph{LMF} und \emph{PTTF} miteinander kaum bessere Ergebnisse als die \Gls{Heuristik} \emph{LMF} alleine. Eine Ausnahme bildet der \emph{required Typ} $\texttt{KOFGPCProvider}$ (\emph{TEI7}). Dazu ist jedoch zu sagen, dass gerade zu diesem \emph{required Typ} im Vergleich zu den anderen \emph{required Typen} die meisten matchenden \emph{provided Typen} existieren. Insofern darf dieser scheinbare Ausreißer nicht unterschätzt werden, weshalb auch die Kombination der oben genannten \Gls{Heuristik}en \emph{LMF} und \emph{PTTF} als sinnvoll anzusehen ist.
\\\\
Ähnliches gilt für die Kombination aller vorgestellten Heuristiken (\emph{LMF + PTTF + BL\_NMC}). Dies ergibt sich ebenfalls aus den vorherigen Auswertungen bzgl. der Synergien in diesem Abschnitt. Bei der Betrachtung der Untersuchungsergebnisse zeigt sich hier ein ähnliches Muster wie zuvor: Die Kombination aller vorgestellten \Gls{Heuristik}en liefert nur für den \emph{required Typ} $\texttt{KOFGPCProvider}$ (\emph{TEI7}) bessere Ergebnisse, als die Kombination der \Gls{Heuristik}en \emph{LMF} und \emph{BL\_NMC}. Aber auch hier darf dieses Ergebnis aufgrund der Eigenschaften von \emph{TEI7} nicht vernachlässigt werden.

\subsection{Erhöhte Komplexität}
Die vorliegende Untersuchung zeigt, dass die Anzahl der zu generierenden und zu prüfenden \emph{Proxies} in dem verwendeten System mit den vorgeschlagenen \Gls{Heuristik}en reduziert werden kann. Allerdings wurden negative Auswirkungen wie bspw. Speichernutzung (\Gls{Speicherkomplexitaet}) oder die benötigte Zeit (\Gls{Zeitkomplexitaet}) für den \emph{Explorationsprozess} nicht untersucht.
\\\\
Die Anwendung der \Gls{Heuristik}en hängt, wie in Abschnitt \ref{sec_heuristics} beschrieben, von Informationen ab, die teilweise aus den für die \emph{Proxies} verwendeten \emph{provided Typen} ermittelt werden müssen (Matcherrating) bzw. nach der Ausführung der Tests über die gesamte restliche Laufzeit des \emph{Explorationsprozesses} verwaltet werden müssen. Von daher ist davon auszugehen, dass sich die Anwendung der \Gls{Heuristik}en durchaus auf den Speicherverbrauch auswirkt.
\\\\
Da die benötigte Zeit für die Verwaltung von Listen, wie sie bei den \Gls{Heuristik}en vorgenommen wird, mit der Anzahl der zu verwaltenden Elemente wächst, kann davon ausgegangen werden, dass die Anwendung der \Gls{Heuristik}en ebenfalls mehr Zeit in Anspruch nimmt, je weiter fortgeschritten der \emph{Explorationsprozess} ist. Dies gilt insbesondere für die \Gls{Heuristik}en \emph{PTTF} und \emph{BL\_NMC}. 

\subsection{Zusammenfassung}
Die Ausführungen der Abschnitt \ref{disc_einzel} und \ref{disc_synergien} lassen vermuten, dass die lediglich die \Gls{Heuristik}en \emph{LMF} und \emph{BL\_NMC} eine Daseinsberechtigung haben. Dies ist nicht korrekt. Die \Gls{Heuristik} \emph{PTTF} liefert zwar schlechtere Ergebnisse, dennoch hat sie die zu generierenden und zu prüfenden \emph{Proxies} im Vergleich zum Ausgangspunkt (siehe Abschnitt \ref{sec_ausgangspunkt}) stark reduziert. Zudem haben die Entwickler*innen bei der Verwendung der \Gls{Heuristik} \emph{PTTF} keinen höheren Aufwand bei der Implementierung der Testfälle. 
\\\\
Dies gilt auch für die \Gls{Heuristik} \emph{LMF}. Diese kann aufgrund dessen, dass sie sich lediglich auf den finalen Durchlauf des \emph{semantischen Evaluation} positiv auswirkt, nur in wenigen Fällen mit der \Gls{Heuristik} \emph{BL\_NMC} mithalten. Allerdings gilt auch hier, dass keine weiteren Anforderungen an die Arbeit der Entwickler*innen gestellt werden. Dazu kommt noch, dass die Ermittlung der \emph{Matcherratings} quasi bei dem Matching der Typen mit abfällt, wodurch die Verwendung dieser \Gls{Heuristik} kaum eine Auswirkung auf die \Gls{Komplexitaet} des \emph{Explorationsprozesses} hat.
\\\\
Die \Gls{Heuristik} \emph{BL\_NMC}, welche sich in dieser Untersuchung häufig als diejenige mit den besten Ergebnissen herausgestellt hat, bedarf einer speziellen Implementierung der Testfälle. Weiterhin ist davon auszugehen, dass diese \Gls{Heuristik} von allen in dieser Arbeit beschriebenen \Gls{Heuristik}en aufgrund der Menge an Informationen, die für diese \Gls{Heuristik} gesammelt werden, den größten negativen Einfluss auf die \Gls{Komplexitaet} des \emph{Explorationsprozesses} hat.

