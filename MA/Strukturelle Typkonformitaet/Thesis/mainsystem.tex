\section{Heiß-System}\label{sec_hotsystem}
Im Heiß-System werden, wie beschrieben, die 889 \emph{provided Typen} verwendet, für die jeweils mindestens eine Implementierung bereitgestellt wurde. Die \emph{provided Typen}, die im Test-System ergänzt wurde, befinden sich nicht im Heiß-System.
\\\\
Für die Evaluation im Test-System werden insgesamt 4 \emph{required Typen} verwendet. Drei dieser \emph{required Typen} wurden ebenfalls im zur Evaluation im Test-System verwendet. Die Deklaration der \emph{required Typen} ist im Anhang \ref{xyz} zu finden.  
\\\\
Die Java-Interfaces, die sich aus dieser Deklaration ableiten lassen, die dazugehörigen Implementierungen der \emph{provided Typen} und die vordefinierten Testfälle sind in Anhang \ref{abc} zu finden.
%\\\\
In \tabref{eIMainShort} sind die Namen der \emph{required Typen} zusammen mit jeweils einem Kürzel aufgeführt. Die Kürzel dienen im weiteren Verlauf der Identifizierung der \emph{required Typen}.
\begin{table}[h!]
\centering
\small
\begin{tabular}{|l|c|}
\hline
\hline
\centering\textbf{required Typ} & \textbf{Kürzel} \\
\hline
\hline
ElerFTFoerderprogrammeProvider & TEI1\\
\hline
FoerderprogrammeProvider & TEI2\\
\hline
MinimalFoerderprogrammeProvider & TEI3\\
\hline
KOFGPCProvider & TEI4\\
\hline
\hline
\end{tabular}
\caption{Kürzel der required Typen für die Evaluation im Heiß-System}
 \label{tab:eIMainShort}
\end{table}
\noindent
%Neben den in Abschnitt \ref{sec_matcher} und \ref{sec_proxies} beschriebenen Matcher und Proxy-Generatoren wird im Heiß-System ein zusätzlicher Matcher sowie ein darauf basierender Generator für Proxies benötigt.
%\\\\
%Der Grund dafür ist, dass im Heiß-System zur Laufzeit Objekte existieren, die über bestimmte Zeichenketten oder numerische Werte eindeutig identifiziert werden. Die Typen solcher Objekte werden im Heiß-System als Domain-Values bezeichnet.
%\\\\
%Die Semantik dieser Domain-Values soll bei der Suche nach einem passenden Proxy zu einem \emph{required Typ} nicht untergraben werden. 


\subsection{Ausgangspunkt}
Für ein \emph{reqiured Typ} können mehrere \emph{provided Typen} gefunden werden, die eine strukturelle Übereinstimmung aufwiesen. \tabref{amountMatchedInterfacesHot} zeigt die Anzahl der strukturell übereinstimmenden \emph{provided Typen} je \emph{reqiured Typ}. Diese kommen einzeln oder in Kombination für die semantische Evaluation in Frage.
\begin{table}[H]
\centering
\small
\singlespacing
			\begin{tabular}[c]{|>{\centering\arraybackslash}p{2cm}|>{\centering\arraybackslash}p{5cm}|}
			\hline
			\hline
				 \textbf{required Interface} & \textbf{Anzahl strukturell übereinstimmender provided Interfaces} \\
				\hline\hline
				TEI1 & 221 \\
				\hline
				TEI2 & 272\\
				\hline
				TEI3 & 268 \\
				\hline
				TEI4 & 348 \\
				\hline
				\hline
			\end{tabular} 
 \caption{Anzahl strukturell übereinstimmender provided Typen je required Typ im Heiß-System}
 \label{tab:amountMatchedInterfacesHot}
\onehalfspacing
\end{table}
\noindent
Die \tabsrefs{hs_start_tei1}{hs_start_tei4_2} zeigen die Vier-Felder-Tafeln, in denen die Ergebnisse der benötigten Iterationen innerhalb des Explorationsalgorithmus für jeden der \emph{required Typen} aus \tabref{amountMatchedInterfacesHot}. Dabei wurden keine Heuristiken verwendet. Somit stellt dies den Ausgangspunkt für die weitere Evaluation im Heiß-System dar.
\begin{multicols}{3}
\vft{1}{$p(44)-1$}{0}{1}{0}{Ausgangspunkt im Heiß-System für TEI1}{hs_start_tei1}\columnbreak
\vft{1}{$p(30)-1$}{0}{1}{0}{Ausgangspunkt im Heiß-System für TEI2}{hs_start_tei2}\columnbreak
\vft{1}{$p(30)-1$}{0}{1}{0}{Ausgangspunkt im Heiß-System für TEI3}{hs_start_tei3}
\end{multicols}
\begin{multicols}{2}
\vft{1}{$p(50)-1$}{0}{1}{0}{Ausgangspunkt im Heiß-System für TEI4 \\1. Durchlauf}{hs_start_tei4_1}\columnbreak
\vft{2}{$p(7714)-1$}{0}{1}{0}{Ausgangspunkt im Heiß-System für TEI4 \\2. Durchlauf}{hs_start_tei4_2}\columnbreak
\end{multicols}
Für den \emph{required Typen} \emph{TEI4} werden zwei Durchläufe benötigt, da die semantischen Test nur von einem Proxy bestanden werden, der aus einer Kombination zweier \emph{provided Typen} erzeugt wurde.

\section{Ergebnisse für die Heuristik PTTF}\label{sec_evalPTTF}
Für die \Gls{Heuristik} \emph{PTTF} gilt es zu evaluieren, ob die Suche nach einem \emph{Proxy}, der die vordefinierten Tests besteht, beschleunigt werden kann. Hierzu wird der \emph{Explorationsprozess} für alle in Tabelle \ref{tab:eIShort} genannten \emph{required Typen} unter der Verwendung der in Abschnitt \ref{sec_pttf} beschriebenen \Gls{Heuristik} durchgeführt.
\\\\
Die folgenden Vier-Felder-Tafeln zeigen die Ergebnisse für die \emph{required Typen} \emph{TEI1}-\emph{TEI7} auf.
\begin{multicols}{3}
\vft{1}{29}{$p_1(44)-30$}{1}{0}{Ergebnisse \emph{PTTF} für TEI1 1.~\mbox{Durchlauf}}{pttf_TEI1_1}
\vft{1}{5544}{$p_1(55)-5545$}{1}{0}{Ergebnisse \emph{PTTF} für TEI2 1.~\mbox{Durchlauf}}{pttf_TEI2_1}
\vft{1}{4761}{$p_1(50)-4762$}{1}{0}{Ergebnisse \emph{PTTF} für TEI3 1.~\mbox{Durchlauf}}{pttf_TEI3_1}
\end{multicols}

\begin{multicols}{2}
\vft{1}{$1174$}{0}{0}{0}{Ergebnisse \emph{PTTF} für TEI4 1.~\mbox{Durchlauf}}{pttf_TEI4_1}
\vft{2}{466}{$p_2(2247)-467$}{1}{0}{Ergebnisse \emph{PTTF} für TEI4 2.~\mbox{Durchlauf}}{pttf_TEI4_2}
\end{multicols}
\pagebreak
\begin{multicols}{2}
\vft{1}{$4984$}{0}{0}{0}{Ergebnisse \emph{PTTF} für TEI5 1.~\mbox{Durchlauf}}{pttf_TEI5_1}
\vft{2}{2172}{$p_2(2775)-2173$}{1}{0}{Ergebnisse \emph{PTTF} für TEI5 2.~\mbox{Durchlauf}}{pttf_TEI5_2}
\end{multicols}

\begin{multicols}{2}
\vft{1}{$1051$}{0}{0}{0}{Ergebnisse \emph{PTTF} für TEI6 1.~\mbox{Durchlauf}}{pttf_TEI6_1}
\vft{2}{13122}{$p_2(1323)-13123$}{1}{0}{Ergebnisse \emph{PTTF} für TEI6 2.~\mbox{Durchlauf}}{pttf_TEI6_2}
\end{multicols}

\begin{multicols}{2}
\vft{1}{$161294$}{0}{0}{0}{Ergebnisse \emph{PTTF} für TEI7 1.~\mbox{Durchlauf}}{pttf_TEI7_1}
\vft{2}{149961}{$p_2(52150)-149962$}{1}{0}{Ergebnisse \emph{PTTF} für TEI7 2.~\mbox{Durchlauf}}{pttf_TEI7_2}
\end{multicols}
\newpage
\noindent
Folgendes kann aus diesen Ergebnissen abgeleitet werden:
\begin{enumerate}
\item Die \Gls{Heuristik} \emph{PTTF} erzielt im Vergleich zum Ausgangspunkt (Abschnitt \ref{sec_ausgangspunkt}) für jeden \emph{required Typ} eine weitere Reduktion der zu prüfenden \emph{Proxies}.

\item Die Heuristik \emph{PTTF} hat keine Auswirkung auf einen Durchlauf, in dem kein \emph{Proxy} erzeugt wird, mit dem die vordefinierten Tests erfolgreich durchgeführt werden können. Dies kann durch einen Vergleich des ersten Durchlaufs für den \emph{required Typ} \emph{TEI4}-\emph{TEI7} im Ausgangspunkt (Tabelle \ref{tab:tmr_start_tei4_1}, \ref{tab:tmr_start_tei5_1}, \ref{tab:tmr_start_tei6_1} und \ref{tab:tmr_start_tei6_1}) mit dem ersten Durchlauf unter Anwendung der Heuristik (Tabellen \ref{tab:pttf_TEI4_1}, \ref{tab:pttf_TEI5_1}, \ref{tab:pttf_TEI6_1} und \ref{tab:pttf_TEI7_1}) festgestellt werden. Aus diesem Grund kommt die in Punkt 1 beschriebene Reduktion erst im jeweils letzten Durchlauf zum Tragen.
\end{enumerate}
\section{Ergebnisse für die Heuristik BL\_NMC}\label{sec_evalBLNMC}
Für die \Gls{Heuristik} \emph{BL\_NMC} gilt es zu evaluieren, ob die Suche nach einem \emph{Proxy}, der die vordefinierten Tests besteht, beschleunigt werden kann. Hierzu wird der \emph{Explorationsprozess} für alle in Tabelle \ref{tab:eIShort}genannten \emph{required Typen} unter der Verwendung der in Abschnitt \ref{sec_bl_nmc} beschriebenen \gls{Heuristik} durchgeführt.
\\\\
Die folgenden Vier-Felder-Tafeln zeigen die Ergebnisse für die \emph{required Typen} \emph{TEI1}-\emph{TEI7} auf.
\begin{multicols}{3}
\vft{1}{105}{$p_1(44)-106$}{1}{0}{Ergebnisse \emph{BL\_NMC} für TEI1 1.~\mbox{Durchlauf}}{blnmc_TEI1_1}
\vft{1}{342}{$p_1(55)-343$}{1}{0}{Ergebnisse \emph{BL\_NMC} für TEI2 1.~\mbox{Durchlauf}}{blnmc_TEI2_1}
\vft{1}{357}{$p_1(50)-358$}{1}{0}{Ergebnisse \emph{BL\_NMC} für TEI3 1.~\mbox{Durchlauf}}{blnmc_TEI3_1}
\end{multicols}

\begin{multicols}{2}
\vft{1}{120}{$1054$}{0}{0}{Ergebnisse \emph{BL\_NMC} für TEI4 1.~\mbox{Durchlauf}}{blnmc_TEI4_1}
\vft{2}{442}{$p_2(2247)-443$}{1}{0}{Ergebnisse \emph{BL\_NMC} für TEI4 2.~\mbox{Durchlauf}}{blnmc_TEI4_2}
\end{multicols}

\begin{multicols}{2}
\vft{1}{550}{$4434$}{0}{0}{Ergebnisse \emph{BL\_NMC} für TEI5 1.~\mbox{Durchlauf}}{blnmc_TEI5_1}
\vft{2}{1304}{$p_2(2775)-1305$}{1}{0}{Ergebnisse \emph{BL\_NMC} für TEI5 2.~\mbox{Durchlauf}}{blnmc_TEI5_2}
\end{multicols}
\pagebreak
\begin{multicols}{2}
\vft{1}{366}{$685$}{0}{0}{Ergebnisse \emph{BL\_NMC} für TEI6 1.~\mbox{Durchlauf}}{blnmc_TEI6_1}
\vft{2}{204}{$p_2(1323)-205$}{1}{0}{Ergebnisse \emph{BL\_NMC} für TEI6 2.~\mbox{Durchlauf}}{blnmc_TEI6_2}
\end{multicols}

\begin{multicols}{2}
\vft{1}{1051}{$160243$}{0}{0}{Ergebnisse \emph{BL\_NMC} für TEI7 1.~\mbox{Durchlauf}}{blnmc_TEI7_1}
\vft{2}{135089}{$p_2(52150)-135090$}{1}{0}{Ergebnisse \emph{BL\_NMC} für TEI7 2.~\mbox{Durchlauf}}{blnmc_TEI7_2}
\end{multicols}

Folgendes kann aus diesen Ergebnissen abgeleitet werden:
\begin{enumerate}
\item Die \Gls{Heuristik} \emph{BL\_NMC} erzielt im Vergleich zum Ausgangspunkt (Abschnitt \ref{sec_ausgangspunkt}) für jeden \emph{required Typ} eine weitere Reduktion der zu prüfenden \emph{Proxies}.

\item Die Heuristik \emph{BL\_NMC} hat das Potential jeden Durchlauf innerhalb der \emph{semantischen Evaluation} zu beschleunigen. Für den jeweils ersten Durchlauf kann dies durch einen Vergleich der Tabellen \ref{tab:tmr_start_tei1}, \ref{tab:tmr_start_tei2}, \ref{tab:tmr_start_tei3}, \ref{tab:tmr_start_tei4_1}, \ref{tab:tmr_start_tei5_1}, \ref{tab:tmr_start_tei6_1} und \ref{tab:tmr_start_tei7_1} zum Ausgangspunkt mit den Tabellen \ref{tab:blnmc_TEI1_1}, \ref{tab:blnmc_TEI2_1}, \ref{tab:blnmc_TEI3_1}, \ref{tab:blnmc_TEI4_1}, \ref{tab:blnmc_TEI5_1}, \ref{tab:blnmc_TEI6_1} und \ref{tab:blnmc_TEI7_1} festgestellt werden. Ein Vergleich der Tabelle \ref{tab:tmr_start_tei4_2}, \ref{tab:tmr_start_tei5_2}, \ref{tab:tmr_start_tei6_2} und \ref{tab:tmr_start_tei7_2} im Ausgangspunkt mit den Tabellen \ref{tab:blnmc_TEI4_2}, \ref{tab:blnmc_TEI5_2}, \ref{tab:blnmc_TEI6_2} und \ref{tab:blnmc_TEI7_2} belegt dies für den zweiten Durchlauf auf.
\end{enumerate}
