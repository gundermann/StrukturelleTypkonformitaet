\chapter{Implementierung}
Die Implementierung der Explorationskomponente besteht aus drei Hauptbestandteilen, die jeweils als separates Java-Projekt umgesetzt wurden. Im weiteren Verlauf werden diese Java-Projekte als Module bezeichnet.
\begin{figure}[h!]
\centering
\includegraphics[scale=0.8]{cd_arch.png}
\caption{Architektur}
\label{cd_arch}
\end{figure}
\noindent
In Abbildung \ref{cd_arch} ist die Architektur der Explorationskomponente aufgeführt. Dieser ist zu entnehmen, dass die Explorationskomponente aus drei Modulen besteht, die im weiteren Verlauf dieses Kapitels beschrieben werden. Das Modul \emph{DesiredComponentSourcerer} ist dabei von den Modulen \emph{ComponentTester} und \emph{SignatureMatching} abhängig, während das Modul \emph{ComponentTester} lediglich vom Modul \emph{SignatureMatching} abhängig ist.
\\\\
Darüber hinaus, werden folgende externe Bibliotheken verwendet:
\begin{itemize}
\item easymock 3.0 \cite{easymock}
\item cglib 3.3.0 \cite{cglib}
\item objenesis 3.1 \cite{objenesis}
\item junit 4.13.0 \cite{junit}
\end{itemize}
Auf die konkrete Verwendung der externen Bibliotheken wird in den detaillierteren Beschreibungen der einzelnen Module in den folgenden Abschnitten eingegangen.
\section{Modul: SignatureMatching}
\begin{figure}[h!]
\includegraphics[scale=0.5]{cd_SigMa.png}
\caption{Modul: SignatureMatching}
\label{fig_cdSigMa}
\end{figure}
\noindent
In diesem Modul befinden sich zum Einen die Implementierungen der Matcher, die in Abschnitt \ref{sec_matcher} formal beschrieben wurden und zum Anderen die Implementierung der Generatoren für die Proxies. In Abbildung \ref{fig_cdSigMa} sind die wichtigsten Klassen und Interfaces dieses Moduls mit ihren Abhängigkeiten zueinander aufgeführt. Die Matcher befinden sich dabei im Package \emph{matching} und die Generatoren für die Proxies in Form der Implememtierungen des des Interfaces $\texttt{ProxyFactory}$ im Package \emph{glue}.
\\\\
Die in Abschnitt \ref{sec_matcher} beschriebenen Matcher und Generatoren wurden teilweise in einer Klasse zusammengefasst. Tabelle \ref{tab_matcher2impl} zeigt die Zuordnung von Matchern zu den jeweiligen Klassen, die die Implementierung dieser darstellen, und den Klassen, die die Implementierung des Generators für den Proxy, der auf Basis des Matchers Anwendung findet, dargestellt.
\begin{table}[h!]
\centering
\begin{tabular}{|l|l|l|}
\hline
\hline
\textbf{Matcher} & \textbf{Matcher-Implementierung} & \textbf{Generator-Implementierung}\\
\hline
ExactTypeMatcher & $\texttt{ExactTypeMatcher}$ & $\texttt{ClassProxyFactory}$ \\
\hline
GenTypeMatcher & $\texttt{GenSpecTypeMatcher}$ & $\texttt{ClassProxyFactory}$\\
\hline
SpecTypeMatcher & $\texttt{GenSpecTypeMatcher}$ & $\texttt{ClassProxyFactory}$\\
\hline
ContentTypeMatcher & $\texttt{ContainerTypeMatcher}$ & $\texttt{ContentProxyFactory}$\\
\hline
ContainerTypeMatcher & $\texttt{ContainerTypeMatcher}$ & $\texttt{ContainerProxyFactory}$\\
\hline
StructuralTypeMatcher & $\texttt{StructuralTypeMatcher}$ & $\texttt{InterfaceProxyFactory}$\\
\hline
\hline
\end{tabular}
\caption{Zuordnung der Matcher zu den Matcher- und Generator-Implementierungen}
\end{table}\label{tab_matcher2impl}
\noindent
Die Klasse $\texttt{StructuralTypeMatcher}$ nimmt dabei eine Sonderstellung ein. Dies ist daran zu erkennen, dass dieser nicht das Interface $\texttt{TypeMatcher}$ implementiert. Dies wird damit begründet, dass es sich bei diesem Matcher um den Einstiegspunkt der strukturellen Evaluation handelt. Analog zum StructuralTypeMatcher aus Abschnitt \ref{sec_matcher} wird in der Klasse $\texttt{StructuralTypeMatcher}$ auf die anderen Matcher bzw. Matcher-Implementierungen zugegriffen, was in Abbildung \ref{fig_cdSigMa} durch die Aggregation zwischen der Klasse $\texttt{StructuralTypeMatcher}$ und dem Interface $\texttt{TypeMatcher}$ angedeutet wird.
\\\\
Die übrigen Matcher-Klassen implementieren das Interface $\texttt{TypeMatcher}$ und können über die Methode $\texttt{combine}$ aus der Klasse $\texttt{MatcherCombinator}$ miteinander kombiniert werden\footnote{Ein Beispiel für die Kombination von Matchern ist im Anhang \ref{app_matchercombination} zu finden.}. 
%TODO ANHANG: Kombination (siehe unten)
So kann eine Kombination mehrerer $\texttt{TypeMatcher}$, die wiederum von Typ $\texttt{TypeMatcher}$ ist, in der Klasse $\texttt{StructuralTypeMatcher}$ verwendet werden. Die konkrete $\texttt{TypeMatcher}$-Kombination, die im $\texttt{StructuralTypeMatcher}$ instanziiert wird, orientiert sich an den Ausführungen in Abschnitt \ref{sec_matcher}. Es ist aber zu erwähnen, dass die Verwendung weitere Matcher, die in dieser Arbeit nicht definiert wurden, denkbar ist. Eine solche Erweiterung ließe sich leicht in dieses Modul über die Implementierung des Interfaces $\texttt{TypeMatcher}$ und die Verwendung der Klasse $\texttt{MatcherCombiner}$ bewerkstelligen.
\\\\
Alle Matcher-Implementierungen bieten die Möglichkeit, zu ermitteln, ob ein Matching zwischen zwei Typen besteht (siehe Klassendiagramme in Abbildungen \ref{fig_cdMatchingInfo} und \ref{fig_cdSingleMatchingInfo}). Dies erfolgt jeweils über die Methode $\texttt{matchesType}$. Über die Methoden $\texttt{calculateMatchingInfos}$ bzw. $\texttt{calculateMatchingInfo}$ werden die Informationen bzgl. der Methodendelegationen zwischen den beiden gemachten Typen ermittelt. Diese Informationen werden in einem Objekt der Klasse $\texttt{SingleMatchingInfo}$ bzw. $\texttt{MatchingInfo}$ zusammengetragen, welche in Abbildung \ref{fig_cdMatchingInfo} und \ref{fig_cdMatchingInfo} detailliert dargestellt werden.
\begin{figure}[h!]
\includegraphics[scale=0.8]{cd_matchinginfo.png}
\caption{Klassendiagramm: $\texttt{StructuralTypeMatcher}$ und $\texttt{MatchingInfos}$}
\label{fig_cdMatchingInfo}
\end{figure}
\begin{figure}[h!]
\includegraphics[scale=0.7]{cd_singlematchinginfo.png}
\caption{Klassendiagramm: $\texttt{TypeMatcher}$ und $\texttt{SingleMatchingInfo}$}
\label{fig_cdSingleMatchingInfo}
\end{figure}
\noindent
Diese beiden Klassen unterscheiden sich lediglich bzgl. des Attributs in dem die Delegationsmethoden hinterlegt sind. Dabei handelt es sich auf Seiten der $\texttt{SingleMatchingInfo}$ um das Attribut $\texttt{methodMatchingInfos}$ und auf Seiten der $\texttt{MatchingInfo}$ um das Attribut $\texttt{methodMatchingSupplier}$. 
\\\\
Während ein Objekt der Klasse $\texttt{MatchingInfo}$ mehrere Delegationsmethoden zu einer aufgerufenen Methoden enthalten kann, darf ein Objekt der Klasse $\texttt{SingleMatchingInfo}$ lediglich eine Delegationsmethode zu einer aufgerufenen Methode enthalten (vgl. auch Abschnitt \ref{sec_matcher}). Zusätzlich zu erwähnen ist, dass die Informationen über die Delegationsmethoden aus einer $\texttt{MatchingInfo}$ über in einem $\texttt{MethodSupplier}$ überliefert wird.
\\\\
Eine Instanz der Klasse $\texttt{MethodSupplier}$ enthält zum Einen ein $\texttt{MatcherRating}$ welches Informationen bzgl. des in Abschnitt \ref{sec_lmf} beschriebenen Matcher-Ratings beinhaltet. Zum Anderen werden im Attribut $\texttt{methodMatchingInfo}$ in einem Objekt der Klasse $\texttt{MethodMatchingInfo}$ (siehe Abbildung \ref{cd_methodMatchingInfo}) die Informationen bzgl. der Delegation der aufgerufenen Methode an die Delegationsmethode hinterlegt. 
\\\\
Bezüglich der Klasse $\texttt{SingleMatchingInfo}$ ist noch das Attribut $\texttt{proxyFactoryCreator}$ zu beschreiben. Darin werden Informationen bzgl. der strukturellen Verbindung von zwischen den gematchten Typen gehalten. Für den \emph{ExactTypeMatcher}, den \emph{GenTypeMatcher} und den \emph{SpecTypeMatcher} wird dabei ein $\texttt{ProxyFactoryCreator}$ erzeugt, der in der Lage ist, eine $\texttt{ProxyFactory}$ für Typen zu erzeugen, die in einer nominalen Beziehung \footnote{Identität, Generalisierung, Spezialisierung} stehen. Für den \emph{ContentTypeMatcher} und den \emph{ContainedTypeMatcher} hingegen, wird ein $\texttt{ProxyFactoryCreator}$ erzeugt, der in der Lage ist, eine $\texttt{ProxyFactory}$ für Typen zu erzeugen, bei denen der eine Typ ein Attribut von Typ des anderen enthält (vgl. mit Tabelle \ref{tab_matcher2impl}). Die erzeugten Objekte vom Typ $\texttt{ProxyFactory}$ werden bei der Generierung der Proxies unter der Zuhilfenahme der Bibliotheken \emph{cglib} und \emph{objenesis} verwendet\footnote{Diese beiden Frameworks wurden verwendet, da die Erzeugung der Proxies mit ihnen komfortabler ist, als mit den Mitteln die das JKD zur Verfügung steht. Dies gilt insbesondere für die Erzeugung von Proxies für Klassen, die mit dem Schlüsselwort $\texttt{final}$ versehen sind.}.
\\\\
Der $\texttt{ProxyFactoryCreator}$ stellt damit eines der Bindeglieder zwischen der Package \emph{matching} und dem Package \emph{glue} innerhalb des Moduls her. Das zweite Artefakt, welches als Bindeglied fungiert, ist die oben bereits erwähnt Klasse $\texttt{MethodMatchingInfo}$, deren Aufbau dem Klassendiagramm aus Abbildung \ref{cd_methodMatchingInfo} zu entnehmen ist.
\begin{figure}[h!]
\includegraphics[scale=1.0]{cd_methodmatchinginfo.png}
\caption{Klassendiagramm: $\texttt{MethodMatchingInfo}$}
\label{cd_methodMatchingInfo}
\end{figure}
\noindent
\\\\
Ein Objekt der Klasse $\texttt{MethodMatchingInfo}$ enthält in den Attributen $\texttt{source}$ und $\texttt{target}$ je eine Methode. Dabei ist im Attribut $\texttt{source}$ die aufgerufene Methode der Methoden-Delegation und im Attribut $\texttt{target}$ die Delegationsmethode enthalten. Darüber hinaus wird im Attribut $\texttt{returnTypeMatchingInfo}$ ein Objekt der Klasse $\texttt{SingleMatchingInfo}$ gehalten , welches alle notwendigen Informationen für das Erzeugen eines Proxies des Rückgabetyp der aufgerufenen Methode aus dem Rückgabetyp der Delegationsmethode.
\\\\
Analog dazu wird im Attribut $\texttt{argumentTypeMatchingInfos}$ eine Map, bestehend aus weiteren Objekten der Klasse $\texttt{SingleMatchingInfo}$ und jeweils einem Objekt der Klasse $\texttt{ParamPosition}$, gehalten. Diese Map enthält alle notwendigen Information für das Erzeugen eines Proxies für die Parametertypen der Delegationsmethoden aus den Parametertypen der aufgerufenen Methode, sowie der Anpassung der Übergabeposition bei der Delegation der aufgerufenen Methode (siehe auch Abschnitt \ref{sec:proxygram}).
\\\\
Um die Methoden-Delegationen zu koordinieren, wird bei der Erzeugung des Proxies in der jeweiligen $\texttt{ProxyFactory}$ für das Proxy-Objekt ein $\texttt{InvocationHandler}$ instanziiert (vgl. \cite{invocationhandler}). Dieses Interface wird im \emph{glue}-Package durch die Klasse $\texttt{BehaviourDelegateInvocationHandler}$ implementiert, in der letztendlich die Koordination der Methoden-Delegationen auf Basis der jeweiligen $\texttt{MethodMatchingInfo}$ spezifiziert ist.
\\\\
Um einen Proxy basierend auf dem Matching zweier Typen zu erzeugen steht die Klasse $\texttt{TypeConverter}$ zur Verfügung (siehe Abbildung \ref{cd_typeconverter}). Die Zugriffe innerhalb des Packages \emph{glue} als auch die Zugriff von außerhalb benötigen jeweils ein Objekt der Klasse $\texttt{ConvertableBundle}$. Diese Klasse beschreibt eine Kombination mehrerer Objekte vom Typ $\texttt{ConvertableComponent}$, die als Delegationsziele des zu erzeugenden Proxy-Objektes fungieren sollen. Ein Objekt der Klasse $\texttt{ConvertableComponent}$ enthält eine Liste von Objekten vom Typ $\texttt{SingleMatchingInfo}$, die wie bereits erwähnt beschreiben, am welche Methode die Delegation erfolgen soll. Das Objekt im Attribut $\texttt{convertableObject}$ der $\texttt{ModuleMatchingInfo}$ beinhaltet das Objekt, auf dem die Delegationsmethode aufgerufen werden soll.
\begin{figure}[h!]
\includegraphics[scale=0.7]{cd_typeconverter.png}
\caption{Klassendiagramm: $\texttt{TypeConverter}$}
\label{cd_typeconverter}
\end{figure}
%TODO ANHANG: Kombination (siehe unten)
%Die Matcher-Klassen $\texttt{ExactTypeMatcher}$, $\texttt{GenSpecTypeMatcher}$ und $\texttt{WrappedTypeMatcher}$ implementieren auch das von $\texttt{TypeMatcher}$ erbende Interfaces $\texttt{CombinalbeTypeMatcher}$. Klassen, die dieses Interface implementieren können über die Klasse $\texttt{MatcherCombiner}$ zu einem neuen $\texttt{TypeMatcher}$-Objekt kombiniert werden. Ein solcher kombinierte $\texttt{TypeMatcher}$ versucht beim Aufruf der Methode $\texttt{matchesType(S,T)}$ die beiden Typen $S$ und $T$ über einen der kombinierten Matcher zu matchen. Abbildung \ref{sd_matchercombiner} zeigt das Sequenzdiagramm für diesen Aufruf. Dabei liefert die Methode $\texttt{getSortedMatcher}$ eine sortiert Liste der kombinierten Matcher. Die Sortierung wird aufsteigend entsprechend dem Matcherrating der kombinierten Matcher vorgenommen.
%\begin{figure}
%\end{figure}\label{sd_matchercombiner}
%\noindent
%Darüber hinaus gibt es noch das von $\texttt{TypeMatcher}$ erbende Interface $\texttt{PartlyTypeMatcher}$. Dieses Interface wird nur von dem $\texttt{StructuralTypeMatcher}$ implementiert, welcher u.a. als Schnittstelle zwischen dem Modul \emph{SignatureMatching} und \emph{DesiredComponentSourcerer} fungiert. Wie der Name des Interfaces bereits impliziert, bieten die Implementierungen des Interfaces $\texttt{PartlyTypeMatcher}$ die Möglichkeit, zwei Typen nur teilweise zu Matchen. Das bildet die Grundlage für die Ermittlung der Typen, aus denen die Proxies für die semantische Evaluation erzeugt werden können (vgl. Abschnitt \ref{sec_ergStructEval}). So stellen die Objekte, die über die Methode $\texttt{calculatePartlyTypeMatchingInfos}$ erzeugten wurden, auf formaler Ebene die Elemente der Mengen, die in Abschnitt \ref{sec_ergStructEval} über Funktion $\texttt{cover}$ beschrieben wurden, dar.

\section{Modul: ComponentTester}
Dieses Modul ist für die Ausführung der vordefinierten Tests zuständig. Darüber hinaus bietet es die Möglichkeit, die vordefinierten Tests mit den Interfaces, die den dazugehörigen required Typdarstellen, zu Verbinden. Dabei sei davon auszugehen, dass ein required Typ $R$ in Form eines Interfaces existiert. Um Tests für $R$ zu definieren, können eine oder mehrere Testklassen implementiert werden. Die Testklassen werden dabei in dem Interface $R$ über das Attribut $\texttt{testClasses}$ der Annotation $\texttt{RequiredTypeTestReference}$ angegeben (siehe Abbildung \ref{fig_cdCompTester} Package: \emph{API}). Ein Beispiel für die Deklaration eines required Typ in Form eines Java-Interfaces und den dazugehörigen Testklassen ist im Anhang zu finden.
%TODO beispiel im Anhang
\\\\
Damit die Testmethoden in den Testklassen die in Abschnitt \ref{sec_testanforderungen} beschriebenen Eigenschaften aufweisen und durch das \emph{ComponentTester}-Modul ausfindig gemacht werden können, stehen mehrere Artefakte in dem \emph{API}- und dem \emph{SPI}-Package des \emph{ComponentTester}-Moduls bereit (siehe Abbildung \ref{fig_cdCompTester}).
\\\\
So muss jede Testklasse eine Methode bereitstellen, über die ein Objekt vom Typ $R$ in die Instanz der Testklasse injiziert werden kann.\footnote{auch genannt: Setter-Injection (vgl. \cite{setterinfjection})} Diese Methode wird von dem \emph{ComponentTester}-Modul über die Annotation $\texttt{RequiredTypeInstanceSetter}$ gefunden. Von daher muss die Methode mit eben dieser Annotation markiert werden.
\begin{figure}[h!]
\centering
\includegraphics[scale=0.6]{pics/cd_ComponentTester.png}
\caption{Modul: ComponentTester}
\label{fig_cdCompTester}
\end{figure}
\noindent
Die Testmethoden müssen von der Sichtbarkeit her öffentlich ($\texttt{public}$) sein. Weiterhin dürfen die Testmethoden keine Parameter erwarten und müssen mit der Annotation $\texttt{RequiredTypeTest}$ markiert sein. Die Erwartungen innerhalb der Testmethoden müssen über die in JUnit 4 zur Verfügung stehenden Methoden aus der Klasse $\texttt{Assert}$ (vgl. \cite{junit_api}) deklariert werden. Testdaten, die für alle Testmethoden innerhalb einer Testklasse zur Verfügung stehen sollen, können diese innerhalb von Methoden erzeugt werden, die mit den in JUnit 4 bereitgestellten Annotationen $\texttt{Before}$ und $\texttt{After}$ (vgl. \cite{junit_api}) markiert wurden.
\\\\
Um die Reihenfolge der versuchten Aufrufe der Methoden, die von $R$ angeboten werden, zu verwalten, muss die Testklasse das Interface $\texttt{TriedMethodCallsInfo}$ implementieren (siehe Abbildung \ref{fig_cdCompTester} Package: \emph{spi}). Dadurch wird die Implementierung der Methoden $\texttt{addTriedMethodCall}$ und $\texttt{getTriedMethodCalls}$ erzwungen. Die Methode $\texttt{getMethod}$ kann mit der Defaultimplementierung übernommen werden, sofern die in $R$ deklarierten Methoden über den Namen identifiziert werden können.
\\\\
Die Implementierung der Methoden $\texttt{addTriedMethodCall}$ und $\texttt{getTriedMethodCalls}$ hat so zu erfolgen, dass bei einem Aufruf der Methode $\texttt{addTriedMethodCall}$ der übergebene Parameter an eine Liste angefügt wird. Der Aufruf der Methode $\texttt{getTriedMethodCalls}$ liefert eben diese Liste als Rückgabewert. Weiterhin ist sicherzustellen, dass vor dem Aufruf einer Methode $m$ aus $R$ die Methode $\texttt{addTriedMethodCall}$ mit $m$ als Parameter aufgerufen wird. Im Anhang ist ein Beispiel für die korrekte Implementierung einer Testklasse zu finden.
%TODO Beispiel im Anhang
\\\\
Der Test eines Proxies für $R$ wird über eine Instanz der Klasse $\texttt{ComponentTester}$ gestartet (siehe Abbildung \ref{fig_cdCompTester} Package: \emph{Tester}). In Abhängigkeit der in $R$ deklarierten Testklassen werden alle darin befindlichen Testmethoden durchgeführt, bis einer dieser Testfälle fehlschlägt. Der Aufrufer erhält dabei ein Objekt der Klasse $\texttt{TestResult}$ zurück (siehe Abbildung \ref{fig_cdCompTester}). In diesem Objekt sind die für die Auswertung des Testergebnisses relevanten Informationen vorhanden, auf die die Heuristiken \emph{PTTF} (siehe Abschnitt \ref{sec_pttf}) und \emph{BL\_NMC} (siehe Abschnitt \ref{sec_bl_nmc}) angewiesen sind.
\section{Modul: DesiredComponentSourcerer}\label{sec_impl_descos}
In diesem Modul ist die Implementierung der Exploration zu finden. Zum Starten der Exploration für ein \emph{required Typ} $R$ in Form eines Interfaces muss zuerst eine Instanz der Klasse $\texttt{DesiredComponentFinder}$ erzeugt werden (genannt: \emph{Finder}). Dies erfolgt über einen Konstruktor, der ein Objekt der Klasse $\texttt{DesiredComponentFinderConfig}$ (genannt: \emph{Konfig}) erwartet (siehe Abbildung \ref{cd_descos}). 
\begin{figure}[h!]
\centering
\includegraphics[scale=0.5]{cd_descos.png}
\caption{Modul: DesiredComponentSourcerer}
\label{cd_descos}
\end{figure}
\noindent
Die Erzeugung einer solchen \emph{Konfig} erfolgt über einen Builder. Dabei müssen zum Einen die Angabe aller \emph{provided Typen} in Form einer Liste von Interfaces. Zum Anderen wird eine Funktion ($\texttt{java.util.Function}$) gefordert, über die die Implementierungen der im Parameter übergebenen Interfaces ermittelt werden können.
\\\\
Zum Zweck der gezielten Evaluation der Heuristiken in Kapitel \ref{chap_evaluation} kann über die \emph{Konfig} gesteuert werden, welche der in Abschnitt \ref{sec_heuristics} beschriebenen Heuristiken bei der Exploration verwendet werden sollen. Dies erfolgt über die in Abbildung \ref{cd_finderCreation} ersichtlichen Methoden mit den Präfix $\texttt{useHeuristic*}$.
\\\\
Nach der Erzeugung des \emph{Finders} kann die Exploration über die Methode $\texttt{getDesiredComponent}$ mit der Übergabe des \emph{desired Interface} $R$ als Parameter gestartet werden. Im Anschluss wird die syntaktische Evaluation für alle \emph{provided Interfaces} durchgeführt. Auf formaler ebene gleicht dieser Schritt der Ausführung der Funktion $\mathit{cover(R,L)}$, wobei die in $L$ befindlichen \emph{provided Typen} auf die an der \emph{Finder} übergebenen \emph{provided Interfaces} beschränkt sind.
\\\\
Hierzu wird ein Objekt vom $\texttt{StructuralTypeMatcher}$ aus dem \emph{SignatureMatching}-Modul verwendet\footnote{Dieses Objekt wird beim Instanziieren des \emph{Finders} erzeugt.} und versucht die \emph{provided Typ} mit dem \emph{required Typ} zu matchen (siehe Abbildung \ref{sd_descos_structeval}).
\\\\
Nach der syntaktischen Evaluation, wird gemäß Abschnitt \ref{sec_semEval} die semantische Evaluation durchgeführt. Dabei werden zuerst die Proxies aus den Kombinationen der gematchten \emph{provided Typ}\footnote{Diese Kombinationen sind mit den Elementen der Mengen aus $\mathit{cover(R,L)}$ gleichzusetzen.} erzeugt, welche im Anschluss hinsichtlich der vordefinierten Tests zum \emph{required Typ} evaluiert werden. Dabei werden die Heuristiken, die in der \emph{Konfig} hinterlegt wurden, angewendet. Sofern bei der Exploration ein Proxy erfolgreich evaluiert wurde, wird dieser als Ergebnis des Aufrufs der Methode $\texttt{getDesiredComponent}$ zurückgegeben. 
