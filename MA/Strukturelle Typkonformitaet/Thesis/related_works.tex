\section{Verwandte Arbeiten}
Ein solcher Ansatz wurde bereits in \cite{sourcerer} von Bajaracharya et al.  verfolgt. Diese Gruppe entwickelte eine Search Engine namens Sourcerer, welche Suche von Open Source Code im Internet ermöglichte. Darauf aufbauend wurde von derselben Gruppe in \cite{Lemos} ein Tool namens CodeGenie entwickelt, welches einem Softwareentwickler die Code Suche über ein Eclipse-Plugin ermöglicht. In diesem Zusammenhang wurde erstmals der Begriff der Test-Driven Code Search (TDCS) etabliert. Parallel dazu wurde in Verbindung mit der Dissertation Oliver Hummel \cite{hummel08} ebenfalls eine Weiterentwicklung von Sourcerer veröffentlicht, welche unter dem Namen Merobase bekannt ist, welches ebenfalls das Konzept der TDCS verfolgt. TDCS beruht grundlegend darauf, dass der Entwickler Testfälle spezifiziert, die im Anschluss verwendet werden, um relevanten Source Code aus einem Repository hinsichtlich dieser Testfälle zu evaluieren. Damit kann das jeweilige Tool dem Entwickler Vorschläge für die Wiederverwendung bestehenden Codes unterbreiten.
\\\\
Bezogen auf die am Ende des vorherigen Abschnitts formulierte Überlegung ermöglichen die genannten Search Engines, das Internet nach bestehendem Source Code zu durchsuchen und damit bereits bestehende Implementierungen für eine nachfragende Komponente zu ermitteln. 