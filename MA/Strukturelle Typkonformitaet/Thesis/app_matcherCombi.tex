\chapter{Kombination von Matchern}\label{app_matchercombination}
Wie aus der formalen Beschreibung zum \emph{StructuralTypeMatcher} im Abschnitt \ref{subsec_structmatcher} hervorgeht, ist dieser von den übrigen Matchern abhängig. Die Implementierung der dazugehörigen Klasse $\texttt{StructuralTypeMatcher}$ verlangt zur Erzeugung eines Objektes dieser Klasse eine $\texttt{TypeMatcher}$. Dieser $\texttt{TypeMatcher}$ muss laut der formalen Beschreibung die Implementierung der Matchingrelation $\Rightarrow_{internStruct}$ darstellen.
\\\\
Zu diesem Zweck müssen die übrigen Matcher bzw. die dafür implementierten Klassen miteinander kombiniert werden, wie es in der Definition zur Matchingrelation $\Rightarrow_{internStruct}$ der Fall ist. Für solche Kombinationen steht die Klasse $\texttt{MatcherCombiner}$ im Modul \emph{SignatureMatching} bereit (siehe Listing \ref{lst_matchercombiner}).
\\\\
Diese Klasse erlaubt die Kombination von Objekten vom Typ $\texttt{TypeMatcher}$. 
Die Matcher-Klassen $\texttt{ExactTypeMatcher}$, $\texttt{GenSpecTypeMatcher}$ und $\texttt{WrappedTypeMatcher}$ implementieren alle dieses \Gls{Interface}. 
\\\\
Über die Methode $\texttt{combine}$ in der $\texttt{MatcherCombiner}$ wird bei der Kombination ein Supplier-Objekt erzeuge, welches über die $\texttt{get}$-Methode ein Objekt vom Typ $\texttt{TypeMatcher}$ liefern kann. Dieses $\texttt{TypeMatcher}$-Objekt versucht beim Aufruf der Methode $\texttt{matchesType(S,T)}$ die beiden Typen $S$ und $T$ über einen der kombinierten Matcher zu matchen (siehe  Listing \ref{lst_matchercombiner}). Dabei liefert die Methode $\texttt{getSortedMatcher}$ eine sortiert Liste der kombinierten Matcher. Die Sortierung wird aufsteigend entsprechend dem Basisrating (siehe auch Abschnitt \ref{sec_lmf}) der kombinierten Matcher vorgenommen .
\begin{lstlisting}[style = java, caption = Klasse: MatcherCombiner, captionpos = b, label = lst_matchercombiner]
package de.fernuni.hagen.ma.gundermann.signaturematching.matching;

import java.util.ArrayList;
import java.util.Arrays;
import java.util.Collection;
import java.util.Collections;
import java.util.List;
import java.util.function.Supplier;

import de.fernuni.hagen.ma.gundermann.signaturematching.SingleMatchingInfo;
import de.fernuni.hagen.ma.gundermann.signaturematching.matching.types.TypeMatcher;

public final class MatcherCombiner {
 private MatcherCombiner() {
 }

 public static Supplier<TypeMatcher> combine(TypeMatcher... matcher) {
  return () -> new TypeMatcher() {

   @Override
   public boolean matchesType(Class<?> checkType, Class<?> queryType) {
	for (TypeMatcher m : getSortedMatcher()) {
	 if (m.matchesType(checkType, queryType)) {
	  return true;
	 }
	}
	return false;
   }

   @Override
   public Collection<SingleMatchingInfo> calculateTypeMatchingInfos(Class<?> checkType, Class<?> queryType) {
	for (TypeMatcher m : getSortedMatcher()) {
	 if (m.matchesType(checkType, queryType)) {
	  return m.calculateTypeMatchingInfos(checkType, queryType);
	 }
	}
	return new ArrayList<>();
   }

   @Override
   public MatcherRate matchesWithRating(Class<?> checkType, Class<?> queryType) {
	for (TypeMatcher m : getSortedMatcher()) {
	 MatcherRate rating = m.matchesWithRating(checkType, queryType);
	 if (rating != null) {
	  return rating;
	 }
	}
	return null;
   }

   @Override
   public double getTypeMatcherRate() {
	// irrelevant, weil matchesWithRating ueberschrieben wurde.
	return -0;
   }

   private Collection<TypeMatcher> getSortedMatcher() {
	List<TypeMatcher> matcherList = Arrays.asList(matcher);
	Collections.sort(matcherList,(l1, l2) -> Double.compare(l1.getTypeMatcherRate(), l2.getTypeMatcherRate()));
	return matcherList;
   }

  };
 }
}
\end{lstlisting}
\noindent
Bei der Exploration wird letztendlich immer ein Objekt des der Klasse $\texttt{StructuralTypeMatcher}$ zur Ermittlung des Matchings verwendet. Listing \ref{lst_matchermanager} zeigt die Instanziierung dieses Objektes unter Verwendung der Klasse $\texttt{MatcherCombiner}$.
\begin{lstlisting}[style = java, caption = Default-Instanziierung des StructuralTypeMatchers im DesiredComponentFinder, captionpos = b, label = lst_matchermanager]
TypeMatcher exactTM = new ExactTypeMatcher();
TypeMatcher genSpecTM = new GenSpecTypeMatcher();
TypeMatcher combinedGenSpecExactTM = MatcherCombiner.combine(genSpecTM, exactTM).get();
TypeMatcher wrappedTM = new WrappedTypeMatcher(() -> combinedGenSpecExactTM);
TypeMatcher combinedWrappedGenSpecExact = MatcherCombiner.combine(genSpecTM, exactTM, wrappedTM).get();
StructuralTypeMatcher structWrappedGenSpecExactTM = new StructuralTypeMatcher(
				() -> combinedWrappedGenSpecExact);
\end{lstlisting}