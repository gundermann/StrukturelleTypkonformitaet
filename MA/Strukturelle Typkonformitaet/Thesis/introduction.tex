\chapter{Einleitung}
\section{Motivation}
In größeren Software-Systemen ist es üblich, dass mehrere \gls{komponente}n miteinander über Schnittstellen kommunizieren. In der Regel werden diese Schnittstellen so konzipiert, dass sie Informationen oder Services anbieten, die von anderen \gls{komponente}n abgefragt und benutzt werden können. Dabei wird zwischen der \gls{komponente}, welche die Schnittstelle implementiert - als angebotene \Gls{komponente} - und der \gls{komponente}, welche die Schnittstelle nutzen soll - als nachfragende \gls{komponente} - unterschieden (siehe \abbref{motiv}). 
\myScalableFigure[0.6\linewidth]{motiv}{Abhängigkeiten von nachfragenden und angebotenen Komponenten}{motiv}
\noindent
Wird von einer nachfragenden \gls{komponente} eine Information benötigt, die in dieser Form noch nicht angeboten wird, so wird häufig ein neues \Gls{Interface} für diese benötigte Information erstellt, welches dann passend dazu implementiert wird. Dabei muss neben der Anpassung der nachfragenden \gls{komponente} auch eine Anpassung oder Erzeugung der anbietenden \gls{komponente} erfolgen und zusätzlich das neue \Gls{Interface} deklariert werden. Zudem bedingt eine nachträgliche Änderung der neuen Schnittstelle ebenfalls eine Anpassung der drei genannten \gls{artefakt}e.\\\\
In einem großen Software-System mit einer Vielzahl von bestehenden Schnittstellen ist eine gewisse Wahrscheinlichkeit gegeben, dass die Informationen oder Services, die von einer neuen nachfragenden \gls{komponente} benötigt werden, in einer ähnlichen Form bereits existieren. Das Problem ist jedoch, dass die manuelle Evaluation der Schnittstellen mitunter sehr aufwendig bis, aufgrund von unzureichender Dokumentation und Kenntnis über die bestehenden Schnittstellen, unmöglich ist.
\\\\
Weiterhin ist es denkbar, dass ein Software-System auf unterschiedlichen Maschinen verteilt wurde und dadurch Teile des Systems ausfallen können. Das hat zur Folge, dass die Implementierung bestimmter Schnittstellen nicht erreichbar ist. Dadurch, dass eine Schnittstelle durch eine nachfragende \gls{komponente} explizit referenziert wird, kann eine solche \gls{komponente} nicht korrekt arbeiten, wenn die Implementierung der Schnittstelle nicht erreichbar ist, obwohl die benötigten Informationen und Services vielleicht durch andere Schnittstellen, deren Implementierung durchaus zur Verfügung stehen, bereitgestellt werden könnten.
\\\\
Dies führt zu der Überlegung, ob eine nachfragende \gls{komponente} anstelle der Referenzierung einer Schnittstelle eine Spezifizierung der Schnittstelle vornimmt, anhand derer eine angebotene \gls{komponente}, die dieses Spezifikation erfüllt, gefunden werden kann.
\\\\
Ein solches Vorgehen wird bei der testgetriebene Codesuche (testdriven codesearch - \emph{TDCS}) verfolgt, welche als Basis für diese Arbeit herangezogen wird. Dabei stellt der Entwickler eine Menge von Suchparametern zusammen, die er an eine so genannte Source \Gls{Engine} übergibt. Die Suchparameter sind dabei jedoch stark an dem orientiert, was der Entwickler benötigt und weniger daran, was tatsächlich im Repository vorliegt. Diese Source \Gls{Engine} durchsucht anschließend ein Repository nach \gls{komponente}n, die zu den gestellten Suchparametern passen. 
\\\\
Die Suchergebnisse werden aufgelistet und der Entwickler entscheidet letztendlich explizit, welche \gls{komponente} verwenden möchte. Die Verwendung der \gls{komponente} läuft dann jedoch auf eine Referenzierung dieser in der nachfragenden \gls{komponente} hinaus. Somit arbeiten die Source \Gls{Engine}s also nicht zur Laufzeit des Systems, in dem die \gls{komponente}n verwendet werden sollen.
\\\\
In dieser Arbeit soll eine solche Exploration jedoch zur Laufzeit erfolgen, sodass eine explizite Referenzierung der angebotenen \gls{komponente} nicht erfolgen muss. Dabei ist die Zeit als Ressource während der Suche nach einer passenden \gls{komponente} als knapp anzusehen. Aus diesem Grund werden in dieser Arbeit \Gls{Heuristik}en vorgeschlagen, die ein gezieltes Auffinden einer passenden \gls{komponente} ermöglichen und damit die Suche beschleunigen.

\section{Aufbau dieser Arbeit}
Zuerst wird in Kapitel \ref{chap_problem} auf den aktuellen Forschungsstand zur \emph{TDCS} eingegangen. Im Anschluss daran wird beschrieben, wie sich die \emph{TDCS} auf einen Ansatz, in dem zur Laufzeit nach \gls{komponente}n gesucht wird, eingegangen, um so eine Abgrenzung zu den früheren Arbeiten zu schaffen.
\\\\
In Kapitel \ref{chap_foundation} werden die einzelnen Schnritte, die während der Exploration durchgeführt werden, sowie die zu evaluierenden \Gls{Heuristik}en formal beschrieben.
\\\\
Kapitel \ref{chap_impl} gibt einen kurzen Überblick über die Implementierung der in Kapitel \ref{chap_foundation} genannten Aspekte.
\\\\
In Kapitel \ref{chap_evaluation} werden die Untersuchungsergebnisse, die unter Anwendung der \Gls{Heuristik}en im Einzelnen und in Kombination zusammengetragen wurden, vorgestellt. 
\\\\
Die Auswertung dieser Ergebnisse erfolgt in Kapitel \ref{chap_disc} zusammen mit einer kritischen Betrachtung des in der Arbeit vorgestellten Ansatzes, sowie einer kurzen Betrachtung möglicher Erweiterungen für diesen Ansatz.
\\\\
Komplettiert wird die Arbeit durch eine kurzen Zusammenfassung der Ergebnisse und einem Ausblick in Kapitel \ref{chap_finish}.
%\subsection{Gegenstand dieser Arbeit}
In dieser Arbeit soll jedoch nicht das gesamte Internet als Quelle oder Repository f�r die Codesuche dienen. Vielmehr wird der Suchbereich weiter eingeschr�nkt. \\\\
Es wird von einem System ausgegangen, in dem ein EJB-Container zur Verf�gung steht. Die Suche soll sich auf die Menge der angemeldeten Bean-Implementierungen beschr�nken. Die angemeldeten Bean-Implementierungen stellen damit die Menge der angebotenen Komponenten dar. Dabei wird eine angebotenen Komponente als Kombination eines Interfaces, welches die Schnittstelle f�r die Aufrufer definiert, und einer Implementierung dieses Interfaces beschrieben. Das Interfaces einer angebotenen Komponente wird im Folgenden auch als angebotenes Interfaces bezeichnet. Die Beans werden bspw. als Provider f�r Informationen oder im weitesten Sinne  auch als Services verwenden, die von unterschiedlichen Komponenten des Systems verwendet werden. Bei der Entwicklung bzw. Weiterentwicklung einer Komponente kann es zu folgendem Szenario kommen, welches durch die unten aufgef�hrten Annahmen charakterisiert wird:
\begin{itemize}
\item Es werden Informationen und Services ben�tigt, bei denen der Entwickler davon ausgehen kann, dass es innerhalb des Systems angebotene Komponenten gibt, die diese Informationen liefern k�nnen bzw. die Services erf�llen.
\item Der Entwickler wei� nicht, �ber welche konkreten angebotenen Komponenten er die Informationen abfragen bzw. die Services in Anspruch nehmen kann.
\end{itemize}
\subsubsection{Funktionale Anforderungen}
In dieser Arbeit soll ein Konzept entwickelt werden, welches dem Entwickler erm�glicht ,die Erwartungen an die angebotenen Komponenten zu spezifizieren. Darauf aufbauend soll ein Algorithmus vorgeschlagen werden, welcher die angebotenen Komponenten zur Laufzeit hinsichtlich der spezifizierten Erwartungen des Entwicklers evaluiert und eine Auswahl derer trifft, die diese Erwartungen erf�llen. Da die Evaluation zur Laufzeit durchgef�hrt wird, kann der Entwickler, anders als bei den oben genannten Arbeiten, nicht aus einer Liste von Vorschl�gen ausw�hlen, welche der evaluierten Komponenten letztendlich verwendet werden soll. Diese Entscheidung ist durch den Algorithmus zu treffen.
\subsubsection{Nichtfunktionale Anforderungen}
Aufgrund bestimmter Konfigurationen des Gesamtsystems gibt es folgende weitere nichtfunktionale Anforderungen:
\begin{itemize}
\item Die Suche muss innerhalb des Transaktionstimeouts zu einem Ergebnis f�hren. (Im verwendeten System ist dieses auf 5 Minuten festgesetzt.)
\item Die Suche soll hinsichtlich der Besonderheiten des System, in dem sie verwendet wird, angepasst werden k�nnen. (Bspw. bei der Verwendung bestimmter Typen, deren Fachlogik bei der Suche nicht untergraben werden darf.)
\item Bei einem Fehlschlag der Suche, sollen dem Entwickler Informationen zur Verf�gung gestellt werden, die eine zielgerichtete Anpassung seiner spezifizierten Erwartungen erlauben.
\end{itemize}