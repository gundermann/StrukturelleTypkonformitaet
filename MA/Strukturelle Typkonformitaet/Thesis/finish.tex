\chapter{Schlussbemerkung}\label{chap_finish}
\section{Zusammenfassung}
Zusammenfassend ist zu sagen, dass die vorgestellten \Gls{Heuristik}en ihren Zweck erfüllen und gemessen an der Anzahl der zu generierenden und zu prüfenden \emph{Proxies} eine schnellere Exploration nach einem passenden \emph{Proxy} ermöglichen. Dabei konnten auch Synergieeffekte zwischen den einzelnen \Gls{Heuristik}en festgestellt werden.
\\\\
Weiterhin wurde gezeigt, dass die testgetriebene Exploration von \emph{EJBs} zur Laufzeit grundlegend funktioniert. Dennoch gibt es Szenarien, in denen von diesem Verfahren eher abzuraten ist. Das betrifft insbesondere solche \emph{EJBs}, durch deren Methodenaufrufe eine Änderung an ihrem inneren Zustand bezweckt wird. Es wurden jedoch Möglichkeiten aufgezeigt, wie mit solchen Fällen umgegangen werden kann.
\\\\
Ob der Ansatz der testgetriebenen Exploration zur Laufzeit im Allgemeinen einen Nutzen verspricht wurde nicht geklärt. Wenn dies überhaupt der Fall ist, dann hängt der Nutzen vermutlich mit dem Wissen der Entwickler*innen zusammen, welches sie über das vorliegende System aufweisen können.
\\\\
Unabhängig davon wurde in dieser Arbeit eine allgemeine formale Beschreibung für Matcher von \Gls{wrappertype}en gegeben (\emph{ContentTypeMatcher} und \emph{ContainerTypeMatcher}). 
\\\\
Zudem können die entwickelten \Gls{Modul}e, welche in Kapitel \ref{chap_impl} beschrieben wurden, in unterschiedlichen Systemen verwendet werden. Hinsichtlich des Repositories haben die Entwickler*innen sehr viel Freiraum und sind nicht auf einen \emph{EJB-Container} beschränkt. Weiterhin können neue Matcher durch die Implementierung der dafür vorgesehenen \Gls{Interface}s in die \Gls{Modul}e integriert werden, was den Nutzen des Ansatzes in einem System individuell steigern kann.
\section{Ausblick}
Die Heuristiken wurden zwar im Rahmen der Exploration zur Laufzeit entworfen. In einem nächsten Schritt könnte versucht werden, diese Heuristiken in bestehende Search \Gls{Engine}s wie Merobase oder CodeGenie zu integriert, um so den Nutzen der Heuristiken für diese \Gls{Engine}s zu untersuchen.
\\\\
Weiterhin wäre es interessant zu untersuchen, ob und wie dieser Ansatz der Exploration von Komponenten zur Laufzeit in anderen Systemtypen wie bspw. Self-Contained-Systems funktioniert. Mitunter ergeben sich bei diesen Untersuchungen weitere Vorteile oder Probleme dieses Ansatzes.
\\\\
Darüber hinaus bieten die in Abschnitt \ref{sec_discApproach} aufgestellten Thesen bzgl. der höheren Verfügbarkeit (Abschnitt \ref{sec_sideeffects}) und dem Nutzen des Ansatzes für den Entwickler im Verhältnis zu dessen Wissen über das System das Potential für weitere Untersuchungen.
