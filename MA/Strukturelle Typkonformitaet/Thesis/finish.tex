\chapter{Schlussbemerkung}\label{chap_finish}
\section{Zusammenfassung}
Zusammenfassend ist zu sagen, dass die vorgestellten Heuristiken ihren Zweck erfüllen und gemessen an der Anzahl der zu generierenden und zu evaluierenden Proxies eine schnellere Exploration nach einem passenden Proxy ermöglichen. Dabei konnten auch Synergieeffekte zwischen den einzelnen Heuristiken festgestellt werden.
\\\\
Weiterhin wurde gezeigt, dass die testgetriebe Exploration von EJBs zur Laufzeit grundlegend funktioniert. Dennoch gibt es Szenarien, in denen von diesem Verfahren eher abzuraten ist. Das betrifft insbesondere solche EJBs, durch deren Methodenaufrufe eine Änderung an ihrem inneren Zustand bezweckt wird. Es wurden jedoch Möglichkeiten aufgezeigt, wie mit solchen Konstellationen umgegangen werden kann.
\\\\
Ob der Ansatz der testgetriebenen Exploration zur Laufzeit im Allgemeinen einen Nutzen verspricht wurde nicht geklärt. Wenn dies überhaupt der Fall ist, dann hängt der Nutzen sicherlich mit dem Wissen des jeweiligen Entwicklers zusammen, das er über das vorliegende System hat.
\\\\
Unabhängig davon wurde in dieser Arbeit eine allgemeine formale Beschreibung für Typen gegeben, die in anderen Typen enthalten sind (ContentTypeMatcher bzw. ContainerTypeMatcher). 
\\\\
Zudem können die entwickelten Module, welche in Kapitel \ref{chap_impl} vorgestellt wurden, in unterschiedlichen Systemen verwendet werden. Hinsichtlich des Repositories hat der Entwickler sehr viel Freiraum und ist nicht auf einen EJB-Container beschränkt. Weiterhin können neue Matcher durch die Implementierung der dafür vorgesehenen Interfaces in die Module integriert werden, was den Nutzen für ein System individuell steigern kann.
\section{Ausblick}
Die Heuristiken wurden zwar im Rahmen der Exploration zur Laufzeit entworfen. In einem nächsten Schritt könnte versucht werden, diese Heuristiken in bestehende Search \Gls{Engine}s wie Merobase oder CodeGenie zu integriert, um so den Nutzen der Heuristiken für diese \Gls{Engine}s zu untersuchen.
\\\\
Weiterhin wäre es interessant zu untersuchen, ob und wie dieser Ansatz der Exploration von Komponenten zur Laufzeit in anderen Systemtypen wie bspw. Self-Contained-Systems funktioniert. Mitunter ergeben sich bei diesen Untersuchungen weitere Vorteile oder Probleme dieses Ansatzes.
\\\\
Darüber hinaus bieten die in Abschnitt \ref{sec_discApproach} aufgestellten Thesen bzgl. der höheren Verfügbarkeit (Abschnitt \ref{sec_sideeffects}) und dem Nutzen des Ansatzes für den Entwickler im Verhältnis zu dessen Wissen über das System das Potential für weitere Untersuchungen.
