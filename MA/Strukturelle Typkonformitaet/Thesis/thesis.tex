\documentclass[a4paper,12pt]{book}
\usepackage[ansinew]{inputenc}
\usepackage[german]{babel}
\usepackage[T1]{fontenc} %Umlaute
\usepackage{lmodern}
\usepackage[top=2cm, left=2cm, bottom=2cm, right=2cm]{geometry}
\usepackage{fancybox, graphicx}
\usepackage{float}
\usepackage{listings} 
\usepackage{multirow}
\usepackage{multicol}
\usepackage{color}		 % f�r Farben im allgemeinen
\usepackage{colortbl}
\usepackage{cite}
\usepackage{bibgerm}
\usepackage{palatino}
\usepackage{chngpage}
\usepackage{chngcntr}
\usepackage{amsmath}
\usepackage{setspace}
\counterwithout{figure}{chapter}
\counterwithout{table}{chapter}
\renewcommand{\floatpagefraction}{0.85}


\usepackage{fancyhdr}
\pagestyle{fancy}

\definecolor{rot}{rgb}{1,0.3,0}
\definecolor{gelb}{rgb}{1,1,0}
\definecolor{gruen}{rgb}{0,1,0.4}


\fancyhf{}
\fancyhead[RE]{\slshape \nouppercase{\leftmark}}    % Even page header: "page   chapter"
\fancyhead[LO]{\slshape \nouppercase{\rightmark}}   % Odd  page header: "section   page"
\fancyhead[RO,LE]{\bfseries \thepage} 
\renewcommand{\headrulewidth}{1pt}    % Underline headers
\renewcommand{\footrulewidth}{0pt}    

\fancypagestyle{plain}{               % No chapter+section on chapter start pages
\fancyhf{}
\fancyhead[RO,LE]{\bfseries \thepage}
\renewcommand{\headrulewidth}{1pt}
\renewcommand{\footrulewidth}{0pt}
}

% Left headings: "1  INTRODUCTION"
%\renewcommand{\chaptermark}[1]{%
%\markboth{\thechapter\ \ \ \ #1}{}}

% Right headings: "1.1  Basics"
\renewcommand{\sectionmark}[1]{%
\markright{\thesection\ \ \ \ #1}{}}

%\lstset{language=java} 
\lstset{basicstyle=\scriptsize}
\lstset{numbers=left, numberstyle=\tiny, numbersep=2pt, breaklines=true} 
\lstset{ numberbychapter=false}
\graphicspath{{pics/}}
\newcommand{\listoflolentryname}{\lstlistingname} 
\usepackage{float}
%\newcommand{\fullcite}{\citep} %for "Author [1980]"
\usepackage[pdftex,plainpages=false,pdfpagelabels]{hyperref}
\usepackage[nonumberlist]{glossaries}

\makeglossaries
\loadglsentries{glossary}

% --- Farbdefinitionen ----------------------------------------
\definecolor{rot}{rgb}{1,0.3,0}
\definecolor{gelb}{rgb}{1,1,0}
\definecolor{gruen}{rgb}{0,1,0.4}
\definecolor{darkblue}{rgb}{0.2,0.3,1}
\definecolor{lightblue}{rgb}{0.6,0.7,1}
\definecolor{white}{rgb}{1,1,1}
\definecolor{pblue}{rgb}{0.13,0.13,1}
\definecolor{pgreen}{rgb}{0,0.5,0}
\definecolor{pred}{rgb}{0.9,0,0}
\definecolor{pgrey}{rgb}{0.46,0.45,0.48}

\onehalfspacing
\linespread{1.5}

\bibliographystyle{geralpha}

\renewcommand{\floatpagefraction}{0.85}

\usepackage[figuresright]{rotating}
\usepackage{geometry}
\geometry{a4paper,left=20mm,right=20mm} 

\newlength{\fullwidth} % Width of text plus margin notes
\setlength{\fullwidth}{\textwidth}


\lstset{language=Java,
  showspaces=false,
  showtabs=false,
  breaklines=true,
  showstringspaces=false,
  breakatwhitespace=true,
  commentstyle=\color{pgreen},
  keywordstyle=\color{pblue},
  stringstyle=\color{pred},
  basicstyle=\fontsize{9}{10}\selectfont\ttfamily,
  moredelim=[il][\textcolor{pgrey}]{$ $},
  moredelim=[is][\textcolor{pgrey}]{\%\%}{\%\%}
}


\usepackage{varwidth}
\newcommand\tabrotate[1]{\begin{turn}{90}\rlap{#1}\end{turn}}
\newcommand\tabvarwidth[2][3cm]{\begin{varwidth}[b]{#1}\centering #2\end{varwidth}}
\newcommand{\myparagraph}[1]{\paragraph{#1}\mbox{}\\}

%----------------------------------------------------------------------------------
% \vft		{ NUMBER }
%			{ UPPER_LEFT }
%			{ UPPER_RIGHT }
%			{ LOWER_LEFT }
%			{ LOWER_RIGHT }
%			{ CAPTION }
%			{ LABEL }
%
% Gleichung, die ein Matching darstellt
\newcommand{\vft}[7]{
\begin{table}[H]
\centering
\small
\doublespacing
\begin{tabular}[c]{|c|c|c|}
\hline
#1 & \cellcolor{gruen}\textbf{positiv} & \cellcolor{rot}\textbf{negativ} \\
\hline
\cellcolor{rot}&\cellcolor{rot}&\cellcolor{rot}\\
\tabrotate{\cellcolor{rot}\textbf{falsch}} & \multirow{-2}{*}{\cellcolor{rot}#2}&\multirow{-2}{*}{\cellcolor{rot}#3}\\
\hline
\cellcolor{gruen}&\cellcolor{gruen}&\cellcolor{rot}\\
\tabrotate{\cellcolor{gruen}richtig} &\multirow{-2}{*}{\cellcolor{gruen}#4}&\multirow{-2}{*}{\cellcolor{rot}#5} \\
\hline
\end{tabular}
\singlespacing
\caption{#6}
 \label{Tab_#7}
\end{table}
}
%----------------------------------------------------------------------------------
% \matchTyp		{ LEFT }
%				{ MATCHERVAR }
%				{ RIGHT }
%
% Gleichung, die ein Matching darstellt
\newcommand{\matchTyp}[3]
{
#1 \equiv_{#2} #3
}

%----------------------------------------------------------------------------------
% \inhTyp		{ CHILD }
%				{ PARENT }
%
% Gleichung, die eine Vererbung darstellt
\newcommand{\inhTyp}[2]
{
#1 \leq #2
}

%----------------------------------------------------------------------------------
% \selTyp		{ WRAPPER }
%				{ ATTR }
%
% Gleichung, die eine Selektion darstellt
\newcommand{\selTyp}[2]
{
#1 \# #2
}

%----------------------------------------------------------------------------------
% \delegate		{ SOURCE }
%				{ TARGET }
%
% Gleichung, die eine Delegation darstellt
\newcommand{\delegate}[2]
{
#1 \Rightarrow #2
}

%----------------------------------------------------------------------------------
% \applyMatcher		{ MATCHER }
%					{ PARAM }
%
% Gleichung, die Applikation eines Matchers auf einen Parameter 
\newcommand{\applyMatcher}[2]
{
(#1)#2
}

%----------------------------------------------------------------------------------
% matcherEquivDef		{ MATCHER }
%						
%
% Umgebung f�r die Definition der �bereinstimmung eines Matchers
\newenvironment{matcherEquivDef}[1]{
\begin{addmargin}[1in]{0cm}
\underline{�bereinstimmung (#1)}
\end{addmargin}
\begin{eqnarray*}
}
{
\end{eqnarray*}
}

%----------------------------------------------------------------------------------
% matcherConvDef		{ MATCHER }
%						{ ANNAHME }
%
% Umgebung f�r die Definition der �bereinstimmung eines Matchers
\newenvironment{matcherConvDef}[2]{
\begin{addmargin}[1in]{0cm}
\underline{Konvertierung (#1)}\newline
#2
\end{addmargin}
\begin{eqnarray*}
}
{
\end{eqnarray*}
}

%----------------------------------------------------------------------------------
% \myFigure	[ LABEL_PREFIX (optional) ]
%				{ FILENAME (without extension) }
%				{ CAPTION TEXT }
%				{ SHORT VERSION OF CAPTION TEXT }
%
%Bild wird in Originalgroesse gesetzt
%picture using full width of the page
\newcommand{\myFigure}[3]
{
\begin{figure}[H]
	\center{
	\begin{minipage}{\fullwidth}
	\center{
		\includegraphics{#1}
		\caption{#2}
		\label{Abb_#3}
		}
	\end{minipage}
	}
\end{figure}
}

%----------------------------------------------------------------------------------
% \myBigFigure	[ LABEL_PREFIX (optional) ]
%				{ FILENAME (without extension) }
%				{ CAPTION TEXT }
%				{ SHORT VERSION OF CAPTION TEXT }
%
%Bild wird in kompletter Breite gesetzt
%picture using full width of the page
\newcommand{\myBigFigure}[4][Abb]
{
\begin{figure}[H]
	\center{
	\begin{minipage}{\fullwidth}
		\includegraphics[width= \fullwidth]{#2}
		\caption{#3}
		\label{#1_#4}
	\end{minipage}
	}
\end{figure}
}


%----------------------------------------------------------------------------------
% \myBigFigure	[ LABEL_PREFIX (optional) ]
%				{ FILENAME (without extension) }
%				{ CAPTION TEXT }
%				{ SHORT VERSION OF CAPTION TEXT }
%
%Bild wird in kompletter Breite gesetzt
%picture using full width of the page
\newcommand{\myBigFigureCited}[5][Abb]
{
\begin{figure}[H]
	\center{
	\begin{minipage}{\fullwidth}
		\includegraphics[width= \fullwidth]{#2}
		\caption[#3]{#3#4}
		\label{#1_#5}
	\end{minipage}
	}
\end{figure}
}


%----------------------------------------------------------------------------------
% \dcite	{ Text }
%				{ source }
%				{ page }
%
%Direktes Zitat
\newcommand{\dcite}[3]
{
\emph{\glqq#1\grqq}\cite[S.#3]{#2}}

%----------------------------------------------------------------------------------
% \dcite	{ Text }
%				{ source }
%				{ page }
%
%Direktes Zitat
\newcommand{\simpledcite}[2]
{
\emph{\glqq#1\grqq}\cite{#2}}


%----------------------------------------------------------------------------------
% \vcite	
%				{ source }
%				{ page }
%
%Direktes Zitat
\newcommand{\vcite}[2]
{
(vgl. \cite[S.#2]{#1})}


%----------------------------------------------------------------------------------
% \vcite	
%				{ source }
%				{ page }
%
%Direktes Zitat
\newcommand{\simplevcite}[1]
{
(vgl. \cite{#1})}

%----------------------------------------------------------------------------------
% \myHUGEFigure	[ LABEL_PREFIX (optional) ]
%				{ FILENAME (without extension) }
%				{ CAPTION TEXT }
%				{ SHORT VERSION OF CAPTION TEXT }
%
%Bild wird rotiert und quer in kompletter Breite gesetzt
%landscape picture using the full width of the rotated page
\newcommand{\myHugeFigure}[4][Abb]
{
\begin{sidewaysfigure}[H]
	
		\includegraphics[width= \textheight]{#2}
		\caption{#3}
		\label{#1_#4}
	
\end{sidewaysfigure}
}


%----------------------------------------------------------------------------------
% \myHUGEFigure	[ LABEL_PREFIX (optional) ]
%				{ FILENAME (without extension) }
%				{ CAPTION TEXT }
%				{ SHORT VERSION OF CAPTION TEXT }
%
%Bild wird rotiert und quer in kompletter Breite gesetzt
%landscape picture using the full width of the rotated page
\newcommand{\myHugeFigureCited}[5][Abb]
{
\begin{sidewaysfigure}[t!bp]
	
		\includegraphics[width= \textheight]{#2}
		\caption[#3]{#3#4}
		\label{#1_#5}
	
\end{sidewaysfigure}
}


%-----------------------------------------------------------------------
% \abbref		{ PIC REFERENCE }
%
%Verweis auf eine Abbildung
\newcommand{\abbref}[1]{Abb. \ref{Abb_#1}}

%-----------------------------------------------------------------------
% \listref		{ PIC REFERENCE }
%
%Verweis auf ein Listing
\newcommand{\lstref}[1]{Listing \ref{#1}}


\newcommand{\gloss}[1]{\emph{\gls{#1}}}

\newcommand{\glossLink}[2]{\emph{\glslink{#1}{#2}}}

\newcommand{\Ks}{\emph{Basiskomponenten} }

\newcommand{\K}{\emph{Basiskomponente} }
\newcommand{\knks}{\emph{komplexen Komponenten} }

\newcommand{\kks}{\emph{komplexe Komponenten} }

\newcommand{\guis}{\emph{\glspl{GUI}} }

\newcommand{\gui}{\emph{\acrshort{GUI}} }

\newcommand{\g}{\emph{GUI-DSL} }

\renewcommand{\deg}{\emph{data experts GmbH} }

\newcommand{\gk}{\emph{GUI-Komponente} }
\newcommand{\gks}{\emph{GUI-Kompo\-nen\-ten} }

\newcommand{\MCF}{\emph{MCF} }
\newcommand{\pcs}{\emph{profil c/s} }

\newcommand{\DSL}{\emph{DSL} }


\usepackage{glossaries}

\makeglossaries

\mywork{Niels Gundermann}{Evaluation von Heuristiken für die testgetriebene Exploration von Enterprise-Java-Beans}
\begin{document}
\mymastertitle{Univ.\ Prof.\ Dr.\ Friedrich Steimann}


\masterabschlusserklaerung{20.11.2021}
\frontmatter

\section*{Abstract}
Mit dem Verfahren der testgetriebenen Codesuche ist ein*e Software-Entwickler*in in der Lage bestehenden Code in einem Repository nach vorgegebenen Kriterien zu durchsuchen. Die Kriterien beinhalten dabei Testfälle, die auf den bestehenden Code im Repository angewendet werden. Ausgehend davon, dass eine solche Suche während der Laufzeit innerhalb eines Systems möglich ist, wird die Zeit, die dafür zur Verfügung steht zu einem kritischen Aspekt.
\\\\
Daher zielt diese Arbeit darauf ab, Heuristiken zu evaluieren, durch die die testgetriebene Codesuche beschleunigt werden kann. Dazu wird die Exploration im Kontext der Arbeit formal beschrieben. Aufbauend auf dieser formalen Beschreibung werden drei Heuristiken vorgestellt, die bei der Exploration in einem bestehenden System evaluiert werden. Das Repository bildet dabei ein EJB-Container mit ca. 900 EJBs innerhalb des Systems. 
\\\\
Die Untersuchungsergebnisse zeigen, dass alle drei Heuristiken - wenn auch mit Abstufungen - das Potential haben, die Exploration zu beschleunigen.
\newpage




\newglossaryentry{komponente}
{
    name=Komponente,
    description={Eine Komponente beschreibt in der Softwarearchitektur im Allgemeinen ein Teil eines Softwaresystems. Die Definition dieses Begriffs wird in speziellen Frameworks weiter spezifiziert. Bezogen auf das in der Arbeit verwendete EJB-Framework, werden bspw. die Beans als Komponenten betrachtet (vgl. \cite{ejbspec})}
}
\newglossaryentry{artefakt}
{
    name=Artefakt,
    description={Ein Artefakt beschreibt in der Software-Entwicklung die Spezifikation einer physischen Informationseinheit als Ergebnis des Software-Entwicklungsprozesses oder dem Deployment bzw. der Ausführung eines Systems. In der UML Spezifikation 2.1.2 \cite{uml} werden u.a. folgende konkrete Beispiele für Artefakte genannt:
    \begin{itemize}
    \item Dateien in denen Source Code enthalten ist
    \item Skripte
    \item Datenbanktabellen    
    \end{itemize}
    \noindent
    Im Kontext dieser Arbeit sind insbesondere die Dateien, in denen Source Code enthalten ist, allgemein als Artefakt bezeichnet}
}


\newglossaryentry{Engine}
{
    name=Engine,
    description={Eine Engine beschreibt eine Software oder einen Teil einer Software, der für eine spezifische Aufgabe verantwortlich ist (vgl. \cite{pcmag}). Die Aufgabe, die die in der Arbeit beschriebenen Source Engines erfüllen, wird in Abschnitt \ref{sec_tdcs} beschrieben}
}


\newglossaryentry{Interface}
{
    name=Interface,
    description={Ein Interface hat im Allgemeinen eine Übersetzungs- oder Vermittlungsfunktion zwischen gekoppelten Systemen (vgl. \cite{interfaces}). Die Bedeutung des Begriffs in dieser Arbeit bezieht sich jedoch auf den Kontext der objektorientierten Programmierung. In diesem Zusammenhang beschreibt ein Interface die Methoden, die in den Klassen, die dieses Interface erfüllen, vorhanden sein müssen}
}


\newglossaryentry{wrappertype}
{
    name=Wrapper-Typ,
    description={}
}

\newglossaryentry{jndi}
{
    name=JNDI,
    description={}
}


\newglossaryentry{injection}
{
    name=Injection,
    description={}
}

\newglossaryentry{attributgrammatik}
{
    name=Attributgrammatik,
    description={}
}

\newglossaryentry{substitutionsprinzip}
{
    name=Substitutionsprinzip,
    description={}
}

\newacronym{ast}{AST}{\Gls{abstractSyntaxtree}}


\newglossaryentry{abstractSyntaxtree}
{
    name=Abstrakter Syntaxbaum,
    description={}
}

\newglossaryentry{downcast}
{
    name=Down-Cast,
    description={}
}



%\phantomsection
\tableofcontents
%\newpage

%\phantomsection
\addcontentsline{toc}{chapter}{\listfigurename}
\listoffigures
%\newpage

%\phantomsection
\addcontentsline{toc}{chapter}{\listtablename}
\listoftables
%\newpage

%\phantomsection
\addcontentsline{toc}{chapter}{Listings}
\lstlistoflistings
%\newpage



\mainmatter
\chapter{Einleitung}
\section{Motivation}
In größeren Software-Systemen ist es üblich, dass mehrere Komponenten miteinander über Schnittstellen kommunizieren. In der Regel werden diese Schnittstellen so konzipiert, dass sie Informationen oder Services anbieten, die von anderen Komponenten abgefragt und benutzt werden können. Dabei wird zwischen der Komponente, welche die Schnittstelle implementiert - als angebotene Komponente - und der Komponente, welche die Schnittstelle nutzen soll - als nachfragende Komponente - unterschieden (siehe \abbref{motiv}). 
\myScalableFigure[0.6\linewidth]{motiv}{Abhängigkeiten von nachfragenden und angebotenen Komponenten}{motiv}
\noindent
Wird von einer nachfragenden Komponente eine Information benötigt, die in dieser Form noch nicht angeboten wird, so wird häufig ein neues Interface für diese benötigte Information erstellt, welches dann passend dazu implementiert wird. Dabei muss neben der Anpassung der nachfragenden Komponente auch eine Anpassung oder Erzeugung der anbietenden Komponente erfolgen und zusätzlich das neue Interface deklariert werden. Zudem bedingt eine nachträgliche Änderung der neuen Schnittstelle ebenfalls eine Anpassung der drei genannten Artefakte.\\\\
In einem großen Software-System mit einer Vielzahl von bestehenden Schnittstellen ist eine gewisse Wahrscheinlichkeit gegeben, dass die Informationen oder Services, die von einer neuen nachfragenden Komponente benötigt werden, in einer ähnlichen Form bereits existieren. Das Problem ist jedoch, dass die manuelle Evaluation der Schnittstellen mitunter sehr aufwendig bis, aufgrund von unzureichender Dokumentation und Kenntnis über die bestehenden Schnittstellen, unmöglich ist.
\\\\
Weiterhin ist es denkbar, dass ein Software-System auf unterschiedlichen Maschinen verteilt wurde und dadurch Teile des Systems ausfallen können. Das hat zur Folge, dass die Implementierung bestimmter Schnittstellen nicht erreichbar ist. Dadurch, dass eine Schnittstelle durch eine nachfragende Komponente explizit referenziert wird, kann eine solche Komponente nicht korrekt arbeiten, wenn die Implementierung der Schnittstelle nicht erreichbar ist, obwohl die benötigten Informationen und Services vielleicht durch andere Schnittstellen, deren Implementierung durchaus zur Verfügung stehen, bereitgestellt werden könnten.
\\\\
Dies führt zu der Überlegung, ob eine nachfragende Komponente anstelle der Referenzierung einer Schnittstelle eine Spezifizierung der Schnittstelle vornimmt, anhand derer eine angebotene Komponente, die dieses Spezifikation erfüllt, gefunden werden kann.
\\\\
Ein solches Vorgehen wird bei der testgetriebene Codesuche (testdriven codesearch - \emph{TDCS}) verfolgt, welche als Basis für diese Arbeit herangezogen wird. Dabei stellt der Entwickler eine Menge von Suchparametern zusammen, die er an eine so genannte Source Engine übergibt. Die Suchparameter sind dabei jedoch stark an dem orientiert, was der Entwickler benötigt und weniger daran, was tatsächlich im Repository vorliegt. Diese Source Engine durchsucht anschließend ein Repository nach Komponenten (im weitesten Sinne), die zu den gestellten Suchparametern passen. 
\\\\
Die Suchergebnisse werden aufgelistet und der Entwickler entscheidet letztendlich explizit, welche Komponente verwenden möchte. Die Verwendung der Komponente läuft dann jedoch auf eine Referenzierung dieser in der nachfragenden Komponente hinaus. Somit arbeiten die Source Engines also nicht zur Laufzeit des Systems, in dem die Komponenten verwendet werden sollen.
\\\\
In dieser Arbeit soll eine solche Exploration jedoch zur Laufzeit erfolgen, sodass eine explizite Referenzierung der angebotenen Komponente nicht erfolgen muss. Dabei ist die Zeit als Ressource während der Suche nach einer passenden Komponente als knapp anzusehen. Aus diesem Grund werden in dieser Arbeit Heuristiken vorgeschlagen, die ein gezieltes Auffinden einer passenden Komponente ermöglichen und damit die Suche beschleunigen.

\section{Aufbau dieser Arbeit}
Zuerst wird in Kapitel \ref{chap_problem} auf den aktuellen Forschungsstand zur \emph{TDCS} eingegangen. Im Anschluss daran wird beschrieben, wie sich die \emph{TDCS} auf einen Ansatz, in dem zur Laufzeit nach Komponenten gesucht wird, eingegangen, um so eine Abgrenzung zu den früheren Arbeiten zu schaffen.
\\\\
In Kapitel \ref{chap_foundation} werden die einzelnen Schnritte, die während der Exploration durchgeführt werden, sowie die zu evaluierenden Heuristiken formal beschrieben.
% Dies teilt sich in vier Bereiche. Zuerst wird beschrieben, wie das Matching zwischen den angebotenen und den erwarteten Komponenten  hergestellt wird. Darauf aufbauend wird beschrieben, wie die matchenden angebotenen Komponenten miteinander kombiniert werden und somit neue Komponenten (so genannte Proxies) bilden. Im dritten Teil (Semantische Evaluation) wird das grundsätzliche Vorgehen bei der Applikation der Testfälle auf eben diese Proxies beschrieben. Und der letzte Teil beinhaltet die Beschreibung der Heuristiken und deren Integration in die semantische Evaluation.
\\\\
Kapitel \ref{chap_impl} gibt einen kurzen Überblick über die Implementierung der in Kapitel \ref{chap_foundation} genannten Aspekte.
\\\\
In Kapitel \ref{chap_evaluation} werden die Untersuchungsergebnisse, die unter Anwendung der Heuristiken im Einzelnen und in Kombination zusammengetragen wurden, vorgestellt. 
\\\\
Die Auswertung dieser Ergebnisse erfolgt in Kapitel \ref{chap_disc} zusammen mit einer kritischen Betrachtung des in der Arbeit vorgestellten Ansatzes, sowie einer kurzen Betrachtung möglicher Erweiterungen für diesen Ansatz.
\\\\
Komplettiert wird die Arbeit durch eine kurzen Zusammenfassung der Ergebnisse und einem Ausblick in Kapitel \ref{chap_finish}.
%\section{Gegenstand dieser Arbeit}
In dieser Arbeit soll jedoch nicht das gesamte Internet als Quelle oder Repository f�r die Codesuche dienen. Vielmehr wird der Suchbereich weiter eingeschr�nkt. \\\\
Es wird von einem System ausgegangen, in dem ein EJB-Container zur Verf�gung steht. Die Suche soll sich auf die Menge der angemeldeten Bean-Implementierungen beschr�nken. Die angemeldeten Bean-Implementierungen stellen damit die Menge der angebotenen Komponenten dar. Dabei wird eine angebotenen Komponente als Kombination eines Interfaces, welches die Schnittstelle f�r die Aufrufer definiert, und einer Implementierung des Interfaces. Das Interfaces einer angebotenen Komponenten wird im Folgenden auch als angebotenes Interfaces bezeichnet. Die Beans werden bspw. als Provider f�r Informationen oder im weitesten Sinne  auch als Services verwenden, die von unterschiedlichen Komponenten des Systems verwendet werden. Bei der Entwicklung bzw. Weiterentwicklung einer Komponente kann es zu folgenden Szenario kommen, welches durch die Erf�llung der unten aufgef�hrten Annahmen charakterisiert wird:
\begin{itemize}
\item Es werden Informationen und Services ben�tigt, bei denen der Entwickler davon ausgehen kann, dass es innerhalb des Systems angebotene Komponenten gibt, die diese Informationen liefern k�nnen bzw. die Services erf�llen.
\item Der Entwickler wei� nicht, �ber welche konkreten angebotenen Komponenten er die Informationen abfragen bzw. die Services in Anspruch nehmen kann.
\end{itemize}
\subsection{Funktionale Anforderungen}
In dieser Arbeit soll ein Konzept entwickelt werden, welches dem entwickler erm�glicht die Erwartungen an die angebotenen Komponenten zu spezifizieren. Darauf aufbauend soll ein Algorithmus vorgeschlagen werden, welcher die angebotenen Komponenten zur Laufzeit hinsichtlich der spezifizierten Erwartungen des Entwicklers evaluiert und eine Auswahl derer trifft, die diese Erwartungen erf�llen. Da die Evaluation zur Laufzeit durchgef�hrt wird, kann der Entwickler anders als bei den oben genannten Arbeiten nicht aus einer Liste von Vorschl�gen ausw�hlen, welche der evaluierten Komponenten letztendlich verwendet werden soll. Diese Entscheidung ist durch den Algorithmus zu treffen.
\subsection{Nichtfunktionale Anforderungen}
Aufgrund bestimmter Konfigurationen des Gesamtsystem gibt es folgende weitere nichtfunktionale Anforderungen:
\begin{itemize}
\item Die Suche muss innerhalb des Transaktionstimeouts von 5 Minuten zu einem Ergebnis f�hren.
\item Die Suche soll hinsichtlich der Besonderheiten des System, in dem sie verwendet wird, angepasst werden k�nnen. (Bspw. bei der Verwendung bestimmter Typen, deren Fachlogik bei der Suche nicht untergraben werden darf.)
\item Bei einem Fehlschlag der Suche, sollen dem Entwickler Informationen zur Verf�gung gestellt werden, die eine zielgerichtete Anpassung seiner spezifizierten Erwartungen erlauben.
\end{itemize}
\chapter{Forschungsziel}
\section{Testgetriebene Codesuche}
Die Idee der testgetriebenen Codesuche (testdriven codesearch - TDCS) beruht im Grunde auf dem Ziel der Wiederverwendung von Software, welches 1992 von Krueger wie folgt beschrieben wurde:
\emph{\glqq Software resure is the process of creating software systems from existing software rather than building software systems from scratch.\grqq{}} \cite{krueger} In der TDCS soll dieses Ziel in Verbindung mit dem Prozess der testgetriebenen Software-Entwicklung (testdriven development - TDD) erreicht werden. \cite{hummel08} 
\\\\
TDCS beruht grundlegend darauf, dass der Entwickler Anforderungen spezifiziert, die im Anschluss verwendet werden, um relevanten Source Code aus einem Repository hinsichtlich dieser Anforderungen zu ermittelt. Darauf aufbauend kann das jeweilige Tool dem Entwickler Vorschläge für die Wiederverwendung bestehenden Codes unterbreiten.
\\\\
Der Prozess der TDCS kann grundlegend wie in Abbildung \ref{} dargestellt werden (vgl. \cite{hummel08}).
\begin{figure}[h!]

\end{figure}
Somit entwirft der Entwickler zuerst ein Design in einem vorgegebenen Format, auf das er in den daraus zu spezifizierenden Tests zurückgreifen kann. Ausgehend von diesen Anforderungen wird das Repository nach Codeteilen durchsucht, die zu dem vorgegebenen Design passen. Die gefundenen Codeteile werden im Anschluss daran mithilfe der vorgegebenen Tests validiert\footnote{Durch die Vorgabe von Tests lässt sich die Parallele zum TDD ziehen.}.
\\\\
%\subsection{Verwandte Arbeiten}
Solche Ansätze wurden bereits in \cite{sourcerer} von Bajaracharya et al.  verfolgt. Diese Gruppe entwickelte eine Search Engine namens Sourcerer, welche Suche von Open Source Code im Internet ermöglichte. Darauf aufbauend wurde von derselben Gruppe in \cite{Lemos} ein Tool namens CodeGenie entwickelt, welches einem Softwareentwickler die Code Suche über ein Eclipse-Plugin ermöglicht. In diesem Zusammenhang wurde erstmals der Begriff der Test-Driven Code Search etabliert. Parallel dazu wurde in Verbindung mit der Dissertation Oliver Hummel \cite{hummel08} ebenfalls eine Weiterentwicklung von Sourcerer veröffentlicht, welche unter dem Namen Merobase bekannt ist, welches ebenfalls das Konzept der TDCS verfolgt.
\\\\
%\subsection{Voraussetzungen}
In Bezug auf die TDCS wurden von Hummel\cite{hummel08} dabei drei Voraussetzungen identifiziert:
\begin{enumerate}
\item Ein Software-Repository, in dem die wiederverwendbaren Softwareteile enthalten sind.
\item Ein Format für die Repräsentation dieser Softwareteile.
\item Ein Mechanismus, welcher in der Lage ist, das Repository zu durchsuchen.
\end{enumerate}
\noindent
Als Software-Repository wurden in den früheren Arbeiten im Internet bestehende Code-Repositiories verwendet. Die Repräsentation konnte dabei je nach Repository unterschiedliche Formen haben. Und die Mechanismen, die für die Suche verwendet wurden, waren ebenfalls vielfältig.

\section{Testgetriebene Exploration von EJBs}
Diese Arbeit legt den Fokus auf die Suche von Enterprise-Java-Beans (EJBs). Hummel hat EJBs bereits in \cite{hummel08} als Client-Server-Architektur für Software-Systeme, welche die Kommunikation zwischen Komponenten, die auf physikalisch unterschiedlichen Maschinen laufen, koordinieren bzw. unterstützen können (vgl. auch \cite{ejbspec}). Dazu wird das jeweilige Software-System auf einem Applikationsserver deployed, der die Enterprise-Java-Beans Spezifikation \cite{ejbspec} erfüllt.
\\\\
Bei einer Bean handelt es sich grundlegend um eine Java-Klasse, die eine vordefinierte Struktur hat. Seit der Version 3 kann die Struktur durch ein Java-Interface vorgegeben werden\cite{ejbspec}.
\\\\
Die Beans können über einen EJB-Container abgerufen werden. Zu diesem Zweck publiziert der EJB-Container die Interfaces der deployten Beans, sodass diese auf den Clients über JNDI oder Dependency Injection zur Verfügung stehen \cite{ejbspec}.
\\\\
Bezogen auf die in \cite{hummel08} beschriebenen Voraussetzungen für die TDCS wird der EJB-Container in dieser Arbeit als Software-Repository angesehen. Die einzelnen Softwareteile (EJBs) liegen in Form von Java-Interfaces repräsentiert. Und der Mechanismus zum Durchsuchen des Repositories wird durch die Publikation der Java-Interfaces der EJBs durch den EJB-Container bereitgestellt.
\\\\
Bezogen auf die Exploration von EJBs sieht der Prozess, der in Abbildung \ref{} aufgezeigt wurde vor, dass das Design der gesuchten Beans und die Tests, mit denen die zum Design passenden Beans validiert werden, vorgegeben wird. Für die Definition des Designs bietet sich dabei die Repräsentationsform der EJBs an - ein Java-Interface. Als Tests können wiederum Java-Klassen verwendet werden, die über ihre Methoden eine Validierung der EJBs erlauben.
\\\\
Somit kann der Prozess der testgetriebenen Exploration von EJBs in Anlehnung an der Beschreibung zu Abbildung \ref{} in Abbildung \ref{} etwas spezifiziert werden.
\begin{figure}[h!]

\end{figure}
\\\\
Die \emph{Strukturelle Evaluation} beschreibt die Ermittlung aller EJBs, die zu dem vorgegebenen Interface passen.
In Anlehnung an \cite{hummel08} werden diese EJBs auf der Basis des Signature-Matching Ansatzes ermittelt. Dieser Ansatz wurde ursprünglich von Zaremski und Wing \cite{moormann} etabliert. Er basiert darauf, dass lediglich die Methoden-Signaturen der Klassen bzw. Interfaces miteinander abgeglichen werden. Der Abgleich erfolgt sowohl in \cite{moormann} als auch in dieser Arbeit auf der Basis von Matchern, die in Abschnitt \ref{sec_matcher} genauer beschrieben werden.
\\\\
Da bei der Ermittlung der Beans lediglich die Methoden-Signaturen eine Rolle spiele, besteht die Möglichkeit, dass die Methoden einer einzelnen Bean nur zu einem Teil der Methoden des vorgegebenen Interfaces passen. In diesem Fall kann der Ansatz dazu verwendet werden, für die übrigen Methoden eine andere Bean zu finden, die dafür passende Methoden bereitstellt. Damit müssten die beiden Beans jedoch miteinander kombiniert werden, um das vorgegebene Interface in Gänze zu matchen.
\\\\
Dieses Problem soll in dieser Arbeit ebenfalls adressiert werden. Die Kombination der Beans soll über ein Proxy-Objekt erreicht werden, welches bei der Exploration im Anschluss an die Strukturelle Evaluation generiert wird. Das Proxy-Objekt muss dann zum einen in der Lage sein, die Methodenaufrufe wie in den Methoden-Signaturen den vorgegebenen Interfaces entgegenzunehmen und diese dann zum Anderen an die entsprechende Bean, die eine dazu passende Methode bereitstellt, delegieren.
\\\\
Dieser Schritt ordnet sich in den Gesamtprozess der testgetriebenen Exploration von EJBs, wie folgt ein (siehe Abbildung \ref{}):
\begin{figure}[h!]

\end{figure}
\noindent
Der letzte Schritt beinhaltet die Validierung der generierten Proxies durch die vorgegebenen Testklassen. Da die Exploration mit den oben genannten Voraussetzungen nur zur Laufzeit durchgeführt werden kann\footnote{Anderenfalls steht der EJB-Container gar nicht zur Verfügung.}, sollte die Suche abgebrochen werden, sofern ein generierter Proxy erfolgreich validiert wurde. Anderenfalls kann es bspw. zu unnötigen Timeouts laufender Transaktionen kommen. Um darüber hinaus ein schnelles Auffinden eines validierten Proxies zu gewährleisten, werden in diesem letzten Schritt Heuristiken verwendet, welche die Generierung von Proxies und der positiven Validierung eines dieser Proxies beschleunigen. Die vorliegende Arbeit dient hauptsächlich der Evaluation solcher Heuristiken.


\chapter{Theoretische Grundlagen}\label{chap_foundation}
In den folgenden Abschnitten wird der Explorationsprozess und dessen Grundlagen formal beschrieben sowie zum besseren Verständnis mit entsprechenden Beispielen untermalt. Die einzelnen Schritte des Prozesses finden sich damit in den  Überschriften der Abschnitte wieder.

\section{Strukturelle Evaluation}
\subsection{Struktur für die Definition von Typen}\label{sec:strukturTypen}
Die Typen seien in einer Bibliothek $\text{L}$ in folgender Form zusammengefasst:
\begin{table}[H]
\centering
\begin{tabular}{|p{5.5cm}|p{8.5cm}|}
\hline
\hline
\centering\textbf{Regel} & \textbf{Erläuterung} \\
\hline
\hline
$\mathit{L} ::= \mathit{TD}\text{*}$ & Eine Bibliothek \emph{L} besteht aus einer Menge von Typdefinitionen.\\
\hline
$\mathit{TD} ::= \mathit{PD} | \mathit{RD}$ & Eine Typdefinition kann entweder die Definition eines provided Typen (PD) oder eines required Typen (RD) sein.\\
\hline
$\mathit{PD} ::= \newline\texttt{provided }T \texttt{ extends } T' \newline  \texttt{\{} \mathit{FD}\text{*} \mathit{MD}\text{*}\texttt{\}}$& Die Definition eines provided Typen besteht aus dem Namen des Typen \emph{T}, dem Namen des Super-Typs \emph{T'} von \emph{T} sowie mehreren Feld- und Methodendeklarationen.\\
\hline
$\mathit{RD} ::= \texttt{required } T \texttt{ \{}\mathit{MD}\text{*}\texttt{\}}$ & Die Definition eines required Typen besteht aus dem Namen des Typen \emph{T} sowie mehreren Methodendeklarationen.\\
\hline
$\mathit{FD} ::= T \texttt{ }\mathit{f}$ & Eine Felddeklaration besteht aus dem Namen des Feldes \emph{f} und dem Namen seines Typs \emph{T}.\\
\hline
$\mathit{MD} ::= \mathit{T'}\texttt{ }\mathit{m(T)}$ & Eine Methodendeklaration besteht aus dem Namen der Methode \emph{m}, dem Namen des Parameter-Typs \emph{T} und dem Namen des Rückgabe-Typs \emph{T'}.\\
\hline
\hline
\end{tabular}
\caption{Struktur für die Definition einer Bibliothek von Typen}
 \label{tab:eIShort}
\end{table}
\noindent
Weiterhin sei die Relation $<$ auf Typen durch folgende Regeln definiert:
\begin{gather*}
\frac{\texttt{provided }T \texttt{ extends } T' \in L}{T < T'}
\end{gather*}
\begin{gather*}
\frac{\texttt{provided } T \texttt{ extends } T'' \in L \wedge T'' < T'}{T < T'}
\end{gather*}

Darüber hinaus seien folgende Funktionen definiert:
\begin{gather*}
\mathit{felder(T)} :=  \left\{ 
				\begin{array}{l|l}
					T \texttt{ }\mathit{f} & T \texttt{ }\mathit{f}\text{ ist Felddeklaration von }T
				\end{array}
              \right\}\\
\mathit{methoden(T)} := \left\{ 
				\begin{array}{l|l}
					T'' \text{ }m(T') & T'' \text{ }m(T') \text{ ist Methodendeklaration von }T
				\end{array}
              \right\}\\
\mathit{feldTyp(f,T)} := 
				\begin{array}{l|l}
					T' & T' \texttt{ }\mathit{f}\text{ ist Felddeklatation von }T
				\end{array}              
\end{gather*}

\subsubsection{Beispiel-Bibliothek}



\begin{lstlisting}[style = dsl]
provided Fire extends Object{}
\end{lstlisting}

\begin{lstlisting}[style = dsl]
provided ExtFire extends Fire{}
\end{lstlisting}


\begin{lstlisting}[style = dsl]
provided FireState extends Object{
	boolean isActive
}
\end{lstlisting}

\begin{lstlisting}[style = dsl]
provided Medicine extends Object{
	String getDescription()
}
\end{lstlisting}

\begin{lstlisting}[style = dsl]
provided Injured extends Object{
	void heal(Medicine med)	
}
\end{lstlisting}


\begin{lstlisting}[style = dsl]
provided Patient extends Injured{
	String getName()
}
\end{lstlisting}
\begin{lstlisting}[style = dsl]
provided FireFighter extends Object{
	FireState extinguishFire(Fire fire)
}
\end{lstlisting}

\begin{lstlisting}[style = dsl]
provided Doctor extends Object{	
	void heal( Patient pat, Medicine med )
}
\end{lstlisting}


\begin{lstlisting}[style = dsl]
provided InverseDoctor extends Object{	
	void heal( Medicine med, Patient pat )
}
\end{lstlisting}

\begin{lstlisting}[style = dsl]
provided MedCabinet extends Object{
	Medicine med
}
\end{lstlisting}

\begin{lstlisting}[style = dsl]
required PatientMedicalFireFighter {
	void heal( Patient patient, MedCabinet med )
	boolean extinguishFire( ExtFire fire )	
}
\end{lstlisting}

\begin{lstlisting}[caption={Bibliothek \emph{ExampLe} von Typen},captionpos=b, style = dsl]
required MedicalFireFighter {
	void heal( Injured injured, MedCabinet med )
	boolean extinguishFire( ExtFire fire )	
}
\end{lstlisting}\label{lst:libEx}

\newpage

\subsection{Struktur für die Definition von Proxies}\label{sec:proxygram}
Die Konvertierung eines Typs $T$ aus einer Menge von provided Typen $P$ wird durch \emph{Proxies} beschrieben. Die Grammatikregeln für einen Proxies sind Tabelle \ref{tab:grProxies} zu entnehmen.
\begin{table}[H]
\centering
\begin{tabular}{|p{5cm}|p{9cm}|}
\hline
\hline
\centering\textbf{Regel} & \textbf{Erläuterung} \\
\hline
\hline
$\mathit{PROXY} ::=$\newline
$\texttt{proxy } \texttt{for } T$\newline
$ \texttt{with [}\mathit{P_1},...,\mathit{P_n}\texttt{]}$ \newline
$\texttt{\{}\mathit{MDEL_1},...,\mathit{MDEL_k} \texttt{\}}$
 & Ein Proxy wird für ein Typ $T$ als Source-Typ mit einer Mengen von provided Typen $P = \{P_1,...,P_n\}$ als Target-Typen, einer Menge von Methoden-Delegationen erzeugt.\\
\hline
$\mathit{MDEL} ::=$\newline
$CALLM \rightarrow DELM $  & Eine \emph{Methodendelegation} besteht aus einer \emph{aufgerufenen Methode} und aus einem \emph{Delegationsziel}.\\
\hline
$\mathit{CALLM} ::=$\newline 
$\mathit{REF}.\mathit{m(\mathit{CP_1},...,\mathit{CP_n}):CR} $  & Eine aufgerufene Methode besteht aus dem Namen der Methode $m$, dem Rückgabetyp $\mathit{CR}$ und einer Menge von Parametertypen $\{\mathit{CP_1},...,\mathit{CP_n}\}$.\\
\hline
$\mathit{DELM} ::=$\newline 
$\mathit{REF}.\mathit{n(\mathit{DP_1},...,\mathit{DP_n}):DR} $  
& Die erste Variante eines Delegationsziels besteht aus  dem Namen der \emph{Delegationsmethode} $n$, dem Rückgabetyp $\mathit{DR}$ und einer Menge von Parametertypen $\{\mathit{DP_1},...,\mathit{DP_n}\}$.\\
\hline
$\mathit{DELM} ::=$\newline
$\texttt{posModi(} \mathit{I_1},...,\mathit{I_n} \texttt{)}$\newline
$\mathit{REF}.\mathit{n(\mathit{DP_1},...,\mathit{DP_n}):DR} $  
& Die zweite Variante eines Delegationsziels besteht aus einer Menge von Indizies $\{\mathit{I_1},...,\mathit{I_n}\}$, einer \emph{Referenz}, dem Namen der Delegationsmethode $n$, dem Rückgabetyp $\mathit{DR}$ und einer Menge von Parametertypen $\{\mathit{DP_1},...,\mathit{DP_n}\}$.\\
\hline
$\mathit{DELM} ::= \texttt{err} $  
& Die dritte Variante eines Delegationsziels enthält keine weiteren Bestandteile. Das Terminal $\texttt{err}$ weist darauf hin, dass die Delegation innerhalb des Proxies nicht möglich ist und zu einem Fehler führt.\\
\hline
$\mathit{REF} ::= \mathit{P_i}$
& Die erste Variante einer Referenz besteht aus einem Typ $P_i$ .\\
\hline
$\mathit{REF} ::= \mathit{P_i}\texttt{.}\mathit{f}$
& Die zweite Variante einer Referenz besteht aus einem Typ $P_i$ und einem Feldnamen $f$.\\
\hline
\end{tabular}
\caption{Grammatikregeln mit Erläuterungen für die Definition eines Proxies}
 \label{tab:grProxies}
\end{table}
\noindent
Es handelt sich dabei um Produktionsregeln einer Attributgrammatik. Die dazugehörigen Attribute sind der Tabelle \ref{tab:attrGrProxies} zu entnehmen. Dazu sei zusätzlich festgelegt, dass die Notation $\mathit{NT}\texttt{.}\text{*}$ in der Spalte \emph{Attribute} eine Key-Value-Liste aller Attribute des Nonterminals $\mathit{NT}$ beschreibt, wobei der Attributname als Key und dessen Wert als Value innerhalb der Liste verwendet wird. Weiterhin sei ein Attribut, dass in der Spalte \emph{Attribute} zu einem Nonterminal nicht aufgeführt ist, wird mit dem Wert \emph{none} belegt.
\begin{table}[h!]
\centering
\begin{tabular}{|p{6cm}|p{8cm}|}
\hline
\hline
\centering\textbf{Regel} & \textbf{Attribute} \\
\hline
\hline
$\mathit{PROXY} ::=$\newline
$\texttt{proxy } \texttt{for } T$\newline
$ \texttt{with [}\mathit{P_1},...,\mathit{P_n}\texttt{]}$ \newline
$\texttt{\{}\mathit{MDEL_1},...,\mathit{MDEL_k} \texttt{\}}$
& 
$\texttt{type} = \mathit{T}$\newline
$\texttt{targets} = [\mathit{P_1},...,\mathit{P_n}]$\newline
$\texttt{dels} = [\mathit{MDEL_1}\texttt{.}\text{*},...,\mathit{MDEL_k}\texttt{.}\text{*}]$
\\
\hline
$\mathit{MDEL} ::=$\newline
$\mathit{CALLM} \rightarrow \mathit{DELM} $  
& 
$\texttt{call} = \mathit{CALLM}.*$\newline
$\texttt{del} = \mathit{DELM}.*$
\\
\hline
$\mathit{CALLM} ::=$\newline 
$\mathit{REF}.\mathit{m(\mathit{CP_1},...,\mathit{CP_n}):CR}$
& 
$\texttt{source} = \mathit{REF.\texttt{mainType}}$\newline
$\texttt{delType} = \mathit{REF.\texttt{delType}}$\newline
$\texttt{name} = \mathit{m}$\newline
$\texttt{paramTypes} = \mathit{[CP_1},...,\mathit{CP_n]}$\newline
$\texttt{returnType} = \mathit{CR}$\newline
$\texttt{field} = \mathit{REF}\texttt{.field}$\newline
$\texttt{paramCount} = n$
\\
\hline
$\mathit{DELM} ::=$\newline 
$\mathit{REF}\texttt{.}n(\mathit{DP_1},...,\mathit{DP_n}):DR $  
&
$\texttt{target} = \mathit{REF}.\texttt{mainType}$\newline
$\texttt{delType} = \mathit{REF}.\texttt{delType}$\newline
$\texttt{posModi} = [0,...,\mathit{n}-1]$\newline
$\texttt{name} = \mathit{n}$\newline
$\texttt{paramTypes} = \mathit{[DP_1},...,\mathit{DP_n]}$\newline
$\texttt{returnType} = \mathit{DR}$\newline
$\texttt{field} = \mathit{REF}\texttt{.field}$
\\
\hline
$\mathit{DELM} ::=\texttt{posModi(} \mathit{I_1},...,\mathit{I_n} \texttt{)}$\newline
$\mathit{REF}\texttt{.}n(\mathit{DP_1},...,\mathit{DP_n}):DR $  
&
$\texttt{target} = \mathit{REF}.\texttt{mainType}$\newline
$\texttt{delType} = \mathit{REF}.\texttt{delType}$\newline
$\texttt{posModi} = \mathit{[I_1},...,\mathit{I_n]}$\newline
$\texttt{name} = \mathit{n}$\newline
$\texttt{paramTypes} = \mathit{[DP_1},...,\mathit{DP_n]}$\newline
$\texttt{returnType} = \mathit{DR}$\newline
$\texttt{field} = \mathit{REF}\texttt{.field}$
\\
\hline
$\mathit{DELM} ::= \texttt{err} $  
&
\\
\hline
$\mathit{REF} ::= \mathit{P}$
& 
$\texttt{mainType} = \mathit{P}$\newline
$\texttt{field} = \texttt{self}$\newline
$\texttt{delType} = \mathit{P}$
\\
\hline
$\mathit{REF} ::= \mathit{P}\texttt{.}\mathit{f}$
&
$\texttt{mainType} = \mathit{P}$\newline
$\texttt{field} = \mathit{f}$\newline
$\texttt{delType} = \mathit{feldTyp(f,P)}$
\\
\hline
\end{tabular}
\caption{Grammatikregeln mit Attributen für die Definition eines Proxies}
 \label{tab:attrGrProxies}
\end{table}
\noindent
Ein Proxy bietet alle Methoden des Source-Typen an. Einige dieser Methoden werden an eine Methode delegiert, die von einem der Target-Typ des Proxies angeboten wird. Eine solche Delegation wird durch eine Methoden-Delegation (siehe Nontermial $\mathit{MDEL}$) definiert.
\paragraph{Beispiel} So beschreibt die folgende Methoden-Delegation, dass die Methode $\texttt{extinguishFire}$, die vom Source-Typ $\texttt{Patient}$ - und damit auch vom Proxy - angeboten wird, an die Methoden $\texttt{heal}$, die der Target-Typ $\texttt{Injured}$ anbietet, delegiert wird.
\begin{lstlisting}[style = dsl, caption = Einfache Methoden-Delegation, captionpos = b]
	Patient.heal(Medicine):void --> Injured.heal(Medicine):void
\end{lstlisting}
\noindent
Die Delegation einer aufgerufenen Methode an ein Delegationsziel, erfolgt in drei Schritten.
\begin{enumerate}
\item Parameterübergabe\\
Dabei werden die Parameter, mit denen die vom Proxy angebotene Methode, aufgerufen wird, an die Delegationsmethode des Delegationsziels übergeben. Dabei sind zwei Dinge zu beachten. Zum Einen müssen die Typen der übergebenen Parameter zu den Typen der von der Delegationsmethode erwarteten Parameter passen. Zum Anderen muss die Reihenfolge, in der die Parameter übergeben wurden, an die erwartete Reihenfolge der Delegationsmethode angepasst werden.
\item Ausführung\\
Dieser Schritt meint die Durchführung der Delegationsmethode mit den übergeben Parametern aus Schritt 1. Dies schließt auch die Ermittlung des Rückgabewertes der Delegationsmethode ein.
\item Übergabe des Rückgabewertes\\
Ähnlich wie bei der Parameterübergabe, muss auch der Rückgabewert, der bei der Ausführung in Schritt 2 ermittelt wurde, an die aufgerufenen Methode, die vom Proxy angeboten wird, übergeben werden. Hier muss ebenfalls sichergestellt werden, dass die beiden Rückgabetypen der beiden Methoden zueinander passen.
\end{enumerate}
Die Delegation aus dem oben genannten Beispiel kann schematisch wie in Abbildung \ref{fig:DEL_heal} dargestellt werden. Die Übergabe der Parameter- und Rückgabewerte wird durch die gestrichelten Pfeile symbolisiert.
\begin{figure}[h!]
\includegraphics[width=\linewidth]{MDEL_heal}
\caption{Delegation der Methode $\texttt{heal}$}
\label{fig:DEL_heal}
\end{figure}
\noindent
An diesem Beispiel sind sowohl die Parameter- als auch die Rückgabe-Typen der aufgerufenen Methode und der Delegationsmethode identisch sind. Weiterhin spielt die Reihenfolge der Parameter in diesem Beispiel keine Rolle, da es nur einen Parameter gibt. Daher stellt die Übergabe der Parameter- und Rückgabewerte kein Problem dar.\\\\
Folgendes Beispiel soll zeigen, wie mit unterschiedlichen Reihenfolgen bzgl. der Parameter bei einer Methoden-Delegation umzugehen ist.
\paragraph{Beispiel} Die Methoden-Delegation aus Listing \ref{lst:methdel2} ist ein Beispiel für einen solchen Fall. Hier wird die aufgerufene Methode $\texttt{heal}$ mit den Parametern $\texttt{Patient}$ und $\texttt{MedCabinet}$ aus dem Typ $\texttt{PatientMedicalFireFighter}$ an die gleichnamige Methode aus dem Typ $\texttt{InverseDoctor}$ delegiert. Die Delegationsmethoden verwendet zwar identische Parameter-Typen, aber die Reihenfolge, in der die Parameter übergeben werden, ist unterschiedlich.
\begin{lstlisting}[style = dsl, caption = Methoden-Delegation mit Parametern in unterschiedlicher Reihenfolge, captionpos = b]
	PatientMedicalFireFighter.heal(Patient, MedCabinet):void --> posModi(1,0)  InverseDoctor.heal(MedCabinet,Patient):void
\end{lstlisting}\label{lst:methdel2}
\noindent
Um die Reihenfolge der Parameter aus dem ursprünglichen Aufruf zu variieren, wird das Schlüsselwort $\texttt{posModi}$ verwendet. Dort werden eine Reihe von Indizes angegeben. Die Anzahl der angegebenen Indizes muss mit der Anzahl der Parameter übereinstimmen. Ein Index beschreibt die Position des in der aufgerufenen Methode angegebenen Parameter. Weiterhin spielt die Reihenfolge der Indizes eine wichtige Rolle. Diese ist mit der Reihenfolge der Parameter der Delegationsmethoden gleichzusetzen.\\\\
So wird in dem o.g. Beispiel der erste Parameter der aufgerufenen Methoden (Index = 0) der Delegationsmethode als zweiter Parameter übergeben. Dementsprechende wird er zweite Parameter der aufgerufenen Methoden (Index = 1) der Delegationsmethode als erster Parameter übergeben (siehe Abbildung \ref{fig:DEL_healInverse}). 
\begin{figure}[H]
\includegraphics[width=\linewidth]{MDEL_healInverse}
\caption{Delegation der Methode $\texttt{heal}$ mit Parametern in unterschiedlicher Reihenfolge}
\label{fig:DEL_healInverse}
\end{figure}
\noindent
Ein weiteres Beispiel soll zeigen, wie mit übergebenen Typen umzugehen ist, die nicht ohne Probleme übergeben werden können. Dafür ist jedoch vorab zu klären, wann dies der Fall ist.\\\\
Dass identische Typen keine Probleme bei der Übergabe zwischen aufgerufener Methode und Delegationsmethode darstellen, wurde in den oben genannten Beispielen gezeigt.\\\\
Darüber hinaus können Typen aber auch dann ohne Probleme übergeben werden, wenn sie sich aufgrund des Substitutionsprinzips austauschen lassen. Daher kann ein Typ $T$ anstelle eines Typs $T'$ verwendet werden, sofern $T \leq T'$ gilt.
\paragraph{Beispiel} In folgendem Listing ist eine Methoden-Delegation aufgerührt, bei der sowohl die Parameter- als auch die Rückgabe-Typen der aufgerufenen Methode und der Delegationsmethode nicht auf Basis des Substitionsprinzips übergeben werden können.
\begin{lstlisting}[style = dsl, caption = Methoden-Delegation mit Typkonvertierung, captionpos = b]
	MedicalFireFighter.extinguishFire(ExtFire):boolean --> FireFigher.extinguishFire(Fire):FireState
\end{lstlisting}\label{lst:methdel3}
\noindent
In einem solchen Fall müssen die Parameter-Typen der aufgerufenen Methoden in die Parameter-Typen der Delegationsmethode konvertiert werden. Analog dazu muss der Rückgabetyp der Delegationsmethode in den Rückgabetyp der aufgerufenen Methoden konvertiert werden.\\\\
Angenommen, die Funktion $\mathit{proxies(S,T)}$ beschreibt eine Menge von Proxies, mit $S$ als Source-Typ und $T$ als Menge der Target-Typen. Dann müssten bezogen auf die Methoden-Delegation aus Listing 4 für die Parameter-Typen einer der Proxies aus der Menge $\mathit{proxies(\texttt{Fire}, \{\texttt{ExtFire}\})}$ an die Delegationsmethode übergeben werden. Nach der Ausführung der Delegationsmethode müsste ein Proxy aus der Menge $\mathit{proxies(\texttt{boolean},\{\texttt{FireState}\})}$ an die aufgerufenen Methode als Rückgabetyp übergeben werden. Der Sachverhalt wird in Abbildung \ref{fig:DEL_extinguishFire} schematisch dargestellt.
\begin{figure}[H]
\includegraphics[width=\linewidth]{MDEL_extinguishFire}
\caption{Delegation der Methode $\texttt{extinguishFire}$ mit Typkonvertierungen}
\label{fig:DEL_extinguishFire}
\end{figure}
\noindent
Wie die Proxies generiert werden, wird im folgenden Abschnitt beschrieben.

\subsection{Generierung der Proxies auf Basis von Matchern}
Ein Proxy wird in Abhängigkeit vom Matching zwischen dem Source- und den Target-Typen erzeugt. Im Folgenden werden zuerst die Matcher beschrieben. Im Anschluss wird auf die Generierung der Proxies eingegangen.
\subsubsection{Matcher}
Ein Matcher definiert das Matching eines Typs $T$ zu einem Typ $T'$ durch die asymmetrische Relation $T \Rightarrow T'$.
\paragraph{ExactTypeMatcher}\label{sec:exacttypematcher}
Der \emph{ExactTypeMatcher} stellt ein Matching von einem Typ $T$ zu demselben Typ $T$ her. Die dazugehörige Matchingrelation $\Rightarrow_{exact}$ wird durch folgende Regel beschrieben:
\begin{gather*}
\frac{}{T \Rightarrow_{exact} T}
\end{gather*}
\paragraph{GenTypeMatcher}\label{sec:gentypematcher}
Der \emph{GenTypeMatcher} stellt ein Matching von einem Typ $T$ zu einem Typ $T'$ mit $T > T'$ her. Die dazugehörige Matchingrelation $\Rightarrow_{gen}$ wird durch folgende Regel beschrieben:
\begin{gather*}
\frac{T > T'}{T \Rightarrow_{gen} T'}
\end{gather*}
\paragraph{SpecTypeMatcher}
Der \emph{SpecTypeMatcher} stellt im Verhältnis zum \emph{GenTypeMatcher} das Matching in die entgegengesetzte Richtung dar. Die dazugehörige Matchingrelation $\Rightarrow_{spec}$ wird durch folgende Regel beschrieben: 
\begin{gather*}
\frac{T < T'}{T \Rightarrow_{spec} T'}
\end{gather*}
\\\\
Die oben genannten Matchingrelationen werden für die Definition weiterer Matcher zusammengefasst, wodurch sich die Matchingrelation $\Rightarrow_{internCont}$ ergibt:
\begin{gather*}
\frac{T \Rightarrow_{exact} T' \vee T \Rightarrow_{gen} T' \vee
T \Rightarrow_{spec} T'  }{T \Rightarrow_{internCont} T'}
\end{gather*}
\paragraph{ContentTypeMatcher}
Der \emph{ContentTypeMatcher} matcht einen Typ $T$ auf einen Typ $T'$, wobei $T'$ ein Feld enthält, auf dessen Typ $T''$ der Typ $T$ über die Matchingrelation $\Rightarrow_{internCont}$ gematcht werden kann. So kann bspw. der Typ $\texttt{boolean}$ aus Listing 1 auf den Typ $\texttt{FireState}$ gematcht werden.
\\\\
Die dazugehörige Matchingrelation $\Rightarrow_{content}$ wird durch folgende Regel beschrieben:
\begin{gather*}
\frac{\exists \mathit{T''\text{ }f}\in felder(T'): T \Rightarrow_{internCont} T''}{T \Rightarrow_{content} T'}
\end{gather*}
\noindent
So würde für die Typen $\texttt{boolean}$ und $\texttt{FireState}$ gelten: 
\begin{gather*}
\texttt{boolean} \Rightarrow_{content} \texttt{FireState}
\end{gather*}
\paragraph{ContainerTypeMatcher}
Der \emph{ContainerTypeMatcher} stellt im Verhältnis zum \emph{ContentTypeMatcher} das Matching in die entgegengesetzte Richtung dar. So kann bspw. auch der Typ $\texttt{FireState}$ auf den Typ $\texttt{booealn}$ aus Listing 1 gematcht werden.
\\\\
Die dazugehörige Matchingrelation $\Rightarrow_{container}$ wird durch folgende Regel beschrieben:
\begin{gather*}
\frac{\exists \mathit{T''\text{ }f}\in felder(T): T'' \Rightarrow_{internCont} T'}{T \Rightarrow_{container} T'}
\end{gather*}
\noindent
So gilt für die Typen $\texttt{FireState}$ und $\texttt{boolean}$: 
\begin{gather*}
\texttt{FireState} \Rightarrow_{container} \texttt{boolean}
\end{gather*}
\\\\
Zur Definition des letzten Matchers werden die Matchingrelationen der oben genannten Matcher noch einmal zusammengefasst. Dabei entsteht die Matchingrelation $\Rightarrow_{internStruct}$, welche durch folgende Regel beschrieben wird:
\begin{gather*}
\frac{T \Rightarrow_{internCont}T' \vee T \Rightarrow_{container} T' \vee T \Rightarrow_{content} T'}{T \Rightarrow_{internStruct}T'}
\end{gather*}
\paragraph{StructuralTypeMatcher} 
Der \emph{StructuralTypeMatcher} matcht einen \emph{required Typ} $R$ auf einen \emph{provided Typ} $P$ auf der Basis struktureller Eigenschaften der Methoden, die in den Typen deklariert sind. 
\\\\
Somit soll bspw. der Typ $\texttt{MedicalFireFighter}$ auf den Typ $\texttt{FireFighter}$ (siehe Listing 1) gematcht werden. Als ein weiteres Beispiel, bezogen auf die Typen aus Listing 1, kann das Matching des Typs $\texttt{MedicalFireFighter}$ auf den Typ $\texttt{Doctor}$ angebracht werden.
\\\\
Damit ein required Typ $R$ auf einen provided Typ $P$ über den \emph{StrukturalTypeMatcher} gematcht werden kann, muss mindestens eine Methode aus $R$ zu einer Methode aus $P$ gematcht werden. Die Menge der gematchten Methoden aus $R$ in $P$ wird wie folgt beschrieben:
\begin{gather*}
structM(R,P) := \left\{ 
				\begin{array}{l|l}
				& \mathit{T'\text{ }m(T)} \in \mathit{methoden(R)} \wedge \mathit{ }\\
T'\text{ }m(T)	& \exists \mathit{S'\text{ }n(S)} \in \mathit{methoden(P)} :\\								 				&  S\Rightarrow_{internStruct}T \wedge T' \Rightarrow_{internStruct}S' 
				\end{array}
              \right\}
\end{gather*}
Da die Notation es nicht hergibt, ist zusätzlich zu erwähnen, dass, sofern in $m$ und $n$ mehrere Parameter verwendet werden, deren Reihenfolge irrelevant ist.\\\\
Die Matchingrelation für die \emph{StructuralTypeMatcher} wird durch folgende Regel beschrieben:
\begin{gather*}
\frac{structM(R,P) \neq \emptyset}{R \Rightarrow_{struct}P}
\end{gather*}
\subsubsection{Generierung von Proxies}
Wie im Abschnitt \ref{sec:proxygram} bereits erwähnt, soll die Menge der Proxies für einen Source-Typ $S$ und einer Menge von Target-Typen $T$ über die Funktion $\mathit{proxies(S,T)}$ beschrieben werden.\\\\
In Abhängigkeit von dem Matching zwischen dem Source-Typ und den Target-Typen werden unterschiedliche Arten von Proxies generiert. Für die unterschiedlichen Proxy-Arten gibt es ebenfalls Funktionen, die eine Menge von Proxies zu einem Source-Typen $S$ und einer Menge von Target-Typen $T$ beschreiben.\\\\
In den folgenden Abschnitten werden diese Funktionen für die einzelnen Proxy-Arten beschrieben. Dabei ist davon auszugehen, dass die Proxies eine allgemeine Struktur haben, die in Abschnitt \ref{sec:proxygram} aufgeführt ist. Um die Regeln für die Generierung der Proxies zu beschreiben, soll davon ausgegangen werden, dass jedes Listen-Attribut ($\mathit{NT.}\text{*}$) aus Tabelle \ref{tab:attrGrProxies} ein Attribut $\texttt{len}$ enthält in dem die Anzahl der in der Liste befindlichen Elemente abgelegt ist.


\paragraph{Sub-Proxy}
Die Voraussetzung für die Erzeugung eines \emph{Sub-Proxies} vom Typ $T$ aus einem Target-Typ $T'$ ist $T \Rightarrow_{spec} T'$. Damit ist der \emph{SpecTypeMatcher} der Basis-Matcher für den Sub-Proxy.
\paragraph{Beispiel}
Als Beispiel soll  der Typ $\texttt{Patient}$ als Source-Typ und der Typ $\texttt{Injured}$ als Target-Typ verwendet werden. Da $\texttt{Patient} \Rightarrow_{spec} \texttt{Injured}$ gilt, kann ein \emph{Sub-Proxy} für diese Konstellation erzeugt werden. Der resultierende \emph{Sub-Proxy} ist im folgenden Listing aufgeführt.
\begin{lstlisting}[style = dsl, caption = Sub-Proxy für Patient, captionpos = b]
proxy for Patient with [Injured]{
	Patient.heal(Medicine):void --> Injured.heal(Medicine):void
	Patient.getName():String --> err
}
\end{lstlisting}
Der abstrakte Syntaxbaum mit den dazugehörigen Attributen ist Abbildung \ref{fig:ASTSUB} zu entnehmen. \footnote{Es wurden nur die Nonterminale mit den dazugehörigen Attributen aufgeführt.}
\begin{figure}[h!]
\includegraphics[width=\linewidth]{AST_SubExample}
\caption{AST für das Beispiel zum Sub-Proxy}
\label{fig:ASTSUB}
\end{figure}
\noindent
Der Proxy bietet alle Methoden an, die auch von dessen Source-Typ angeboten werden. Die Methodendelegationen innerhalb des Proxies, beschreiben, was beim Aufruf der jeweiligen aufgerufenen Methoden passiert. So wird ein Aufruf der Methode $\texttt{heal}$ an die Methode $\texttt{heal}$ aus dem Target-Typ delegiert. Ein Aufruf der Methode $\texttt{getName}$ hingegen führt zu einem Fehler, weil keine Delegationsmethode zur Verfügung steht.\\\\
Im Hinblick darauf, dass eine Konvertierung von einem Super-Typ und einen Sub-Typ (Down-Cast) ebenfalls dazu führt, dass bestimmte Methoden, wie in diesem Fall $\texttt{getName}$ nicht ausgeführt werden können, spiegelt der \emph{Sub-Proxy} dieses Verhalten wieder.
\paragraph{Formalisierung}
Formal wird ein \emph{Sub-Proxy} durch die Regeln beschrieben, die im Folgenden vorgestellt werden. Ein \emph{Sub-Proxy} enthält genau einen Target-Typ. Für einen Proxy $P$ wird dieser Sachverhalt durch die folgende Regel dargestellt.
\begin{gather*}
\frac{|\mathit{P.targets}| = 1 \wedge \forall \mathit{T'} \in \mathit{P.targets}: T = T'}{\mathit{targets_{single}(P,T)}}
\end{gather*}
Darüber hinaus enthält ein \emph{Sub-Proxy} $P$ eine bestimmte Menge von Methoden-Delegationen. Dabei muss in allen Methodendelegationen das Attribut $\texttt{field}$ der aufgerufenen Methoden mit dem der Delegationsmethoden übereinstimmen. Folgende Regel stellt diesen Sachverhalt für eine Menge von Methoden-Delegationen $\mathit{MDList}$ dar.
\begin{gather*}
\frac{\splitfrac{\forall \mathit{MD_1}\in \mathit{MDList}: \neg(\exists \mathit{MD_2} \in \mathit{MDList}:\mathit{MD_1.call.field} \neq \mathit{MD_2.call.field}}{ \vee \mathit{MD_1.del.field} \neq \mathit{MD_2.del.field} )}}
{\mathit{equalRefs(MDList)}}
\end{gather*}
Für jede einzelne Methoden-Delegation $\mathit{MD}$ gilt weiterhin, dass die aufgerufene Methode und die Delegationsmethode denselben Namen haben.
\begin{gather*}
\frac{\mathit{MD.call.name} = \mathit{MD.del.name}}
{\mathit{methDel_{nominal}(MD)}}
\end{gather*}
Die aufgerufene Methode muss dabei generell im Typ aus dem Attribut $\texttt{call.delType}$ deklariert sein und die Delegationsmethode im Typ aus dem Attribut $\texttt{del.delType}$.
\begin{gather*}
\frac{\exists \mathit{T'\text{ } m(T)} \in \mathit{methoden(MD.call.delType)}: \mathit{MD.call.name} = m}
{\mathit{callMethod_{simple}(MD)}}
\end{gather*}
\begin{gather*}
\frac{\exists \mathit{T'\text{ }m(T)} \in \mathit{methoden(MD.del.delType)}: \mathit{MD.del.name} = m}
{\mathit{delMethod_{simple}(MD)}}
\end{gather*}
Zusätzlich muss das Attribut $\texttt{field}$ im Attribut $\texttt{call}$ mit dem Wert $\texttt{self}$ belegt und das Attribut $\texttt{mainType}$ mit dem Source-Typ des Proxies belegt sein.
\begin{gather*}
\frac{\mathit{MD.call.mainType} = \mathit{P.type} \wedge \mathit{MD.call.field} = \mathit{self}}
{\mathit{callMethodDelType_{simple}(MD, P)}}
\end{gather*}
Damit ist auch automatisch gewährleistet, dass die Attribute $\texttt{mainType}$ und $\texttt{delType}$ im Attribut $\texttt{call}$ übereinstimmen. (siehe Tabelle \ref{tab:attrGrProxies})\\\\
Ähnliches gilt für die Attribute $\texttt{field}$ und $\texttt{mainType}$ im Attribut $\texttt{del}$. Hierbei muss der Wert des Attributs $\texttt{mainType}$ jedoch mit dem Target-Typ des Proxies übereinstimmen.
\begin{gather*}
\frac{\mathit{MD.del.field} = \mathit{self} \wedge  \mathit{MD.del.mainType} \in \mathit{P.targets} }
{\mathit{delMethodDelType_{simple}(MD, P)}}
\end{gather*}
Damit ist wiederum automatisch gewährleistet, dass die Attribute $\texttt{mainType}$ und $\texttt{delType}$ im Attribut $\texttt{del}$ übereinstimmen. (siehe Tabelle \ref{tab:attrGrProxies})\\\\
Die Regeln für die linke Seite einer Methoden-Delegation $\mathit{MD}$ innerhalb eines \emph{Sub-Proxies} $P$ können damit in folgender Regel zusammengefasst werden:
\begin{gather*}
\frac{\mathit{callMethod_{simple}(MD)} \wedge \mathit{callMethodDelType_{simple}(MD,P)}}
{\mathit{call_{simple}(MD,P)}}
\end{gather*}
Analog dazu können auch die Regeln für die rechte Seite einer Methoden-Delegation $\mathit{MD}$ innerhalb eines \emph{Sub-Proxies} $P$ zusammengefasst werden:
\begin{gather*}
\frac{\mathit{delMethod_{simple}(MD)} \wedge \mathit{delMethodDelType_{simple}(MD,P)}}
{\mathit{del_{simple}(MD,P)}}
\end{gather*}
Im \emph{Sub-Proxy} ist darüber hinaus noch die Methoden-Delegation zu beachten, die bei einem Aufruf zu einem Fehler führt. Dieser Fall wird für eine Methoden-Delegation $\mathit{MD}$ wie folgt beschrieben:
\begin{gather*}
\frac{\mathit{MD.del.name} = \mathit{none}}
{\mathit{del_{err}(MD)}}
\end{gather*}
Die genannten Regeln für eine Methoden-Delegation $\mathit{MD}$ in einem \emph{Sub-Proxy} lassen sich über die beiden folgenden Regeln beschreiben:
\begin{gather*}
\frac{\mathit{call_{simple}(MD,P)} \wedge \mathit{del_{simple}(MD,P) \wedge \mathit{methDel_{nominal}(MD)}}}
{\mathit{methDel_{sub}(MD,P)}}
\end{gather*}
\begin{gather*}
\frac{\mathit{call_{simple}(MD,P)}\wedge\mathit{del_{err}(MD)}
}
{\mathit{methDel_{sub}(MD,P)}}
\end{gather*}
Innerhalb eines \emph{Sub-Proxies} gibt es für jede Methode $m$ des Source-Typ genau eine Methoden-Delegation mit der Methode $m$ als aufgerufene Methode. Damit lässt sich für einen Proxy $P$ in Bezug auf alle seine Methoden-Delegationen folgende Regeln formulieren:
\begin{gather*}
\frac{\splitfrac{\mathit{M} = \mathit{methoden(P.type)}\wedge|\mathit{M}| = |P.dels| \wedge \forall \mathit{T'\text{ }m(T)} \in \mathit{M}:}{\exists \mathit{MD} \in \mathit{P.dels}:m = \mathit{MD.call.name} \wedge \mathit{methDel_{sub}(MD,P)
 }}
}
{\mathit{methDelList_{sub}(P)}}
\end{gather*}
Für einen Proxy $P$ kann die Regel $\mathit{equalRefs(P)}$ im Allgemeinen mit der Bedingung zusammengefasst werden, die besagt, dass ein Proxy immer einen bestimmten Source-Typ $S$ haben muss. Die zusammengefasste Regel lautet:
\begin{gather*}
\frac{\mathit{P.type} = \mathit{S} \wedge \mathit{equalRefs(P)}}{\mathit{proxy(P,S)}}
\end{gather*}
\noindent
Die Menge der \emph{Sub-Proxies}, die mit dem Source-Typ $T$ und dem Target-Typ $T'$ erzeugt werden, wird durch die folgende Funktion beschrieben.
\begin{gather*}
\mathit{proxies_{sub}(T,T')} := 
\left\{\begin{array}{l|l}
		& \mathit{proxy(P,T)}\wedge \mathit{ } \\
	P	& \mathit{targets_{single}(P,T')} \wedge \mathit{ } \\
		& \mathit{methDelList_{sub}(P)}
		 \end{array}
\right\}
\end{gather*}


\paragraph{Content-Proxy}
Die Voraussetzung für die Erzeugung eines \emph{Content-Proxies} vom Typ $T$ aus einem Target-Typ $T'$ ist $T \Rightarrow_{content} T'$. Damit ist der \emph{ContentTypeMatcher} der Basis-Matcher für den \emph{Content-Proxy}.
\paragraph{Beispiel} Als Beispiel sollen die Typen $\texttt{Medicine}$ und $\texttt{MedCabinet}$ verwendet werden, welche ein Matching der Form $\texttt{Medicine} \Rightarrow_{content} \texttt{MedCabinet}$ aufweisen. Daher kann ein \emph{Content-Proxy} für diese Konstellation erzeugt werden. Ein resultierender \emph{Content-Proxy} ist in folgendem Listing aufgeführt.
\begin{lstlisting}[style = dsl, caption = Content-Proxy für Medicine, captionpos = b]
proxy for Medicine with [MedCabinet]{
	Medicine.getDesciption():String --> MedCabinet.med.getDesciption():String
}
\end{lstlisting}
Durch die Methoden-Delegation dieses \emph{Content-Proxies} wird die Methode $\texttt{getDescription}$ an das Feld $\texttt{med}$ des Target-Typen $\texttt{MedCabniet}$ delegiert.\\\\
Der abstrakte Syntaxbaum mit den dazugehörigen Attributen ist Abbildung \ref{fig:ASTCONTENT} zu entnehmen. \footnote{Es wurden nur die Nonterminale mit den dazugehörigen Attributen aufgeführt.}
\begin{figure}[h!]
\centering
\includegraphics[width=0.5\linewidth]{AST_ContentExample}
\caption{AST für das Beispiel zum Content-Proxy}
\label{fig:ASTCONTENT}
\end{figure}
\noindent
\paragraph{Formalisierung}
Formal wird ein \emph{Content-Proxy} durch die Regeln beschrieben, die im Folgenden vorgestellt werden.\\\\
Ein \emph{Content-Proxy} enthält, wie auch der \emph{Sub-Proxy}, genau einen Target-Typ. Ebenfalls identisch zum \emph{Sub-Proxy} sind die Bedingungen hinsichtlich der aufgerufenen Methoden in den einzelnen Methoden-Delegationen.\\\\
In den Delegationsmethoden einer einzelnen Methoden-Delegation $\mathit{MD}$ dürfen die Attribute $\texttt{mainType}$ und $\texttt{delType}$ im \emph{Content-Proxy} nicht identisch sein. Dementsprechend darf das Attribut $\texttt{field}$ nicht mit dem Wert $\texttt{self}$ belegt sein. Vielmehr muss für das Attribut $\texttt{delTyp}$ und den Source-Typ $T$ des Proxies ein Matching der Form $T \Rightarrow_{internCont} \mathit{MD.del.delTyp}$ gelten. Daher gilt für den \emph{Content-Proxy} die folgende Regel:
\begin{gather*}
\frac{\mathit{P.type} \Rightarrow_{internCont} \mathit{MD.del.delType}  \wedge \mathit{MD.del.mainType} \in \mathit{P.targets}}
{\mathit{delMethodDelType_{content}(MD,P)}}
\end{gather*}
\noindent
Damit kann eine zusammenfassende Regel für die Delegationsmethoden einer Methoden-Delegation $\mathit{MD}$ wie folgt definiert werden:
\begin{gather*}
\frac{\mathit{delMethod_{simple}(MD)} \wedge \mathit{delMethodDelType_{content}(MD,P)}}
{\mathit{del_{content}(MD,P)}}
\end{gather*}
Die zusammenfassende Regel für eine einzelne Methoden-Delegation $\mathit{MD}$ innerhalb eines \emph{Content-Proxies} hat die folgende Form:
\begin{gather*}
\frac{\mathit{call_{simple}(MD,P)} \wedge \mathit{del_{content}(MD,P) \wedge \mathit{methDel_{nominal}(MD)}}}
{\mathit{methDel_{content}(MD,P)}}
\end{gather*}
Wie auch im \emph{Sub-Proxy} gibt es im \emph{Content-Proxy} für jede Methode $m$ des Source-Typen genau eine Methoden-Delegation mit der Methode $m$ als aufgerufene Methode. Daraus ergibt sich für alle Methoden-Delegationen aus einem \emph{Content-Proxy} $P$ folgende Regel:
\begin{gather*}
\frac{\splitfrac{M = \mathit{methoden(P.type) }\wedge|\mathit{M}| = |\mathit{P.dels}| \wedge \forall \mathit{T' \text{ }m(T)} \in \mathit{M}:}{ \exists \mathit{MD} \in \mathit{P.dels}:m = \mathit{MD.call.name} \wedge \mathit{methDel_{content}(MD,P)
 }
}}
{methDelList_{content}(P)}
\end{gather*}
Die Menge der \emph{Content-Proxies}, die mit dem Source-Typ $T$ und dem Target-Typ $T'$ erzeugt werden, wird durch die folgende Funktion beschrieben.
\begin{gather*}
\mathit{proxies_{content}(T,T')} := 
\left\{\begin{array}{l|l}
		& \mathit{proxy(P,T)} \wedge \mathit{ } \\
	P	& \mathit{targets_{single}(P,T')} \wedge \mathit{ }\\
		& \mathit{methDelList_{content}(P)} 
		 \end{array}
\right\}
\end{gather*}
\paragraph{Container-Proxy}
Die Voraussetzung für die Erzeugung eines \emph{Container-Proxies} vom Typ $T$ aus einem Target-Typ $T'$ ist $T \Rightarrow_{container} T'$. Damit ist der \emph{ContainerTypeMatcher} der Basis-Matcher für den \emph{Container-Proxy}.
\paragraph{Beispiel}
Als Beispiel werden wiederum die Typen $\texttt{Medicine}$ und $\texttt{MedCabinet}$ verwendet, welche ein Matching der Form $\texttt{MedCabinet} \Rightarrow_{container} \texttt{Medicine}$ aufweisen. Daher kann ein \emph{Content-Proxy} für diese Konstellation erzeugt werden. Ein resultierender \emph{Content-Proxy} ist in folgendem Listing aufgeführt.
\begin{lstlisting}[style = dsl, caption = Container-Proxy für MedCabniet, captionpos = b ]
proxy for MedCabinet with [Medicine]{
	MedCabinet.med.getDesciption():String --> Medicine.getDesciption():String
}
\end{lstlisting}
Durch die Methoden-Delegation dieses \emph{Container-Proxies} findet eine Delegation nur dann statt, wenn die Methoden $\texttt{getDescription}$ auf dem Feld $\texttt{med}$ des Source-Typ aufgerufen wird. Diese wird dann an den Target-Typen $\texttt{MedCabniet}$ delegiert.\\\\
Der abstrakte Syntaxbaum mit den dazugehörigen Attributen ist Abbildung \ref{fig:ASTCONTAINER} zu entnehmen. \footnote{Es wurden nur die Nonterminale mit den dazugehörigen Attributen aufgeführt.}
\begin{figure}[h!]
\centering
\includegraphics[width=0.5\linewidth]{AST_ContainerExample}
\caption{AST für das Beispiel zum Container-Proxy}
\label{fig:ASTCONTAINER}
\end{figure}
\noindent
\paragraph{Formalisierung}
Formal wird ein \emph{Container-Proxy} durch die Regeln beschrieben, die im Folgenden vorgestellt werden.\\\\
Ein \emph{Container-Proxy} enthält, wie die vorher beschriebenen Proxies, genau einen Target-Typ. Die Eigenschaften der Delegationsmethoden innerhalb der einzelnen Methoden-Delegationen gleichen denen aus dem \emph{Sub-Proxy}.\\\\
In den angerufenen Methoden einer einzelnen Methoden-Delegation $\mathit{MD}$ dürfen die Attribute $\texttt{mainType}$ und $\texttt{delType}$ im \emph{Container-Proxy} nicht übereinstimmen. Dementsprechend darf das Attribut $\texttt{field}$ nicht mit dem Wert $\texttt{self}$ belegt sein. Vielmehr müssen der Wert des Attributs $\texttt{delTyp}$ und der Target-Typ $T$ des Proxies ein Matching der Form $T \Rightarrow_{internCont} \texttt{delTyp}$ ausweisen. Daher gilt für den \emph{Container-Proxy} $P$ folgende Regel.
\begin{gather*}
\frac{\splitfrac{\mathit{MD.call.mainType} = \mathit{P.type} \wedge \forall \mathit{T} \in \mathit{P.targets}:}
{  \mathit{T} \Rightarrow_{internCont} \mathit{MD.call.delType}}
}
{\mathit{callMethodDelType_{container}(MD,P)}}
\end{gather*}
\noindent
Damit kann eine zusammenfassende Regel für die aufgerufenen Methoden wie folgt definiert werden:
\begin{gather*}
\frac{\mathit{callMethod_{simple}(MD)} \wedge \mathit{callMethodDelType_{container}(MD,P)}}
{\mathit{call_{container}(MD,P)}}
\end{gather*}
Die zusammenfassende Regel für eine einzelne Methoden-Delegation $\mathit{MD}$ innerhalb eines \emph{Container-Proxies} hat die folgende Form:
\begin{gather*}
\frac{\mathit{call_{container}(MD,P)} \wedge \mathit{del_{simple}(MD,P)} \wedge \mathit{methDel_{nominal}(MD)}}
{\mathit{methDel_{container}(MD,P)}}
\end{gather*}
Für einen \emph{Container-Proxy} $P$ gilt ebenfalls die Regel $\mathit{equalRefs(P.dels)}$. Daher müssen die Werte des Attributs $\texttt{call.delType}$ aller Methoden-Delegationen des Proxies $P$ übereinstimmen. Ferner muss es für jede Methode $m$ des Typen aus $\texttt{call.delType}$ genau eine Methoden-Delegation mit der Methode $m$ als aufgerufene Methode existieren. Daraus ergibt sich für alle Methoden-Delegationen aus einem \emph{Content-Proxy} $P$ folgende Regel:
\begin{gather*}
\frac{\splitfrac{\mathit{M} = \mathit{methoden(P.dels[0].call.delType)} \wedge |\mathit{M}| = |P.dels| \wedge \forall \mathit{T' \text{ } m(T)} \in \mathit{M}:}
{\exists \mathit{MD} \in P.dels:m = \mathit{MD.call.name} \wedge \mathit{methDel_{container}(MD,P)}
 }}
{\mathit{methDelList_{container}(P)}}
\end{gather*}
Die Menge der \emph{Container-Proxies}, die mit dem Source-Typ $T$ und dem Target-Typ $T'$ erzeugt werden, wird durch die folgende Funktion beschrieben.
\begin{gather*}
\mathit{proxies_{container}(T,T')} := 
\left\{\begin{array}{l|l}
		& \mathit{proxy(P,T)}  \wedge \mathit{ } \\
	P	& \mathit{target_{single}(P,T')} \wedge \mathit{ } \\
		& \mathit{methDelList_{container}(P)} 
		 \end{array}
\right\}
\end{gather*}

\paragraph{Struktureller Proxy}
Die Voraussetzung für die Erzeugung eines \emph{strukturellen Proxies} vom \emph{required Typ} $R$ aus einem Target-Typ $T$ ist $R \Rightarrow_{struct} T$. Damit ist der \emph{StructuralTypeMatcher} der Basis-Matcher für den \emph{strukturellen Proxy}.\\\\
Der \emph{strukturelle Proxy} ist der einzige Proxy, der mit mehreren Target-Typen erzeugt werden kann. 
\paragraph{Beispiel}
Als Beispiel werden die Typen $\texttt{MedicalFireFighter}$, $\texttt{Doctor}$ und $\texttt{FireFighter}$ verwendet. Dabei ist $\texttt{MedicalFireFighter}$ der Source-Typ des Proxies und die Menge der anderen beiden Typen bilden die Target-Typen des Proxies. Da der Source-Typ zu den Target-Typen ein Matching der Form $\texttt{MedicalFireFighter} \Rightarrow_{struct} \texttt{FireFighter}$ bzw. $\texttt{MedicalFireFighter} \Rightarrow_{struct} \texttt{Doctor}$ aufweist, kann ein \emph{struktureller Proxy} erzeugt werden. Ein solcher ist in folgendem Listing aufgeführt.
\begin{lstlisting}[style = dsl, caption = Struktureller Proxy für MedicalFireFighter, captionpos = b]
proxy for MedicalFireFighter with [Doctor, FireFighter]{
	MedicalFireFighter.heal(Patient, MedCabinet):void --> Doctor.heal(Patient, Medicine):void
	MedicalFireFighter.extinguishFire(ExtFire):boolean --> FireFighter.extinguishFire(Fire):FireState
}
\end{lstlisting}
In diesem Beispiel wird der Methodenaufruf der Methode $\texttt{heal}$ auf dem Proxy an die Methode $heal$ des Typs $\texttt{Doctor}$ delegiert. Analog dazu würde ein Aufruf der Methode $\texttt{extinguishFire}$ auf dem Proxy an die Methode $extinguishFire$ des Typs $\texttt{FireFighter}$ delegiert werden. Die Methoden stimmen jeweils strukturell überein.\\\\
Der abstrakte Syntaxbaum mit den dazugehörigen Attributen ist Abbildung \ref{fig:ASTSTRUCT} zu entnehmen. \footnote{Es wurden nur die Nonterminale mit den dazugehörigen Attributen aufgeführt.}
\begin{figure}[h!]
\centering
\includegraphics[width=\linewidth]{AST_StructExample}
\caption{AST für das Beispiel zum strukturellen Proxy}
\label{fig:ASTSTRUCT}
\end{figure}
\noindent
\paragraph{Formalisierung}
Ein \emph{struktureller Proxy} wird formal durch die folgenden Regeln beschrieben.\\\\
Ein \emph{struktureller Proxy} kann, wie bereits erwähnt, mehrere Target-Typen enthalten.
Für jeden Target-Typ $T$ muss dabei jedoch wenigstens eine Delegationsmethode im Proxy mit einem Attribut $\texttt{target} = T$ existiert. Dadurch gilt die für einen \emph{strukturellen Proxy} Proxy $P$:
\begin{gather*}
\frac{\forall \mathit{T} \in \mathit{P.targets}:\exists \mathit{MD} \in \mathtt{P.dels}:\mathit{MD.del.target} = T}{\mathit{targets_{struct}(P, T)}}
\end{gather*}
Für die aufgerufene Methode und die Delegationsmethode einer einzelnen Methoden-Delegation $\mathit{M}$ gelten im \emph{strukturellen Proxy} dieselben Regeln wie für den \emph{Sub-Proxy}. Die Namen der aufgerufenen Methode und der Delegationsmethode müssen dabei jedoch nicht übereinstimmen. Dafür müssen diese beiden Methode jedoch ein strukturelles Matching aufweisen. Bezogen auf die Rückgabe-Typen einer aufgerufenen Methode $\mathit{C}$ und der Delegationsmethode $\mathit{D}$ aus einer Methoden-Delegation muss daher Folgendes gelten.
\begin{gather*}
\frac{\mathit{D.returnType} \Rightarrow_{internStruct} \mathit{C.returnType}}{\mathit{return_{struct}(C,D)}}
\end{gather*} 
Weiterhin muss für die Parameter-Typen gelten:
\begin{gather*}
\frac{\mathit{C.paramCount} = 0}{\mathit{params_{struct}(C,D)}}
\end{gather*} 
\begin{gather*}
\frac{\splitfrac{\forall \mathit{i} \in \{0,...,\mathit{C.paramCount}-1\}:}
{ \mathit{C.paramTypes}[i] \Rightarrow_{internStruct} \mathit{D.paramTypes}[\mathit{D.posModi}[i]]
}}{\mathit{params_{struct}(C,D)}}
\end{gather*} 
Für eine einzelne Methoden-Delegation $\mathit{MD}$ eines \emph{strukturellen Proxies} $P$ kann dann folgende Regel aufgestellt werden.
\begin{gather*}
\frac{\splitfrac{\mathit{call_{simple}(MD,P)} \wedge \mathit{del_{simple}(MD,P)} \wedge} {\mathit{return_{struct}(MD.call, MD.del)} \wedge \mathit{params_{struct}(MD.call, MD.del)}}
}
{\mathit{methDel_{struct}(MD,P)}}
\end{gather*}
In einem \emph{strukturellen Proxy} muss für jede Methode $m$ des Source-Typen genau eine Methoden-Delegation mit der Methode $m$ als aufgerufene Methode existieren. Daraus ergibt sich für alle Methoden-Delegationen aus einem \emph{strukturellen Proxy} $P$ folgende Regel:
\begin{gather*}
\frac{\splitfrac{\mathit{M} = \mathit{methoden(P.type)} \wedge |\mathit{M}| = |\mathit{P.dels}| \wedge \forall \mathit{T' \text{ }m(T)} \in \mathit{M}:}{\exists \mathit{MD} \in \mathit{P.dels}:\mathit{MD.call.name} = m \wedge \mathit{methDel_{struct}(MD,P)}
 }
}
{\mathit{methDelList_{struct}(P)}}
\end{gather*}
Wie in Abschnitt 
Die Menge der \emph{strukturellen Proxies}, die mit dem Source-Typ $R$ und der Menge von Target-Typen $T$ erzeugt werden, wird durch die folgende Funktion beschrieben.
\begin{gather*}
\mathit{proxies_{struct}(R,T)} := 
\left\{\begin{array}{l|l}
		& \mathit{proxy(P,R)}\wedge \mathit{ }\\
	P	& \mathit{targets_{struct}(P,T)} \wedge \mathit{ }\\
		& \mathit{methDelList_{struct}(P)}  
		 \end{array}
\right\}
\end{gather*}

\paragraph{Allgemeine Generierung von Proxies}
Die Proxy-Funktion der einzelnen Proxy-Arten werden zur Beschreibung einer allgemeine Funktion für die Generierung der Proxies verwendet. Dazu sind die Proxy-Arten zusammen mit den dazugehörigen Matchingrelationen und Proxy-Fukntionen in Tabelle \ref{tab:baseMatcher} noch einmal aufgeführt.

\begin{table}[H]
\centering
\begin{tabular}{|c|c|c|}
\hline
\hline
\centering\textbf{Proxy-Art} & \textbf{Matchingrelation} & \textbf{Funktionsname}\\
\hline
\hline
Sub-Proxy
&  
$\Rightarrow_{spec}$
& 
$\mathit{proxies_{sub}}$
\\
\hline
Content-Proxy
& 
$\Rightarrow_{content}$
& 
$\mathit{proxies_{content}}$
\\
\hline
Container-Proxy
& 
$\Rightarrow_{container}$
& 
$\mathit{proxies_{container}}$
\\
\hline
struktureller Proxy
&
$\Rightarrow_{struct}$
& 
$\mathit{proxies_{struct}}$
 \\
\hline
\hline
\end{tabular}
\caption{Proxy-Arten mit Matchingrelationen und Proxy-Funktionen}
 \label{tab:baseMatcher}
\end{table}
\noindent
Die im Abschnitt \ref{sec:proxygram} erwähnte Funktion $\mathit{proxies(S,T)}$ kann darauf aufbauend für einen Source-Typ $S$ und eine Menge von Target-Typen $T$ wie folgt beschrieben werden.
\begin{gather*}
\mathit{proxies(S,T)} := 
\left\{\begin{array}{ll}
\mathit{proxy_{sub}(S,T)}	& \text{wenn } |T| = 1 \wedge \mathit{ }\\
& \forall T' \in T: S \Rightarrow_{sub} T'\\	
&\\
\mathit{proxy_{content}(S,T)}	& \text{wenn } |T| = 1 \wedge \mathit{ }\\
& \forall T' \in T: S \Rightarrow_{content} T' \\
&\\
\mathit{proxy_{container}(S,T)} & \text{wenn } |T| = 1 \wedge \mathit{ } \\
& \forall T' \in T: S \Rightarrow_{container} T' \\
&\\
\mathit{proxy_{struct}(S,T)} & \text{wenn } |T| > 0 \wedge \mathit{ } \\
&\forall T' \in T: S \Rightarrow_{struct} T'
		 \end{array}
\right\}
\end{gather*}
\subsubsection{Anzahl möglicher Proxies innerhalb einer Bibliothek}
Innerhalb einer Bibliothek $L$ kann für einen required Typ $R$ mitunter eine Vielzahl von \emph{Proxies} erzeugt werden. 
Die folgende Funktion $\mathit{cover}$ beschreibt eine Menge von Mengen von provided Typen aus der Bibliothek $L$, die für die Erzeugung eines Proxies für $R$ verwendet werden können.
\begin{gather*}
cover(R,L) := 
\left\{\begin{array}{l|l}
					& T_1 \in L \wedge \text{...} \wedge T_n \in L 								\wedge \mathit{ }\\
\{T_1,...,T_n\}		& \mathit{methoden(R)} = \mathit{structM(R,T_1)}							\cup \mathit{ }\\
					& \texttt{...} \cup \mathit{structM(R, T_n)} 								\wedge \mathit{ }\\
					& \forall T \in \{T_1,...,T_n\}:											\mathit{structM(R,T)} \neq \emptyset
\end{array}\right\}
\end{gather*}
\noindent



\paragraph{Beispiel}
Sei folgende Bibliothek $L$ gegeben.
\begin{lstlisting}[style = dsl]
provided Come extends Object{
	String hello()
	String goodMorning()
}

provided Leave extends Object{
	String bye()
}

required Greeting{
	String hello()
	String bye()
}
\end{lstlisting}
Über die Funktion $\mathit{cover}$ werden folgenden Mengen von Target-Typen für die Bildung von Proxies für den required Typ $\texttt{Greeting}$ ermittelt.
\begin{gather*}
\mathit{cover(\texttt{Greeting},L)} = \{
	\{\texttt{Come}\},\{\texttt{Leave}, \texttt{Come}\}
\}
\end{gather*}
\noindent
Mit einer Menge $T \in \mathit{cover(R,L)}$ können durchaus mehrere Proxies erzeugt werden. Das ist dann der Fall, wenn mehrere der Methoden, die in den provided Typen aus $T$ deklariert wurden, mit einer Methode des required Typs $R$ strukturell übereinstimmen.
Die Anzahl der möglichen Proxies für ein required Typ $R$ mit einer bestimmten Mengen von Target-Typen $T_1,...,T_k$ ist somit von der Anzahl der Methoden abhängig, die in einem der Target-Typen des Proxies deklariert wurden und strukturell mit den Methoden aus $R$ übereinstimmen. 
\\\\
Die Menge der Methoden der provided Typen aus einer Menge $T$, die strukturell mit einer Methoden mit der Struktur $\methodForm{A}{m}{P}$ übereinstimmen, wird über die Funktion $\mathit{structM_{target}}$ beschrieben.
\begin{gather*}
\mathit{structM_{target}(\methodForm{A}{m}{P}, T)} := 
\left\{\begin{array}{l|l}
						& \exists \mathit{T_i} \in T :\\
\methodForm{A'}{n}{P'}	& \methodForm{A'}{n}{P'} 													\in	\mathit{methoden(T_i)} \wedge 
							\mathit{ }\\
						& P' \Rightarrow_{internStruct} P \wedge 									\mathit{ } \\
						& A \Rightarrow_{internStruct} A' 
\end{array}\right\}
\end{gather*}
\noindent
Sei $R$ ein required Typ und $T$ eine Menge von provided Typen innerhalb einer Bibliothek $L$ mit $T \in \mathit{cover(R,L)}$. Sei weiterhin $\{m_1,...,m_n\} = \mathit{methoden(R)}$.
Dann bilden $M_1,...,M_n$ wie folgt die Mengen der Methoden der Target-Typen in $T$, die mit jeweils einer Methode $m_i \in \mathit{methoden(R)}$  strukturell übereinstimmen.
\begin{gather*}
M_1 = \mathit{structM_{target}(m_1,T)}\\
...\\
M_n = \mathit{structM_{target}(m_n,T)}
\end{gather*}
Für jede Kombination von jeweils einem Element aus jeder der Mengen $M_1,...M_n$ kann ein Proxy für $R$ mit der Menge der Target-Typen $T$ erzeugt werden.

\paragraph{Beispiel}
Aufbauend auf dem vorherigen Beispiel ergeben sich für die Menge der Target-Typen  $\{\texttt{Leave}, \texttt{Come}\}$ und die beiden Methoden des required Typs $\texttt{Greeting}$ folgende Menge von übereinstimmenden Methoden über die Funktion $\mathit{structM_{target}}$:
\begin{gather*}
\mathit{structM_{target}(\methodForm{String}{hello}{},\{\texttt{Leave}, \texttt{Come}\})} = 
\left\{
\begin{array}{l}
\methodForm{String}{hello}{},\\
\methodForm{String}{goodMorning}{},\\
\methodForm{String}{bye}{}
\end{array}
\right\}\\
\mathit{structM_{target}(\methodForm{String}{bye}{},\{\texttt{Leave}, \texttt{Come}\})} = 
\left\{
\begin{array}{l}
\methodForm{String}{hello}{},\\
\methodForm{String}{goodMorning}{},\\
\methodForm{String}{bye}{}
\end{array}
\right\}
\end{gather*}
\noindent
Darauch aufbauend lassen sich die folgenden vier Proxies mit den Target-Typen $\texttt{Leave}$ und $\texttt{Come}$ erzeugen.
\begin{lstlisting}[style = dsl]
proxy Greeting with [Come, Leave]{
	Greeting.hello():String --> Come.hello():String
	Greeting.bye():String --> Leave.bye():String
}
\end{lstlisting}
\begin{lstlisting}[style = dsl]
proxy Greeting with [Come, Leave]{
	Greeting.hello():String --> Come.goodMorning():String
	Greeting.bye():String --> Leave.bye():String
}
\end{lstlisting}
\begin{lstlisting}[style = dsl]
proxy Greeting with [Come, Leave]{
	Greeting.hello():String --> Leave.bye():String
	Greeting.bye():String --> Come.hello():String
}
\end{lstlisting}
\begin{lstlisting}[style = dsl]
proxy Greeting with [Come, Leave]{
	Greeting.hello():String --> Leave.bye():String
	Greeting.bye():String --> Come.goodMorning():String
}
\end{lstlisting}
\noindent
\\\\
Für die Bildung eines Proxies wird aus jeder der oben genannten Menge $M_1,...,M_n$ genau ein Element als Delegationsmethode verwendet werden. Die Anzahl aller möglichen Proxies für ein required Typ $R$ aus einer Menge von Target-Typen $T$ und unter der Annahme, dass $\{m_1,...,m_n\} = \mathit{methoden(R)}$, sei über die Funktion $\mathit{proxyCount(R,T)}$ ausgedrückt. Für $\mathit{proxyCount(R,T)}$ ist zu beachten, dass es sich dabei lediglich um eine Annäherung an die tatsächliche Anzahl der Proxies handelt, die unter den oben beschriebenen Bedingungen erzeugt werden können. Dies liegt daran, dass eine Delegationsmethoden $dm \in M_1 \cup ... \cup M_n$ innerhalb eines Proxy maximal einmal verwendet werden darf. Es ist jedoch möglich, dass es zwischen den oben genannten Mengen 
$M_1,...,M_n$ Überschneidungen gibt (siehe vorheriges Beispiel). Daher gelten für die Funktion $\mathit{proxyCount}$ folgende Regeln unter den oben genannten Modalitäten:
\begin{gather*}
\frac{M_1 \cap ... \cap M_n = \emptyset}{\mathit{proxyCount(R,T)} = \prod\limits_{i=1}^{n}|M_i| }
\\\\
\frac{M_1 \cap ... \cap M_n \neq \emptyset}{\mathit{proxyCount(R,T)} < \prod\limits_{i=1}^{n}|M_i| }
\end{gather*}
\noindent
Im Allgemeinen gilt demnach:
\begin{gather*}
\mathit{proxyCount(R,T)} \leq 
\begin{array}{l|l}
\prod\limits_{i=1}^{n}|\mathit{structM_{target}(m_i, T)}|
&
\left\{
\begin{array}{l}
m_1,\\
...,\\
m_n
\end{array}
\right\}
= \mathit{methoden(R)}
\end{array}
\end{gather*}
Da innerhalb einer Bibliothek $L$ mehrere Mengen von Target-Typen zur Bildung eines Proxies für einen required Typ $R$ infrage kommen (siehe Funktion $\mathit{cover}$) muss die Anzahl der Proxies über die Funktion $\mathit{proxyCount}$ für alle Elemente aus $\mathit{cover(R,L)}$ ermittelt und summiert werden. Die folgende Funktion beschreibt diesen Sachverhalt für einen required Typ $R$ aus einer Bibliothek $L$.
\begin{gather*}
\mathit{libProxyCount(R,L)} = 
\begin{array}{l|l}
\sum_{i=1}^{n}\mathit{proxyCount(R,c_i)}
&
\left\{
\begin{array}{l}
c_1,\\
...,\\
c_n
\end{array}
\right\} = \mathit{cover(R,L)}
\end{array}
\end{gather*}


\section{Generierung der Proxies auf Basis von Matchern}\label{sec_proxygen}
Ein Proxy ist ein Objekt, das stellvertretend für ein anderes Objekt verwendet wird und den Zugang - in diesem Fall den Methodenaufruf - auf dieses Objekt kontrolliert (vgl. \cite{patterns}). Dadurch ist es dem Proxy möglich, die Methodenaufrufe an andere Objekte zu delegieren. Diese Eigenschaft wird sich in dieser Arbeit zunutze gemacht, um einen Aufruf einer Methode, die in einem \emph{required Typ} deklariert wurde, an eine Methode zu delegieren, die in einem \emph{provided Typ} deklariert wurde.
\\\\
Dabei wird zwischen einem \emph{Source-Typen} und einem oder mehrerer \emph{Target-Typen} unterschieden. Als \emph{Source-Typ} wird immer der Typ betrachtet, für den der Proxy generiert und stellvertretend eingesetzt wird. Bei den \emph{Target-Typen} handelt es sich um die provided Typen an deren Methoden die Methodenaufrufe delegiert werden.
\\\\
Zur Beschreibung der Generierung von Proxies wird im Folgenden zuerst vorgegeben, wie sich ein Proxy deklarieren lässt. Darauf aufbauend werden die Generatoren, die in Abhängigkeit des Matchings zwischen \emph{Source-} und \emph{Target-Typen} Anwendung finden, beschrieben.
\subsection{Struktur für die Definition von Proxies}\label{sec:proxygram}
Die Konvertierung eines Typs $T$ aus einer Menge von \emph{provided Typen} $P$ wird durch \emph{Proxies} beschrieben. Die Grammatikregeln für die Deklaration eines Proxies sind Tabelle \ref{tab:grProxies} zu entnehmen.
\begin{table}[H]
\centering
\begin{tabular}{|p{5cm}|p{9cm}|}
\hline
\hline
\centering\textbf{Regel} & \textbf{Erläuterung} \\
\hline
\hline
$\mathit{PROXY} ::=$\newline
$\texttt{proxy } \texttt{for } T$\newline
$ \texttt{with [}\mathit{P_1},...,\mathit{P_n}\texttt{]}$ \newline
$\texttt{\{}\mathit{MDEL_1},...,\mathit{MDEL_k} \texttt{\}}$
 & Ein Proxy wird für ein Typ $T$ als Source-Typ mit einer Mengen von provided Typen $P = \{P_1,...,P_n\}$ als Target-Typen, einer Menge von Methoden-Delegationen erzeugt.\\
\hline
$\mathit{MDEL} ::=$\newline
$CALLM \rightarrow DELM $  & Eine \emph{Methodendelegation} besteht aus einer \emph{aufgerufenen Methode} und aus einem \emph{Delegationsziel}.\\
\hline
$\mathit{CALLM} ::=$\newline 
$\mathit{REF}.\mathit{m(\mathit{CP_1},...,\mathit{CP_n}):CR} $  & Eine aufgerufene Methode besteht aus dem Namen der Methode $m$, dem Rückgabetyp $\mathit{CR}$ und einer Menge von Parametertypen $\{\mathit{CP_1},...,\mathit{CP_n}\}$.\\
\hline
$\mathit{DELM} ::=$\newline 
$\mathit{REF}.\mathit{n(\mathit{DP_1},...,\mathit{DP_n}):DR} $  
& Die erste Variante eines Delegationsziels besteht aus  dem Namen der \emph{Delegationsmethode} $n$, dem Rückgabetyp $\mathit{DR}$ und einer Menge von Parametertypen $\{\mathit{DP_1},...,\mathit{DP_n}\}$.\\
\hline
$\mathit{DELM} ::=$\newline
$\texttt{posModi(} \mathit{I_1},...,\mathit{I_n} \texttt{)}$\newline
$\mathit{REF}.\mathit{n(\mathit{DP_1},...,\mathit{DP_n}):DR} $  
& Die zweite Variante eines Delegationsziels besteht aus einer Menge von Indizies $\{\mathit{I_1},...,\mathit{I_n}\}$, einer \emph{Referenz}, dem Namen der Delegationsmethode $n$, dem Rückgabetyp $\mathit{DR}$ und einer Menge von Parametertypen $\{\mathit{DP_1},...,\mathit{DP_n}\}$.\\
\hline
$\mathit{DELM} ::= \texttt{err} $  
& Die dritte Variante eines Delegationsziels enthält keine weiteren Bestandteile. Das Terminal $\texttt{err}$ weist darauf hin, dass die Delegation innerhalb des Proxies nicht möglich ist und zu einem Fehler führt.\\
\hline
$\mathit{REF} ::= \mathit{P_i}$
& Die erste Variante einer Referenz besteht aus einem Typ $P_i$ .\\
\hline
$\mathit{REF} ::= \mathit{P_i}\texttt{.}\mathit{f}$
& Die zweite Variante einer Referenz besteht aus einem Typ $P_i$ und einem Feldnamen $f$.\\
\hline
\end{tabular}
\caption{Grammatikregeln mit Erläuterungen für die Deklaration eines Proxies}
 \label{tab:grProxies}
\end{table}
\noindent
Es handelt sich dabei um Produktionsregeln einer Attributgrammatik. Die dazugehörigen Attribute sind der Tabelle \ref{tab:attrGrProxies} zu entnehmen. Dazu sei zusätzlich festgelegt, dass die Notation $\mathit{NT}\texttt{.}\text{*}$ in der Spalte \emph{Attribute} eine Key-Value-Liste aller Attribute des Nonterminals $\mathit{NT}$ beschreibt, wobei der Attributname als Key und dessen Wert als Value innerhalb der Liste verwendet wird. Weiterhin sei ein Attribut, dass in der Spalte \emph{Attribute} zu einem Nonterminal nicht aufgeführt ist, wird mit dem Wert \emph{none} belegt.
\begin{table}[h!]
\centering
\begin{tabular}{|p{6cm}|p{8cm}|}
\hline
\hline
\centering\textbf{Regel} & \textbf{Attribute} \\
\hline
\hline
$\mathit{PROXY} ::=$\newline
$\texttt{proxy } \texttt{for } T$\newline
$ \texttt{with [}\mathit{P_1},...,\mathit{P_n}\texttt{]}$ \newline
$\texttt{\{}\mathit{MDEL_1},...,\mathit{MDEL_k} \texttt{\}}$
& 
$\texttt{type} = \mathit{T}$\newline
$\texttt{targets} = [\mathit{P_1},...,\mathit{P_n}]$\newline
$\texttt{dels} = [\mathit{MDEL_1}\texttt{.}\text{*},...,\mathit{MDEL_k}\texttt{.}\text{*}]$
\\
\hline
$\mathit{MDEL} ::=$\newline
$\mathit{CALLM} \rightarrow \mathit{DELM} $  
& 
$\texttt{call} = \mathit{CALLM}.*$\newline
$\texttt{del} = \mathit{DELM}.*$
\\
\hline
$\mathit{CALLM} ::=$\newline 
$\mathit{REF}.\mathit{m(\mathit{CP_1},...,\mathit{CP_n}):CR}$
& 
$\texttt{source} = \mathit{REF.\texttt{mainType}}$\newline
$\texttt{delType} = \mathit{REF.\texttt{delType}}$\newline
$\texttt{name} = \mathit{m}$\newline
$\texttt{paramTypes} = \mathit{[CP_1},...,\mathit{CP_n]}$\newline
$\texttt{returnType} = \mathit{CR}$\newline
$\texttt{field} = \mathit{REF}\texttt{.field}$\newline
$\texttt{paramCount} = n$
\\
\hline
$\mathit{DELM} ::=$\newline 
$\mathit{REF}\texttt{.}n(\mathit{DP_1},...,\mathit{DP_n}):DR $  
&
$\texttt{target} = \mathit{REF}.\texttt{mainType}$\newline
$\texttt{delType} = \mathit{REF}.\texttt{delType}$\newline
$\texttt{posModi} = [0,...,\mathit{n}-1]$\newline
$\texttt{name} = \mathit{n}$\newline
$\texttt{paramTypes} = \mathit{[DP_1},...,\mathit{DP_n]}$\newline
$\texttt{returnType} = \mathit{DR}$\newline
$\texttt{field} = \mathit{REF}\texttt{.field}$
\\
\hline
$\mathit{DELM} ::=\texttt{posModi(} \mathit{I_1},...,\mathit{I_n} \texttt{)}$\newline
$\mathit{REF}\texttt{.}n(\mathit{DP_1},...,\mathit{DP_n}):DR $  
&
$\texttt{target} = \mathit{REF}.\texttt{mainType}$\newline
$\texttt{delType} = \mathit{REF}.\texttt{delType}$\newline
$\texttt{posModi} = \mathit{[I_1},...,\mathit{I_n]}$\newline
$\texttt{name} = \mathit{n}$\newline
$\texttt{paramTypes} = \mathit{[DP_1},...,\mathit{DP_n]}$\newline
$\texttt{returnType} = \mathit{DR}$\newline
$\texttt{field} = \mathit{REF}\texttt{.field}$
\\
\hline
$\mathit{DELM} ::= \texttt{err} $  
&
\\
\hline
$\mathit{REF} ::= \mathit{P}$
& 
$\texttt{mainType} = \mathit{P}$\newline
$\texttt{field} = \texttt{self}$\newline
$\texttt{delType} = \mathit{P}$
\\
\hline
$\mathit{REF} ::= \mathit{P}\texttt{.}\mathit{f}$
&
$\texttt{mainType} = \mathit{P}$\newline
$\texttt{field} = \mathit{f}$\newline
$\texttt{delType} = \mathit{feldTyp(f,P)}$
\\
\hline
\end{tabular}
\caption{Grammatikregeln mit Attributen für die Deklaration eines Proxies}
 \label{tab:attrGrProxies}
\end{table}
\noindent
Ein Proxy bietet alle Methoden des Source-Typen an. Einige dieser Methoden werden an eine Methode delegiert, die von einem Target-Typ des Proxies angeboten wird. Eine solche Delegation wird durch eine Methoden-Delegation (siehe Nontermial $\mathit{MDEL}$) definiert.
\paragraph{Beispiel} So beschreibt die folgende Methoden-Delegation, dass die Methode $\texttt{extinguishFire}$, die vom Source-Typ $\texttt{Patient}$ - und damit auch vom Proxy - angeboten wird, an die Methoden $\texttt{heal}$, die der Target-Typ $\texttt{Injured}$ anbietet, delegiert wird.
\begin{lstlisting}[style = dsl, caption = Einfache Methoden-Delegation, captionpos = b]
	Patient.heal(Medicine):void --> Injured.heal(Medicine):void
\end{lstlisting}
\noindent
Die Delegation einer aufgerufenen Methode an ein Delegationsziel, erfolgt in drei Schritten.
\begin{enumerate}
\item Parameterübergabe\\
Dabei werden die Parameter, mit denen die vom Proxy angebotene Methode aufgerufen wird, an die Delegationsmethode des Delegationsziels übergeben. Dabei sind zwei Dinge zu beachten. Zum Einen müssen die Typen der übergebenen Parameter zu den Typen der von der Delegationsmethode erwarteten Parameter passen. Zum Anderen muss die Reihenfolge, in der die Parameter übergeben wurden, an die erwartete Reihenfolge der Delegationsmethode angepasst werden.
\item Ausführung\\
Dieser Schritt meint die Durchführung der Delegationsmethode mit den übergeben Parametern aus Schritt 1. Dies schließt auch die Ermittlung des Rückgabewertes der Delegationsmethode ein.
\item Übergabe des Rückgabewertes\\
Ähnlich wie bei der Parameterübergabe, muss auch der Rückgabewert, der bei der Ausführung in Schritt 2 ermittelt wurde, an die aufgerufenen Methode, die vom Proxy angeboten wird, übergeben werden. Hier muss ebenfalls sichergestellt werden, dass die beiden Rückgabetypen der beiden Methoden zueinander passen.
\end{enumerate}
Die Delegation aus dem oben genannten Beispiel kann schematisch wie in Abbildung \ref{fig:DEL_heal} dargestellt werden. Die Übergabe der Parameter- und Rückgabewerte wird durch die gestrichelten Pfeile symbolisiert.
\begin{figure}[h!]
\includegraphics[width=\linewidth]{MDEL_heal}
\caption{Delegation der Methode $\texttt{heal}$}
\label{fig:DEL_heal}
\end{figure}
\noindent
In diesem Beispiel sind sowohl die Parameter- als auch die Rückgabe-Typen der aufgerufenen Methode und der Delegationsmethode identisch. Weiterhin spielt die Reihenfolge der Parameter in diesem Beispiel keine Rolle, da es nur einen Parameter gibt. Daher stellt die Übergabe der Parameter- und Rückgabewerte kein Problem dar.\\\\
Folgendes Beispiel soll zeigen, wie mit unterschiedlichen Reihenfolgen bzgl. der Parameter bei einer Methoden-Delegation umzugehen ist.
\paragraph{Beispiel} Die Methoden-Delegation aus Listing \ref{lst:methdel2} ist ein Beispiel für einen solchen Fall. Hier wird die aufgerufene Methode $\texttt{heal}$ mit den Parametern $\texttt{Patient}$ und $\texttt{MedCabinet}$ aus dem Typ $\texttt{PatientMedicalFireFighter}$ an die gleichnamige Methode aus dem Typ $\texttt{InverseDoctor}$ delegiert. Die Delegationsmethode verwendet zwar identische Parameter-Typen, aber die Reihenfolge, in der die Parameter übergeben werden, ist unterschiedlich.
\begin{lstlisting}[style = dsl, caption = Methoden-Delegation mit Parametern in unterschiedlicher Reihenfolge, captionpos = b]
	PatientMedicalFireFighter.heal(Patient, MedCabinet):void --> posModi(1,0)  InverseDoctor.heal(MedCabinet,Patient):void
\end{lstlisting}\label{lst:methdel2}
\noindent
Um die Reihenfolge der Parameter aus dem ursprünglichen Aufruf zu variieren, wird das Schlüsselwort $\texttt{posModi}$ verwendet. Dort werden eine Reihe von Indizes angegeben. Die Anzahl der angegebenen Indizes muss mit der Anzahl der Parameter übereinstimmen. Ein Index beschreibt die Position des in der aufgerufenen Methode angegebenen Parameter. Weiterhin spielt die Reihenfolge der Indizes eine wichtige Rolle. Diese ist mit der Reihenfolge der Parameter der Delegationsmethoden gleichzusetzen.
\\\\
So wird in dem o.g. Beispiel der erste Parameter der aufgerufenen Methoden (Index = 0) der Delegationsmethode als zweiter Parameter übergeben. Dementsprechend wird der zweite Parameter der aufgerufenen Methoden (Index = 1) der Delegationsmethode als erster Parameter übergeben (siehe Abbildung \ref{fig:DEL_healInverse}). 
\begin{figure}[H]
\includegraphics[width=\linewidth]{MDEL_healInverse}
\caption{Delegation der Methode $\texttt{heal}$ mit Parametern in unterschiedlicher Reihenfolge}
\label{fig:DEL_healInverse}
\end{figure}
\noindent
Ein weiteres Beispiel soll zeigen, wie mit übergebenen Typen umzugehen ist, die nicht ohne Probleme übergeben werden können. Dafür ist jedoch vorab zu klären, wann dies der Fall ist.
\\\\
Dass identische Typen keine Probleme bei der Übergabe zwischen aufgerufener Methode und Delegationsmethode darstellen, wurde in den oben genannten Beispielen gezeigt.\\\\
Darüber hinaus können Typen aber auch dann ohne Probleme übergeben werden, wenn sie sich aufgrund des Substitutionsprinzips austauschen lassen. Daher kann ein Typ $T$ anstelle eines Typs $T'$ verwendet werden, sofern $T \leq T'$ gilt.
\paragraph{Beispiel} In folgendem Listing ist eine Methoden-Delegation aufgerührt, bei der sowohl die Parameter- als auch die Rückgabe-Typen der aufgerufenen Methode und der Delegationsmethode nicht auf Basis des Substitionsprinzips übergeben werden können.
\begin{lstlisting}[style = dsl, caption = Methoden-Delegation mit Typkonvertierung, captionpos = b]
	MedicalFireFighter.extinguishFire(ExtFire):boolean --> FireFigher.extinguishFire(Fire):FireState
\end{lstlisting}\label{lst:methdel3}
\noindent
In einem solchen Fall müssen die Parameter-Typen der aufgerufenen Methoden in die Parameter-Typen der Delegationsmethode konvertiert werden. Analog dazu muss der Rückgabetyp der Delegationsmethode in den Rückgabetyp der aufgerufenen Methoden konvertiert werden.\\\\
Angenommen, die Funktion $\mathit{proxies(S,T)}$ beschreibt eine Menge von Proxies, mit $S$ als Source-Typ und $T$ als Menge der Target-Typen. Dann müssten bezogen auf die Methoden-Delegation aus Listing 4 für die Parameter-Typen einer der Proxies aus der Menge $\mathit{proxies(\texttt{Fire}, \{\texttt{ExtFire}\})}$ an die Delegationsmethode übergeben werden. Nach der Ausführung der Delegationsmethode müsste ein Proxy aus der Menge $\mathit{proxies(\texttt{boolean},\{\texttt{FireState}\})}$ an die aufgerufenen Methode als Rückgabetyp übergeben werden. Der Sachverhalt wird in Abbildung \ref{fig:DEL_extinguishFire} schematisch dargestellt.
\begin{figure}[H]
\includegraphics[width=\linewidth]{MDEL_extinguishFire}
\caption{Delegation der Methode $\texttt{extinguishFire}$ mit Typkonvertierungen}
\label{fig:DEL_extinguishFire}
\end{figure}
\noindent
Wie die Proxies generiert werden, wird im folgenden Abschnitt beschrieben.

\subsection{Generierung von Proxies}\label{sec_proxyGen}
Wie im Abschnitt \ref{sec:proxygram} bereits erwähnt, soll die Menge der Proxies für einen Source-Typ $S$ und einer Menge von Target-Typen $T$ über die Funktion $\mathit{proxies(S,T)}$ beschrieben werden.\\\\
In Abhängigkeit von dem Matching zwischen dem Source-Typ und den Target-Typen werden unterschiedliche Arten von Proxies generiert. Für die unterschiedlichen Proxy-Arten gibt es ebenfalls Funktionen, die eine Menge von Proxies zu einem Source-Typen $S$ und einer Menge von Target-Typen $T$ beschreiben.\\\\
In den folgenden Abschnitten werden diese Funktionen für die einzelnen Proxy-Arten beschrieben. Dabei ist davon auszugehen, dass die Proxies eine allgemeine Struktur haben, die in Abschnitt \ref{sec:proxygram} aufgeführt ist. Um die Regeln für die Generierung der Proxies zu beschreiben, soll davon ausgegangen werden, dass jedes Listen-Attribut ($\mathit{NT.}\text{*}$) aus Tabelle \ref{tab:attrGrProxies} ein Attribut $\texttt{len}$ enthält in dem die Anzahl der in der Liste befindlichen Elemente abgelegt ist.


\subsubsection{Sub-Proxy}
Die Voraussetzung für die Erzeugung eines \emph{Sub-Proxies} vom Typ $T$ aus einem Target-Typ $T'$ ist $T \Rightarrow_{spec} T'$. Damit ist der \emph{SpecTypeMatcher} der Basis-Matcher für den Sub-Proxy.
\paragraph{Beispiel}
Als Beispiel soll  der Typ $\texttt{Patient}$ als Source-Typ und der Typ $\texttt{Injured}$ als Target-Typ verwendet werden. Da $\texttt{Patient} \Rightarrow_{spec} \texttt{Injured}$ gilt, kann ein \emph{Sub-Proxy} für diese Konstellation erzeugt werden. Der resultierende \emph{Sub-Proxy} ist im folgenden Listing aufgeführt.
\begin{lstlisting}[style = dsl, caption = Sub-Proxy für Patient, captionpos = b]
proxy for Patient with [Injured]{
	Patient.heal(Medicine):void --> Injured.heal(Medicine):void
	Patient.getName():String --> err
}
\end{lstlisting}
Der abstrakte Syntaxbaum mit den dazugehörigen Attributen ist Abbildung \ref{fig:ASTSUB} zu entnehmen. \footnote{Es wurden nur die Nonterminale mit den dazugehörigen Attributen aufgeführt.}
\begin{figure}[h!]
\includegraphics[width=\linewidth]{AST_SubExample}
\caption{AST für das Beispiel zum Sub-Proxy}
\label{fig:ASTSUB}
\end{figure}
\noindent
Der Proxy bietet alle Methoden an, die auch von dessen Source-Typ angeboten werden. Die Methodendelegationen innerhalb des Proxies beschreiben, was beim Aufruf der jeweiligen aufgerufenen Methoden passiert. So wird ein Aufruf der Methode $\texttt{heal}$ an die Methode $\texttt{heal}$ aus dem Target-Typ delegiert. Ein Aufruf der Methode $\texttt{getName}$ hingegen führt zu einem Fehler, weil keine Delegationsmethode zur Verfügung steht.
\\\\
Im Hinblick darauf, dass eine Konvertierung von einem Super-Typ und einen Sub-Typ (Down-Cast) ebenfalls dazu führt, dass bestimmte Methoden, wie in diesem Fall $\texttt{getName}$ nicht ausgeführt werden können, spiegelt der \emph{Sub-Proxy} dieses Verhalten wieder.
\paragraph{Formalisierung}
Formal wird ein \emph{Sub-Proxy} durch die Regeln beschrieben, die im Folgenden vorgestellt werden. Ein \emph{Sub-Proxy} enthält genau einen Target-Typ. Für einen Proxy $P$ wird dieser Sachverhalt durch die folgende Regel dargestellt.
\begin{gather*}
\frac{|\mathit{P.targets}| = 1 \wedge \forall \mathit{T'} \in \mathit{P.targets}: T = T'}{\mathit{targets_{single}(P,T)}}
\end{gather*}
Darüber hinaus enthält ein \emph{Sub-Proxy} $P$ eine bestimmte Menge von Methoden-Delegationen. Dabei muss in allen Methodendelegationen das Attribut $\texttt{field}$ der aufgerufenen Methoden mit dem der Delegationsmethoden übereinstimmen. Folgende Regel stellt diesen Sachverhalt für eine Menge von Methoden-Delegationen $\mathit{MDList}$ dar.
\begin{gather*}
\frac{\splitfrac{\forall \mathit{MD_1}\in \mathit{MDList}: \neg(\exists \mathit{MD_2} \in \mathit{MDList}:\mathit{MD_1.call.field} \neq \mathit{MD_2.call.field}}{ \vee \mathit{MD_1.del.field} \neq \mathit{MD_2.del.field} )}}
{\mathit{equalRefs(MDList)}}
\end{gather*}
Für jede einzelne Methoden-Delegation $\mathit{MD}$ gilt weiterhin, dass die aufgerufene Methode und die Delegationsmethode denselben Namen haben.
\begin{gather*}
\frac{\mathit{MD.call.name} = \mathit{MD.del.name}}
{\mathit{methDel_{nominal}(MD)}}
\end{gather*}
Die aufgerufene Methode muss dabei generell im Typ aus dem Attribut $\texttt{call.delType}$ deklariert sein und die Delegationsmethode im Typ aus dem Attribut $\texttt{del.delType}$.
\begin{gather*}
\frac{\exists \mathit{T'\text{ } m(T)} \in \mathit{methoden(MD.call.delType)}: \mathit{MD.call.name} = m}
{\mathit{callMethod_{simple}(MD)}}
\end{gather*}
\begin{gather*}
\frac{\exists \mathit{T'\text{ }m(T)} \in \mathit{methoden(MD.del.delType)}: \mathit{MD.del.name} = m}
{\mathit{delMethod_{simple}(MD)}}
\end{gather*}
Zusätzlich muss das Attribut $\texttt{field}$ im Attribut $\texttt{call}$ mit dem Wert $\texttt{self}$ belegt und das Attribut $\texttt{mainType}$ mit dem Source-Typ des Proxies belegt sein.
\begin{gather*}
\frac{\mathit{MD.call.mainType} = \mathit{P.type} \wedge \mathit{MD.call.field} = \mathit{self}}
{\mathit{callMethodDelType_{simple}(MD, P)}}
\end{gather*}
Damit ist auch automatisch gewährleistet, dass die Attribute $\texttt{mainType}$ und $\texttt{delType}$ im Attribut $\texttt{call}$ übereinstimmen. (siehe Tabelle \ref{tab:attrGrProxies})\\\\
Ähnliches gilt für die Attribute $\texttt{field}$ und $\texttt{mainType}$ im Attribut $\texttt{del}$. Hierbei muss der Wert des Attributs $\texttt{mainType}$ jedoch mit dem Target-Typ des Proxies übereinstimmen.
\begin{gather*}
\frac{\mathit{MD.del.field} = \mathit{self} \wedge  \mathit{MD.del.mainType} \in \mathit{P.targets} }
{\mathit{delMethodDelType_{simple}(MD, P)}}
\end{gather*}
Damit ist wiederum automatisch gewährleistet, dass die Attribute $\texttt{mainType}$ und $\texttt{delType}$ im Attribut $\texttt{del}$ übereinstimmen. (siehe Tabelle \ref{tab:attrGrProxies})\\\\
Die Regeln für die linke Seite einer Methoden-Delegation $\mathit{MD}$ innerhalb eines \emph{Sub-Proxies} $P$ können damit in folgender Regel zusammengefasst werden:
\begin{gather*}
\frac{\mathit{callMethod_{simple}(MD)} \wedge \mathit{callMethodDelType_{simple}(MD,P)}}
{\mathit{call_{simple}(MD,P)}}
\end{gather*}
Analog dazu können auch die Regeln für die rechte Seite einer Methoden-Delegation $\mathit{MD}$ innerhalb eines \emph{Sub-Proxies} $P$ zusammengefasst werden:
\begin{gather*}
\frac{\mathit{delMethod_{simple}(MD)} \wedge \mathit{delMethodDelType_{simple}(MD,P)}}
{\mathit{del_{simple}(MD,P)}}
\end{gather*}
Im \emph{Sub-Proxy} ist darüber hinaus noch die Methoden-Delegation zu beachten, die bei einem Aufruf zu einem Fehler führt. Dieser Fall wird für eine Methoden-Delegation $\mathit{MD}$ wie folgt beschrieben:
\begin{gather*}
\frac{\mathit{MD.del.name} = \mathit{none}}
{\mathit{del_{err}(MD)}}
\end{gather*}
Die genannten Regeln für eine Methoden-Delegation $\mathit{MD}$ in einem \emph{Sub-Proxy} lassen sich über die beiden folgenden Regeln beschreiben:
\begin{gather*}
\frac{\mathit{call_{simple}(MD,P)} \wedge \mathit{del_{simple}(MD,P) \wedge \mathit{methDel_{nominal}(MD)}}}
{\mathit{methDel_{sub}(MD,P)}}
\end{gather*}
\begin{gather*}
\frac{\mathit{call_{simple}(MD,P)}\wedge\mathit{del_{err}(MD)}
}
{\mathit{methDel_{sub}(MD,P)}}
\end{gather*}
Innerhalb eines \emph{Sub-Proxies} gibt es für jede Methode $m$ des Source-Typ genau eine Methoden-Delegation mit der Methode $m$ als aufgerufene Methode. Damit lässt sich für einen Proxy $P$ in Bezug auf alle seine Methoden-Delegationen folgende Regeln formulieren:
\begin{gather*}
\frac{\splitfrac{\mathit{M} = \mathit{methoden(P.type)}\wedge|\mathit{M}| = |P.dels| \wedge \forall \mathit{T'\text{ }m(T)} \in \mathit{M}:}{\exists \mathit{MD} \in \mathit{P.dels}:m = \mathit{MD.call.name} \wedge \mathit{methDel_{sub}(MD,P)
 }}
}
{\mathit{methDelList_{sub}(P)}}
\end{gather*}
Für einen Proxy $P$ kann die Regel $\mathit{equalRefs(P)}$ im Allgemeinen mit der Bedingung zusammengefasst werden, die besagt, dass ein Proxy immer einen bestimmten Source-Typ $S$ haben muss. Die zusammengefasste Regel lautet:
\begin{gather*}
\frac{\mathit{P.type} = \mathit{S} \wedge \mathit{equalRefs(P)}}{\mathit{proxy(P,S)}}
\end{gather*}
\noindent
Die Menge der \emph{Sub-Proxies}, die mit dem Source-Typ $T$ und dem Target-Typ $T'$ erzeugt werden, wird durch die folgende Funktion beschrieben.
\begin{gather*}
\mathit{proxies_{sub}(T,T')} := 
\left\{\begin{array}{l|l}
		& \mathit{proxy(P,T)}\wedge \mathit{ } \\
	P	& \mathit{targets_{single}(P,T')} \wedge \mathit{ } \\
		& \mathit{methDelList_{sub}(P)}
		 \end{array}
\right\}
\end{gather*}


\subsubsection{Content-Proxy}
Die Voraussetzung für die Erzeugung eines \emph{Content-Proxies} vom Typ $T$ aus einem Target-Typ $T'$ ist $T \Rightarrow_{content} T'$. Damit ist der \emph{ContentTypeMatcher} der Basis-Matcher für den \emph{Content-Proxy}.
\paragraph{Beispiel} Als Beispiel sollen die Typen $\texttt{Medicine}$ und $\texttt{MedCabinet}$ verwendet werden, welche ein Matching der Form $\texttt{Medicine} \Rightarrow_{content} \texttt{MedCabinet}$ aufweisen. Daher kann ein \emph{Content-Proxy} für diese Konstellation erzeugt werden. Ein resultierender \emph{Content-Proxy} ist in folgendem Listing aufgeführt.
\begin{lstlisting}[style = dsl, caption = Content-Proxy für Medicine, captionpos = b]
proxy for Medicine with [MedCabinet]{
	Medicine.getDesciption():String --> MedCabinet.med.getDesciption():String
}
\end{lstlisting}
Durch die Methoden-Delegation dieses \emph{Content-Proxies} wird die Methode $\texttt{getDescription}$ an das Feld $\texttt{med}$ des Target-Typen $\texttt{MedCabniet}$ delegiert.\\\\
Der abstrakte Syntaxbaum mit den dazugehörigen Attributen ist Abbildung \ref{fig:ASTCONTENT} zu entnehmen. \footnote{Es wurden nur die Nonterminale mit den dazugehörigen Attributen aufgeführt.}
\begin{figure}[h!]
\centering
\includegraphics[width=0.5\linewidth]{AST_ContentExample}
\caption{AST für das Beispiel zum Content-Proxy}
\label{fig:ASTCONTENT}
\end{figure}
\noindent
\paragraph{Formalisierung}
Formal wird ein \emph{Content-Proxy} durch die Regeln beschrieben, die im Folgenden vorgestellt werden.\\\\
Ein \emph{Content-Proxy} enthält, wie auch der \emph{Sub-Proxy}, genau einen Target-Typ. Ebenfalls identisch zum \emph{Sub-Proxy} sind die Bedingungen hinsichtlich der aufgerufenen Methoden in den einzelnen Methoden-Delegationen.\\\\
In den Delegationsmethoden einer einzelnen Methoden-Delegation $\mathit{MD}$ dürfen die Attribute $\texttt{mainType}$ und $\texttt{delType}$ im \emph{Content-Proxy} nicht identisch sein. Dementsprechend darf das Attribut $\texttt{field}$ nicht mit dem Wert $\texttt{self}$ belegt sein. Vielmehr muss für das Attribut $\texttt{delTyp}$ und den Source-Typ $T$ des Proxies ein Matching der Form $T \Rightarrow_{internCont} \mathit{MD.del.delTyp}$ gelten. Daher gilt für den \emph{Content-Proxy} die folgende Regel:
\begin{gather*}
\frac{\mathit{P.type} \Rightarrow_{internCont} \mathit{MD.del.delType}  \wedge \mathit{MD.del.mainType} \in \mathit{P.targets}}
{\mathit{delMethodDelType_{content}(MD,P)}}
\end{gather*}
\noindent
Damit kann eine zusammenfassende Regel für die Delegationsmethoden einer Methoden-Delegation $\mathit{MD}$ wie folgt definiert werden:
\begin{gather*}
\frac{\mathit{delMethod_{simple}(MD)} \wedge \mathit{delMethodDelType_{content}(MD,P)}}
{\mathit{del_{content}(MD,P)}}
\end{gather*}
Die zusammenfassende Regel für eine einzelne Methoden-Delegation $\mathit{MD}$ innerhalb eines \emph{Content-Proxies} hat die folgende Form:
\begin{gather*}
\frac{\mathit{call_{simple}(MD,P)} \wedge \mathit{del_{content}(MD,P) \wedge \mathit{methDel_{nominal}(MD)}}}
{\mathit{methDel_{content}(MD,P)}}
\end{gather*}
Wie auch im \emph{Sub-Proxy} gibt es im \emph{Content-Proxy} für jede Methode $m$ des Source-Typen genau eine Methoden-Delegation mit der Methode $m$ als aufgerufene Methode. Daraus ergibt sich für alle Methoden-Delegationen aus einem \emph{Content-Proxy} $P$ folgende Regel:
\begin{gather*}
\frac{\splitfrac{M = \mathit{methoden(P.type) }\wedge|\mathit{M}| = |\mathit{P.dels}| \wedge \forall \mathit{T' \text{ }m(T)} \in \mathit{M}:}{ \exists \mathit{MD} \in \mathit{P.dels}:m = \mathit{MD.call.name} \wedge \mathit{methDel_{content}(MD,P)
 }
}}
{methDelList_{content}(P)}
\end{gather*}
Die Menge der \emph{Content-Proxies}, die mit dem Source-Typ $T$ und dem Target-Typ $T'$ erzeugt werden, wird durch die folgende Funktion beschrieben.
\begin{gather*}
\mathit{proxies_{content}(T,T')} := 
\left\{\begin{array}{l|l}
		& \mathit{proxy(P,T)} \wedge \mathit{ } \\
	P	& \mathit{targets_{single}(P,T')} \wedge \mathit{ }\\
		& \mathit{methDelList_{content}(P)} 
		 \end{array}
\right\}
\end{gather*}
\subsubsection{Container-Proxy}
Die Voraussetzung für die Erzeugung eines \emph{Container-Proxies} vom Typ $T$ aus einem Target-Typ $T'$ ist $T \Rightarrow_{container} T'$. Damit ist der \emph{ContainerTypeMatcher} der Basis-Matcher für den \emph{Container-Proxy}.
\paragraph{Beispiel}
Als Beispiel werden wiederum die Typen $\texttt{Medicine}$ und $\texttt{MedCabinet}$ verwendet, welche ein Matching der Form $\texttt{MedCabinet} \Rightarrow_{container} \texttt{Medicine}$ aufweisen. Daher kann ein \emph{Content-Proxy} für diese Konstellation erzeugt werden. Ein resultierender \emph{Content-Proxy} ist in folgendem Listing aufgeführt.
\begin{lstlisting}[style = dsl, caption = Container-Proxy für MedCabniet, captionpos = b ]
proxy for MedCabinet with [Medicine]{
	MedCabinet.med.getDesciption():String --> Medicine.getDesciption():String
}
\end{lstlisting}
Durch die Methoden-Delegation dieses \emph{Container-Proxies} findet eine Delegation nur dann statt, wenn die Methoden $\texttt{getDescription}$ auf dem Feld $\texttt{med}$ des Source-Typ aufgerufen wird. Diese wird dann an den Target-Typen $\texttt{MedCabniet}$ delegiert.
\\\\
Der abstrakte Syntaxbaum mit den dazugehörigen Attributen ist Abbildung \ref{fig:ASTCONTAINER} zu entnehmen. \footnote{Es wurden nur die Nonterminale mit den dazugehörigen Attributen aufgeführt.}
\begin{figure}[h!]
\centering
\includegraphics[width=0.5\linewidth]{AST_ContainerExample}
\caption{AST für das Beispiel zum Container-Proxy}
\label{fig:ASTCONTAINER}
\end{figure}
\noindent
\paragraph{Formalisierung}
Formal wird ein \emph{Container-Proxy} durch die Regeln beschrieben, die im Folgenden vorgestellt werden.
\\\\
Ein \emph{Container-Proxy} enthält, wie die vorher beschriebenen Proxies, genau einen Target-Typ. Die Eigenschaften der Delegationsmethoden innerhalb der einzelnen Methoden-Delegationen gleichen denen aus dem \emph{Sub-Proxy}.
\\\\
In den angerufenen Methoden einer einzelnen Methoden-Delegation $\mathit{MD}$ dürfen die Attribute $\texttt{mainType}$ und $\texttt{delType}$ im \emph{Container-Proxy} nicht übereinstimmen. Dementsprechend darf das Attribut $\texttt{field}$ nicht mit dem Wert $\texttt{self}$ belegt sein. Vielmehr müssen der Wert des Attributs $\texttt{delTyp}$ und der Target-Typ $T$ des Proxies ein Matching der Form $T \Rightarrow_{internCont} \texttt{delTyp}$ aufweisen. Daher gilt für den \emph{Container-Proxy} $P$ folgende Regel.
\begin{gather*}
\frac{\splitfrac{\mathit{MD.call.mainType} = \mathit{P.type} \wedge \forall \mathit{T} \in \mathit{P.targets}:}
{  \mathit{T} \Rightarrow_{internCont} \mathit{MD.call.delType}}
}
{\mathit{callMethodDelType_{container}(MD,P)}}
\end{gather*}
\noindent
Damit kann eine zusammenfassende Regel für die aufgerufenen Methoden wie folgt definiert werden:
\begin{gather*}
\frac{\mathit{callMethod_{simple}(MD)} \wedge \mathit{callMethodDelType_{container}(MD,P)}}
{\mathit{call_{container}(MD,P)}}
\end{gather*}
Die zusammenfassende Regel für eine einzelne Methoden-Delegation $\mathit{MD}$ innerhalb eines \emph{Container-Proxies} hat die folgende Form:
\begin{gather*}
\frac{\mathit{call_{container}(MD,P)} \wedge \mathit{del_{simple}(MD,P)} \wedge \mathit{methDel_{nominal}(MD)}}
{\mathit{methDel_{container}(MD,P)}}
\end{gather*}
Für einen \emph{Container-Proxy} $P$ gilt ebenfalls die Regel $\mathit{equalRefs(P.dels)}$. Daher müssen die Werte des Attributs $\texttt{call.delType}$ aller Methoden-Delegationen des Proxies $P$ übereinstimmen. Ferner muss es für jede Methode $m$ des Typen aus $\texttt{call.delType}$ genau eine Methoden-Delegation mit der Methode $m$ als aufgerufene Methode existieren. Daraus ergibt sich für alle Methoden-Delegationen aus einem \emph{Content-Proxy} $P$ folgende Regel:
\begin{gather*}
\frac{\splitfrac{\mathit{M} = \mathit{methoden(P.dels[0].call.delType)} \wedge |\mathit{M}| = |P.dels| \wedge \forall \mathit{T' \text{ } m(T)} \in \mathit{M}:}
{\exists \mathit{MD} \in P.dels:m = \mathit{MD.call.name} \wedge \mathit{methDel_{container}(MD,P)}
 }}
{\mathit{methDelList_{container}(P)}}
\end{gather*}
Die Menge der \emph{Container-Proxies}, die mit dem Source-Typ $T$ und dem Target-Typ $T'$ erzeugt werden, wird durch die folgende Funktion beschrieben.
\begin{gather*}
\mathit{proxies_{container}(T,T')} := 
\left\{\begin{array}{l|l}
		& \mathit{proxy(P,T)}  \wedge \mathit{ } \\
	P	& \mathit{target_{single}(P,T')} \wedge \mathit{ } \\
		& \mathit{methDelList_{container}(P)} 
		 \end{array}
\right\}
\end{gather*}

\subsubsection{Struktureller Proxy}
Die Voraussetzung für die Erzeugung eines \emph{strukturellen Proxies} vom \emph{required Typ} $R$ aus einem Target-Typ $T$ ist $R \Rightarrow_{struct} T$. Damit ist der \emph{StructuralTypeMatcher} der Basis-Matcher für den \emph{strukturellen Proxy}.
\\\\
Der \emph{strukturelle Proxy} ist der einzige Proxy, der mit mehreren Target-Typen erzeugt werden kann. 
\paragraph{Beispiel}
Als Beispiel werden die Typen $\texttt{MedicalFireFighter}$, $\texttt{Doctor}$ und $\texttt{FireFighter}$ verwendet. Dabei ist $\texttt{MedicalFireFighter}$ der Source-Typ des Proxies und die Menge der anderen beiden Typen bilden die Target-Typen des Proxies. Da der Source-Typ zu den Target-Typen ein Matching der Form $\texttt{MedicalFireFighter} \Rightarrow_{struct} \texttt{FireFighter}$ bzw. $\texttt{MedicalFireFighter} \Rightarrow_{struct} \texttt{Doctor}$ aufweist, kann ein \emph{struktureller Proxy} erzeugt werden. Ein solcher ist in folgendem Listing aufgeführt.
\begin{lstlisting}[style = dsl, caption = Struktureller Proxy für MedicalFireFighter, captionpos = b]
proxy for MedicalFireFighter with [Doctor, FireFighter]{
	MedicalFireFighter.heal(Patient, MedCabinet):void --> Doctor.heal(Patient, Medicine):void
	MedicalFireFighter.extinguishFire(ExtFire):boolean --> FireFighter.extinguishFire(Fire):FireState
}
\end{lstlisting}
In diesem Beispiel wird der Methodenaufruf der Methode $\texttt{heal}$ auf dem Proxy an die Methode $heal$ des Typs $\texttt{Doctor}$ delegiert. Analog dazu würde ein Aufruf der Methode $\texttt{extinguishFire}$ auf dem Proxy an die Methode $extinguishFire$ des Typs $\texttt{FireFighter}$ delegiert werden. Die Methoden stimmen jeweils strukturell überein.
\\\\
Der abstrakte Syntaxbaum mit den dazugehörigen Attributen ist Abbildung \ref{fig:ASTSTRUCT} zu entnehmen. \footnote{Es wurden nur die Nonterminale mit den dazugehörigen Attributen aufgeführt.}
\begin{figure}[h!]
\centering
\includegraphics[width=\linewidth]{AST_StructExample}
\caption{AST für das Beispiel zum strukturellen Proxy}
\label{fig:ASTSTRUCT}
\end{figure}
\noindent
\paragraph{Formalisierung}
Ein \emph{struktureller Proxy} wird formal durch die folgenden Regeln beschrieben.
\\\\
Ein \emph{struktureller Proxy} kann, wie bereits erwähnt, mehrere Target-Typen enthalten.
Für jeden Target-Typ $T$ muss dabei jedoch wenigstens eine Delegationsmethode im Proxy mit einem Attribut $\texttt{target} = T$ existieren. Dadurch gilt die für einen \emph{strukturellen Proxy} Proxy $P$:
\begin{gather*}
\frac{\forall \mathit{T} \in \mathit{P.targets}:\exists \mathit{MD} \in \mathtt{P.dels}:\mathit{MD.del.target} = T}{\mathit{targets_{struct}(P, T)}}
\end{gather*}
Für die aufgerufene Methode und die Delegationsmethode einer einzelnen Methoden-Delegation $\mathit{M}$ gelten im \emph{strukturellen Proxy} dieselben Regeln wie für den \emph{Sub-Proxy}. Die Namen der aufgerufenen Methoden und der Delegationsmethode müssen dabei jedoch nicht übereinstimmen. Dafür müssen diese beiden Methode jedoch ein strukturelles Matching aufweisen. Bezogen auf die Rückgabe-Typen einer aufgerufenen Methode $\mathit{C}$ und der Delegationsmethode $\mathit{D}$ aus einer Methoden-Delegation muss daher Folgendes gelten.
\begin{gather*}
\frac{\mathit{D.returnType} \Rightarrow_{internStruct} \mathit{C.returnType}}{\mathit{return_{struct}(C,D)}}
\end{gather*} 
Weiterhin muss für die Parameter-Typen gelten:
\begin{gather*}
\frac{\mathit{C.paramCount} = 0}{\mathit{params_{struct}(C,D)}}
\end{gather*} 
\begin{gather*}
\frac{\splitfrac{\forall \mathit{i} \in \{0,...,\mathit{C.paramCount}-1\}:}
{ \mathit{C.paramTypes}[i] \Rightarrow_{internStruct} \mathit{D.paramTypes}[\mathit{D.posModi}[i]]
}}{\mathit{params_{struct}(C,D)}}
\end{gather*} 
Für eine einzelne Methoden-Delegation $\mathit{MD}$ eines \emph{strukturellen Proxies} $P$ kann dann folgende Regel aufgestellt werden.
\begin{gather*}
\frac{\splitfrac{\mathit{call_{simple}(MD,P)} \wedge \mathit{del_{simple}(MD,P)} \wedge} {\mathit{return_{struct}(MD.call, MD.del)} \wedge \mathit{params_{struct}(MD.call, MD.del)}}
}
{\mathit{methDel_{struct}(MD,P)}}
\end{gather*}
In einem \emph{strukturellen Proxy} muss für jede Methode $m$ des Source-Typen genau eine Methoden-Delegation mit der Methode $m$ als aufgerufene Methode existieren. Daraus ergibt sich für alle Methoden-Delegationen aus einem \emph{strukturellen Proxy} $P$ folgende Regel:
\begin{gather*}
\frac{\splitfrac{\mathit{M} = \mathit{methoden(P.type)} \wedge |\mathit{M}| = |\mathit{P.dels}| \wedge \forall \mathit{T' \text{ }m(T)} \in \mathit{M}:}{\exists \mathit{MD} \in \mathit{P.dels}:\mathit{MD.call.name} = m \wedge \mathit{methDel_{struct}(MD,P)}
 }
}
{\mathit{methDelList_{struct}(P)}}
\end{gather*}
Die Menge der \emph{strukturellen Proxies}, die mit dem Source-Typ $R$ und der Menge von Target-Typen $T$ erzeugt werden, wird durch die folgende Funktion beschrieben.
\begin{gather*}
\mathit{proxies_{struct}(R,T)} := 
\left\{\begin{array}{l|l}
		& \mathit{proxy(P,R)}\wedge \mathit{ }\\
	P	& \mathit{targets_{struct}(P,T)} \wedge \mathit{ }\\
		& \mathit{methDelList_{struct}(P)}  
		 \end{array}
\right\}
\end{gather*}

\subsubsection{Allgemeine Generierung von Proxies}
Die Proxy-Funktion der einzelnen Proxy-Arten werden zur Beschreibung einer allgemeine Funktion für die Generierung der Proxies verwendet. Dazu sind die Proxy-Arten zusammen mit den dazugehörigen Matchingrelationen und den Namen der Funktionen zur Generierung des jeweiligen Proxies in Tabelle \ref{tab:baseMatcher} noch einmal aufgeführt.

\begin{table}[H]
\centering
\begin{tabular}{|c|c|c|}
\hline
\hline
\centering\textbf{Proxy-Art} & \textbf{Matchingrelation} & \textbf{Funktionsname}\\
\hline
\hline
Sub-Proxy
&  
$\Rightarrow_{spec}$
& 
$\mathit{proxies_{sub}}$
\\
\hline
Content-Proxy
& 
$\Rightarrow_{content}$
& 
$\mathit{proxies_{content}}$
\\
\hline
Container-Proxy
& 
$\Rightarrow_{container}$
& 
$\mathit{proxies_{container}}$
\\
\hline
struktureller Proxy
&
$\Rightarrow_{struct}$
& 
$\mathit{proxies_{struct}}$
 \\
\hline
\hline
\end{tabular}
\caption{Proxy-Arten mit Matchingrelationen und Proxy-Funktionen}
 \label{tab:baseMatcher}
\end{table}
\noindent
Die im Abschnitt \ref{sec:proxygram} erwähnte Funktion $\mathit{proxies(S,T)}$ kann darauf aufbauend für einen Source-Typ $S$ und eine Menge von Target-Typen $T$ wie folgt beschrieben werden.
\begin{gather*}
\mathit{proxies(S,T)} := 
\left\{\begin{array}{ll}
\mathit{proxy_{sub}(S,T)}	& \text{wenn } |T| = 1 \wedge \mathit{ }\\
& \forall T' \in T: S \Rightarrow_{spec} T'\\	
&\\
\mathit{proxy_{content}(S,T)}	& \text{wenn } |T| = 1 \wedge \mathit{ }\\
& \forall T' \in T: S \Rightarrow_{content} T' \\
&\\
\mathit{proxy_{container}(S,T)} & \text{wenn } |T| = 1 \wedge \mathit{ } \\
& \forall T' \in T: S \Rightarrow_{container} T' \\
&\\
\mathit{proxy_{struct}(S,T)} & \text{wenn } |T| > 0 \wedge \mathit{ } \\
&\forall T' \in T: S \Rightarrow_{struct} T'
		 \end{array}
\right\}
\end{gather*}
\subsection{Anzahl struktureller Proxies innerhalb einer Bibliothek}\label{sec_anzahlProxies}
Die Generierung der strukturellen Proxies für ein \emph{required Typ} $R$ aus der Bibliothek $L$ erfolgt während der Exploration mit den Mengen von \emph{provided Typen} aus $\mathit{cover(R,L)}$ (siehe Abschnitt \ref{sec_ergStructEval}). Mit einer Menge $\mathit{TM} \in \mathit{cover(R,L)}$ können durchaus mehrere Proxies erzeugt werden. Das ist dann der Fall, wenn mehrere der Methoden, die in den \emph{provided Typen} aus $\mathit{TM}$ deklariert wurden, mit einer Methode aus $R$ strukturell übereinstimmen.
\\\\
Die Anzahl der strukturellen Proxies für einen \emph{required Typ} $R$ mit einer bestimmten Menge von \emph{Target-Type}n ist somit von der Anzahl der Methoden abhängig, die in einem der \emph{Target-Typen} deklariert wurden und strukturell mit den Methoden aus $R$ übereinstimmen. 
\\\\
Die Menge der Methoden eines \emph{provided Typs} $T$, die strukturell mit einer Methode $m$ übereinstimmen, wird über die Funktion $\mathit{structM_{target}}$ beschrieben.
\begin{gather*}
\mathit{structM_{target}(m, T)} := 
\left\{\begin{array}{l|l}
m'	& m' \in \mathit{methoden(T)} \wedge  m \Rightarrow_{method} m'
\end{array}
\right\}
\end{gather*}
\noindent
Darauf aufbauend wird die Menge der Methoden einer Menge von \emph{provided Typen} $\mathit{TM}$, die strukturell mit einer Methode $m$ übereinstimmen, über die Funktion $\mathit{structM_{targetset}}$ beschrieben.
\begin{gather*}
\mathit{structM_{targetset}(m, \mathit{TM})} := 
\left\{\begin{array}{l|l}
m'	& \exists T \in \mathit{TM}: m' \in \mathit{structM_{target}(m,T)}
\end{array}
\right\}
\end{gather*}
\noindent

\begin{example}{bsp_structmtarget}
Aufbauend auf dem vorherigen Beispiel \ref{bsp_cover} ergeben sich für die Menge der \emph{Target-Typen}  $\{\texttt{Leave}, \texttt{Come}\}$ und die beiden Methoden des \emph{required Typs} $\texttt{Greeting}$ folgende Mengen von übereinstimmenden Methoden über die Funktion $\mathit{structM_{targetset}}$:
\begin{gather*}
\mathit{structM_{targetset}(\methodForm{String}{hello}{},\{\texttt{Leave}, \texttt{Come}\})} = 
\left\{
\begin{array}{l}
\methodForm{String}{hello}{},\\
\methodForm{String}{goodMorning}{},\\
\methodForm{String}{bye}{}
\end{array}
\right\}\\
\mathit{structM_{targetset}(\methodForm{String}{bye}{},\{\texttt{Leave}, \texttt{Come}\})} = 
\left\{
\begin{array}{l}
\methodForm{String}{hello}{},\\
\methodForm{String}{goodMorning}{},\\
\methodForm{String}{bye}{}
\end{array}
\right\}
\end{gather*}
\end{example}
\noindent
\\
Sei $R$ ein \emph{required Typ} und $\mathit{TM}$ eine Menge von \emph{provided Typen} innerhalb einer Bibliothek $L$ mit $\mathit{TM} \in \mathit{cover(R,L)}$. Dann bildet die Funktion $\mathit{structMSets}$ die Menge von Mengen der Methoden aus den Elementen aus $\mathit{TM}$ ab, die mit jeweils einer Methode aus $R$ gematcht werden können.
\begin{gather*}
\mathit{structMSets(R,\mathit{TM})} := 
\left\{M
\begin{array}{l|l}
&\exists \mathit{m} \in \mathit{methoden(R)} : 
\\
&M = \mathit{structM_{targetset}(m,\mathit{TM})}
\end{array}
\right\}
\end{gather*}
\noindent
%Für jede Kombination von jeweils einem Element aus jeder der Mengen aus $\mathit{structMSets(R,T)}$ kann ein Proxy für $R$ mit der Menge der Target-Typen $T$ erzeugt werden.
Für die Bildung eines Proxies wird aus jedem Element der Menge $structMSets(R,\mathit{TM})$ genau ein Element als Delegationsmethode verwendet werden. 
\begin{example}{bsp_proxycreation}
Ausgehend von Beispiel \ref{bsp_structmtarget} lassen sich die folgenden vier Proxies mit den \emph{Target-Typen} $\texttt{Leave}$ und $\texttt{Come}$ erzeugen.
\begin{lstlisting}[style = dsl]
proxy Greeting with [Come, Leave]{
	Greeting.hello():String --> Come.hello():String
	Greeting.bye():String --> Leave.bye():String
}
\end{lstlisting}
\begin{lstlisting}[style = dsl]
proxy Greeting with [Come, Leave]{
	Greeting.hello():String --> Come.goodMorning():String
	Greeting.bye():String --> Leave.bye():String
}
\end{lstlisting}
\begin{lstlisting}[style = dsl]
proxy Greeting with [Come, Leave]{
	Greeting.hello():String --> Leave.bye():String
	Greeting.bye():String --> Come.hello():String
}
\end{lstlisting}
\begin{lstlisting}[style = dsl]
proxy Greeting with [Come, Leave]{
	Greeting.hello():String --> Leave.bye():String
	Greeting.bye():String --> Come.goodMorning():String
}
\end{lstlisting}
\end{example}
\noindent
\\
Die Anzahl aller möglichen Proxies für ein \emph{required Typ} $R$ aus einer Menge von \emph{Target-Typen} $\mathit{TM}$ sei näherungsweise über die Funktion $\mathit{proxyCount(R,\mathit{TM})}$ beschrieben. Dass es sich hierbei lediglich um eine Annäherung handelt liegt daran, dass eine Methode $\mathit{dm}$ mit $\mathit{dm} \in M_1 \cup ... \cup M_n$ und $\{M_1,...,M_n\} = \mathit{structMSets(R,TM)}$ innerhalb eines Proxy maximal einmal als Delegationsmethode verwendet werden darf. Es ist jedoch möglich, dass es zwischen den Mengen 
$M_1,...,M_n$ Überschneidungen gibt (siehe vorheriges Beispiel). Daher gelten für die Funktion $\mathit{proxyCount}$ folgende Regeln:
\begin{gather*}
\frac{\{M_1,...,M_n\} = \mathit{structMSets(R,TM)} \wedge M_1 \cap ... \cap M_n = \emptyset}{\mathit{proxyCount(R,TM)} = \prod\limits_{i=1}^{n}|M_i| }
\\\\
\frac{\{M_1,...,M_n\} = \mathit{structMSets(R,TM)} \wedge M_1 \cap ... \cap M_n \neq \emptyset}{\mathit{proxyCount(R,TM)} < \prod\limits_{i=1}^{n}|M_i| }
\end{gather*}
\noindent
Im Allgemeinen gilt demnach:
\begin{gather*}
\mathit{proxyCount(R,TM)} \leq 
\begin{array}{l|l}
\prod\limits_{i=1}^{n}|\mathit{structM_{targetset}(m_i, TM)}|
&
\left\{
\begin{array}{l}
m_1,\\
...,\\
m_n
\end{array}
\right\}
= \mathit{methoden(R)}
\end{array}
\end{gather*}
Da innerhalb einer Bibliothek $L$ mehrere Mengen von \emph{Target-Typen} zur Bildung eines Proxies für einen \emph{required Typ} $R$ infrage kommen (siehe Funktion $\mathit{cover}$) muss die Anzahl der strukturellen Proxies über die Funktion $\mathit{proxyCount}$ für alle Elemente aus $\mathit{cover(R,L)}$ ermittelt und summiert werden. Die folgende Funktion beschreibt diesen Sachverhalt:
\begin{gather*}
\mathit{libProxyCount(R,L)} = 
\begin{array}{l|l}
\sum_{i=1}^{n}\mathit{proxyCount(R,c_i)}
&
\left\{
\begin{array}{l}
c_1,\\
...,\\
c_n
\end{array}
\right\} = \mathit{cover(R,L)}
\end{array}
\end{gather*}
\section{Semantische Evaluation}\label{sec_semEval}
Das Ziel der \emph{semantischen Evaluation} ist es, einen der \emph{Proxies}, die aus den Mengen von \emph{Target-Typen}, die im Rahmen der \emph{strukturellen Evaluation} erzeugt werden können, hinsichtlich der vordefinierten Testfälle zu evaluieren. Da der gesamte \emph{Explorationsprozess} zur Laufzeit des jeweiligen Programms durchgeführt wird, ist dieser hinsichtlich der nicht-funktionalen Anforderungen als zeitkritisch einzustufen.
\\\\
Da die Anforderungen an den gesuchten \emph{Proxy} mit Bedacht spezifiziert werden müssen, ist es irrelevant, ob es mehrere \emph{Proxies} gibt, die hinsichtlich der vordefinierten Testfällen positiv geprüft werden können. Es ist ausreichend lediglich ein \emph{Proxy} zu finden, dessen Semantik zu positiven Ergebnissen hinsichtlich aller vordefinierten Testfälle führt.
\subsection{Besonderheiten der Testfälle}\label{sec_testanforderungen}
Bei den vordefinierten Tests handelt es sich auf formaler Ebene um Typen, die eine $\texttt{eval}$-Methode mit der Struktur $\texttt{boolean eval( proxy )}$ anbieten, welche einen \emph{Proxy} als Parameter erwartet und ein Objekt vom Typ $\texttt{boolean}$ zurückgibt. Weiterhin verfügt ein Test über ein Attribut $\texttt{triedMethodCalls}$, in dem eine Liste von Methodennamen, die bei der Durchführung der $\texttt{eval}$-Methode auf den \emph{Proxies} aufgerufen wurden, hinterlegt ist.
\\\\
Die Implementierung der $\texttt{eval}$-Methode ist an folgende Bedingungen geknüpft:
\begin{enumerate}
\item Vor dem Aufruf einer Methode auf dem als Parameter übergebenen \emph{Proxy}, wird der Name dieser Methode in der Liste im Feld $\texttt{triedMethodCalls}$ ergänzt.
\item Wenn der \emph{Proxy} den Test besteht, wird der Wert $\texttt{true}$ zurückgegeben. Anderenfalls wird der Wert $\texttt{false}$ zurückgegeben.
\end{enumerate}

\begin{example}{xmpl_evalMethode}
In folgendem Listing \ref{lst_examEval} ist eine $\texttt{eval}$-Methode aufgeführt, die die oben genannten Bedingungen erfüllt. Es sei davon auszugehen, dass der als Parameter übergebene \emph{Proxy} eine Methode mit der Struktur $\methodForm{Integer}{add}{Integer, Integer}$
anbietet.
\newpage
\begin{lstlisting}[style = pseudo, label = lst_examEval, caption = Beispielhafte Implementierung einer $\texttt{eval}$-Methode, captionpos = b]
function eval( proxy ){
 res = 0	
 triedMethodCalls.add( "add" )
 res = proxy.add(1, 1)
 return res == 2;
}
\end{lstlisting}
\end{example}

\subsection{Algorithmus für die semantische Evaluation}\label{sec_semEvalAlgo}
Während des \emph{Explorationsprozesses} soll aus den \emph{provided Typen} in einer Bibliothek $L$ zu einem vorgegebenen \emph{required Type} $R$ ein Proxy generiert und evaluiert werden. Die Mengen der \emph{Target-Typen} auf deren Basis mehrere \emph{Proxies} erzeugt werden können, wurden in Abschnitt \ref{sec_anzahlProxies} mithilfe der Funktion $\mathit{cover(R,L)}$ beschrieben. In diesem Zusammenhang wurde in Lemma \ref{lemma_targetcount} eine Restriktion bzgl. der Anzahl möglicher \emph{Target-Typen} eines \emph{Proxies} beschrieben.
Darauf aufbauend, kann die maximale Anzahl von \emph{Target-Typen} eines \emph{Proxies} für $R$ wie folgt definiert werden:
%TODO Daraus ergibt sich folgender Satz:
\begin{gather*}
\mathit{maxTargets(R)} := |\mathit{methods(R)}|
\end{gather*}
\noindent
Das in dieser Arbeit beschriebene Konzept basiert auf der Annahme, dass der gesamte Anwendungsfall - oder Teile davon - , der mit der vordefinierten Struktur (\emph{required Typ}) und den vordefinierten Tests abgebildet werden soll, schon einmal genauso oder so ähnlich in dem gesamten System implementiert wurde. Aus diesem Grund kann für die \emph{semantische Evaluation} davon ausgegangen werden, dass die erfolgreiche Durchführung aller relevanten Tests umso wahrscheinlicher ist, je weniger \emph{Target-Typen} im \emph{Proxy} enthalten sind.
\\\\
Die Mengen innerhalb einer Menge $\mathit{C}$ mit einer Mächtigkeit $a$ seien durch folgende Funktion beschrieben:
\begin{gather*}
\mathit{targetSets(\mathit{C},a)} := 
\left\{\begin{array}{l|l}	
				\mathit{TM} & \mathit{TM} \in \mathit{C} \wedge |\mathit{TM}| = a
		 \end{array}
\right\}
\end{gather*}
\noindent
Ausgehend von einer Bibliothek $L$ kann der Algorithmus für die \emph{semantische Evaluation} der \emph{Proxies}, die für einen \emph{required Typ} $R$ (Parameter $\texttt{R}$) mit den Mengen der \emph{Target-Typen} $\mathit{cover(R, L)}$ (Parameter $\texttt{T}$) erzeugt werden können, und einer Menge von Tests (Parameter $\texttt{tests}$) über die Methode $\texttt{semanticEval}$ wie in Listring \ref{lst_semEval} im Pseudo-Code beschrieben werden. Die globale Variable $\texttt{passedTests}$ enthält dabei die Anzahl der für den aktuell zu überprüfenden \emph{Proxy} erfolgreich durchgeführten Tests. Außerdem sei davon auszugehen, dass die Funktionen aus Abschnitt \ref{sec_proxyGen} wie beschrieben definiert sind.
\\\\
Die Dauer der Laufzeit der in Listing \ref{lst_semEval} definierten Funktionen hängt maßgeblich von der Anzahl der \emph{Proxies} ab, die für den \emph{required Typ} $R$ in der Bibliothek $L$ erzeugt werden können (siehe auch Abschnitt \ref{sec_anzahlProxies} Funktion $\mathit{libProxyCount}$). Im schlimmsten Fall müssen alle \emph{Proxies} generiert werden und hinsichtlich der vordefinierten Tests geprüft werden. Um die Anzahl dieser \emph{Proxies} zu reduzieren, werden die im folgenden Abschnitt beschriebenen \Gls{Heuristik}en verwendet.
\newpage
\begin{lstlisting}[style = pseudo, caption = Semantische Evaluation ohne Heuristiken, captionpos = b, label = lst_semEval]
passedTests = 0

function semanticEval( R, T, tests ){
 for( anzahl = 1; anzahl <= $\mathit{maxTargets( R )}$; i++ ){
  for( targets : $\mathit{targetSets( T, anzahl )}$ ){
   relProxies = $\mathit{proxies( R, targets )}$
   proxy = evalProxies( relProxies, tests )	
   if( proxy != null ){
    return proxy
   }
  }
 }
 return null;
}

function evalProxies(proxies, tests){
 for( proxy : proxies ){
  passedTests = 0
  evalProxy(proxy, tests)
  if( passedTests == tests.size ){
   return proxy
  }
 }
 return null
}

function evalProxy(proxy, tests){
 for( test : tests ){
  if( !test.eval( proxy ) ){
   return
  }
  passedTests = passedTests + 1
 }
}
\end{lstlisting}
\newpage

\chapter{Implementierung}
Die Implementierung der Explorationskomponente besteht aus drei Hauptbestandteilen, die jeweils als separates Java-Projekt umgesetzt wurden. Im weiteren Verlauf werden diese Java-Projekte als Module bezeichnet.
\begin{figure}[h!]
\centering
\includegraphics[scale=0.8]{cd_arch.png}
\caption{Architektur}
\label{cd_arch}
\end{figure}
\noindent
In Abbildung \ref{cd_arch} ist die Architektur der Explorationskomponente aufgeführt. Dieser ist zu entnehmen, dass die Explorationskomponente aus drei Modulen besteht, die im weiteren Verlauf dieses Kapitels beschrieben werden. Das Modul \emph{DesiredComponentSourcerer} ist dabei von den Modulen \emph{ComponentTester} und \emph{SignatureMatching} abhängig, während das Modul \emph{ComponentTester} lediglich vom Modul \emph{SignatureMatching} abhängig ist.
\\\\
Darüber hinaus, werden folgende externe Bibliotheken verwendet:
\begin{itemize}
\item easymock 3.0 \cite{easymock}
\item cglib 3.3.0 \cite{cglib}
\item objenesis 3.1 \cite{objenesis}
\item junit 4.13.0 \cite{junit}
\end{itemize}
Auf die konkrete Verwendung der externen Bibliotheken wird in den detaillierteren Beschreibungen der einzelnen Module in den folgenden Abschnitten eingegangen.
\section{Modul: SignatureMatching}
\begin{figure}[h!]
\includegraphics[scale=0.5]{cd_SigMa.png}
\caption{Modul: SignatureMatching}
\label{fig_cdSigMa}
\end{figure}
\noindent
In diesem Modul befinden sich zum Einen die Implementierungen der Matcher, die in Abschnitt \ref{sec_matcher} formal beschrieben wurden und zum Anderen die Implementierung der Generatoren für die Proxies. In Abbildung \ref{fig_cdSigMa} sind die wichtigsten Klassen und Interfaces dieses Moduls mit ihren Abhängigkeiten zueinander aufgeführt. Die Matcher befinden sich dabei im Package \emph{matching} und die Generatoren für die Proxies in Form der Implememtierungen des des Interfaces $\texttt{ProxyFactory}$ im Package \emph{glue}.
\\\\
Die in Abschnitt \ref{sec_matcher} beschriebenen Matcher und Generatoren wurden teilweise in einer Klasse zusammengefasst. Tabelle \ref{tab_matcher2impl} zeigt die Zuordnung von Matchern zu den jeweiligen Klassen, die die Implementierung dieser darstellen, und den Klassen, die die Implementierung des Generators für den Proxy, der auf Basis des Matchers Anwendung findet, dargestellt.
\begin{table}[h!]
\centering
\begin{tabular}{|l|l|l|}
\hline
\hline
\textbf{Matcher} & \textbf{Matcher-Implementierung} & \textbf{Generator-Implementierung}\\
\hline
ExactTypeMatcher & $\texttt{ExactTypeMatcher}$ & $\texttt{ClassProxyFactory}$ \\
\hline
GenTypeMatcher & $\texttt{GenSpecTypeMatcher}$ & $\texttt{ClassProxyFactory}$\\
\hline
SpecTypeMatcher & $\texttt{GenSpecTypeMatcher}$ & $\texttt{ClassProxyFactory}$\\
\hline
ContentTypeMatcher & $\texttt{ContainerTypeMatcher}$ & $\texttt{ContentProxyFactory}$\\
\hline
ContainerTypeMatcher & $\texttt{ContainerTypeMatcher}$ & $\texttt{ContainerProxyFactory}$\\
\hline
StructuralTypeMatcher & $\texttt{StructuralTypeMatcher}$ & $\texttt{InterfaceProxyFactory}$\\
\hline
\hline
\end{tabular}
\caption{Zuordnung der Matcher zu den Matcher- und Generator-Implementierungen}
\end{table}\label{tab_matcher2impl}
\noindent
Die Klasse $\texttt{StructuralTypeMatcher}$ nimmt dabei eine Sonderstellung ein. Dies ist daran zu erkennen, dass dieser nicht das Interface $\texttt{TypeMatcher}$ implementiert. Dies wird damit begründet, dass es sich bei diesem Matcher um den Einstiegspunkt der strukturellen Evaluation handelt. Analog zum StructuralTypeMatcher aus Abschnitt \ref{sec_matcher} wird in der Klasse $\texttt{StructuralTypeMatcher}$ auf die anderen Matcher bzw. Matcher-Implementierungen zugegriffen, was in Abbildung \ref{fig_cdSigMa} durch die Aggregation zwischen der Klasse $\texttt{StructuralTypeMatcher}$ und dem Interface $\texttt{TypeMatcher}$ angedeutet wird.
\\\\
Die übrigen Matcher-Klassen implementieren das Interface $\texttt{TypeMatcher}$ und können über die Methode $\texttt{combine}$ aus der Klasse $\texttt{MatcherCombinator}$ miteinander kombiniert werden\footnote{Ein Beispiel für die Kombination von Matchern ist im Anhang \ref{app_matchercombination} zu finden.}. 
%TODO ANHANG: Kombination (siehe unten)
So kann eine Kombination mehrerer $\texttt{TypeMatcher}$, die wiederum von Typ $\texttt{TypeMatcher}$ ist, in der Klasse $\texttt{StructuralTypeMatcher}$ verwendet werden. Die konkrete $\texttt{TypeMatcher}$-Kombination, die im $\texttt{StructuralTypeMatcher}$ instanziiert wird, orientiert sich an den Ausführungen in Abschnitt \ref{sec_matcher}. Es ist aber zu erwähnen, dass die Verwendung weitere Matcher, die in dieser Arbeit nicht definiert wurden, denkbar ist. Eine solche Erweiterung ließe sich leicht in dieses Modul über die Implementierung des Interfaces $\texttt{TypeMatcher}$ und die Verwendung der Klasse $\texttt{MatcherCombiner}$ bewerkstelligen.
\\\\
Alle Matcher-Implementierungen bieten die Möglichkeit, zu ermitteln, ob ein Matching zwischen zwei Typen besteht (siehe Klassendiagramme in Abbildungen \ref{fig_cdMatchingInfo} und \ref{fig_cdSingleMatchingInfo}). Dies erfolgt jeweils über die Methode $\texttt{matchesType}$. Über die Methoden $\texttt{calculateMatchingInfos}$ bzw. $\texttt{calculateMatchingInfo}$ werden die Informationen bzgl. der Methodendelegationen zwischen den beiden gemachten Typen ermittelt. Diese Informationen werden in einem Objekt der Klasse $\texttt{SingleMatchingInfo}$ bzw. $\texttt{MatchingInfo}$ zusammengetragen, welche in Abbildung \ref{fig_cdMatchingInfo} und \ref{fig_cdMatchingInfo} detailliert dargestellt werden.
\begin{figure}[h!]
\includegraphics[scale=0.8]{cd_matchinginfo.png}
\caption{Klassendiagramm: $\texttt{StructuralTypeMatcher}$ und $\texttt{MatchingInfos}$}
\label{fig_cdMatchingInfo}
\end{figure}
\begin{figure}[h!]
\includegraphics[scale=0.7]{cd_singlematchinginfo.png}
\caption{Klassendiagramm: $\texttt{TypeMatcher}$ und $\texttt{SingleMatchingInfo}$}
\label{fig_cdSingleMatchingInfo}
\end{figure}
\noindent
Diese beiden Klassen unterscheiden sich lediglich bzgl. des Attributs in dem die Delegationsmethoden hinterlegt sind. Dabei handelt es sich auf Seiten der $\texttt{SingleMatchingInfo}$ um das Attribut $\texttt{methodMatchingInfos}$ und auf Seiten der $\texttt{MatchingInfo}$ um das Attribut $\texttt{methodMatchingSupplier}$. 
\\\\
Während ein Objekt der Klasse $\texttt{MatchingInfo}$ mehrere Delegationsmethoden zu einer aufgerufenen Methoden enthalten kann, darf ein Objekt der Klasse $\texttt{SingleMatchingInfo}$ lediglich eine Delegationsmethode zu einer aufgerufenen Methode enthalten (vgl. auch Abschnitt \ref{sec_matcher}). Zusätzlich zu erwähnen ist, dass die Informationen über die Delegationsmethoden aus einer $\texttt{MatchingInfo}$ über in einem $\texttt{MethodSupplier}$ überliefert wird.
\\\\
Eine Instanz der Klasse $\texttt{MethodSupplier}$ enthält zum Einen ein $\texttt{MatcherRating}$ welches Informationen bzgl. des in Abschnitt \ref{sec_lmf} beschriebenen Matcher-Ratings beinhaltet. Zum Anderen werden im Attribut $\texttt{methodMatchingInfo}$ in einem Objekt der Klasse $\texttt{MethodMatchingInfo}$ (siehe Abbildung \ref{cd_methodMatchingInfo}) die Informationen bzgl. der Delegation der aufgerufenen Methode an die Delegationsmethode hinterlegt. 
\\\\
Bezüglich der Klasse $\texttt{SingleMatchingInfo}$ ist noch das Attribut $\texttt{proxyFactoryCreator}$ zu beschreiben. Darin werden Informationen bzgl. der strukturellen Verbindung von zwischen den gematchten Typen gehalten. Für den \emph{ExactTypeMatcher}, den \emph{GenTypeMatcher} und den \emph{SpecTypeMatcher} wird dabei ein $\texttt{ProxyFactoryCreator}$ erzeugt, der in der Lage ist, eine $\texttt{ProxyFactory}$ für Typen zu erzeugen, die in einer nominalen Beziehung \footnote{Identität, Generalisierung, Spezialisierung} stehen. Für den \emph{ContentTypeMatcher} und den \emph{ContainedTypeMatcher} hingegen, wird ein $\texttt{ProxyFactoryCreator}$ erzeugt, der in der Lage ist, eine $\texttt{ProxyFactory}$ für Typen zu erzeugen, bei denen der eine Typ ein Attribut von Typ des anderen enthält (vgl. mit Tabelle \ref{tab_matcher2impl}). Die erzeugten Objekte vom Typ $\texttt{ProxyFactory}$ werden bei der Generierung der Proxies unter der Zuhilfenahme der Bibliotheken \emph{cglib} und \emph{objenesis} verwendet\footnote{Diese beiden Frameworks wurden verwendet, da die Erzeugung der Proxies mit ihnen komfortabler ist, als mit den Mitteln die das JKD zur Verfügung steht. Dies gilt insbesondere für die Erzeugung von Proxies für Klassen, die mit dem Schlüsselwort $\texttt{final}$ versehen sind.}.
\\\\
Der $\texttt{ProxyFactoryCreator}$ stellt damit eines der Bindeglieder zwischen der Package \emph{matching} und dem Package \emph{glue} innerhalb des Moduls her. Das zweite Artefakt, welches als Bindeglied fungiert, ist die oben bereits erwähnt Klasse $\texttt{MethodMatchingInfo}$, deren Aufbau dem Klassendiagramm aus Abbildung \ref{cd_methodMatchingInfo} zu entnehmen ist.
\begin{figure}[h!]
\includegraphics[scale=1.0]{cd_methodmatchinginfo.png}
\caption{Klassendiagramm: $\texttt{MethodMatchingInfo}$}
\label{cd_methodMatchingInfo}
\end{figure}
\noindent
\\\\
Ein Objekt der Klasse $\texttt{MethodMatchingInfo}$ enthält in den Attributen $\texttt{source}$ und $\texttt{target}$ je eine Methode. Dabei ist im Attribut $\texttt{source}$ die aufgerufene Methode der Methoden-Delegation und im Attribut $\texttt{target}$ die Delegationsmethode enthalten. Darüber hinaus wird im Attribut $\texttt{returnTypeMatchingInfo}$ ein Objekt der Klasse $\texttt{SingleMatchingInfo}$ gehalten , welches alle notwendigen Informationen für das Erzeugen eines Proxies des Rückgabetyp der aufgerufenen Methode aus dem Rückgabetyp der Delegationsmethode.
\\\\
Analog dazu wird im Attribut $\texttt{argumentTypeMatchingInfos}$ eine Map, bestehend aus weiteren Objekten der Klasse $\texttt{SingleMatchingInfo}$ und jeweils einem Objekt der Klasse $\texttt{ParamPosition}$, gehalten. Diese Map enthält alle notwendigen Information für das Erzeugen eines Proxies für die Parametertypen der Delegationsmethoden aus den Parametertypen der aufgerufenen Methode, sowie der Anpassung der Übergabeposition bei der Delegation der aufgerufenen Methode (siehe auch Abschnitt \ref{sec:proxygram}).
\\\\
Um die Methoden-Delegationen zu koordinieren, wird bei der Erzeugung des Proxies in der jeweiligen $\texttt{ProxyFactory}$ für das Proxy-Objekt ein $\texttt{InvocationHandler}$ instanziiert (vgl. \cite{invocationhandler}). Dieses Interface wird im \emph{glue}-Package durch die Klasse $\texttt{BehaviourDelegateInvocationHandler}$ implementiert, in der letztendlich die Koordination der Methoden-Delegationen auf Basis der jeweiligen $\texttt{MethodMatchingInfo}$ spezifiziert ist.
\\\\
Um einen Proxy basierend auf dem Matching zweier Typen zu erzeugen steht die Klasse $\texttt{TypeConverter}$ zur Verfügung (siehe Abbildung \ref{cd_typeconverter}). Die Zugriffe innerhalb des Packages \emph{glue} als auch die Zugriff von außerhalb benötigen jeweils ein Objekt der Klasse $\texttt{ConvertableBundle}$. Diese Klasse beschreibt eine Kombination mehrerer Objekte vom Typ $\texttt{ConvertableComponent}$, die als Delegationsziele des zu erzeugenden Proxy-Objektes fungieren sollen. Ein Objekt der Klasse $\texttt{ConvertableComponent}$ enthält eine Liste von Objekten vom Typ $\texttt{SingleMatchingInfo}$, die wie bereits erwähnt beschreiben, am welche Methode die Delegation erfolgen soll. Das Objekt im Attribut $\texttt{convertableObject}$ der $\texttt{ModuleMatchingInfo}$ beinhaltet das Objekt, auf dem die Delegationsmethode aufgerufen werden soll.
\begin{figure}[h!]
\includegraphics[scale=0.7]{cd_typeconverter.png}
\caption{Klassendiagramm: $\texttt{TypeConverter}$}
\label{cd_typeconverter}
\end{figure}
%TODO ANHANG: Kombination (siehe unten)
%Die Matcher-Klassen $\texttt{ExactTypeMatcher}$, $\texttt{GenSpecTypeMatcher}$ und $\texttt{WrappedTypeMatcher}$ implementieren auch das von $\texttt{TypeMatcher}$ erbende Interfaces $\texttt{CombinalbeTypeMatcher}$. Klassen, die dieses Interface implementieren können über die Klasse $\texttt{MatcherCombiner}$ zu einem neuen $\texttt{TypeMatcher}$-Objekt kombiniert werden. Ein solcher kombinierte $\texttt{TypeMatcher}$ versucht beim Aufruf der Methode $\texttt{matchesType(S,T)}$ die beiden Typen $S$ und $T$ über einen der kombinierten Matcher zu matchen. Abbildung \ref{sd_matchercombiner} zeigt das Sequenzdiagramm für diesen Aufruf. Dabei liefert die Methode $\texttt{getSortedMatcher}$ eine sortiert Liste der kombinierten Matcher. Die Sortierung wird aufsteigend entsprechend dem Matcherrating der kombinierten Matcher vorgenommen.
%\begin{figure}
%\end{figure}\label{sd_matchercombiner}
%\noindent
%Darüber hinaus gibt es noch das von $\texttt{TypeMatcher}$ erbende Interface $\texttt{PartlyTypeMatcher}$. Dieses Interface wird nur von dem $\texttt{StructuralTypeMatcher}$ implementiert, welcher u.a. als Schnittstelle zwischen dem Modul \emph{SignatureMatching} und \emph{DesiredComponentSourcerer} fungiert. Wie der Name des Interfaces bereits impliziert, bieten die Implementierungen des Interfaces $\texttt{PartlyTypeMatcher}$ die Möglichkeit, zwei Typen nur teilweise zu Matchen. Das bildet die Grundlage für die Ermittlung der Typen, aus denen die Proxies für die semantische Evaluation erzeugt werden können (vgl. Abschnitt \ref{sec_ergStructEval}). So stellen die Objekte, die über die Methode $\texttt{calculatePartlyTypeMatchingInfos}$ erzeugten wurden, auf formaler Ebene die Elemente der Mengen, die in Abschnitt \ref{sec_ergStructEval} über Funktion $\texttt{cover}$ beschrieben wurden, dar.

\section{Modul: ComponentTester}
Dieses Modul ist für die Ausführung der vordefinierten Tests zuständig. Darüber hinaus bietet es die Möglichkeit, die vordefinierten Tests mit den Interfaces, die den dazugehörigen required Typdarstellen, zu Verbinden. Dabei sei davon auszugehen, dass ein required Typ $R$ in Form eines Interfaces existiert. Um Tests für $R$ zu definieren, können eine oder mehrere Testklassen implementiert werden. Die Testklassen werden dabei in dem Interface $R$ über das Attribut $\texttt{testClasses}$ der Annotation $\texttt{RequiredTypeTestReference}$ angegeben (siehe Abbildung \ref{fig_cdCompTester} Package: \emph{API}). Ein Beispiel für die Deklaration eines required Typ in Form eines Java-Interfaces und den dazugehörigen Testklassen ist im Anhang zu finden.
%TODO beispiel im Anhang
\\\\
Damit die Testmethoden in den Testklassen die in Abschnitt \ref{sec_testanforderungen} beschriebenen Eigenschaften aufweisen und durch das \emph{ComponentTester}-Modul ausfindig gemacht werden können, stehen mehrere Artefakte in dem \emph{API}- und dem \emph{SPI}-Package des \emph{ComponentTester}-Moduls bereit (siehe Abbildung \ref{fig_cdCompTester}).
\\\\
So muss jede Testklasse eine Methode bereitstellen, über die ein Objekt vom Typ $R$ in die Instanz der Testklasse injiziert werden kann.\footnote{auch genannt: Setter-Injection (vgl. \cite{setterinfjection})} Diese Methode wird von dem \emph{ComponentTester}-Modul über die Annotation $\texttt{RequiredTypeInstanceSetter}$ gefunden. Von daher muss die Methode mit eben dieser Annotation markiert werden.
\begin{figure}[h!]
\centering
\includegraphics[scale=0.6]{pics/cd_ComponentTester.png}
\caption{Modul: ComponentTester}
\label{fig_cdCompTester}
\end{figure}
\noindent
Die Testmethoden müssen von der Sichtbarkeit her öffentlich ($\texttt{public}$) sein. Weiterhin dürfen die Testmethoden keine Parameter erwarten und müssen mit der Annotation $\texttt{RequiredTypeTest}$ markiert sein. Die Erwartungen innerhalb der Testmethoden müssen über die in JUnit 4 zur Verfügung stehenden Methoden aus der Klasse $\texttt{Assert}$ (vgl. \cite{junit_api}) deklariert werden. Testdaten, die für alle Testmethoden innerhalb einer Testklasse zur Verfügung stehen sollen, können diese innerhalb von Methoden erzeugt werden, die mit den in JUnit 4 bereitgestellten Annotationen $\texttt{Before}$ und $\texttt{After}$ (vgl. \cite{junit_api}) markiert wurden.
\\\\
Um die Reihenfolge der versuchten Aufrufe der Methoden, die von $R$ angeboten werden, zu verwalten, muss die Testklasse das Interface $\texttt{TriedMethodCallsInfo}$ implementieren (siehe Abbildung \ref{fig_cdCompTester} Package: \emph{spi}). Dadurch wird die Implementierung der Methoden $\texttt{addTriedMethodCall}$ und $\texttt{getTriedMethodCalls}$ erzwungen. Die Methode $\texttt{getMethod}$ kann mit der Defaultimplementierung übernommen werden, sofern die in $R$ deklarierten Methoden über den Namen identifiziert werden können.
\\\\
Die Implementierung der Methoden $\texttt{addTriedMethodCall}$ und $\texttt{getTriedMethodCalls}$ hat so zu erfolgen, dass bei einem Aufruf der Methode $\texttt{addTriedMethodCall}$ der übergebene Parameter an eine Liste angefügt wird. Der Aufruf der Methode $\texttt{getTriedMethodCalls}$ liefert eben diese Liste als Rückgabewert. Weiterhin ist sicherzustellen, dass vor dem Aufruf einer Methode $m$ aus $R$ die Methode $\texttt{addTriedMethodCall}$ mit $m$ als Parameter aufgerufen wird. Im Anhang ist ein Beispiel für die korrekte Implementierung einer Testklasse zu finden.
%TODO Beispiel im Anhang
\\\\
Der Test eines Proxies für $R$ wird über eine Instanz der Klasse $\texttt{ComponentTester}$ gestartet (siehe Abbildung \ref{fig_cdCompTester} Package: \emph{Tester}). In Abhängigkeit der in $R$ deklarierten Testklassen werden alle darin befindlichen Testmethoden durchgeführt, bis einer dieser Testfälle fehlschlägt. Der Aufrufer erhält dabei ein Objekt der Klasse $\texttt{TestResult}$ zurück (siehe Abbildung \ref{fig_cdCompTester}). In diesem Objekt sind die für die Auswertung des Testergebnisses relevanten Informationen vorhanden, auf die die Heuristiken \emph{PTTF} (siehe Abschnitt \ref{sec_pttf}) und \emph{BL\_NMC} (siehe Abschnitt \ref{sec_bl_nmc}) angewiesen sind.
\section{Modul: DesiredComponentSourcerer}\label{sec_impl_descos}
In diesem Modul ist die Implementierung der Exploration zu finden. Zum Starten der Exploration für ein \emph{required Typ} $R$ in Form eines Interfaces muss zuerst eine Instanz der Klasse $\texttt{DesiredComponentFinder}$ erzeugt werden (genannt: \emph{Finder}). Dies erfolgt über einen Konstruktor, der ein Objekt der Klasse $\texttt{DesiredComponentFinderConfig}$ (genannt: \emph{Konfig}) erwartet (siehe Abbildung \ref{cd_descos}). 
\begin{figure}[h!]
\centering
\includegraphics[scale=0.5]{cd_descos.png}
\caption{Modul: DesiredComponentSourcerer}
\label{cd_descos}
\end{figure}
\noindent
Die Erzeugung einer solchen \emph{Konfig} erfolgt über einen Builder. Dabei müssen zum Einen die Angabe aller \emph{provided Typen} in Form einer Liste von Interfaces. Zum Anderen wird eine Funktion ($\texttt{java.util.Function}$) gefordert, über die die Implementierungen der im Parameter übergebenen Interfaces ermittelt werden können.
\\\\
Zum Zweck der gezielten Evaluation der Heuristiken in Kapitel \ref{chap_evaluation} kann über die \emph{Konfig} gesteuert werden, welche der in Abschnitt \ref{sec_heuristics} beschriebenen Heuristiken bei der Exploration verwendet werden sollen. Dies erfolgt über die in Abbildung \ref{cd_finderCreation} ersichtlichen Methoden mit den Präfix $\texttt{useHeuristic*}$.
\\\\
Nach der Erzeugung des \emph{Finders} kann die Exploration über die Methode $\texttt{getDesiredComponent}$ mit der Übergabe des \emph{desired Interface} $R$ als Parameter gestartet werden. Im Anschluss wird die syntaktische Evaluation für alle \emph{provided Interfaces} durchgeführt. Auf formaler ebene gleicht dieser Schritt der Ausführung der Funktion $\mathit{cover(R,L)}$, wobei die in $L$ befindlichen \emph{provided Typen} auf die an der \emph{Finder} übergebenen \emph{provided Interfaces} beschränkt sind.
\\\\
Hierzu wird ein Objekt vom $\texttt{StructuralTypeMatcher}$ aus dem \emph{SignatureMatching}-Modul verwendet\footnote{Dieses Objekt wird beim Instanziieren des \emph{Finders} erzeugt.} und versucht die \emph{provided Typ} mit dem \emph{required Typ} zu matchen (siehe Abbildung \ref{sd_descos_structeval}).
\\\\
Nach der syntaktischen Evaluation, wird gemäß Abschnitt \ref{sec_semEval} die semantische Evaluation durchgeführt. Dabei werden zuerst die Proxies aus den Kombinationen der gematchten \emph{provided Typ}\footnote{Diese Kombinationen sind mit den Elementen der Mengen aus $\mathit{cover(R,L)}$ gleichzusetzen.} erzeugt, welche im Anschluss hinsichtlich der vordefinierten Tests zum \emph{required Typ} evaluiert werden. Dabei werden die Heuristiken, die in der \emph{Konfig} hinterlegt wurden, angewendet. Sofern bei der Exploration ein Proxy erfolgreich evaluiert wurde, wird dieser als Ergebnis des Aufrufs der Methode $\texttt{getDesiredComponent}$ zurückgegeben. 

\chapter{Evaluierung}
Die Evaluierung erfolgt mit Systemen, in denen mindestens 889 angebotene Interfaces existieren. Es wird zwischen einem Test-System und einem Hei�-System unterschieden.\\\\
Das Test-System wurde vorrangig f�r die Evaluation der Type-Matcher Rating basierten Heuristiken verwendet, da f�r diese Heuristiken keine Implementierungen der angebotenen Interfaces vorliegen m�ssen.\\\\
Das Hei�-System wurde vorrangig f�r die Evaluation der testergebnis basierten Heuristiken verwendet, da hier zu jedem der 889 angebotenen Interfaces eine Implementierung existiert. Die angebotenen Komponenten wurden  im Hei�-System als Java Enterprise Beans umgesetzt.
\section{Test-System}
Wie bereits erw�hnt werden im Test-System die 889 angebotenen Interfaces verwendet, die auch im Hei�-System verwendet werden. Dar�ber hinaus wurden noch 6 weitere angebotene Interfaces dem Test-System hinzugef�gt, um bestimmte Konstellationen gezielter zu evaluieren. Damit stehen in diesem System f�r die Evaluation der insgesamt wurde 6 erwartete Interfaces f�r die Evaluation verwendet. Die 6 erwarteten Interfaces wurden wie folgt deklariert (siehe Abbildung 16 - 21).
\subsection{Type-Matcher Rating basierte Heuristiken}
\subsubsection{Ausgangspunkt}
F�r ein erwarteten Interfaces konnten mehrere angebotene Interfaces gefunden werden, die eine strukturelle �bereinstimmung aufwiesen. Tabelle 1 zeigt die Anzahl der strukturell �bereinstimmenden angebotenen Interfaces je erwartetes Interface.
\begin{table}[H]
\centering
\small
\singlespacing
			\begin{tabular}[c]{|>{\centering\arraybackslash}p{2cm}|>{\centering\arraybackslash}p{5cm}|}
			\hline
			\hline
				 \textbf{erwartetes Interface} & \textbf{Anzahl strukturell �bereinstimmender angebotener Interfaces} \\
				\hline\hline
				TEI1 & 169 \\
				\hline
				TEI2 & 179\\
				\hline
				TEI3 & 187\\
				\hline
				TEI4 & 62\\
				\hline
				TEI5 & 60\\
				\hline
				TEI6 & 33\\
				\hline
				\hline
			\end{tabular} 
 \caption{Anzahl strukturell �bereinstimmender angebotener Interfaces je erwartetes Interfaces}
 \label{tab:amountMatchedInterfaces}
\onehalfspacing
\end{table}
\noindent
Die \tabsrefs{tmr_start_tei1}{tmr_start_tei6_2} zeigen Vier-Felder-Tafeln f�r die Durchl�ufe des Explorationsalgorithmus f�r die Suche nach einer jeweils passenden ben�tigten Komponente der erwarteten Interfaces aus \tabref{amountMatchedInterfaces} ohne die Verwendung von Heuristiken. Dies stellt somit den Ausgangspunkt f�r die weitere Evaluation dar.
\begin{multicols}{3}
\vft{1}{$mk(169)$}{0}{1}{0}{Ausgangspunkt Test-System TMR f�r TEI1}{tmr_start_tei1}\columnbreak
\vft{1}{$mk(179)$}{0}{1}{0}{Ausgangspunkt Test-System TMR f�r TEI2}{tmr_start_tei2}\columnbreak
\vft{1}{$mk(187)$}{0}{1}{0}{Ausgangspunkt Test-System TMR f�r TEI3}{tmr_start_tei3}
\end{multicols}
\begin{multicols}{3}
\vft{1}{$mk(62)$}{0}{0}{0}{Ausgangspunkt Test-System TMR f�r TEI4 1. Durchlauf}{tmr_start_tei4_1}\columnbreak
\vft{1}{$mk(60)$}{0}{0}{0}{Ausgangspunkt Test-System TMR f�r TEI5 1. Durchlauf}{tmr_start_tei5_1}\columnbreak
\vft{1}{$mk(33)$}{0}{0}{0}{Ausgangspunkt Test-System TMR f�r TEI6 1. Durchlauf}{tmr_start_tei6_1}
\end{multicols}
\begin{multicols}{3}
\vft{2}{$mk(1891)$}{0}{1}{0}{Ausgangspunkt Test-System TMR f�r TEI4 2. Durchlauf}{tmr_start_tei4_2}\columnbreak
\vft{2}{$mk(1770)$}{0}{1}{0}{Ausgangspunkt Test-System TMR f�r TEI5 2. Durchlauf}{tmr_start_tei5_2}\columnbreak
\vft{2}{$mk(528)$}{0}{1}{0}{Ausgangspunkt Test-System TMR f�r TEI6 2. Durchlauf}{tmr_start_tei6_2}
\end{multicols}
\noindent
F�r die Interfaces TEI4 - TEI6 werden zwei Durchl�ufe ben�tigt, da die semantischen Test nur von einer ben�tigten Komponente bestanden werden, die auch einer Kombination zweier Typ-Konvertierungsvarianten erzeugt wurde.\\\\
Die Typ-Matcher Rating basierten Heuristiken sollten zu einer Reduktion der Anzahl von erzeugten Kombinationen von Methoden-Konvertierungsvarianten, die die semantischen Tests nicht bestehen w�rden, f�hren (positiv und falsch).
\subsubsection{Ergebnisse TMR\_Quant}
Durch die Verwendung der Heuristik TMR\_Quant kann f�r die ersten 3 erwarteten Interfaces eine Besserung erzielt werden. Der Grund daf�r ist, dass die ben�tigte Komponente, die letztendlich alle semantischen Tests besteht auf der Basis genau einer Typ-Konvertierungsvariante erzeugt wurde. Damit ben�tigt der Explorationsalgorithmus lediglich einen Durchlauf. TMR\_Quant sorgt dennoch daf�r, dass die erzeugten Kombinationen von Typ-Konvertierungsvarianten im 2. Schritt reduziert werden, da solche, die ein quantitatives Type-Matcher Rating von < 100\% haben nicht in die Ergebnismenge des 2. Schrittes einflie�en. Die unten aufgef�hrten Tafeln zeigen die Auswirkung auf die ersten drei erwarteten Interfaces (TEI1 - TEI3).
\begin{multicols}{3}
\vft{1}{$mk(29)$}{$mk(140)$}{1}{0}{TMR\_Quant Test-System TMR f�r TEI1}{tmr_quant_tei1}\columnbreak
\vft{1}{$mk(22)$}{$mk(157)$}{1}{0}{TMR\_Quant Test-System TMR f�r TEI2}{tmr_quant_tei2}\columnbreak
\vft{1}{$mk(24)$}{$mk(163)$}{1}{0}{TMR\_Quant Test-System TMR f�r TEI3}{tmr_quant_tei3}
\end{multicols}
\noindent
F�r die anderen erwarteten Interfaces (TEI4 - TEI6) kann durch diese Heuristik h�chstens f�r den ersten Durchlauf eine eine Verbesserung erzielen. Die unteren Tafeln zeigen, dass sich diese Verbesserung signifikant nur auf das erwartete Interface TEI4 auswirkt.
\begin{multicols}{3}
\vft{1}{$mk(30)$}{$mk(32)$}{0}{0}{TMR\_Quant Test-System TMR f�r TEI4}{tmr_quant_tei4}\columnbreak
\vft{1}{$mk(30)$}{0}{0}{0}{TMR\_Quant Test-System TMR f�r TEI5}{tmr_quant_tei5}\columnbreak
\vft{1}{$mk(31)$}{$mk(2)$}{0}{0}{TMR\_Quant Test-System TMR f�r TEI6}{tmr_quant_tei6}
\end{multicols}
\subsubsection{Ergebnisse TMR\_Qual}
F�r die Heuristik TMR\_Qual gibt es drei Aspekte, deren Konfiguration zu unterschiedlichen Ergebnissen f�hren kann:
\begin{enumerate}
\item Die Wahl des Basiswertes der einzelnen Type-Matcher
\item Die Ermittlung des akkumulierten qualitativen Type-Matcher Ratings einer Typ-Konvertierungsvariante
\item Die Ermittlung des akkumulierten qualitativen Type-Matcher Ratings einer Methoden-Konvertierungsvariante
\end{enumerate}
\myparagraph{Auswahl der Basiswerte} 
Die Basiswerte wurden bei den Untersuchungen konstant gelassen. Die konkreten Basiswerte, die f�r die Untersuchungen verwendet wurden, sind der Tabelle 2 zu entnehmen.\\\\
Die Werte bilden meiner Meinung nach die Wertigkeit der einzelnen Type-Matcher in Hinblick auf die Typisierung innerhalb der Sprache Java ab. So ist der ExactTypeMatcher, der nur zwei identische Typen als �bereinstimmend bewertet, mit dem niedrigsten Wert und damit der h�chsten Qualit�t hinsichtlich TMR\_Qual zu konfigurieren. Gleich dahinter folgt der GenSpecTypeMatcher, der Typen als �bereinstimmend bewertet, wenn sie innerhalb der Sprache auch miteinander substituiert werden k�nnen. An dritter Stelle kommt meiner Meinung nach der WrappedTypeMatcher, da dieser immerhin eine vollst�ndige �bereinstimmung von Typen fordert (auch wenn ein Typen innerhalb eines anderes enthalten ist), w�hrend der StructuralTypeMatcher lediglich einen Teil der deklarierten Methoden f�r eine �bereinstimmung fordert.
\begin{table}[H]
\centering
\small
			\begin{tabular}[c]{|c|c|}
			\hline
			\hline
				 \textbf{Type-Matcher} & \textbf{Basiswert} \\
				\hline\hline
				ExactTypeMatcher & 100 \\
				\hline
				ExactTypeMatcher & 200\\
				\hline
				WrappedTypeMatcher & 300\\
				\hline
				StructuralTypeMatcher & 400\\
				\hline
				\hline
			\end{tabular} 
 \caption{Type-Matcher mit Basiswerten
}
 \label{tab_basevalues}
\end{table}
\noindent
\myparagraph{Auswahl des Akkumulationsverfahrens des Type-Matcher Ratings einer\\Typ-Konvertierungsvariante bzw. Methoden-Konvertierungsvariante}
Das Akkumulationsverfahren f�r das qualitative Type-Matcher Rating einer Typ-Konvertierungsvariante  $TMR_{TK}$ ist von dem Type-Matcher Rating der verwendeten Type-Matcher abh�ngig. Das Akkumulationsverfahren f�r das qualitative Type-Matcher Rating einer Methoden-Konvertierungsvariante  $TMR_{MK}$ ist von dem qualitativen Type-Matcher Rating der verwendeten Type-Matcher f�r den R�ckgabe- und den Parametertypen der Methode abh�ngig abh�ngig. Somit kann das qualitative Type-Matcher Rating als Funktion von einer Typ- bzw. Methoden-Konvertierungsvariante $tmr_{Qual}(v)$ beschrieben werden.
Das Type-Matcher Rating der verwendeten Type-Matcher wird als Funktion $tmr_{Base}(m)$ beschrieben. Dabei stellt m den jeweiligen Type-Matcher dar. Die Funktion $tmr_{Base}(m)$ ist durch die Tabelle 2 definiert.\\\\
F�r einen Menge von Type-Matcher $m_1, m_2, ..., m_i$, die zur Erzeugung einer Typ-Konvertierungsvariante bzw. Methoden-Konvertierungsvariante $v$ verwendet wurden, werden folgende Akkumulationsverfahren f�r das Type-Matcher Rating der Typ-Konvertierungsvariante bzw. Methoden-Konvertierungsvariante im weiteren Verlauf evaluiert:
\begin{enumerate}
\item Wahl des Durchschnitts
\begin{equation*}
tmr_{Qual}(v) = \frac{ \sum_{n=1}^{i} tmr_{Base}(m_n)}{i}
\end{equation*}
\item Wahl des Maximums
\begin{equation*}
tmr_{Qual}(v) = max(tmr_{Base}(m_1), ..., tmr_{Base}(m_i))
\end{equation*}
\item Wahl des Minimums
\begin{equation*}
tmr_{Qual}(v) = min(tmr_{Base}(m_1), ..., tmr_{Base}(m_i))
\end{equation*}
\item Wahl des Durchschnitts aus Minimum und Maximum
\begin{equation*}
tmr_{Qual}(v) = \frac{min(tmr_{Base}(m_1), ..., tmr_{Base}(m_i)) +  max(tmr_{Base}(m_1), ..., tmr_{Base}(m_i))}{2}
\end{equation*}

\end{enumerate}
\noindent
Die folgenden Abschnitte stellen eine Auswahl der Ergebnisse hinsichtlich der Kombinationen der oben genannten Akkumulationsverfahren dar. Die Ergebnisse von Kombinationen, deren Ergebnisse nicht dargestellt wurden, sind mit den Ergebnissen einer der dargestellten Kombinationen gleichzusetzen. An entsprechender Stelle wird darauf verwiesen.\\\\
An den �berschriften der folgenden Abschnitte ist abzulesen, welche Akkumulationsverfahren miteinander kombiniert wurden. Dabei haben die �berschriften die Form ``Typ: T Methoden: M'' wobei f�r ``T'' und ``M'' die Nummern der oben genannten Akkumulationsverfahren eingesetzt werden.
\myparagraph{Typ: 1 Methoden: 2}\label{tmrquant_1-2}
\begin{multicols}{3}
\vft{1}{$mk(48)$}{$mk(121)$}{1}{0}{TMR\_Qual Test-System TMR f�r TEI1 mit 1-2}{tmr_qual_2_2_tei1}\columnbreak

\vft{1}{$mk(47)$}{$mk(132)$}{1}{0}{TMR\_Qual Test-System TMR f�r TEI2 mit 1-2}{tmr_qual_2_2_tei2}\columnbreak
\vft{1}{$mk(46)$}{$mk(141)$}{1}{0}{TMR\_Qual Test-System TMR f�r TEI3 mit 1-2}{tmr_qual_2_2_tei3}
\end{multicols}

\begin{multicols}{3}
\vft{1}{$mk(62)$}{0}{0}{0}{TMR\_Qual Test-System TMR f�r TEI4 mit 1-2 1. Durchlauf}{tmr_qual_2_2_tei4_1}\columnbreak
\vft{1}{$mk(60)$}{0}{0}{0}{TMR\_Qual Test-System TMR f�r TEI5 mit 1-2 1. Durchlauf}{tmr_qual_2_2_tei5_1}\columnbreak
\vft{1}{$mk(33)$}{0}{1}{0}{TMR\_Qual Test-System TMR f�r TEI6 mit 1-2 1. Durchlauf}{tmr_qual_2_2_tei6_1}
\end{multicols}
\newpage
\begin{multicols}{3}
\vft{2}{$mk(1)$}{$mk(1890)$}{1}{0}{TMR\_Qual Test-System TMR f�r TEI4 mit 1-2 2. Durchlauf}{tmr_qual_2_2_tei4_2}\columnbreak
\vft{2}{$mk(1)$}{$mk(1769)$}{1}{0}{TMR\_Qual Test-System TMR f�r TEI5 mit 1-2 2. Durchlauf}{tmr_qual_2_2_tei5_2}\columnbreak
\vft{2}{$mk(1)$}{$mk(527)$}{1}{0}{TMR\_Qual Test-System TMR f�r TEI6 mit 1-2 2. Durchlauf}{tmr_qual_2_2_tei6_2}
\end{multicols}


\myparagraph{Typ: 3 Methoden: 2}\label{tmrquant_3-2}
\begin{multicols}{3}
\vft{1}{$mk(49)$}{$mk(120)$}{1}{0}{TMR\_Qual Test-System TMR f�r TEI1 mit 3-2}{tmr_qual_3_2_tei1}\columnbreak
\vft{1}{$mk(49)$}{$mk(130)$}{1}{0}{TMR\_Qual Test-System TMR f�r TEI2 mit 3-2}{tmr_qual_3_2_tei2}\columnbreak
\vft{1}{$mk(48)$}{$mk(139)$}{1}{0}{TMR\_Qual Test-System TMR f�r TEI3 mit 3-2}{tmr_qual_3_2_tei3}
\end{multicols}


\begin{multicols}{3}
\vft{1}{$mk(62)$}{0}{0}{0}{TMR\_Qual Test-System TMR f�r TEI4 mit 3-2 1. Durchlauf}{tmr_qual_3_2_tei4_1}\columnbreak
\vft{1}{$mk(60)$}{0}{0}{0}{TMR\_Qual Test-System TMR f�r TEI5 mit 3-2 1. Durchlauf}{tmr_qual_3_2_tei5_1}\columnbreak
\vft{1}{$mk(33)$}{0}{1}{0}{TMR\_Qual Test-System TMR f�r TEI6 mit 3-2 1. Durchlauf}{tmr_qual_3_2_tei6_1}
\end{multicols}

\newpage
\begin{multicols}{3}
\vft{2}{$mk(1)$}{$mk(1890)$}{1}{0}{TMR\_Qual Test-System TMR f�r TEI4 mit 3-2 2. Durchlauf}{tmr_qual_3_2_tei4_2}\columnbreak
\vft{2}{$mk(1)$}{$mk(1769)$}{1}{0}{TMR\_Qual Test-System TMR f�r TEI5 mit 3-2 2. Durchlauf}{tmr_qual_3_2_tei5_2}\columnbreak
\vft{2}{$mk(1)$}{$mk(527)$}{1}{0}{TMR\_Qual Test-System TMR f�r TEI6 mit 3-2 2. Durchlauf}{tmr_qual_3_2_tei6_2}
\end{multicols}

\myparagraph{Typ: 4 Methoden: 3}\label{tmrquant_4-3}
\begin{multicols}{3}
\vft{1}{$mk(52)$}{$mk(117)$}{1}{0}{TMR\_Qual Test-System TMR f�r TEI1 mit 4-3}{tmr_qual_4_3_tei1}\columnbreak
\vft{1}{$mk(62)$}{$mk(117)$}{1}{0}{TMR\_Qual Test-System TMR f�r TEI2 mit 4-3}{tmr_qual_4_3_tei2}\columnbreak
\vft{1}{$mk(62)$}{$mk(125)$}{1}{0}{TMR\_Qual Test-System TMR f�r TEI3 mit 4-3}{tmr_qual_4_3_tei3}
\end{multicols}

\begin{multicols}{3}
\vft{1}{$mk(62)$}{0}{0}{0}{TMR\_Qual Test-System TMR f�r TEI4 mit 4-3 1. Durchlauf}{tmr_qual_4_3_tei4_1}\columnbreak
\vft{1}{$mk(60)$}{0}{0}{0}{TMR\_Qual Test-System TMR f�r TEI5 mit 4-3 1. Durchlauf}{tmr_qual_4_3_tei5_1}\columnbreak
\vft{1}{$mk(33)$}{0}{1}{0}{TMR\_Qual Test-System TMR f�r TEI6 mit 4-3 1. Durchlauf}{tmr_qual_4_3_tei6_1}
\end{multicols}
\newpage
\begin{multicols}{3}
\vft{2}{$mk(1891)$}{0}{1}{0}{TMR\_Qual Test-System TMR f�r TEI4 mit 4-3 2. Durchlauf}{tmr_qual_4_3_tei4_2}\columnbreak
\vft{2}{$mk(1770)$}{0}{1}{0}{TMR\_Qual Test-System TMR f�r TEI5 mit 4-3 2. Durchlauf}{tmr_qual_4_3_tei5_2}\columnbreak
\vft{2}{$mk(528)$}{0}{1}{0}{TMR\_Qual Test-System TMR f�r TEI6 mit 4-3 2. Durchlauf}{tmr_qual_4_3_tei6_2}
\end{multicols}
\noindent
Die \tabref{akkuverfahren} zeigt durch die Markierung mit einem ``x'', welche Kombinationen der oben genannten Akkumulationsverfahren hinsichtlich der Testergebnisse mit denen gleichzusetzen sind, die oben ausf�hrlich aufgef�hrt wurden. Die Kombinationen werden in der Tabelle �hnlich wie in den vorherigen �berschriften beschrieben. Die Notation ``1-4'' beschreibt die Kombination des 1. Akkumulationsverfahrens f�r die Typ-Konvertierungsvarianten und den 4. Akkumulationsverfahrens f�r die Methoden-Konvertierungsvarianten.
\begin{table}[H]
\centering
\begin{tabular}[c]{|c|c|c|c|}
\hline\hline
\textbf{Kombination} & \textbf{1-2} & \textbf{3-2} & \textbf{4-3} \\
\hline
1-1 & x& & \\
\hline
1-3 & & & x\\
\hline
1-4 & x& & \\
\hline
2-1 & x& & \\
\hline
2-2 & x& & \\
\hline
2-3 & & &x \\
\hline
2-4 & x& & \\
\hline
3-1 & &x & \\
\hline
3-3 & & & x\\
\hline
3-4 & &x & \\
\hline
4-1 & x& & \\
\hline
4-2 & x& & \\
\hline
4-4 & x& & \\
\hline\hline
\end{tabular}
\caption{Kombinationen von Akkumulationsverfahren mit gleichen Ergebnissen}
\label{tab:akkuverfahren}
\end{table}

Aus diesen Ergebnissen l�sst sich folgendes ableiten:
\begin{enumerate}
\item Das Akkumulationsverfahren Nummer 3. (Minimum) f�hrt sowohl f�r die Typ- und Methoden-Konvertierungsvarianten zu schlechteren Ergebnissen als die anderen drei Akkumulationsverfahren. Es sollte daher f�r die Heuristik TMR\_Quant nicht verwendet werden.
\item Die Ergebnisse von 1-2 und 3-2 unterscheiden sich nur geringf�gig, obwohl bei 3-2 das Akkumulationsverfahren Nummer 3. zum Einsatz kam. Dies konnte auch bei anderen Kombinationen festgestellt werden, bei denen das 3. Akkumulationsverfahren f�r die Akkumulation des Type-Matcher Ratings der Typ-Konvertierungsvariante verwendet wurde. Das l�sst vermuten, dass die Beachtung des Type-Matcher Ratings einer ganzen Typ-Konvertierungsvariante weitgehend unerheblich f�r die Heuristik TMR\_Quant ist, wenn das Type-Matcher Rating je Methoden-Konvertierungsvarianten �ber ein entsprechend gutes Akkumulationsverfahren ermittelt wurde. Dies ist jedoch darauf zur�ckzuf�hren, dass das Type-Matcher Rating je Methoden-Konvertierungsvariante die Parameter f�r die Ermittlung des Type-Matcher Ratings einer Typ-Konvertierungsvariante darstellen.
\item An den Ergebnissen zu den erwarteten Interfaces TEI4-TEI6 ist zu erkennen, dass die Heuristik TMR\_Quant keinen Einfluss auf den 1. Durchlauf hat. Daraus kann geschlussfolgert werden, dass die Heuristik nur in dem Durchlauf einen Gewinn bringt, in dem auch eine passende ben�tigte Komponente gefunden werden kann. 
\end{enumerate}
Aufgrund der Ergebnisse stehen f�r die weitere Verwendung der Heuristik TMR\_Qual mehrere Kombinationen von Akkumulationsverfahren zur Auswahl. Die Entscheidung f�llt aufgrund der etwas geringeren Komplexit�t auf die Kombination 1-2. 

\myparagraph{TMR\_Quant und TMR\_Qual in Kombination}
Bei der Kombination der beiden Heuristiken TMR\_Quant und TMR\_Qual ist vor allem f�r die erwarteten Interfaces TEI4-TEI6 eine weitere Verbesserung zu erwarten. Der Grund daf�r ist, dass die Heuristik TMR\_Qual keinen Einfluss auf den ersten Durchlauf des Explorationsalgorithmus f�r diese erwarteten Interfaces hat, die Heuristik TMR\_Quant hingegen schon. Die \tabsrefs{tmr_quant+qual:tei1}{tmr_quant+qual:tei6_2} zeigen wiederum die bekannten Vier-Felder-Tafeln f�r den jeweiligen Durchlauf und dem jeweiligen erwarteten Interface.
\begin{multicols}{3}
\vft{1}{$mk(2)$}{$mk(167)$}{1}{0}{TMR\_Quant + TMR\_Qual Test-System TMR f�r TEI1}{tmr_quant+qual:tei1}\columnbreak
\vft{1}{$mk(2)$}{$mk(177)$}{1}{0}{TMR\_Quant + TMR\_Qual Test-System TMR f�r TEI2}{tmr_quant+qual:tei2}\columnbreak
\vft{1}{$mk(1)$}{$mk(186)$}{1}{0}{TMR\_Quant + TMR\_Qual Test-System TMR f�r TEI3}{tmr_quant+qual:tei3}
\end{multicols}
\newpage
\begin{multicols}{3}
\vft{1}{$mk(30)$}{$mk(32)$}{0}{0}{TMR\_Quant + TMR\_Qual Test-System TMR f�r TEI4 1. Durchlauf}{tmr_quant+qual:tei4_1}\columnbreak
\vft{1}{$mk(60)$}{0}{0}{0}{TMR\_Quant + TMR\_Qual Test-System TMR f�r TEI5 1. Durchlauf}{tmr_quant+qual:te5_1}\columnbreak
\vft{1}{$mk(31)$}{$mk(2)$}{0}{0}{TMR\_Quant + TMR\_Qual Test-System TMR f�r TEI6 1. Durchlauf}{tmr_quant+qual:tei6_1}
\end{multicols}

\begin{multicols}{3}
\vft{1}{$mk(1)$}{$mk(1890)$}{1}{0}{TMR\_Quant + TMR\_Qual Test-System TMR f�r TEI4 2. Durchlauf}{tmr_quant+qual:tei4_2}\columnbreak
\vft{1}{$mk(1)$}{$mk(1769)$}{1}{0}{TMR\_Quant + TMR\_Qual Test-System TMR f�r TEI5 2. Durchlauf}{tmr_quant+qual:te5_2}\columnbreak
\vft{1}{$mk(1)$}{$mk(527)$}{1}{0}{TMR\_Quant + TMR\_Qual Test-System TMR f�r TEI6 2. Durchlauf}{tmr_quant+qual:tei6_2}
\end{multicols}

\noindent
Wie man diesen Ergebnissen zu erkennen ist, wird der Explorationsalgorithmus f�r eine Suche nach einer passenden ben�tigten Komponente f�r TEI1 und TEI2 lediglich f�r zwei angebotene Interfaces bzw. Typ-Konvertierungsvarianten durchlaufen. In Bezug auf TEI3 ist es sogar nur noch eine Typ-Konvertierungsvariante. F�r diese erwarteten Interfaces erf�llen die Heuristiken die Erwartungen.\\\\
Bei der Betrachtung der Ergebnisse f�r die erwarteten Interfaces TEI4-TEI6 zeigt sich gut, wie sich die beiden Heuristiken gegenseitig erg�nzen. So wirkt die Heuritik TMR\_Quant grunds�tzlich nur auf den ersten Durchlauf des Explorationsalgorithmus aus. Die Heuristik TMR\_Qual hingegen erweist ihre St�rke erst in dem Durchlauf, in dem auch eine passende ben�tigte Komponente gefunden wird. Die Evaluationsergebnisse best�tigen also auch hier die Annahmen.\\\\
Im Allgemeinen kann festgehalten werden, dass die passenden ben�tigten Komponenten trotz der Kombination der beiden Heuristiken gefunden werden konnten. Die Reduktion der notwendigen Durchl�ufe des Explorationsalgorithmus ist jedoch haupts�chlich auf die Heuristik TMR\_Qual zur�ckzuf�hren. 



\section{Heiß-System}\label{sec_hotsystem}
Im Heiß-System werden, wie beschrieben, die 889 \emph{provided Typen} verwendet, für die jeweils mindestens eine Implementierung bereitgestellt wurde. Die \emph{provided Typen}, die im Test-System ergänzt wurde, befinden sich nicht im Heiß-System.
\\\\
Für die Evaluation im Test-System werden insgesamt 4 \emph{required Typen} verwendet. Drei dieser \emph{required Typen} wurden ebenfalls im zur Evaluation im Test-System verwendet. Die Deklaration der \emph{required Typen} ist im Anhang \ref{xyz} zu finden.  
\\\\
Die Java-Interfaces, die sich aus dieser Deklaration ableiten lassen, die dazugehörigen Implementierungen der \emph{provided Typen} und die vordefinierten Testfälle sind in Anhang \ref{abc} zu finden.
%\\\\
In \tabref{eIMainShort} sind die Namen der \emph{required Typen} zusammen mit jeweils einem Kürzel aufgeführt. Die Kürzel dienen im weiteren Verlauf der Identifizierung der \emph{required Typen}.
\begin{table}[h!]
\centering
\small
\begin{tabular}{|l|c|}
\hline
\hline
\centering\textbf{required Typ} & \textbf{Kürzel} \\
\hline
\hline
ElerFTFoerderprogrammeProvider & TEI1\\
\hline
FoerderprogrammeProvider & TEI2\\
\hline
MinimalFoerderprogrammeProvider & TEI3\\
\hline
KOFGPCProvider & TEI4\\
\hline
\hline
\end{tabular}
\caption{Kürzel der required Typen für die Evaluation im Heiß-System}
 \label{tab:eIMainShort}
\end{table}
\noindent
%Neben den in Abschnitt \ref{sec_matcher} und \ref{sec_proxies} beschriebenen Matcher und Proxy-Generatoren wird im Heiß-System ein zusätzlicher Matcher sowie ein darauf basierender Generator für Proxies benötigt.
%\\\\
%Der Grund dafür ist, dass im Heiß-System zur Laufzeit Objekte existieren, die über bestimmte Zeichenketten oder numerische Werte eindeutig identifiziert werden. Die Typen solcher Objekte werden im Heiß-System als Domain-Values bezeichnet.
%\\\\
%Die Semantik dieser Domain-Values soll bei der Suche nach einem passenden Proxy zu einem \emph{required Typ} nicht untergraben werden. 


\subsection{Ausgangspunkt}
Für ein \emph{reqiured Typ} können mehrere \emph{provided Typen} gefunden werden, die eine strukturelle Übereinstimmung aufwiesen. \tabref{amountMatchedInterfacesHot} zeigt die Anzahl der strukturell übereinstimmenden \emph{provided Typen} je \emph{reqiured Typ}. Diese kommen einzeln oder in Kombination für die semantische Evaluation in Frage.
\begin{table}[H]
\centering
\small
\singlespacing
			\begin{tabular}[c]{|>{\centering\arraybackslash}p{2cm}|>{\centering\arraybackslash}p{5cm}|}
			\hline
			\hline
				 \textbf{required Interface} & \textbf{Anzahl strukturell übereinstimmender provided Interfaces} \\
				\hline\hline
				TEI1 & 221 \\
				\hline
				TEI2 & 272\\
				\hline
				TEI3 & 268 \\
				\hline
				TEI4 & 348 \\
				\hline
				\hline
			\end{tabular} 
 \caption{Anzahl strukturell übereinstimmender provided Typen je required Typ im Heiß-System}
 \label{tab:amountMatchedInterfacesHot}
\onehalfspacing
\end{table}
\noindent
Die \tabsrefs{hs_start_tei1}{hs_start_tei4_2} zeigen die Vier-Felder-Tafeln, in denen die Ergebnisse der benötigten Iterationen innerhalb des Explorationsalgorithmus für jeden der \emph{required Typen} aus \tabref{amountMatchedInterfacesHot}. Dabei wurden keine Heuristiken verwendet. Somit stellt dies den Ausgangspunkt für die weitere Evaluation im Heiß-System dar.
\begin{multicols}{3}
\vft{1}{$p(44)-1$}{0}{1}{0}{Ausgangspunkt im Heiß-System für TEI1}{hs_start_tei1}\columnbreak
\vft{1}{$p(30)-1$}{0}{1}{0}{Ausgangspunkt im Heiß-System für TEI2}{hs_start_tei2}\columnbreak
\vft{1}{$p(30)-1$}{0}{1}{0}{Ausgangspunkt im Heiß-System für TEI3}{hs_start_tei3}
\end{multicols}
\begin{multicols}{2}
\vft{1}{$p(50)-1$}{0}{1}{0}{Ausgangspunkt im Heiß-System für TEI4 \\1. Durchlauf}{hs_start_tei4_1}\columnbreak
\vft{2}{$p(7714)-1$}{0}{1}{0}{Ausgangspunkt im Heiß-System für TEI4 \\2. Durchlauf}{hs_start_tei4_2}\columnbreak
\end{multicols}
Für den \emph{required Typen} \emph{TEI4} werden zwei Durchläufe benötigt, da die semantischen Test nur von einem Proxy bestanden werden, der aus einer Kombination zweier \emph{provided Typen} erzeugt wurde.

\section{Ergebnisse für die Heuristik PTTF}\label{sec_evalPTTF}
Für die \Gls{Heuristik} \emph{PTTF} gilt es zu evaluieren, ob die Suche nach einem \emph{Proxy}, der die vordefinierten Tests besteht, beschleunigt werden kann. Hierzu wird der \emph{Explorationsprozess} für alle in Tabelle \ref{tab:eIShort} genannten \emph{required Typen} unter der Verwendung der in Abschnitt \ref{sec_pttf} beschriebenen \Gls{Heuristik} durchgeführt.
\\\\
Die folgenden Vier-Felder-Tafeln zeigen die Ergebnisse für die \emph{required Typen} \emph{TEI1}-\emph{TEI7} auf.
\begin{multicols}{3}
\vft{1}{29}{$p_1(44)-30$}{1}{0}{Ergebnisse \emph{PTTF} für TEI1 1.~\mbox{Durchlauf}}{pttf_TEI1_1}
\vft{1}{5544}{$p_1(55)-5545$}{1}{0}{Ergebnisse \emph{PTTF} für TEI2 1.~\mbox{Durchlauf}}{pttf_TEI2_1}
\vft{1}{4761}{$p_1(50)-4762$}{1}{0}{Ergebnisse \emph{PTTF} für TEI3 1.~\mbox{Durchlauf}}{pttf_TEI3_1}
\end{multicols}

\begin{multicols}{2}
\vft{1}{$1174$}{0}{0}{0}{Ergebnisse \emph{PTTF} für TEI4 1.~\mbox{Durchlauf}}{pttf_TEI4_1}
\vft{2}{466}{$p_2(2247)-467$}{1}{0}{Ergebnisse \emph{PTTF} für TEI4 2.~\mbox{Durchlauf}}{pttf_TEI4_2}
\end{multicols}
\pagebreak
\begin{multicols}{2}
\vft{1}{$4984$}{0}{0}{0}{Ergebnisse \emph{PTTF} für TEI5 1.~\mbox{Durchlauf}}{pttf_TEI5_1}
\vft{2}{2172}{$p_2(2775)-2173$}{1}{0}{Ergebnisse \emph{PTTF} für TEI5 2.~\mbox{Durchlauf}}{pttf_TEI5_2}
\end{multicols}

\begin{multicols}{2}
\vft{1}{$1051$}{0}{0}{0}{Ergebnisse \emph{PTTF} für TEI6 1.~\mbox{Durchlauf}}{pttf_TEI6_1}
\vft{2}{13122}{$p_2(1323)-13123$}{1}{0}{Ergebnisse \emph{PTTF} für TEI6 2.~\mbox{Durchlauf}}{pttf_TEI6_2}
\end{multicols}

\begin{multicols}{2}
\vft{1}{$161294$}{0}{0}{0}{Ergebnisse \emph{PTTF} für TEI7 1.~\mbox{Durchlauf}}{pttf_TEI7_1}
\vft{2}{149961}{$p_2(52150)-149962$}{1}{0}{Ergebnisse \emph{PTTF} für TEI7 2.~\mbox{Durchlauf}}{pttf_TEI7_2}
\end{multicols}
\newpage
\noindent
Folgendes kann aus diesen Ergebnissen abgeleitet werden:
\begin{enumerate}
\item Die \Gls{Heuristik} \emph{PTTF} erzielt im Vergleich zum Ausgangspunkt (Abschnitt \ref{sec_ausgangspunkt}) für jeden \emph{required Typ} eine weitere Reduktion der zu prüfenden \emph{Proxies}.

\item Die Heuristik \emph{PTTF} hat keine Auswirkung auf einen Durchlauf, in dem kein \emph{Proxy} erzeugt wird, mit dem die vordefinierten Tests erfolgreich durchgeführt werden können. Dies kann durch einen Vergleich des ersten Durchlaufs für den \emph{required Typ} \emph{TEI4}-\emph{TEI7} im Ausgangspunkt (Tabelle \ref{tab:tmr_start_tei4_1}, \ref{tab:tmr_start_tei5_1}, \ref{tab:tmr_start_tei6_1} und \ref{tab:tmr_start_tei6_1}) mit dem ersten Durchlauf unter Anwendung der Heuristik (Tabellen \ref{tab:pttf_TEI4_1}, \ref{tab:pttf_TEI5_1}, \ref{tab:pttf_TEI6_1} und \ref{tab:pttf_TEI7_1}) festgestellt werden. Aus diesem Grund kommt die in Punkt 1 beschriebene Reduktion erst im jeweils letzten Durchlauf zum Tragen.
\end{enumerate}
\section{Ergebnisse für die Heuristik BL\_NMC}\label{sec_evalBLNMC}
Für die \Gls{Heuristik} \emph{BL\_NMC} gilt es zu evaluieren, ob die Suche nach einem \emph{Proxy}, der die vordefinierten Tests besteht, beschleunigt werden kann. Hierzu wird der \emph{Explorationsprozess} für alle in Tabelle \ref{tab:eIShort}genannten \emph{required Typen} unter der Verwendung der in Abschnitt \ref{sec_bl_nmc} beschriebenen \gls{Heuristik} durchgeführt.
\\\\
Die folgenden Vier-Felder-Tafeln zeigen die Ergebnisse für die \emph{required Typen} \emph{TEI1}-\emph{TEI7} auf.
\begin{multicols}{3}
\vft{1}{105}{$p_1(44)-106$}{1}{0}{Ergebnisse \emph{BL\_NMC} für TEI1 1.~\mbox{Durchlauf}}{blnmc_TEI1_1}
\vft{1}{342}{$p_1(55)-343$}{1}{0}{Ergebnisse \emph{BL\_NMC} für TEI2 1.~\mbox{Durchlauf}}{blnmc_TEI2_1}
\vft{1}{357}{$p_1(50)-358$}{1}{0}{Ergebnisse \emph{BL\_NMC} für TEI3 1.~\mbox{Durchlauf}}{blnmc_TEI3_1}
\end{multicols}

\begin{multicols}{2}
\vft{1}{120}{$1054$}{0}{0}{Ergebnisse \emph{BL\_NMC} für TEI4 1.~\mbox{Durchlauf}}{blnmc_TEI4_1}
\vft{2}{442}{$p_2(2247)-443$}{1}{0}{Ergebnisse \emph{BL\_NMC} für TEI4 2.~\mbox{Durchlauf}}{blnmc_TEI4_2}
\end{multicols}

\begin{multicols}{2}
\vft{1}{550}{$4434$}{0}{0}{Ergebnisse \emph{BL\_NMC} für TEI5 1.~\mbox{Durchlauf}}{blnmc_TEI5_1}
\vft{2}{1304}{$p_2(2775)-1305$}{1}{0}{Ergebnisse \emph{BL\_NMC} für TEI5 2.~\mbox{Durchlauf}}{blnmc_TEI5_2}
\end{multicols}
\pagebreak
\begin{multicols}{2}
\vft{1}{366}{$685$}{0}{0}{Ergebnisse \emph{BL\_NMC} für TEI6 1.~\mbox{Durchlauf}}{blnmc_TEI6_1}
\vft{2}{204}{$p_2(1323)-205$}{1}{0}{Ergebnisse \emph{BL\_NMC} für TEI6 2.~\mbox{Durchlauf}}{blnmc_TEI6_2}
\end{multicols}

\begin{multicols}{2}
\vft{1}{1051}{$160243$}{0}{0}{Ergebnisse \emph{BL\_NMC} für TEI7 1.~\mbox{Durchlauf}}{blnmc_TEI7_1}
\vft{2}{135089}{$p_2(52150)-135090$}{1}{0}{Ergebnisse \emph{BL\_NMC} für TEI7 2.~\mbox{Durchlauf}}{blnmc_TEI7_2}
\end{multicols}

Folgendes kann aus diesen Ergebnissen abgeleitet werden:
\begin{enumerate}
\item Die \Gls{Heuristik} \emph{BL\_NMC} erzielt im Vergleich zum Ausgangspunkt (Abschnitt \ref{sec_ausgangspunkt}) für jeden \emph{required Typ} eine weitere Reduktion der zu prüfenden \emph{Proxies}.

\item Die Heuristik \emph{BL\_NMC} hat das Potential jeden Durchlauf innerhalb der \emph{semantischen Evaluation} zu beschleunigen. Für den jeweils ersten Durchlauf kann dies durch einen Vergleich der Tabellen \ref{tab:tmr_start_tei1}, \ref{tab:tmr_start_tei2}, \ref{tab:tmr_start_tei3}, \ref{tab:tmr_start_tei4_1}, \ref{tab:tmr_start_tei5_1}, \ref{tab:tmr_start_tei6_1} und \ref{tab:tmr_start_tei7_1} zum Ausgangspunkt mit den Tabellen \ref{tab:blnmc_TEI1_1}, \ref{tab:blnmc_TEI2_1}, \ref{tab:blnmc_TEI3_1}, \ref{tab:blnmc_TEI4_1}, \ref{tab:blnmc_TEI5_1}, \ref{tab:blnmc_TEI6_1} und \ref{tab:blnmc_TEI7_1} festgestellt werden. Ein Vergleich der Tabelle \ref{tab:tmr_start_tei4_2}, \ref{tab:tmr_start_tei5_2}, \ref{tab:tmr_start_tei6_2} und \ref{tab:tmr_start_tei7_2} im Ausgangspunkt mit den Tabellen \ref{tab:blnmc_TEI4_2}, \ref{tab:blnmc_TEI5_2}, \ref{tab:blnmc_TEI6_2} und \ref{tab:blnmc_TEI7_2} belegt dies für den zweiten Durchlauf auf.
\end{enumerate}

\chapter{Diskussion}
In den folgenden Abschnitten werden sowohl die Untersuchungsergebnisse aus Kapitel \ref{chap_evaluation} ausgewertet als auch die Vor- und Nachteile des Ansatzes zur Exploration von \emph{EJBs} zur Laufzeit gegenüber gestellt. Darüber hinaus werden Erweiterungsmöglichkeiten bzgl. der Deklaration von \emph{required Typen} und der Matcher, sowie deren zu erwartende Auswirkung auf die Exploration beschrieben.
\section{Auswertung der Untersuchungsergebnisse}
\subsection{Einzelbetrachtung}\label{disc_einzel}
Die in Kapitel \ref{chap_evaluation} beschriebenen Untersuchungsergebnisse zeigen, dass die Heuristiken die Anzahl der zu generierenden und zu evaluierenden Proxies reduzieren. Dabei zeigt sich, dass sich die Heuristiken nicht auf alle Explorationsdurchläufe positiv auswirken. So kann für die Heuristiken \emph{LMF} und \emph{PTTF} festgehalten werden, dass diese nur in dem Durchlauf eine positive Wirkung erzielt, in dem ein passender Proxy auch gefunden wird.
\\\\
Die Heuristik \emph{BL\_NMC} hingegen wirkt sich auf jeden der durchgeführten Durchläufe aus. Dies liegt zum einen daran, dass die Menge der Informationen, auf deren Basis sie arbeitet, während eines Durchlaufs anwächst. Bei der Heuristik LMF ist dies nicht der Fall. Allerdings weist die Heuristik \emph{PTTF} ebenfalls dieses Merkmal auf.
\\\\
Ein weiterer Grund ist, dass die Heuristik \emph{BL\_NMC} dafür sorgt, dass Proxies bei der Evaluierung mitunter übersprungen werden, oder diese gar nicht erst generiert werden. Die anderen Heuristiken hingegen sorgen lediglich für eine Umsortierung der zu generierenden bzw. zu evaluierenden Proxies. Somit müssen unter der Verwendung der Heuristiken \emph{LMF} und \emph{PTTF} im Zweifelsfall alle Proxies generiert und erzeugt werden, auch wenn kein passender Proxy ausgemacht werden kann.
\\\\
Weiterhin ist festzuhalten, dass mit der Heuristik \emph{BL\_NMC} scheinbar die besten Ergebnisse erzielt werden. Eine Ausnahme bildet hier lediglich die Exploration zum required Typ $\texttt{ElerFTFoerderprogrammeProvider}$ (\emph{TEI1}). Die Ursache dafür liegt darin begründet, dass die in den Methoden von TEI1 verwendeten \emph{provided Typen} mit denen des erwarteten \emph{provided Typen}, auf dessen Basis ein passender Proxy erzeugt wird, genau übereinstimmen. Damit wird ein vergleichsweise geringes Matcherrating für das Matching dieser beiden Typen ermittelt, wodurch der Proxy sehr früh während der Exploration generiert und evaluiert wird.
\subsection{Synergien}\label{disc_synergien}
Neben der Einzelbetrachtung der Heuristiken wurden in Abschnitt \ref{sec_evalKombis} auch die Kombinationen der drei Heuristiken untersucht. Aus den Feststellungen in Abschnitt \ref{disc_einzel} lässt sich ableiten, dass eine Kombination mit der Heuristik \emph{BL\_NMC} durchaus sinnvoll ist, egal ob sie mit der Heuristik \emph{LMF} oder \emph{PTTF} kombiniert wird. Der Grund dafür liegt wiederum in der Tatsache, dass die Heuristiken \emph{LMF} und \emph{PTTF} lediglich auf einen der Explorationsdurchläufe einen positiven Effekt haben. Aus diesem Grund kann in Kombination mit der Heuristik \emph{BL\_NMC} wenigstens in den anderen Durchläufen eine positive Auswirkung festgestellt werden.
\\\\
Dementgegen liefert die Kombination der Heuristiken \emph{LMF} und \emph{PTTF} miteinander kaum bessere Ergebnisse als die Heuristik LMF alleine. Eine Ausnahme bildet der required Typ $\texttt{KOFGPCProvider}$ (\emph{TEI7}). Dazu ist jedoch zu sagen, dass gerade zu diesem \emph{required Typ} im Vergleich zu den anderen required Typen die meisten matchenden \emph{provided Typen} existieren. Insofern darf dieser scheinbare Ausreißer nicht unterschätzt werden, weshalb auch die Kombination der oben genannten Heuristiken sinnvoll ist.
\\\\
Ähnliches gilt für die Kombination aller vorgestellten Heuristiken (\emph{LMF + PTTF + BL\_NMC}). Dies ergibt sich ebenfalls aus den vorherigen Auswertungen bzgl. der Synergien in diesem Abschnitt. Bei der Betrachtung der Untersuchungsergebnisse zeigt sich hier ein ähnliches Muster wie zuvor: Die Kombination aller vorgestellten Heuristiken liefert nur für den \emph{required Typ} $\texttt{KOFGPCProvider}$ (\emph{TEI7}) bessere Ergebnisse, als die Kombination der Heuristiken \emph{BL\_NMC} und LMF. Aber auch hier darf dieses Ergebnis aufgrund der Eigenschaften von \emph{TEI7} nicht vernachlässigt werden.

\subsection{Erhöhte Komplexität}
Die vorliegende Untersuchung zeigt, dass die Anzahl der zu evaluierenden Proxies in dem verwendeten System mit den vorgeschlagenen Heuristiken reduziert werden können. Allerdings wurden negative Auswirkungen wie bspw. Speichernutzung (Speicherkomplexität) oder die benötigte Zeit (Zeitkomplexität) für die Exploration nicht untersucht.
\\\\
Die Anwendung der Heuristiken hängt, wie in Abschnitt \ref{sec_heuristics} beschrieben, von Informationen ab, die teilweise aus den für die Proxies verwendeten \emph{provided Typen} ermittelt werden müssen (Matcherrating) bzw. nach der Ausführung der Tests über die gesamte restliche Laufzeit der Exploration verwaltet werden müssen. Von daher ist davon auszugehen, dass sich die Anwendung der Heuristiken durchaus auf den Speicherverbrauch auswirkt.
\\\\
Da die benötigte Zeit für die Verwaltung von Listen, wie sie bei den Heuristiken vorgenommen wird, mit der Anzahl der zu verwaltenden Elemente wächst, kann davon ausgegangen werden, dass die Anwendung der Heuristiken ebenfalls mehr Zeit in Anspruch nimmt, je weiter fortgeschritten die Exploration ist. Dies gilt insbesondere für die Heuristiken PTTF und BL\_NMC. 
\\\\
Aufgrund dessen, dass in dieser Arbeit lediglich die Anzahl der zu evaluierenden Proxies während der Exploration untersucht wurden, ist es auch nicht auszuschließen, dass die verwendete Implementierung kein Optimierungspotential besitzt.

\subsection{Zusammenfassung}
Die Ausführungen der Abschnitt \ref{disc_einzel} und \ref{disc_synergien} lassen vermuten, dass lediglich die Heuristiken \emph{LMF} und\emph{ BL\_NMC} eine Daseinsberechtigung haben. Dies ist nicht korrekt. Die Heuristik \emph{PTTF} liefert zwar schlechtere Ergebnisse, dennoch hat sie die zu generierenden und zu prüfenden Proxies im Vergleich zum schlimmst Fall ohne Heuristiken stark reduziert. Allerdings hat der Entwickler keinen höheren Aufwand bei der Implementierung der Testfälle. Die Heuristik \emph{BL\_NMC}, welche sich in dieser Untersuchung häufig als diejenige mit den besten Ergebnissen herausgestellt hat, bedarf einer speziellen Implementierung der Testfälle.
\\\\
Dasselbe gilt für die Heuristik \emph{LMF}. Diese liefert zwar bessere Ergebnisse als die Heuristik PTTF, kann aber aufgrund dessen, dass sie sich lediglich auf den finalen Explorationsdurchlauf positiv auswirkt, nur in wenigen Fällen mit der Heuristik BL\_NMC mithalten. Allerdings gilt auch hier, dass keine weiteren Anforderungen an die Arbeit des Entwicklers gestellt werden. Dazu kommt noch, dass die Ermittlung der Matcherratings quasi bei dem Matching der Typen mit abfällt, wodurch die Verwendung dieser Heuristik kaum eine Auswirkung auf die Komplexität der Exploration hat.


\section{Kritik am Ansatz}
%Die Aussagekraft der Ergebnisse ist aufgrund der gewählten required Typen eher gering. Hier spielt auch der Abstraktionsgrad der Typen mit rein, die in den Methoden als Parameter- oder Rückgabetypen verwendet werden. Eine These ist, dass die Anzahl der zu evaluierenden Proxies steigt, je weiter der Abstraktionsgrad der in den required Typen und den provided Typen verwendeten Parameter- und Rückgabetypen auseinandergeht.

\subsection{Seiteneffekte durch Testevaluation}\label{sec_sideeffects}
Die Exploration erfordert die Ausführung der vordefinierten Testfälle zur Laufzeit. Sofern diese Testfälle eine Änderung des Zustands bestimmter Objekte bewirken, kann dies auch Auswirkungen auf die Funktionsweise des Systems haben. 
\\\\
Um dieses Problem zu beheben könnte man sicherstellen, dass die Generierung der Proxies nur auf Basis von \emph{provided Typen} erfolgt, die solche Seiteneffekte nicht aufweise. Diese Eigenschaft kann jedoch nur durch den Entwickler festgestellt werden und entsprechend markiert werden (bspw. über Annotationen). Während der Exploration könnten solche \emph{provided Typen} über solche Markierungen erkannt werden. Dieser Ansatz reduziert jedoch die Anzahl der \emph{provided Typen}, die für die Generierung eines Proxies verwendet werden können. Dadurch sinkt auch die Wahrscheinlichkeit, dass ein passender Proxy gefunden wird.
\\\\
Um die zu markierenden EJBs zu identifizieren ist zu prüfen, wie sich die Ausführung der einzelnen Methoden der Bean auf das System auswirken. Es kann festgehalten werden, dass alle Methoden, die den persistenten oder den transienten Zustand von Objekten verändert, das Potential für solche unerwünschten Seiteneffenkte besitzen. 
\\\\
Aufbauend auf der Prüfung einzelner Methoden, kann auch die Markierung von Methoden in Betracht gezogen werden. So dürften markierte Methoden bei der Generierung eines Proxies nicht als Delegationsmethode verwendet werden.

\subsection{Auswirkung auf die Stabilität des Systems}\label{sec_stabliliy}
Ein System gilt als stabil, wenn die enthaltenen Komponenten problemlos zusammenarbeiten \cite{}. Da der Ansatz darauf abzielt, bestimmte Komponenten (EJBs) zur Laufzeit zu kombinieren, hat der vorgestellte Ansatz durchaus eine Auswirkung auf die Stabilität des Systems.
\\\\
Die Auswirkung des Ansatzes auf die Stabilität des Systems wird maßgeblich durch die Güte der vordefinierten Testfälle bestimmt. Sofern die durch die Testfälle sichergestellte Semantik der gefundenen Proxies ausreichend gut spezifiziert wurde, ist es möglich, dass das System auch dann noch stabil ist, wenn Komponenten entfernt wurden.
\\\\
Sofern die Testfälle nicht ausreichend die Semantik sicherstellen, können zwar immer noch passende Komponenten gefunden werden, jedoch muss in Frage gestellt werden, ob das System unter der Verwendung dieser immer noch korrekt arbeitet. Somit hängt die Auswirkung des Ansatzes auf die Stabilität des Systems direkt mit der Sorgfalt des Entwicklers, der dieses Ansatz verwendet, zusammen.
\\\\
Darüber hinaus darf nicht vernachlässigt werden, dass der Ansatz das Finden eines passenden Proxies nicht garantiert. Der Entwickler muss also damit umgehen, dass kein Proxy gefunden wurde.
\subsection{Auswirkung durch Änderungen am System}
An einem System könne vielfältige Änderungen vorgenommen werden. Im folgenden wird zum Einen die Erweiterung vom zusätzliche \emph{provided Typen} und zum Anderen die Entfernung von \emph{provided Typen} betrachtet. Dabei sei angenommen, dass die \emph{required Typen}, zu denen ein passender Proxy gefunden werden soll, nicht verändert werden.
\\\\
Die Erweiterung von Systemen geht in Bezug auf den beschriebenen Ansatz zur testgetriebenen Exploration zur Laufzeit damit einher, dass sich die Anzahl der \emph{provided Typen} verändert. Wie in Abschnitt \ref{sec_anzahlProxies} beschrieben, besteht damit auch die Gefahr, dass die Anzahl der möglichen Proxies steigt. Dazu muss jedoch gelten, dass eine Methode im neuen provided Typ mit einer Methode eines required Typ gematcht werden kann.
\\\\
Mehrere mögliche Proxies haben wiederum einen Einfluss auf die Laufzeit und das Ergebnis der Exploration. So kann nicht davon ausgegangen werden, dass ein passender Proxy zu einem bestimmten required Typ genauso schnell gefunden wird, nachdem das System neu gestartet wurde.
\\\\
Ebenso wirkt sich das Entfernen eines provided Typs, der bei einer früheren Exploration für die Generierung eines Proxies verwendet wurde, auf die Exploraiton nach einem Neustart aus. Dadurch, dass der früher verwendete provied Typ nicht mehr vorhanden ist, muss ein anderer Proxies, der auf anderen provided Typen basiert, erzeugt werden.
\\\\
Da die Exploration beendet wird, sofern ein passender Proxy gefunden wurde, kann es auch unter diesen Umständen dazu kommen, dass die Exploration mitunter länger dauert als vorher. Zudem besteht in diesem Fall die Gefahr, dass die Exploration fehlschlägt.






\subsection{Verantwortung des Entwicklers}
Aus dem oben genannten ergibt sich, dass der Entwickler bei der Verwendung dieses Ansatzes eine große Verantwortung trägt. Dieser kann er meiner Meinung nach umso besser gerecht werden, je besser er das System, in dem der Ansatz verwendet werden soll, kennt. 
\\\\
So kann festgehalten werden, dass ein Entwickler, der das System gut kennt und somit weiß, welche Komponenten innerhalb dessen verwendet werden, diesen Ansatz wohl kaum benötigt. Vielmehr ist es ihm möglich die passenden Komponenten aufgrund seines Wissens explizit zu benennen, wie es im EJB-Framework grundlegend der Fall ist.
\\\\
Ein Entwickler, der das System hingegen weniger kennt, kann von diesem Ansatz profitieren, da er nicht selbst nach einer für ihn passenden EJB (mitunter auch mehreren) suchen muss. Diese kann er über die Deklaration eines required Typen und der Spezifikation dazugehöriger Tests suchen lassen. Dabei ist jedoch zu erwähnen, dass die Exploration insbesondere mit der vorgestellten Heuristik LMF umso schneller ist, je genauer die in den Methoden des required Typs verwendeten Typen mit den Typen, die in den Methoden der provided Typen übereinstimmen (Matcherrating).
\\\\
Ist dem Entwickler das System unbekannt, wird es schwerfallen zum Einen ein \emph{required Typ} so zu definieren, dass die Anzahl der möglichen Proxies nicht zu hoch wird. Und zum Anderen werden sich auch Probleme bei der ausreichenden Spezifikation von Testfällen ergeben. Wobei der zweite Punkt direkt mit der Kopplung der \emph{provided Typen} zusammenhängt. Eine zu starke Kopplung könnte die Gefahr für unerwünschte Seiteneffekte erhöhen \cite{}. Zusammenfassend könnte folge These formuliert werden: These: Je schlechter ein Entwickler das vorliegende System kennt, desto eher wird der o.g. Abstraktionsgrad abweichen. Dies ließe sich mitunter durch eine Umfrage mehrere Mitarbeiter unterschiedlicher Betriebsangehörigkeit in Erfahrung bringen.
\\\\



Das unten stehende Diagramm stellt dar, welche Eigenschaften in Bezug auf den Ansatz zu- bzw. abnehmen. Wobei das linke Ende der Skala als Grundlage das Wissen eines Entwicklers darstellt, der das System nicht kennt und das rechte Ende dementsprechend das Wissen eines Entwicklers darstellt, der das System sehr gut kennt.


Darauf aufbauend ergibt sich die Frage nach der Aussagekraft der vorliegenden Untersuchungsergebnisse.
\section{Erweiterungsmöglichkeiten}
\subsection{Zusätzliche Matcher}
Eine mögliche Erweiterung des Ansatzes wäre die Definition und Implementierung zusätzlicher Matcher. Diese würde es ermöglichen, dass der Abstraktionsgrad zwischen den Typen, die in den Methoden der required und provided Typen verwendet werden, noch weiter auseinandergeht, als es bei den vorgestellten Matchern in Abschnitt \ref{sec_matcher} der Fall ist (Identität, Vererbung, Container).
\\\\
Die vorgestellten Matcher beachten beispielsweise keine impliziten Typumwandlungen (Coercions). Diese können je nach Programmiersprache abweichen, was eine formale und allgemeine Beschreibung wie in Abschnitt \ref{sec_matcher} eines solchen Matchers (CoercionMatcher) erschwert. So müsste ein CoercionMatcher für jede Programmiersprache explizit spezifiziert werden.
\\\\
Die Programmiersprache Java bietet eine Vielzahl solcher impliziten Typumwandlungen an \cite{conversions_and_promotions}. Dabei ist zu beachten, dass es implizite Typumwandlungen gibt, die ohne Informationsverlust vonstatten gehen\footnote{bspw. \emph{Identity Conversion} oder \emph{Widening Primitive Conversion} \cite{conversions_and_promotions}} und solche, bei denen ein Informationsverlust nicht auszuschließen ist\footnote{bspw. \emph{Narrowing Primitive Conversion} \cite{conversions_and_promotions}}. 
\\\\
Typumwandlungen ohne Informationsverlust sind in Bezug auf die weitere Verwendung innerhalb eines Proxies unbedenklich. Diese sind hinsichtlich des Informationsverlustes mit dem GenTypeMatcher vergleichbar, welcher in Abschnitt \ref{sec_matcher} beschrieben wurde. In der Spezifikation des darauf aufbauenden Proxy-Generators sind dementsprechend keine Methodendelegationen zu finden, die zu einem Fehler führen.
\\\\
Anders ist es bei Typumwandlungen mit Informationsverlust. Diese sind mit dem SpecTypeMatcher vergleichbar (siehe Abschnitt \ref{sec_matcher}). In der Spezifikation des darauf aufbauenden Proxy-Generators ist zu erkennen, dass durch eine solche Typumwandlung bestimmte Methodendelegationen in einen Fehler münden. Da sich der SpecTypeMatcher direkt auf die Vererbungsbeziehung der beiden Typen bezieht, kann die Ursache solcher Fehler auf die Methoden zurückgeführt werden, die zwar im Subtyp jedoch nicht im Supertyp implementiert sind. Bei einem CoercionMatcher, der in Abhängigkeit der Programmiersprache spezifiziert wird, kann es weitere Fehlerursachen geben.
\\\\
Aus diesem Grund wäre es sinnvoll, nicht einen einzigen Matcher zu spezifizieren, der alle impliziten Typumwandlungen abdeckt. Vielmehr sollten die in der Programmiersprache definierten Coercions nach dem möglichem Informationsverlust kategorisiert werden und dann je Kategorie ein Matcher spezifiziert werden.
\\\\
Darüber hinaus ist zu beachten, dass die Spezifikation eines Matchers alleine nicht ausreicht, um diesen zu integrieren. Da die Heuristik LMF auf dem Matcherrating aufbaut, ist es ebenso notwendig, den zusätzlichen Matchern ein Basisrating zuzuweisen. Wie in Abschnitt \ref{impl_sigma} beschrieben, wird dieses Basisrating von der Implementierung des Matchers geliefert. Dabei gilt es jedoch zu beachten, dass das Basisrating eines zusätzlichen Matchers im korrekten Verhältnis zu den bestehenden Matchern steht.
\\\\
In Bezug auf den/die CoercionMatcher gibt es hierbei mehrere sinnvolle Möglichkeiten. Beispielsweise könnte man begründen, dass für den/die CoercionMatcher ein Basisrating zwischen 100 und 200 verwendet werden muss. Die untere Schranke von 100 wird dadurch begründet, dass es kein besseres Matching gibt, als die Identität, welche durch den ExactTypeMatcher mit einem Basisrating von 100 beschrieben wird. Die obere Schranke von 200 könnte damit begründet werden, dass es sich um Typumwandlungen handelt, die über die Programmiersprache definiert sind und diese somit sicherer sind als Upcasts, die durch den SpecTypeMatcher mit einem Basisrating von 200 abgedeckt werden.
\subsection{Default-Implementierungen in required Typen}
Im Abschnitt \ref{sec_tdcs_ejb} wurde darauf aufmerksam gemacht, dass die Exploration das Auffinden eines passenden Proxies nicht garantiert. Der Entwickler muss also in einem solchen Fall eine alternative Implementierung bereitstellen.
\\\\
Dass ein passender Proxy nicht gefunden wurde, kann allgemein betrachtet zwei Ursachen haben: Entweder konnte kein Proxy generiert werden, oder keiner der generierten Proxies erfüllt alle vordefinierten Test. 
\\\\
Die Generierung eines Proxies hängt von dem Matching der Methoden des required Typs und der Methoden der provided Typen ab. Aufgrund dessen dass der Entwickler Testfälle für den required Typ spezifizieren muss, hat er eine grundlegende Vorstellung von den Ein- und Ausgabewerten der Methoden, sowie der Verarbeitung dieser. Um nun der Gefahr vorzubeugen, dass gar kein Proxy generiert werden kann, könnte der Entwickler eine Implementierung, die seine Erwartungen zumindest minimal erfüllt, als default-Methoden in dem Interface zum required Typ aufnehmen. Sofern bei der Exploration zu dieser Methode keine passende Methode aus einem provided Typ gefunden wird, kann auf die Default-Implementierung zurückgegriffen werden. Der generierte Proxy, welcher technisch gesehen das Interface zum required Typ implementiert, würde den Methodenaufruf dann an sich selbst bzw. an die default-Methode delegieren.
\\\\
Ein Beispiel für eine solche Konstellation zeigen die folgenden Listings. In Listing \ref{lst_calc} ist der required Typ \emph{Calc} deklariert. Listing \ref{lst_interface_calc} zeigt das dazugehörige Java-Interface mit der default-Implementierung der Methode \emph{div}. Die Implementierung wurde so umgesetzt, dass die Testfälle, welche in der Klasse in Listing \ref{lst_testklasse_calc} enthalten sind, positiv ausfallen.
\begin{lstlisting}[caption={Required Typ \emph{Calc}},captionpos=b, style = dsl, label=lst_calc]
required Calc {
	Float div( int a, int b )	
}
\end{lstlisting}
\begin{lstlisting}[style = java, caption = Interface Calc, captionpos = b, label = lst_interface_calc]
@RequiredTypeTestReference( testClasses = CalcTest.class )
public interface Calc {

  default Float div(int a, int b){
  	if(b == 0)
  		return null;
  	return Float.valueOf(a/b)
  }

}
\end{lstlisting}
\begin{lstlisting}[style = java, caption = Test CalcTest, captionpos = b, label = lst_testklasse_calc]
public class CalcTest {

  private Calc calc;
  
  @RequiredTypeInstanceSetter
  public void setProvider( Calc calc ) {
    this.calc = calc;
  }

  @RequiredTypeTest
  public void testDivByZero() {
    assertThat( calc.dev(1,0), nullValue() );
  }
  
  @RequiredTypeTest
  public void testDiv() {
    assertThat( calc.dev(4,2), equalTo(2) );
  }

}
\end{lstlisting}
\noindent
Dadurch ist zwar immer noch nicht sichergestellt, dass ein passender Proxy in jedem Fall gefunden wird, aber der Entwickler kann ein alternatives Verhalten direkt im Interface zum \emph{required Typ} implementieren, wodurch diese Implementierung einen sehr engen Bezug zum \emph{required Typ} hat. 

\chapter{Schlussbemerkung}\label{chap_finish}
\section{Zusammenfassung}
Zusammenfassend ist zu sagen, dass die vorgestellten \Gls{Heuristik}en ihren Zweck erfüllen und gemessen an der Anzahl der zu generierenden und zu prüfenden \emph{Proxies} eine schnellere Exploration nach einem passenden \emph{Proxy} ermöglichen. Dabei konnten auch Synergieeffekte zwischen den einzelnen \Gls{Heuristik}en festgestellt werden.
\\\\
Weiterhin wurde gezeigt, dass die testgetriebene Exploration von \emph{EJBs} zur Laufzeit grundlegend funktioniert. Dennoch gibt es Szenarien, in denen von diesem Verfahren eher abzuraten ist. Das betrifft insbesondere solche \emph{EJBs}, durch deren Methodenaufrufe eine Änderung an ihrem inneren Zustand bezweckt wird. Es wurden jedoch Möglichkeiten aufgezeigt, wie mit solchen Fällen umgegangen werden kann.
\\\\
Ob der Ansatz der testgetriebenen Exploration zur Laufzeit im Allgemeinen einen Nutzen verspricht wurde nicht geklärt. Wenn dies überhaupt der Fall ist, dann hängt der Nutzen vermutlich mit dem Wissen der Entwickler*innen zusammen, welches sie über das vorliegende System aufweisen können.
\\\\
Unabhängig davon wurde in dieser Arbeit eine allgemeine formale Beschreibung für Matcher von \Gls{wrappertype}en gegeben (\emph{ContentTypeMatcher} und \emph{ContainerTypeMatcher}). 
\\\\
Zudem können die entwickelten \Gls{Modul}e, welche in Kapitel \ref{chap_impl} beschrieben wurden, in unterschiedlichen Systemen verwendet werden. Hinsichtlich des Repositories haben die Entwickler*innen sehr viel Freiraum und sind nicht auf einen \emph{EJB-Container} beschränkt. Weiterhin können neue Matcher durch die Implementierung der dafür vorgesehenen \Gls{Interface}s in die \Gls{Modul}e integriert werden, was den Nutzen des Ansatzes in einem System individuell steigern kann.
\section{Ausblick}
Die \Gls{Heuristik}en wurden für die Exploration zur Laufzeit entworfen. In einem nächsten Schritt könnte versucht werden, diese \gls{Heuristik}en in bestehende Search \Gls{Engine}s wie \emph{Merobase} oder \emph{CodeGenie} zu integrieren und deren Nutzen in diesem Kontext zu untersuchen.
\\\\
Weiterhin wäre es interessant zu untersuchen, ob und wie dieser Ansatz der Exploration von \Gls{komponente}n zur Laufzeit in anderen Systemtypen wie bspw. Self-Contained-Systems funktioniert. Mitunter ergeben sich bei diesen Untersuchungen weitere Vorteile oder Probleme dieses Ansatzes.
\\\\
Darüber hinaus bieten die in Abschnitt \ref{sec_discApproach} aufgestellten Thesen bzgl. der höheren Verfügbarkeit (Abschnitt \ref{sec_sideeffects}) und dem Nutzen des Ansatzes für die Entwickler*innen im Verhältnis zu deren Wissen über das System das Potential für weitere Untersuchungen.

\newpage
\appendix
\addcontentsline{toc}{chapter}{Glossar}
\printglossaries

\chapter{Beweise}\label{app_proofs}
\section{Lemma \ref{lemma_targetcount}}
\begin{lemma*}\label{lemma_targetcount}
Sei $R$ ein \emph{required Typ} innerhalb einer Bibliothek $L$. 
Ein \emph{struktureller Proxy} für $R$ lässt sich nur aus den Mengen $\mathit{TM} \in \mathit{cover(R,L)}$ generieren, für die gilt:
\begin{gather*}
|\mathit{TM}| \leq |\mathit{methods(R)}|
\end{gather*}
\end{lemma*}
\begin{proof}
In Bezug auf alle \emph{strukturellen Proxies} $P \in \mathit{proxies_{struct}(R,TM})$, drückt das Lemma folgendes aus:
\begin{gather*}
\forall \mathit{P} \in \mathit{proxies_{struct}(R,TM}):\mathit{TM}| \leq |\mathit{methods(R)}|
\end{gather*}
\noindent
Da ein \emph{struktureller Proxy} $P \in \mathit{proxies_{struct}(R,TM)}$ der Bedingung $\mathit{targets_{multi}(P,TM)}$ unterliegt (siehe Abschnitt \ref{sssec_structproxy}), muss gelten:
\begin{gather*}
|\mathit{P.targets}| = |\mathit{TM}|
\end{gather*}
Weiterhin gilt aufgrund von $\mathit{targets_{multi}(P,TM)}$, dass für jeden \emph{Target-Typ} eine \emph{Methoden-Delegation} existiert, die diesen \emph{Target-Typ} im Attribut $\texttt{target}$ enthält:
\begin{gather*}
\forall \mathit{T} \in \mathit{P.targets}: \exists \mathit{MD} \in \mathit{P.dels}:\mathit{MD.del.target} = T
\end{gather*}
Daraus folgt für die Mächtigkeit der \emph{Methoden-Delegationen}:
\begin{gather*}
\mathit{P.dels.len} \geq |\mathit{P.targets}|\\ \mathit{P.dels.len} \geq |\mathit{TM}|
\end{gather*}
\noindent
Zusätzlich gilt aufgrund der Regel $\mathit{delegationCount_{struct}(P)}$ (siehe Abschnitt \ref{sssec_structproxy}):
\begin{gather*}
|\mathit{methods(R)}| = \mathit{P.dels.len}
\end{gather*}
Daraus folgt direkt:
\begin{gather*}
\forall \mathit{P} \in \mathit{proxies_{struct}(R,TM}):\mathit{TM}| \leq |\mathit{methods(R)}|
\end{gather*}



\end{proof}
\chapter{Kombination von Matchern}\label{app_matchercombination}
%hier soll der MatcherCombiner als der entwickelten Explorationskomponente beschrieben werden


%TODO ANHANG: Kombination (siehe unten)
%Die Matcher-Klassen $\texttt{ExactTypeMatcher}$, $\texttt{GenSpecTypeMatcher}$ und $\texttt{WrappedTypeMatcher}$ implementieren auch das von $\texttt{TypeMatcher}$ erbende Interfaces $\texttt{CombinalbeTypeMatcher}$. Klassen, die dieses Interface implementieren können über die Klasse $\texttt{MatcherCombiner}$ zu einem neuen $\texttt{TypeMatcher}$-Objekt kombiniert werden. Ein solcher kombinierte $\texttt{TypeMatcher}$ versucht beim Aufruf der Methode $\texttt{matchesType(S,T)}$ die beiden Typen $S$ und $T$ über einen der kombinierten Matcher zu matchen. Abbildung \ref{sd_matchercombiner} zeigt das Sequenzdiagramm für diesen Aufruf. Dabei liefert die Methode $\texttt{getSortedMatcher}$ eine sortiert Liste der kombinierten Matcher. Die Sortierung wird aufsteigend entsprechend dem Matcherrating der kombinierten Matcher vorgenommen.
%\begin{figure}
%\end{figure}\label{sd_matchercombiner}
%\noindent
%Darüber hinaus gibt es noch das von $\texttt{TypeMatcher}$ erbende Interface $\texttt{PartlyTypeMatcher}$. Dieses Interface wird nur von dem $\texttt{StructuralTypeMatcher}$ implementiert, welcher u.a. als Schnittstelle zwischen dem Modul \emph{SignatureMatching} und \emph{DesiredComponentSourcerer} fungiert. Wie der Name des Interfaces bereits impliziert, bieten die Implementierungen des Interfaces $\texttt{PartlyTypeMatcher}$ die Möglichkeit, zwei Typen nur teilweise zu Matchen. Das bildet die Grundlage für die Ermittlung der Typen, aus denen die Proxies für die semantische Evaluation erzeugt werden können (vgl. Abschnitt \ref{sec_ergStructEval}). So stellen die Objekte, die über die Methode $\texttt{calculatePartlyTypeMatchingInfos}$ erzeugten wurden, auf formaler Ebene die Elemente der Mengen, die in Abschnitt \ref{sec_ergStructEval} über Funktion $\texttt{cover}$ beschrieben wurden, dar.

\chapter{Semantische Evaluation mit allen vorgestellten Heuristiken}\label{app_semEvalMitAllenHeuristiken}
Die in den Abschnitten \ref{sec_lmf} - \ref{sec_bl_nmc} vorgestellten Heuristiken können miteinander Kombiniert werden. Listing \ref{lst_heuristikkombination} zeigt die Implementierung der Funktionen, die für diese Kombination auf der Basis von Listing \ref{lst_semEval} angepasst oder ergänzt werden müssen.

%TODO - Die drei Funktionen sollten einzeln erklärt werden
\begin{lstlisting}[style = pseudo, caption = Kombination aller Heuristiken, captionpos = b, label = lst_heuristikkombination]
function evalProxiesMitTarget( proxies, tests ){
	testedProxies = []
	for( proxy : proxies ){
		passedTestcases = 0
		blacklistChanged = false
		evalProxy(proxy, tests)
		if( passedTests == T.size ){
			// passenden Proxy gefunden
			return proxy
		}
		else{
			testedProxies.add(proxy)
			if( passedTests > 0 || blacklistChanged ){
				// noch nicht evaluierte Proxies ermitteln
				optmizedProxies = proxies.removeAll( testedProxies )
				// Heuristik PTTF
				if( passedTests > 0 ){
					priorityTargets.addAll( proxy.targets )
					optmizedProxies = PTTF( optmizedProxies )	
				}
				// Heuristik BL_FFMD und BL_FSMT
				if( blacklistChanged ){
					optmizedProxies = BL( optmizedProxies )	
				}
				return evalProxiesMitTarget( optmizedProxies, tests )
			}
		}
	}
	// kein passenden Proxy gefunden
	return null
}

function evalProxy(proxy, tests){
	for( test : tests ){
		//alle Tests werden durchgefuehrt	
		try{
			if( test.eval( proxy ) ){
				passedTestcases = passedTestcases + 1
			}elseif( test.isSingleMethodTest ){
				methodName = test.testedSingleMethodName
				mDel = getMethodDelegation( proxy, methodName )
				methodDelegationBlacklist.add( mDel )
				blacklistChanged = true
				return
			}
		}
		catch (SigMaGlueException e){
			mDel = e.failedMethodDelegation
			methodDelegationBlacklist.add( mDel )
			blacklistChanged = true
			return
		} 
	}
}

function relevantProxies( proxies, anzahl ){
	relProxies = proxiesMitTargets( proxies, anzahl );
	optimizedLMF = LMF( relProxies )
	optimizedPTTF = PTTF( optimizedLMF )
	return BL( optimizedPTTF )
} 


\end{lstlisting}
\chapter{Deklaration der relevanten Typen}\label{app_evalTypes}
Im Folgenden erfolgt die Deklaration der \emph{required Typen}, mit denen die Evaluation der Heuristiken in Kapitel \ref{chap_evaluation} durchgeführt wird, sowie die Deklaration der \emph{provided Typen}, die als Ergebnis der jeweiligen Exploration für einen \emph{required Typ} einzeln oder in Kombination erwartet, oder innerhalb einer der Deklarationen eines \emph{required Typ} verwendet werden. Dabei ist davon auszugehen, dass diese Typen aus dem JDK als Bibliothek aufbauen.
\\\\
Die Listings \ref{lst_tei1} - \ref{lst_tei7} zeigen die Deklarationen für die \emph{required Typen}.
\begin{lstlisting}[style = dsl, caption = Deklaration von ElerFTFoerderprogrammeProvider, captionpos = b, label = lst_tei1]
required ElerFTFoerderprogrammeProvider{
 Collection getAlleFreigegebenenFPs()
 ElerFTFoerderprogramm getElerFTFoerderprogramm(DvAntragsJahr, DvFoerderprogramm, Date)
}
\end{lstlisting}
\begin{lstlisting}[style = dsl, caption = Deklaration von FoerderprogrammeProvider, captionpos = b, label = lst_tei2]
required FoerderprogrammeProvider{
 Collection getAlleFreigegebenenFPs()
 Foerderprogramm getFoerderprogramm(DvAntragsJahr, DvFoerderprogramm, Date)
}
\end{lstlisting}
\newpage
\begin{lstlisting}[style = dsl, caption = Deklaration von MinimalFoerderprogrammeProvider, captionpos = b, label = lst_tei3]
required MinimalFoerderprogrammeProvider{
 Collection getAlleFreigegebenenFPs()
 Foerderprogramm getFoerderprogramm(String, int, Date)
}
\end{lstlisting}
\begin{lstlisting}[style = dsl, caption = Deklaration von IntubatingFireFighter, captionpos = b, label = lst_tei4]
required IntubatingFireFighter{
 void intubate(Injured)
 FireState extinguishFire(Fire)
}
\end{lstlisting}
\begin{lstlisting}[style = dsl, caption = Deklaration von IntubatingFreeing, captionpos = b, label = lst_tei5]
required IntubatingFreeing{
 void intubate(Injured)
 void free(Injured)
}
\end{lstlisting}
\begin{lstlisting}[style = dsl, caption = Deklaration von IntubatingPatientFireFighter, captionpos = b, label = lst_tei6]
required IntubatingFreeing{
 void intubate(IntubationPatient)
 FireState extinguishFire(Fire)
}
\end{lstlisting}
\begin{lstlisting}[style = dsl, caption = Deklaration von KOFGPCProvider, captionpos = b, label = lst_tei7]
required KOFGPCProvider{
 Collection getKOFGsVonFP(DvFoerderprogramm)
 Collection getPCsZuKOFG(DvFoerdergegenstand, DvAntragsJahr)
}
\end{lstlisting}
\newpage
\noindent
Die Listings \ref{lst_ElerFTFoerderprogramm} - \ref{lst_IntubationPatient} zeigen die \emph{provided Typen}, die in den Deklarationen der \emph{required Typen} verwendet wurden und nicht Teil des JDKs sind.
\begin{lstlisting}[style = dsl, caption = Deklaration von ElerFTFoerderprogramm, captionpos = b, label = lst_ElerFTFoerderprogramm]
provided ElerFTFoerderprogramm extends Foerderprogramm{
 DvFlaeche mindestParzellenGroesse
 DvFlaeche maximaleParzellenGroesse
 int differenzKassenjahrAntragsjahr
 boolean isMehrjaehrig
  
 DvFlaeche getMaximaleParzellengroesse()
 DvFlaeche getMindestParzellenGroesse()
 int getDifferenzKassenjahrAntragsjahr()
 boolean isMehrjaehrig()
}
\end{lstlisting}
\begin{lstlisting}[style = dsl, caption = Deklaration von Foerderprogramm, captionpos = b, label = lst_Foerderprogramm]
provided Foerderprogramm extends Object{
 Long id
 STDGueltigkeit gueltigkeit
 Long fpId
 BigDecimal bagatellbetrag
 BigDecimal bagatellmenge
 List vorgaengeAm15
 Set landesmassnahmen
  
 Long getId()
 boolean isTechnischGueltig(Date)
 DvFoerderprogramm getFoerderprogramm()
 BigDecimal getBagatellmengeFoerd()
 BigDecimal getBagatellbetragFoerd()
 boolean isFachlichGueltig(DvAntragsJahr)
 STDGueltigkeit getGueltigkeit()
 Long getFpId()
}
\end{lstlisting}
\pagebreak
\begin{lstlisting}[style = dsl, caption = Deklaration von DvAntragsJahr, captionpos = b, label = lst_dvantragsjahr]
provided DvAntragsJahr extends AbstractDomainValue{
 int antragsJahr
  
 DvAntragsJahr add(int)
 int compareTo(Object)
 int intValue()
 Object readResolve()
 DvAntragsJahr getVorjahr()
 int differenz(DvAntragsJahr)
 DvAntragsJahr sub(int)
 String toStringImpl()
}
\end{lstlisting}
\begin{lstlisting}[style = dsl, caption = Deklaration von DvFoerderprogramm, captionpos = b, label = lst_DvFoerderprogramm]
provided DvFoerderprogramm extends DvEnumerable{
 long id
 String code
 String fpGruppe
 String bezeichnung
 String bezeichnungLang
 String getName()
  
 Long getId()
 Long getNummer()
 void validateCode(String)
 String getFpGruppe()
 String getBezeichnung()
 String toStringImpl()
 String getCode()
 String getFPNummerExtern()
 String getBezeichnungLang()
}
\end{lstlisting}
\begin{lstlisting}[style = dsl, caption = Deklaration von Injured, captionpos = b, label = lst_Injured]
provided Injured extends Object{
 Collection suffers

 Collection getSuffers()
 void healSuffer(Suffer)
 boolean isStabilized()
}
\end{lstlisting}
\begin{lstlisting}[style = dsl, caption = Deklaration von Fire, captionpos = b, label = lst_Fire]
provided Fire extends Object{
 boolean active

 void extinguish()
 boolean isActive()
}
\end{lstlisting}
\begin{lstlisting}[style = dsl, caption = Deklaration von IntubationPatient, captionpos = b, label = lst_IntubationPatient]
provided IntubationPatient extends Object{
 boolean isIntubated

 boolean isIntubated()
 void setIntubated(boolean)
}
\end{lstlisting}
\noindent
Die Listings \ref{lst_eftstd} - \ref{lst_firefighter} zeigen die Deklarationen der \emph{provided Typen}, aus denen bei der Exploration ein passender Proxy erzeugt werden soll.
\begin{lstlisting}[style = dsl, caption = Deklaration von ElerFTStammdatenAuskunftService, captionpos = b, label = lst_eftstd]
provided ElerFTStammdatenAuskunftService extends Object{
 Collection getAlleElerFTKombiKzFpFoerdergegenstaende()
 Collection getAlleElerFTKoFoerdergegenstaende()
 Collection getFeststellungscodeVerpflichtungList(FeststellungscodeVerpflichtungImplQuery)
 FeststellungscodeVerpflichtungImpl getFeststellungscodeVerpflichtungImpl(FeststellungscodeVerpflichtungImplQuery)
 Collection getAlleElerFTTierFoerdergegenstaende(DvFoerderprogramm, DvAntragsJahr, AntragsVorgangsTyp)
 Collection getAlleFreigegebenenFoerderprogramme(AntragsVorgangsTyp)
 Collection getAlleFreigegebenenFoerderprogramme()
 ElerFTKzFpFoerdergegenstand2Foerderfaehigkeit getElerFTKzFpFoerdergegenstand2Foerderfaehigkeit(DvFoerdergegenstand, DvAntragsJahr)
 FeststellungsCodeVerpflichtung2FP getFeststellungsCodeVerpflichtung2FP(FeststellungsCodeVerpflichtung2FPQuery)
 DvEftOekoFoerdergegenstandGruppe getOekoFgGruppe2Foerdergegenstand(DvFoerdergegenstand)
 Collection getAlleElerFTKzFpFoerdergegenstaende()
 VerpflichtungsGegenstandImpl getVerpflichtungsGegenstandImpl(VerpflichtungsGegenstandImplQuery)
 ElerFTVorhaben getVorhaben2Foerdergegenstand(DvFoerdergegenstand, DvAntragsJahr)
 Verpflichtungszeitraum getVerpflichtungszeitraum(DvFoerderprogramm, DvAntragsJahr)
 int getMaxStandardAnzahlZahlungen(DvFoerderprogramm, DvAntragsJahr)
 DvZusatzInfoTyp getZusatzInfo2Foerdergegenstand(DvFoerdergegenstand, DvAntragsJahr)
 int getStandardAnzahlZahlungen(DvUntermassnahme, DvAntragsJahr)
 int getStandardAnzahlZahlungen(Landesmassnahme, DvAntragsJahr)
 Collection getElerFTKoFoerdergegenstaende(DvFoerderprogramm, DvUntermassnahme, DvAntragsJahr)
 Collection getElerFTKoFoerdergegenstaende(DvFoerderprogramm)
 Collection getAlleFg2ZusatzInfo(DvZusatzInfoTyp, DvAntragsJahr)
 int getDifferenzJahrVerpflbeginnEAJ(DvFoerderprogramm, DvAntragsJahr)
 Collection getVerpflichtungsGegenstandList(VerpflichtungsGegenstandImplQuery)
 Collection getAenderungscodePropertiesList(AenderungscodePropertiesQuery)
 Collection getAlleFg2OekoFgGruppe(DvEftOekoFoerdergegenstandGruppe)
 ElerFTFoerderprogramm getFoerderprogramm(ElerFTFoerderprogrammQuery)
 ElerFTFoerderprogramm getFoerderprogramm(DvAntragsJahr, DvFoerderprogramm, Date)
 Collection getElerFTAenderung2ElerFTFP(DvFoerderprogramm)
 Collection getElerFTAenderung2ElerFTFP(ElerFTAenderung)
 ElerFTAenderung2ElerFTFP getElerFTAenderung2ElerFTFP(ElerFTAenderung, DvFoerderprogramm)
 Collection getFoerdergegenstaende(AbstractElerFTFoerdergegenstandQuery)
 Collection getElerFTTierFoerdergegenstaende(DvFoerderprogramm, DvUntermassnahme, DvAntragsJahr)
 Collection getFoerderprogramme(ElerFTFoerderprogrammQuery)
 Collection getFoerderprogramme(Date)
 Collection getAlleFoerderprogramme()
 Collection getElerFTKzFpFoerdergegenstaende(DvFoerderprogramm, DvUntermassnahme, DvAntragsJahr)
 Collection getElerFTKzFpFoerdergegenstaende(ElerFTKombiKzFpFoerdergegenstand)
 Collection getElerFTKzFpFoerdergegenstaende(DvFoerderprogramm, Finanzierungsschluessel, DvAntragsJahr)
 Collection getElerFTKzFpFoerdergegenstaende(DvFoerderprogramm, DvAntragsJahr)
 Collection getAlleFg2Vorhaben(ElerFTVorhaben, DvAntragsJahr)
 Map getKzFpJeFg(Collection, DvAntragsJahr)
}
\end{lstlisting}
\begin{lstlisting}[style = dsl, caption = Deklaration von StammdatenAuskunftService, captionpos = b, label = lst_std]
provided StammdatenAuskunftService extends Object{
 Collection getLandesmassnahmen2Foerdergegenstaende(Landesmassnahme2FoerdergegenstandQuery)
 Collection getFoerdergegenstaendeZuFinanzierungsschluessel(DvFoerderprogramm, Finanzierungsschluessel, DvAntragsJahr)
 Landesmassnahme getLandesmassnahme(Long)
 Map getOberFgJeUnterFg(DvAntragsJahr)
 Collection getFoerderprogramme(Date)
 Foerdergegenstand getFoerdergegenstand(FoerdergegenstandQuery)
 Collection getFoerdergegenstaende(DvFoerderprogramm)
 Collection getFoerdergegenstaende(FoerdergegenstandQuery)
 Collection getFoerdergegenstaende(Landesmassnahme)
 Collection getFinanzierungsschluessel(FinanzierungsschluesselQuery)
 Collection getFinanzierungskonfigurationen(FinanzierungskonfigurationQuery)
 Collection getFinanzierungskonfigurationen(Collection, DvAntragsJahr)
 Collection getFinanzierungskonfigurationen(DvAntragsJahr, DvFoerderprogramm, Long)
 Finanzierungskonfiguration getFinanzierungskonfigurationen(DvAntragsJahr, DvFoerderprogramm, DvFoerdergegenstand)
 Map getProduktcodesJeFg(DvFoerderprogramm, DvAntragsJahr, Collection, ProduktcodeArt, Finanzierungsschluessel)
 Foerderprogramm getFoerderprogramm(Foerdergegenstand)
 Foerderprogramm getFoerderprogramm(DvAntragsJahr, DvFoerderprogramm, Date)
 Collection getAblehnungsgrundCodes(Foerderprogramm, DvAntragsJahr, KuerzungsgrundCode)
 Collection getUnterFoerdergegenstaende(DvAntragsJahr, Collection)
 Collection getFoerdergegenstandGruppenZuFgs(DvAntragsJahr, Collection)
 Collection getLandesmassnahmen(DvAntragsJahr, DvFoerderprogramm)
 Collection getLandesmassnahmen(DvAntragsJahr, Foerdergegenstand)
 Collection getLandesmassnahmen(LandesmassnahmeQuery)
 Produktcode getProduktcode(ProduktcodeQuery)
 Produktcode getProduktcode(DvAntragsJahr, DvFoerdergegenstand, ProduktcodeArt)
 Produktcode getProduktcode(DvAntragsJahr, DvFoerdergegenstand, ProduktcodeArt, Finanzierungsschluessel)
 BigDecimal getBeihilfesatz(DvAntragsJahr, DvFoerdergegenstand, Integer)
 Collection getProduktcodes(DvAntragsJahr, Finanzierungsschluessel)
 Collection getProduktcodes(DvAntragsJahr, DvFoerdergegenstand, Finanzierungsschluessel)
 Collection getProduktcodes(ProduktcodeQuery)
 Collection getProduktcodes(DvAntragsJahr, DvFoerderprogramm)
 Collection getProduktcodes(DvAntragsJahr, DvFoerdergegenstand)
 Collection getProduktcodes(Collection)
 BigDecimal getKappungBetrag(DvFoerdergegenstand, DvAntragsJahr)
 Collection getVorgaenge(Date, DvFoerderprogramm)
 Collection getVorgaenge(AntragsVorgangsTyp)
 Collection getVorgaenge(Date, AntragsVorgangsTyp)
 Collection getVorgaenge()
 Collection getVorgaenge(DvFoerderprogramm, Date, AntragsVorgangsTyp)
 BigDecimal getKappungMenge(DvFoerdergegenstand, DvAntragsJahr)
 Vorgang getVorgang(DvAntragsJahr, DvFoerderprogramm, Date, AntragsVorgangsTyp, DvAntragsJahr)
 Vorgang getVorgang(DvFoerderprogramm, Date, AntragsVorgangsTyp, DvAntragsJahr)
}
\end{lstlisting}
\pagebreak
\begin{lstlisting}[style = dsl, caption = Deklaration von Doctor, captionpos = b, label = lst_doctor]
provided Doctor extends Object{
 void provideHeartbeatMassage(Injured)
 void stablilizeBrokenBones(Injured)
 void healWithMed(Injured, Medicine)
 void placeInfusion(Injured)
 void nurseWounds(Injured)
 void intubate(Injured)
}
\end{lstlisting}
\begin{lstlisting}[style = dsl, caption = Deklaration von FireFighter, captionpos = b, label = lst_firefighter]
provided FireFighter extends Object{
 void stabilizeBrokenBones(Injured)
 void provideHeartbeatMassage(Injured)
 FireState extinguishFire(Fire)
 void free(Injured)
 void nurseWounds(Injured)
}
\end{lstlisting}
\chapter{Interfaces und Test-Implementierungen}\label{app_interfacesAndTests}
Im Folgenden werden zum einen die \Gls{Interface}s, die sich aus den Deklarationen der \emph{required Typen} aus dem Anhang \ref{app_evalTypes} ableiten lassen, aufgeführt. Zum anderen werden die Implementierungen der Testklassen, auf die die oben genannten \Gls{Interface}s über die Annotation $\texttt{RequiredTypeTestReference}$ verweisen, dargelegt. 
\\\\
Die Listings \ref{lst_interfaces_tei1} - \ref{lst_interfaces_tei7} zeigen dabei die Deklarationen der Java-Interfaces\footnote{Auf die Import-Anweisungen wurde verzichtet.} für die \emph{required Typen} aus Tabelle \ref{tab:eIShort} aus Kapitel \ref{chap_evaluation}.
\begin{lstlisting}[style = java, caption = Interface ElerFTFoerderprogrammeProvider, captionpos = b, label = lst_interfaces_tei1]
@RequiredTypeTestReference( testClasses = ElerFTFoerderprogrammProviderTest.class )
public interface ElerFTFoerderprogrammeProvider {

  Collection<ElerFTFoerderprogramm> getAlleFreigegebenenFPs();

  ElerFTFoerderprogramm getElerFTFoerderprogramm( DvAntragsJahr jahr, DvFoerderprogramm fp, Date date );
  
}
\end{lstlisting}
\begin{lstlisting}[style = java, caption = Interface FoerderprogrammeProvider, captionpos = b, label = lst_interfaces_tei2]
@RequiredTypeTestReference( testClasses = FoerderprogrammProviderTest.class )
public interface FoerderprogrammeProvider {

 Collection<Foerderprogramm> getAlleFreigegebenenFPs();

   Foerderprogramm getFoerderprogramm( DvFoerderprogramm fp, DvAntragsJahr jahr, Date date );
   
}
\end{lstlisting}
\begin{lstlisting}[style = java, caption = Interface MinimalFoerderprogrammeProvider, captionpos = b, label = lst_interfaces_tei3]
@RequiredTypeTestReference( testClasses = MinimalFoerderprogrammProviderTest.class )
public interface MinimalFoerderprogrammeProvider {

  Collection<String> getAlleFreigegebenenFPs();

  Foerderprogramm getFoerderprogramm( String fp, int jahr, Date date );
  
}
\end{lstlisting}
\begin{lstlisting}[style = java, caption = Interface IntubatingFireFighter, captionpos = b, label = lst_interfaces_tei4]
@RequiredTypeTestReference( testClasses = IntubatingFireFighterTest.class )
public interface IntubatingFireFighter {

  public void intubate( Injured injured );

  public FireState extinguishFire( Fire fire );
  
}
\end{lstlisting}
\begin{lstlisting}[style = java, caption = Interface IntubatingFreeing, captionpos = b, label = lst_interfaces_tei5]
@RequiredTypeTestReference( testClasses = IntubatingFreeingTest.class )
public interface IntubatingFreeing {

  public void intubate( Injured injured );

  public void free( Injured injured );
  
}
\end{lstlisting}
\begin{lstlisting}[style = java, caption = Interface IntubatingPatientFireFighter, captionpos = b, label = lst_interfaces_tei6]
@RequiredTypeTestReference( testClasses = IntubatingPatientFireFighterTest.class )
public interface IntubatingPatientFireFighter {

  public void intubate( IntubationPartient patient );

  public FireState extinguishFire( Fire fire );
  
}
\end{lstlisting}
\begin{lstlisting}[style = java, caption = Interface KOFGPCProvider, captionpos = b, label = lst_interfaces_tei7]
@RequiredTypeTestReference( testClasses = KOFGPCProviderTest.class )
public interface KOFGPCProvider {

  Collection<ElerFTKoFoerdergegenstand> getKOFGsVonFP( DvFoerderprogramm fp );
 
  Collection<Produktcode> getPCsZuKOFG( DvFoerdergegenstand fg, DvAntragsJahr aj ); 
  
}
\end{lstlisting}
Zu erkennen ist, dass jedes \Gls{Interface}s, wie in Abschnitt \ref{sec_Impl_CT} beschrieben, mit der Annotation $\texttt{RequiredTypeTestReference}$ versehen ist, über die auf eine Java-Klasse verwiesen wird, in der die Tests zu dem jeweiligen \emph{required Typ} implementiert sind.
\\\\
Die Listings \ref{lst_testklassen_tei1} - \ref{lst_testklassen_tei7} zeigen die Implementierungen dieser Testklassen\footnote{Auf die Import-Anweisungen wurde verzichtet.}.
\begin{lstlisting}[style = java, caption = Interface ElerFTFoerderprogrammProviderTest, captionpos = b, label = lst_testklassen_tei1]
public class ElerFTFoerderprogrammProviderTest implements TriedMethodCallsInfo {

  private ElerFTFoerderprogrammeProvider provider;
  private Collection<Method> calledMethods = new ArrayList<Method>();
  
  @RequiredTypeInstanceSetter
  public void setProvider( ElerFTFoerderprogrammeProvider provider ) {
    this.provider = provider;
  }

  @RequiredTypeTest
  public void testEmptyCollection() {
    addTriedMethodCall( getMethod( "getAlleFreigegebenenFPs", ElerFTFoerderprogrammeProvider.class ) );
    Collection<ElerFTFoerderprogramm> alleFreigegebenenFPs = provider.getAlleFreigegebenenFPs();
    assertThat( alleFreigegebenenFPs, notNullValue() );
  }

  @RequiredTypeTest
  public void testMockedFPCollection() {
    DvFoerderprogramm fp = DvFoerderprogramm.Factory.valueOf( DvFoerderprogramm.FP215 );
    addTriedMethodCall( getMethod( "getElerFTFoerderprogramm", ElerFTFoerderprogrammeProvider.class ) );
    ElerFTFoerderprogramm alleFreigegebenenFPs = provider.getElerFTFoerderprogramm( DvAntragsJahr.AJ2020,
        fp, new Date() );
    assertThat( alleFreigegebenenFPs, nullValue() );
  }

  @Override
  public void addTriedMethodCall( Method method ) {
    calledMethods.add( method );
  }

  @Override
  public Collection<Method> getTriedMethodCalls() {
    return calledMethods;
  }

}
\end{lstlisting}
\begin{lstlisting}[style = java, caption = Interface FoerderprogrammProviderTest, captionpos = b, label = lst_testklassen_tei2]
public class FoerderprogrammProviderTest implements TriedMethodCallsInfo {

  private FoerderprogrammeProvider provider;

  private Collection<Method> calledMethods = new ArrayList<Method>();

  @RequiredTypeInstanceSetter
  public void setProvider( FoerderprogrammeProvider provider ) {
    this.provider = provider;
  }

  @RequiredTypeTest
  public void testEmptyCollection() {
    addTriedMethodCall( getMethod( "getAlleFreigegebenenFPs", FoerderprogrammeProvider.class ) );
    Collection<Foerderprogramm> alleFreigegebenenFPs = provider.getAlleFreigegebenenFPs();
    assertThat( alleFreigegebenenFPs, notNullValue() );
  }

  @RequiredTypeTest
  public void testMockedFPCollection() {
    DvFoerderprogramm fp = DvFoerderprogramm.Factory.valueOf( DvFoerderprogramm.FP508 );
    addTriedMethodCall( getMethod( "getFoerderprogramm", FoerderprogrammeProvider.class ) );
    Foerderprogramm relevantFP = provider.getFoerderprogramm( fp, DvAntragsJahr.AJ2020,
        new Date() );
    assertThat( relevantFP, notNullValue() );
  }

  @RequiredTypeTest
  public void testDZFPCollection() {
    DvFoerderprogramm fp = DvFoerderprogramm.Factory.valueOf( DvFoerderprogramm.FP215 );
    addTriedMethodCall( getMethod( "getFoerderprogramm", FoerderprogrammeProvider.class ) );
    Foerderprogramm relevantFP = provider.getFoerderprogramm( fp, DvAntragsJahr.AJ2020,
        new Date() );
    assertThat( relevantFP, notNullValue() );
  }

  @Override
  public void addTriedMethodCall( Method method ) {
    calledMethods.add( method );
  }

  @Override
  public Collection<Method> getTriedMethodCalls() {
    return calledMethods;
  }

}
\end{lstlisting}
\begin{lstlisting}[style = java, caption = Interface MinimalFoerderprogrammProviderTest, captionpos = b, label = lst_testklassen_tei3]
public class MinimalFoerderprogrammProviderTest implements TriedMethodCallsInfo {

  private MinimalFoerderprogrammeProvider provider;
  private Collection<Method> calledMethods = new ArrayList<Method>();

  @RequiredTypeInstanceSetter
  public void setProvider( MinimalFoerderprogrammeProvider provider ) {
    this.provider = provider;
  }

  @RequiredTypeTest
  public void testEmptyCollection() {
    addTriedMethodCall( getMethod( "getAlleFreigegebenenFPs", MinimalFoerderprogrammeProvider.class) );
    Collection<String> alleFreigegebenenFPs = provider.getAlleFreigegebenenFPs();
    assertThat( alleFreigegebenenFPs, notNullValue() );
  }

  @RequiredTypeTest
  public void testGetFoerderprogramm() {
    String fpCode = "215";
    addTriedMethodCall( getMethod( "getFoerderprogramm", MinimalFoerderprogrammeProvider.class) );
    Foerderprogramm fp = provider.getFoerderprogramm( fpCode, 2015, new Date() );
    assertThat( fp, notNullValue() );
    DvFoerderprogramm dvFP = fp.getFoerderprogramm();
    assertThat( dvFP, notNullValue() );

    String code = dvFP.getCode();
    assertThat( fpCode, equalTo( code ) );

  }

  @Override
  public void addTriedMethodCall( Method method ) {
    calledMethods.add( method );
  }

  @Override
  public Collection<Method> getTriedMethodCalls() {
    return calledMethods ;
  }

}
\end{lstlisting}
\begin{lstlisting}[style = java, caption = Interface IntubatingFireFighterTest, captionpos = b, label = lst_testklassen_tei4]
public class IntubatingFireFighterTest implements TriedMethodCallsInfo {

	private IntubatingFireFighter intubatingFireFighter;
	private Collection<Method> calledMethods = new ArrayList<Method>();

	@RequiredTypeInstanceSetter
	public void setProvider(IntubatingFireFighter intubatingFireFighter) {
		this.intubatingFireFighter = intubatingFireFighter;
	}

	@RequiredTypeTest
	public void free() {
		Fire fire = new Fire();
		addTriedMethodCall(getMethod("extinguishFire", IntubatingFireFighter.class));
		FireState fireState = intubatingFireFighter.extinguishFire(fire);
		assertTrue(Objects.equals(fireState.isActive(), fire.isActive()));
		assertFalse(fire.isActive());
	}

	@RequiredTypeTest
	public void intubate() {
		Collection<Suffer> suffer = Arrays.asList(Suffer.BREATH_PROBLEMS);
		Injured patient = new Injured(suffer);
		addTriedMethodCall(getMethod("intubate", IntubatingFireFighter.class));
		intubatingFireFighter.intubate(patient);
		assertTrue(patient.isStabilized());
	}

	@Override
	public void addTriedMethodCall(Method m) {
		calledMethods.add(m);
	}

	@Override
	public Collection<Method> getTriedMethodCalls() {
		return calledMethods;
	}

}
\end{lstlisting}
\begin{lstlisting}[style = java, caption = Interface IntubatingFreeingTest, captionpos = b, label = lst_testklassen_tei5]
public class IntubatingFreeingTest implements TriedMethodCallsInfo {

	private IntubatingFreeing intubatingFreeing;
	private Collection<Method> calledMethods = new ArrayList<Method>();

	@RequiredTypeInstanceSetter
	public void setProvider(IntubatingFreeing intubatingFireFighter) {
		this.intubatingFreeing = intubatingFireFighter;
	}

	@RequiredTypeTest
	public void free() {
		Collection<Suffer> suffer = Arrays.asList(Suffer.LOCKED);
		Injured patient = new Injured(suffer);
		addTriedMethodCall(getMethod("free", IntubatingFreeing.class));
		intubatingFreeing.free(patient);
		assertTrue(patient.isStabilized());
	}

	@RequiredTypeTest
	public void intubate() {
		Collection<Suffer> suffer = Arrays.asList(Suffer.BREATH_PROBLEMS);
		Injured patient = new Injured(suffer);
		addTriedMethodCall(getMethod("intubate", IntubatingFreeing.class));
		intubatingFreeing.intubate(patient);
		assertTrue(patient.isStabilized());
	}

	@Override
	public void addTriedMethodCall(Method m) {
		calledMethods.add(m);
	}

	@Override
	public Collection<Method> getTriedMethodCalls() {
		return calledMethods;
	}
	
}
\end{lstlisting}
\begin{lstlisting}[style = java, caption = Interface IntubatingPatientFireFighterTest, captionpos = b, label = lst_testklassen_tei6]
public class IntubatingPatientFireFighterTest implements TriedMethodCallsInfo {

	private IntubatingPatientFireFighter intubatingPatientFireFighter;
	private Collection<Method> calledMethods = new ArrayList<Method>();

	@RequiredTypeInstanceSetter
	public void setProvider(IntubatingPatientFireFighter intubatingFireFighter) {
		this.intubatingPatientFireFighter = intubatingFireFighter;
	}

	@RequiredTypeTest
	public void extinguishFire() {
		Fire fire = new Fire();
		addTriedMethodCall(getMethod("extinguishFire", IntubatingPatientFireFighter.class));
		FireState fireState = intubatingPatientFireFighter.extinguishFire(fire);
		assertTrue(Objects.equals(fireState.isActive(), fire.isActive()));
		assertFalse(fire.isActive());
	}

	@RequiredTypeTest
	public void intubate() {
		IntubationPartient patient = new IntubationPartient();
		addTriedMethodCall(getMethod("intubate", IntubatingPatientFireFighter.class));
		intubatingPatientFireFighter.intubate(patient);
		assertTrue(patient.isIntubated());
	}

	@Override
	public void addTriedMethodCall(Method m) {
		calledMethods.add(m);
	}

	@Override
	public Collection<Method> getTriedMethodCalls() {
		return calledMethods;
	}

}
\end{lstlisting}
\begin{lstlisting}[style = java, caption = Interface KOFGPCProviderTest, captionpos = b, label = lst_testklassen_tei7]
public class KOFGPCProviderTest implements TriedMethodCallsInfo {

  private KOFGPCProvider provider;

  private Collection<Method> calledMethods = new ArrayList<Method>();

  @RequiredTypeInstanceSetter
  public void setProvider( KOFGPCProvider provider ) {
    this.provider = provider;
  }

  @RequiredTypeTest
  public void testKOFGsCollection() {
    DvFoerderprogramm fp = DvFoerderprogramm.Factory.valueOf( DvFoerderprogramm.FP508 );
    addTriedMethodCall( getMethod( "getKOFGsVonFP", KOFGPCProvider.class ) );
    Collection<ElerFTKoFoerdergegenstand> kofGsVonFP = provider.getKOFGsVonFP( fp );
    assertThat( kofGsVonFP, notNullValue() );
    assertThat( kofGsVonFP.isEmpty(), equalTo( false ) );
    assertThat( kofGsVonFP.stream().anyMatch( fg -> fg.getCode().equals( "KO508" ) ), equalTo( true ) );
  }

  @RequiredTypeTest
  public void testPCsCollection() {
    DvFoerdergegenstand fg = DvFoerdergegenstand.Factory.valueOf( 20155080025L );
    addTriedMethodCall( getMethod( "getPCsZuKOFG", KOFGPCProvider.class ) );
    Collection<Produktcode> pcs = provider.getPCsZuKOFG( fg, DvAntragsJahr.AJ2020 );
    assertThat( pcs, notNullValue() );
    assertThat( pcs.isEmpty(), equalTo( false ) );
  }

  @Override
  public void addTriedMethodCall( Method m ) {
    this.calledMethods.add( m );
  }

  @Override
  public Collection<Method> getTriedMethodCalls() {
    return calledMethods;
  }

}
\end{lstlisting}
\noindent
Hier ist zu erkennen, dass die Testklassen alle das \Gls{Interface}s $\texttt{TriedMethodCallsInfo}$ implementieren, über das die für die Heuristik \emph{BL\_NMC} benötigten Informationen (siehe Abschnitt \ref{sec_bl_nmc}) ermittelt werden. Ebenso ist die Implementierung dieses \Gls{Interface}s in den oben genannten Listings zu erkennen.
\chapter{Ergebnisse für die Heuristik LMF (Ergänzungen)}\label{app_matcherratingEval}
In diesem Abschnitt werden die Untersuchungsergebnisse der Heuristik \emph{LMF} mit allen Varianten zur Bestimmung des Matcherratings aus Abschnitt \ref{sec_lmf} dargelegt. Dieses Kapitel bildet somit eine Ergänzung zu Abschnitt \ref{sec_evalLMF}. Die darin beschriebenen Ergebnisse der Variante \emph{1.1} werden der Vollständigkeit halber in dem vorliegenden Kapitel nochmals aufgeführt.
\\\\
Die folgenden Ergebnisse beziehen sich auf die in Kapitel \ref{chap_evaluation} vorgestellten \emph{required Typen} \emph{TEI1}-\emph{TEI7}.
\pagebreak
\section*{Ergebnisse für Variante 1.1}
\begin{multicols}{3}
\vft{1}{5}{$p(44)-6$}{1}{0}{Ergebnisse \emph{LMF} mit Variante 1.1 für TEI1 1.~\mbox{Durchlauf}}{lmf11_TEI1_1_app}
\vft{1}{1889}{$p(55)-1890$}{1}{0}{Ergebnisse \emph{LMF} mit Variante 1.1 für TEI2 1.~\mbox{Durchlauf}}{lmf11_TEI2_1_app}
\vft{1}{1463}{$p(50)-1464$}{1}{0}{Ergebnisse \emph{LMF} mit Variante 1.1 für TEI3 1.~\mbox{Durchlauf}}{lmf11_TEI3_1_app}
\end{multicols}

\begin{multicols}{2}
\vft{1}{$1174$}{0}{0}{0}{Ergebnisse \emph{LMF} mit Variante 1.1 für TEI4 1. Durchlauf}{lmf11_TEI4_1_app}
\vft{2}{2}{$p(2247)-3$}{1}{0}{Ergebnisse \emph{LMF} mit Variante 1.1 für TEI4 2. Durchlauf}{lmf11_TEI4_2_app}
\end{multicols}
\begin{multicols}{2}
\vft{1}{$4984$}{0}{0}{0}{Ergebnisse \emph{LMF} mit Variante 1.1 für TEI5 1. Durchlauf}{lmf11_TEI5_1_app}
\vft{2}{32}{$p(2775)-33$}{1}{0}{Ergebnisse \emph{LMF} mit Variante 1.1 für TEI5 2. Durchlauf}{lmf11_TEI5_2_app}
\end{multicols}
\begin{multicols}{2}
\vft{1}{$1051$}{0}{0}{0}{Ergebnisse \emph{LMF} mit Variante 1.1 für TEI6 1. Durchlauf}{lmf11_TEI6_1_app}
\vft{2}{0}{$p(1323)-1$}{1}{0}{Ergebnisse \emph{LMF} mit Variante 1.1 für TEI6 2. Durchlauf}{lmf11_TEI6_2_app}
\end{multicols}
\begin{multicols}{2}
\vft{1}{$161294$}{0}{0}{0}{Ergebnisse \emph{LMF} mit Variante 1.1 für TEI7 1. Durchlauf}{lmf11_TEI7_1_app}
\vft{2}{7641}{$p(52150)-7642$}{1}{0}{Ergebnisse \emph{LMF} mit Variante 1.1 für TEI7 2. Durchlauf}{lmf11_TEI7_2_app}
\end{multicols}
\section*{Ergebnisse für Variante 1.2}
\begin{multicols}{3}
\vft{1}{1}{$p(44)-2$}{1}{0}{Ergebnisse \emph{LMF} mit Variante 1.2 für TEI1 1.~\mbox{Durchlauf}}{lmf12_TEI1_1}
\vft{1}{2783}{$p(55)-2784$}{1}{0}{Ergebnisse \emph{LMF} mit Variante 1.2 für TEI2 1.~\mbox{Durchlauf}}{lmf12_TEI2_1}
\vft{1}{1830}{$p(50)-1831$}{1}{0}{Ergebnisse \emph{LMF} mit Variante 1.2 für TEI3 1.~\mbox{Durchlauf}}{lmf12_TEI3_1}
\end{multicols}
\begin{multicols}{2}
\vft{1}{$1174$}{0}{0}{0}{Ergebnisse \emph{LMF} mit Variante 1.2 für TEI4 1. Durchlauf}{lmf12_TEI4_1}
\vft{2}{3}{$p(2247)-4$}{1}{0}{Ergebnisse \emph{LMF} mit Variante 1.2 für TEI4 2. Durchlauf}{lmf12_TEI4_2}
\end{multicols}
\begin{multicols}{2}
\vft{1}{$4984$}{0}{0}{0}{Ergebnisse \emph{LMF} mit Variante 1.2 für TEI5 1. Durchlauf}{lmf12_TEI5_1}
\vft{2}{3}{$p(2775)-4$}{1}{0}{Ergebnisse \emph{LMF} mit Variante 1.2 für TEI5 2. Durchlauf}{lmf12_TEI5_2}
\end{multicols}

\begin{multicols}{2}
\vft{1}{$1051$}{0}{0}{0}{Ergebnisse \emph{LMF} mit Variante 1.2 für TEI6 1. Durchlauf}{lmf12_TEI6_1}
\vft{2}{0}{$p(1323)-1$}{1}{0}{Ergebnisse \emph{LMF} mit Variante 1.2 für TEI6 2. Durchlauf}{lmf12_TEI6_2}
\end{multicols}
\pagebreak
\begin{multicols}{2}
\vft{1}{$161294$}{0}{0}{0}{Ergebnisse \emph{LMF} mit Variante 1.2 für TEI7 1. Durchlauf}{lmf12_TEI7_1}
\vft{2}{161298}{$p(52150)-161299$}{1}{0}{Ergebnisse \emph{LMF} mit Variante 1.2 für TEI7 2. Durchlauf}{lmf12_TEI7_2}
\end{multicols}
\section*{Ergebnisse für Variante 1.3}
\begin{multicols}{3}
\vft{1}{50}{$p(44)-51$}{1}{0}{Ergebnisse \emph{LMF} mit Variante 1.3 für TEI1 1.~\mbox{Durchlauf}}{lmf13_TEI1_1}
\vft{1}{20}{$p(55)-21$}{1}{0}{Ergebnisse \emph{LMF} mit Variante 1.3 für TEI2 1.~\mbox{Durchlauf}}{lmf13_TEI2_1}
\vft{1}{121}{$p(50)-122$}{1}{0}{Ergebnisse \emph{LMF} mit Variante 1.3 für TEI3 1.~\mbox{Durchlauf}}{lmf13_TEI3_1}
\end{multicols}
\begin{multicols}{2}
\vft{1}{$1174$}{0}{0}{0}{Ergebnisse \emph{LMF} mit Variante 1.3 für TEI4 1. Durchlauf}{lmf13_TEI4_1}
\vft{2}{57}{$p(2247)-58$}{1}{0}{Ergebnisse \emph{LMF} mit Variante 1.3 für TEI4 2. Durchlauf}{lmf13_TEI4_2}
\end{multicols}
\begin{multicols}{2}
\vft{1}{$4984$}{0}{0}{0}{Ergebnisse \emph{LMF} mit Variante 1.3 für TEI5 1. Durchlauf}{lmf13_TEI5_1}
\vft{2}{6246}{$p(2775)-6247$}{1}{0}{Ergebnisse \emph{LMF} mit Variante 1.3 für TEI5 2. Durchlauf}{lmf13_TEI5_2}
\end{multicols}

\begin{multicols}{2}
\vft{1}{$1051$}{0}{0}{0}{Ergebnisse \emph{LMF} mit Variante 1.3 für TEI6 1. Durchlauf}{lmf13_TEI6_1}
\vft{2}{5}{$p(1323)-6$}{1}{0}{Ergebnisse \emph{LMF} mit Variante 1.3 für TEI6 2. Durchlauf}{lmf13_TEI6_2}
\end{multicols}

\begin{multicols}{2}
\vft{1}{$161294$}{0}{0}{0}{Ergebnisse \emph{LMF} mit Variante 1.3 für TEI7 1. Durchlauf}{lmf13_TEI7_1}
\vft{2}{121074}{$p(52150)-121075$}{1}{0}{Ergebnisse \emph{LMF} mit Variante 1.3 für TEI7 2. Durchlauf}{lmf13_TEI7_2}
\end{multicols}

\section*{Ergebnisse für Variante 1.4}
\begin{multicols}{3}
\vft{1}{45}{$p(44)-46$}{1}{0}{Ergebnisse \emph{LMF} mit Variante 1.4 für TEI1 1.~\mbox{Durchlauf}}{lmf14_TEI1_1}
\vft{1}{2025}{$p(55)-2026$}{1}{0}{Ergebnisse \emph{LMF} mit Variante 1.4 für TEI2 1.~\mbox{Durchlauf}}{lmf14_TEI2_1}
\vft{1}{1517}{$p(50)-1518$}{1}{0}{Ergebnisse \emph{LMF} mit Variante 1.4 für TEI3 1.~\mbox{Durchlauf}}{lmf14_TEI3_1}
\end{multicols}
\begin{multicols}{2}
\vft{1}{$1174$}{0}{0}{0}{Ergebnisse \emph{LMF} mit Variante 1.4 für TEI4 1. Durchlauf}{lmf14_TEI4_1}
\vft{2}{4}{$p(2247)-5$}{1}{0}{Ergebnisse \emph{LMF} mit Variante 1.4 für TEI4 2. Durchlauf}{lmf14_TEI4_2}
\end{multicols}
\begin{multicols}{2}
\vft{1}{$4984$}{0}{0}{0}{Ergebnisse \emph{LMF} mit Variante 1.4 für TEI5 1. Durchlauf}{lmf14_TEI5_1}
\vft{2}{34}{$p(2775)-35$}{1}{0}{Ergebnisse \emph{LMF} mit Variante 1.4 für TEI5 2. Durchlauf}{lmf14_TEI5_2}
\end{multicols}

\begin{multicols}{2}
\vft{1}{$1051$}{0}{0}{0}{Ergebnisse \emph{LMF} mit Variante 1.4 für TEI6 1. Durchlauf}{lmf14_TEI6_1}
\vft{2}{0}{$p(1323)-1$}{1}{0}{Ergebnisse \emph{LMF} mit Variante 1.4 für TEI6 2. Durchlauf}{lmf14_TEI6_2}
\end{multicols}

\begin{multicols}{2}
\vft{1}{$161294$}{0}{0}{0}{Ergebnisse \emph{LMF} mit Variante 1.4 für TEI7 1. Durchlauf}{lmf14_TEI7_1}
\vft{2}{21068}{$p(52150)-21069$}{1}{0}{Ergebnisse \emph{LMF} mit Variante 1.4 für TEI7 2. Durchlauf}{lmf14_TEI7_2}
\end{multicols}

\section*{Ergebnisse für Variante 2.1}
\begin{multicols}{3}
\vft{1}{8}{$p(44)-9$}{1}{0}{Ergebnisse \emph{LMF} mit Variante 2.1 für TEI1 1.~\mbox{Durchlauf}}{lmf21_TEI1_1}
\vft{1}{3975}{$p(55)-3976$}{1}{0}{Ergebnisse \emph{LMF} mit Variante 2.1 für TEI2 1.~\mbox{Durchlauf}}{lmf21_TEI2_1}
\vft{1}{2933}{$p(50)-2934$}{1}{0}{Ergebnisse \emph{LMF} mit Variante 2.1 für TEI3 1.~\mbox{Durchlauf}}{lmf21_TEI3_1}
\end{multicols}
\begin{multicols}{2}
\vft{1}{$1174$}{0}{0}{0}{Ergebnisse \emph{LMF} mit Variante 2.1 für TEI4 1. Durchlauf}{lmf21_TEI4_1}
\vft{2}{2}{$p(2247)-3$}{1}{0}{Ergebnisse \emph{LMF} mit Variante 2.1 für TEI4 2. Durchlauf}{lmf21_TEI4_2}
\end{multicols}
\begin{multicols}{2}
\vft{1}{$4984$}{0}{0}{0}{Ergebnisse \emph{LMF} mit Variante 2.1 für TEI5 1. Durchlauf}{lmf21_TEI5_1}
\vft{2}{32}{$p(2775)-33$}{1}{0}{Ergebnisse \emph{LMF} mit Variante 2.1 für TEI5 2. Durchlauf}{lmf21_TEI5_2}
\end{multicols}

\begin{multicols}{2}
\vft{1}{$1051$}{0}{0}{0}{Ergebnisse \emph{LMF} mit Variante 2.1 für TEI6 1. Durchlauf}{lmf21_TEI6_1}
\vft{2}{0}{$p(1323)-1$}{1}{0}{Ergebnisse \emph{LMF} mit Variante 2.1 für TEI6 2. Durchlauf}{lmf21_TEI6_2}
\end{multicols}

\begin{multicols}{2}
\vft{1}{$161294$}{0}{0}{0}{Ergebnisse \emph{LMF} mit Variante 2.1 für TEI7 1. Durchlauf}{lmf21_TEI7_1}
\vft{2}{32018037}{$p(52150)-32018038$}{1}{0}{Ergebnisse \emph{LMF} mit Variante 2.1 für TEI7 2. Durchlauf}{lmf21_TEI7_2}
\end{multicols}

\section*{Ergebnisse für Variante 2.2}
\begin{multicols}{3}
\vft{1}{0}{$p(44)-1$}{1}{0}{Ergebnisse \emph{LMF} mit Variante 2.2 für TEI1 1.~\mbox{Durchlauf}}{lmf22_TEI1_1}
\vft{1}{8007}{$p(55)-8008$}{1}{0}{Ergebnisse \emph{LMF} mit Variante 2.2 für TEI2 1.~\mbox{Durchlauf}}{lmf22_TEI2_1}
\vft{1}{7104}{$p(50)-7105$}{1}{0}{Ergebnisse \emph{LMF} mit Variante 2.2 für TEI3 1.~\mbox{Durchlauf}}{lmf22_TEI3_1}
\end{multicols}
\begin{multicols}{2}
\vft{1}{$1174$}{0}{0}{0}{Ergebnisse \emph{LMF} mit Variante 2.2 für TEI4 1. Durchlauf}{lmf22_TEI4_1}
\vft{2}{0}{$p(2247)-1$}{1}{0}{Ergebnisse \emph{LMF} mit Variante 2.2 für TEI4 2. Durchlauf}{lmf22_TEI4_2}
\end{multicols}
\begin{multicols}{2}
\vft{1}{$4984$}{0}{0}{0}{Ergebnisse \emph{LMF} mit Variante 2.2 für TEI5 1. Durchlauf}{lmf22_TEI5_1}
\vft{2}{0}{$p(2775)-1$}{1}{0}{Ergebnisse \emph{LMF} mit Variante 2.2 für TEI5 2. Durchlauf}{lmf22_TEI5_2}
\end{multicols}

\begin{multicols}{2}
\vft{1}{$1051$}{0}{0}{0}{Ergebnisse \emph{LMF} mit Variante 2.2 für TEI6 1. Durchlauf}{lmf22_TEI6_1}
\vft{2}{0}{$p(1323)-1$}{1}{0}{Ergebnisse \emph{LMF} mit Variante 2.2 für TEI6 2. Durchlauf}{lmf22_TEI6_2}
\end{multicols}

\begin{multicols}{2}
\vft{1}{$161294$}{0}{0}{0}{Ergebnisse \emph{LMF} mit Variante 2.2 für TEI7 1. Durchlauf}{lmf22_TEI7_1}
\vft{2}{2840500}{$p(52150)-2840501$}{1}{0}{Ergebnisse \emph{LMF} mit Variante 2.2 für TEI7 2. Durchlauf}{lmf22_TEI7_2}
\end{multicols}

\section*{Ergebnisse für Variante 2.3}
\begin{multicols}{3}
\vft{1}{5}{$p(44)-6$}{1}{0}{Ergebnisse \emph{LMF} mit Variante 2.3 für TEI1 1.~\mbox{Durchlauf}}{lmf23_TEI1_1}
\vft{1}{2642}{$p(55)-2643$}{1}{0}{Ergebnisse \emph{LMF} mit Variante 2.3 für TEI2 1.~\mbox{Durchlauf}}{lmf23_TEI2_1}
\vft{1}{1686}{$p(50)-1687$}{1}{0}{Ergebnisse \emph{LMF} mit Variante 2.3 für TEI3 1.~\mbox{Durchlauf}}{lmf23_TEI3_1}
\end{multicols}
\begin{multicols}{2}
\vft{1}{$1174$}{0}{0}{0}{Ergebnisse \emph{LMF} mit Variante 2.3 für TEI4 1. Durchlauf}{lmf23_TEI4_1}
\vft{2}{67}{$p(2247)-68$}{1}{0}{Ergebnisse \emph{LMF} mit Variante 2.3 für TEI4 2. Durchlauf}{lmf23_TEI4_2}
\end{multicols}
\begin{multicols}{2}
\vft{1}{$4984$}{0}{0}{0}{Ergebnisse \emph{LMF} mit Variante 2.3 für TEI5 1. Durchlauf}{lmf23_TEI5_1}
\vft{2}{5413}{$p(2775)-5414$}{1}{0}{Ergebnisse \emph{LMF} mit Variante 2.3 für TEI5 2. Durchlauf}{lmf23_TEI5_2}
\end{multicols}

\begin{multicols}{2}
\vft{1}{$1051$}{0}{0}{0}{Ergebnisse \emph{LMF} mit Variante 2.3 für TEI6 1. Durchlauf}{lmf23_TEI6_1}
\vft{2}{11}{$p(1323)-12$}{1}{0}{Ergebnisse \emph{LMF} mit Variante 2.3 für TEI6 2. Durchlauf}{lmf23_TEI6_2}
\end{multicols}

\begin{multicols}{2}
\vft{1}{$161294$}{0}{0}{0}{Ergebnisse \emph{LMF} mit Variante 2.3 für TEI7 1. Durchlauf}{lmf23_TEI7_1}
\vft{2}{8084753}{$p(52150)-8084754$}{1}{0}{Ergebnisse \emph{LMF} mit Variante 2.3 für TEI7 2. Durchlauf}{lmf23_TEI7_2}
\end{multicols}

\section*{Ergebnisse für Variante 2.4}
\begin{multicols}{3}
\vft{1}{20}{$p(44)-21$}{1}{0}{Ergebnisse \emph{LMF} mit Variante 2.4 für TEI1 1.~\mbox{Durchlauf}}{lmf24_TEI1_1}
\vft{1}{3928}{$p(55)-3929$}{1}{0}{Ergebnisse \emph{LMF} mit Variante 2.4 für TEI2 1.~\mbox{Durchlauf}}{lmf24_TEI2_1}
\vft{1}{3117}{$p(50)-3118$}{1}{0}{Ergebnisse \emph{LMF} mit Variante 2.4 für TEI3 1.~\mbox{Durchlauf}}{lmf24_TEI3_1}
\end{multicols}
\begin{multicols}{2}
\vft{1}{$1174$}{0}{0}{0}{Ergebnisse \emph{LMF} mit Variante 2.4 für TEI4 1. Durchlauf}{lmf24_TEI4_1}
\vft{2}{3}{$p(2247)-4$}{1}{0}{Ergebnisse \emph{LMF} mit Variante 2.4 für TEI4 2. Durchlauf}{lmf24_TEI4_2}
\end{multicols}
\begin{multicols}{2}
\vft{1}{$4984$}{0}{0}{0}{Ergebnisse \emph{LMF} mit Variante 2.4 für TEI5 1. Durchlauf}{lmf24_TEI5_1}
\vft{2}{33}{$p(2775)-34$}{1}{0}{Ergebnisse \emph{LMF} mit Variante 2.4 für TEI5 2. Durchlauf}{lmf24_TEI5_2}
\end{multicols}

\begin{multicols}{2}
\vft{1}{$1051$}{0}{0}{0}{Ergebnisse \emph{LMF} mit Variante 2.4 für TEI6 1. Durchlauf}{lmf24_TEI6_1}
\vft{2}{0}{$p(1323)-1$}{1}{0}{Ergebnisse \emph{LMF} mit Variante 2.4 für TEI6 2. Durchlauf}{lmf24_TEI6_2}
\end{multicols}

\begin{multicols}{2}
\vft{1}{$161294$}{0}{0}{0}{Ergebnisse \emph{LMF} mit Variante 2.4 für TEI7 1. Durchlauf}{lmf24_TEI7_1}
\vft{2}{10899025}{$p(52150)-10899026$}{1}{0}{Ergebnisse \emph{LMF} mit Variante 2.4 für TEI7 2. Durchlauf}{lmf24_TEI7_2}
\end{multicols}

\section*{Ergebnisse für Variante 3.1}
\begin{multicols}{3}
\vft{1}{1037}{$p(44)-1038$}{1}{0}{Ergebnisse \emph{LMF} mit Variante 3.1 für TEI1 1.~\mbox{Durchlauf}}{lmf31_TEI1_1}
\vft{1}{3956}{$p(55)-3957$}{1}{0}{Ergebnisse \emph{LMF} mit Variante 3.1 für TEI2 1.~\mbox{Durchlauf}}{lmf31_TEI2_1}
\vft{1}{3851}{$p(50)-3852$}{1}{0}{Ergebnisse \emph{LMF} mit Variante 3.1 für TEI3 1.~\mbox{Durchlauf}}{lmf31_TEI3_1}
\end{multicols}
\begin{multicols}{2}
\vft{1}{$1174$}{0}{0}{0}{Ergebnisse \emph{LMF} mit Variante 3.1 für TEI4 1. Durchlauf}{lmf31_TEI4_1}
\vft{2}{191}{$p(2247)-192$}{1}{0}{Ergebnisse \emph{LMF} mit Variante 3.1 für TEI4 2. Durchlauf}{lmf31_TEI4_2}
\end{multicols}
\begin{multicols}{2}
\vft{1}{$4984$}{0}{0}{0}{Ergebnisse \emph{LMF} mit Variante 3.1 für TEI5 1. Durchlauf}{lmf31_TEI5_1}
\vft{2}{1608}{$p(2775)-1609$}{1}{0}{Ergebnisse \emph{LMF} mit Variante 3.1 für TEI5 2. Durchlauf}{lmf31_TEI5_2}
\end{multicols}

\begin{multicols}{2}
\vft{1}{$1051$}{0}{0}{0}{Ergebnisse \emph{LMF} mit Variante 3.1 für TEI6 1. Durchlauf}{lmf31_TEI6_1}
\vft{2}{37}{$p(1323)-38$}{1}{0}{Ergebnisse \emph{LMF} mit Variante 3.1 für TEI6 2. Durchlauf}{lmf31_TEI6_2}
\end{multicols}

\begin{multicols}{2}
\vft{1}{$161294$}{0}{0}{0}{Ergebnisse \emph{LMF} mit Variante 3.1 für TEI7 1. Durchlauf}{lmf31_TEI7_1}
\vft{2}{758477}{$p(52150)-758478$}{1}{0}{Ergebnisse \emph{LMF} mit Variante 3.1 für TEI7 2. Durchlauf}{lmf31_TEI7_2}
\end{multicols}

\section*{Ergebnisse für Variante 3.2}
\begin{multicols}{3}
\vft{1}{1097}{$p(44)-1098$}{1}{0}{Ergebnisse \emph{LMF} mit Variante 3.2 für TEI1 1.~\mbox{Durchlauf}}{lmf32_TEI1_1}
\vft{1}{386}{$p(55)-387$}{1}{0}{Ergebnisse \emph{LMF} mit Variante 3.2 für TEI2 1.~\mbox{Durchlauf}}{lmf32_TEI2_1}
\vft{1}{121}{$p(50)-122$}{1}{0}{Ergebnisse \emph{LMF} mit Variante 3.2 für TEI3 1.~\mbox{Durchlauf}}{lmf32_TEI3_1}
\end{multicols}
\begin{multicols}{2}
\vft{1}{$1174$}{0}{0}{0}{Ergebnisse \emph{LMF} mit Variante 3.2 für TEI4 1. Durchlauf}{lmf32_TEI4_1}
\vft{2}{524}{$p(2247)-525$}{1}{0}{Ergebnisse \emph{LMF} mit Variante 3.2 für TEI4 2. Durchlauf}{lmf32_TEI4_2}
\end{multicols}
\begin{multicols}{2}
\vft{1}{$4984$}{0}{0}{0}{Ergebnisse \emph{LMF} mit Variante 3.2 für TEI5 1. Durchlauf}{lmf32_TEI5_1}
\vft{2}{3402}{$p(2775)-3403$}{1}{0}{Ergebnisse \emph{LMF} mit Variante 3.2 für TEI5 2. Durchlauf}{lmf32_TEI5_2}
\end{multicols}

\begin{multicols}{2}
\vft{1}{$1051$}{0}{0}{0}{Ergebnisse \emph{LMF} mit Variante 3.2 für TEI6 1. Durchlauf}{lmf32_TEI6_1}
\vft{2}{115}{$p(1323)-116$}{1}{0}{Ergebnisse \emph{LMF} mit Variante 3.2 für TEI6 2. Durchlauf}{lmf32_TEI6_2}
\end{multicols}

\begin{multicols}{2}
\vft{1}{$161294$}{0}{0}{0}{Ergebnisse \emph{LMF} mit Variante 3.2 für TEI7 1. Durchlauf}{lmf32_TEI7_1}
\vft{2}{379600}{$p(52150)-379601$}{1}{0}{Ergebnisse \emph{LMF} mit Variante 3.2 für TEI7 2. Durchlauf}{lmf32_TEI7_2}
\end{multicols}

\section*{Ergebnisse für Variante 3.3}
\begin{multicols}{3}
\vft{1}{4088}{$p(44)-4089$}{1}{0}{Ergebnisse \emph{LMF} mit Variante 3.3 für TEI1 1.~\mbox{Durchlauf}}{lmf33_TEI1_1}
\vft{1}{2005}{$p(55)-2006$}{1}{0}{Ergebnisse \emph{LMF} mit Variante 3.3 für TEI2 1.~\mbox{Durchlauf}}{lmf33_TEI2_1}
\vft{1}{1776}{$p(50)-1777$}{1}{0}{Ergebnisse \emph{LMF} mit Variante 3.3 für TEI3 1.~\mbox{Durchlauf}}{lmf33_TEI3_1}
\end{multicols}
\begin{multicols}{2}
\vft{1}{$1174$}{0}{0}{0}{Ergebnisse \emph{LMF} mit Variante 3.3 für TEI4 1. Durchlauf}{lmf33_TEI4_1}
\vft{2}{55881}{$p(2247)-55882$}{1}{0}{Ergebnisse \emph{LMF} mit Variante 3.3 für TEI4 2. Durchlauf}{lmf33_TEI4_2}
\end{multicols}
\begin{multicols}{2}
\vft{1}{$4984$}{0}{0}{0}{Ergebnisse \emph{LMF} mit Variante 3.3 für TEI5 1. Durchlauf}{lmf33_TEI5_1}
\vft{2}{239768}{$p(2775)-239769$}{1}{0}{Ergebnisse \emph{LMF} mit Variante 3.3 für TEI5 2. Durchlauf}{lmf33_TEI5_2}
\end{multicols}

\begin{multicols}{2}
\vft{1}{$1051$}{0}{0}{0}{Ergebnisse \emph{LMF} mit Variante 3.3 für TEI6 1. Durchlauf}{lmf33_TEI6_1}
\vft{2}{42748}{$p(1323)-42749$}{1}{0}{Ergebnisse \emph{LMF} mit Variante 3.3 für TEI6 2. Durchlauf}{lmf33_TEI6_2}
\end{multicols}

\begin{multicols}{2}
\vft{1}{$161294$}{0}{0}{0}{Ergebnisse \emph{LMF} mit Variante 3.3 für TEI7 1. Durchlauf}{lmf33_TEI7_1}
\vft{2}{4912200}{$p(52150)-4912201$}{1}{0}{Ergebnisse \emph{LMF} mit Variante 3.3 für TEI7 2. Durchlauf}{lmf33_TEI7_2}
\end{multicols}

\section*{Ergebnisse für Variante 3.4}
\begin{multicols}{3}
\vft{1}{5105}{$p(44)-5106$}{1}{0}{Ergebnisse \emph{LMF} mit Variante 3.4 für TEI1 1.~\mbox{Durchlauf}}{lmf34_TEI1_1}
\vft{1}{3598}{$p(55)-3599$}{1}{0}{Ergebnisse \emph{LMF} mit Variante 3.4 für TEI2 1.~\mbox{Durchlauf}}{lmf34_TEI2_1}
\vft{1}{3421}{$p(50)-3422$}{1}{0}{Ergebnisse \emph{LMF} mit Variante 3.4 für TEI3 1.~\mbox{Durchlauf}}{lmf34_TEI3_1}
\end{multicols}
\begin{multicols}{2}
\vft{1}{$1174$}{0}{0}{0}{Ergebnisse \emph{LMF} mit Variante 3.4 für TEI4 1. Durchlauf}{lmf34_TEI4_1}
\vft{2}{762}{$p(2247)-763$}{1}{0}{Ergebnisse \emph{LMF} mit Variante 3.4 für TEI4 2. Durchlauf}{lmf34_TEI4_2}
\end{multicols}
\begin{multicols}{2}
\vft{1}{$4984$}{0}{0}{0}{Ergebnisse \emph{LMF} mit Variante 3.4 für TEI5 1. Durchlauf}{lmf34_TEI5_1}
\vft{2}{6130}{$p(2775)-6131$}{1}{0}{Ergebnisse \emph{LMF} mit Variante 3.4 für TEI5 2. Durchlauf}{lmf34_TEI5_2}
\end{multicols}

\begin{multicols}{2}
\vft{1}{$1051$}{0}{0}{0}{Ergebnisse \emph{LMF} mit Variante 3.4 für TEI6 1. Durchlauf}{lmf34_TEI6_1}
\vft{2}{141}{$p(1323)-142$}{1}{0}{Ergebnisse \emph{LMF} mit Variante 3.4 für TEI6 2. Durchlauf}{lmf34_TEI6_2}
\end{multicols}

\begin{multicols}{2}
\vft{1}{$161294$}{0}{0}{0}{Ergebnisse \emph{LMF} mit Variante 3.4 für TEI7 1. Durchlauf}{lmf34_TEI7_1}
\vft{2}{788327}{$p(52150)-788328$}{1}{0}{Ergebnisse \emph{LMF} mit Variante 3.4 für TEI7 2. Durchlauf}{lmf34_TEI7_2}
\end{multicols}

\section*{Ergebnisse für Variante 4.1}
\begin{multicols}{3}
\vft{1}{0}{$p(44)-1$}{1}{0}{Ergebnisse \emph{LMF} mit Variante 4.1 für TEI1 1.~\mbox{Durchlauf}}{lmf41_TEI1_1}
\vft{1}{516}{$p(55)-517$}{1}{0}{Ergebnisse \emph{LMF} mit Variante 4.1 für TEI2 1.~\mbox{Durchlauf}}{lmf41_TEI2_1}
\vft{1}{185}{$p(50)-186$}{1}{0}{Ergebnisse \emph{LMF} mit Variante 4.1 für TEI3 1.~\mbox{Durchlauf}}{lmf41_TEI3_1}
\end{multicols}
\begin{multicols}{2}
\vft{1}{$1174$}{0}{0}{0}{Ergebnisse \emph{LMF} mit Variante 4.1 für TEI4 1. Durchlauf}{lmf41_TEI4_1}
\vft{2}{2}{$p(2247)-3$}{1}{0}{Ergebnisse \emph{LMF} mit Variante 4.1 für TEI4 2. Durchlauf}{lmf41_TEI4_2}
\end{multicols}
\begin{multicols}{2}
\vft{1}{$4984$}{0}{0}{0}{Ergebnisse \emph{LMF} mit Variante 4.1 für TEI5 1. Durchlauf}{lmf41_TEI5_1}
\vft{2}{2}{$p(2775)-3$}{1}{0}{Ergebnisse \emph{LMF} mit Variante 4.1 für TEI5 2. Durchlauf}{lmf41_TEI5_2}
\end{multicols}

\begin{multicols}{2}
\vft{1}{$1051$}{0}{0}{0}{Ergebnisse \emph{LMF} mit Variante 4.1 für TEI6 1. Durchlauf}{lmf41_TEI6_1}
\vft{2}{0}{$p(1323)-1$}{1}{0}{Ergebnisse \emph{LMF} mit Variante 4.1 für TEI6 2. Durchlauf}{lmf41_TEI6_2}
\end{multicols}

\begin{multicols}{2}
\vft{1}{$161294$}{0}{0}{0}{Ergebnisse \emph{LMF} mit Variante 4.1 für TEI7 1. Durchlauf}{lmf41_TEI7_1}
\vft{2}{314549}{$p(52150)-314550$}{1}{0}{Ergebnisse \emph{LMF} mit Variante 4.1 für TEI7 2. Durchlauf}{lmf41_TEI7_2}
\end{multicols}

\section*{Ergebnisse für Variante 4.2}
\begin{multicols}{3}
\vft{1}{5}{$p(44)-6$}{1}{0}{Ergebnisse \emph{LMF} mit Variante 4.2 für TEI1 1.~\mbox{Durchlauf}}{lmf42_TEI1_1}
\vft{1}{4132}{$p(55)-4133$}{1}{0}{Ergebnisse \emph{LMF} mit Variante 4.2 für TEI2 1.~\mbox{Durchlauf}}{lmf42_TEI2_1}
\vft{1}{3847}{$p(50)-3848$}{1}{0}{Ergebnisse \emph{LMF} mit Variante 4.2 für TEI3 1.~\mbox{Durchlauf}}{lmf42_TEI3_1}
\end{multicols}
\begin{multicols}{2}
\vft{1}{$1174$}{0}{0}{0}{Ergebnisse \emph{LMF} mit Variante 4.2 für TEI4 1. Durchlauf}{lmf42_TEI4_1}
\vft{2}{0}{$p(2247)-1$}{1}{0}{Ergebnisse \emph{LMF} mit Variante 4.2 für TEI4 2. Durchlauf}{lmf42_TEI4_2}
\end{multicols}
\begin{multicols}{2}
\vft{1}{$4984$}{0}{0}{0}{Ergebnisse \emph{LMF} mit Variante 4.2 für TEI5 1. Durchlauf}{lmf42_TEI5_1}
\vft{2}{0}{$p(2775)-1$}{1}{0}{Ergebnisse \emph{LMF} mit Variante 4.2 für TEI5 2. Durchlauf}{lmf42_TEI5_2}
\end{multicols}

\begin{multicols}{2}
\vft{1}{$1051$}{0}{0}{0}{Ergebnisse \emph{LMF} mit Variante 4.2 für TEI6 1. Durchlauf}{lmf42_TEI6_1}
\vft{2}{0}{$p(1323)-1$}{1}{0}{Ergebnisse \emph{LMF} mit Variante 4.2 für TEI6 2. Durchlauf}{lmf42_TEI6_2}
\end{multicols}

\begin{multicols}{2}
\vft{1}{$161294$}{0}{0}{0}{Ergebnisse \emph{LMF} mit Variante 4.2 für TEI7 1. Durchlauf}{lmf42_TEI7_1}
\vft{2}{445110}{$p(52150)-445111$}{1}{0}{Ergebnisse \emph{LMF} mit Variante 4.2 für TEI7 2. Durchlauf}{lmf42_TEI7_2}
\end{multicols}

\section*{Ergebnisse für Variante 4.3}
\begin{multicols}{3}
\vft{1}{5}{$p(44)-6$}{1}{0}{Ergebnisse \emph{LMF} mit Variante 4.3 für TEI1 1.~\mbox{Durchlauf}}{lmf43_TEI1_1}
\vft{1}{6015}{$p(55)-6016$}{1}{0}{Ergebnisse \emph{LMF} mit Variante 4.3 für TEI2 1.~\mbox{Durchlauf}}{lmf43_TEI2_1}
\vft{1}{6353}{$p(50)-6354$}{1}{0}{Ergebnisse \emph{LMF} mit Variante 4.3 für TEI3 1.~\mbox{Durchlauf}}{lmf43_TEI3_1}
\end{multicols}
\begin{multicols}{2}
\vft{1}{$1174$}{0}{0}{0}{Ergebnisse \emph{LMF} mit Variante 4.3 für TEI4 1. Durchlauf}{lmf43_TEI4_1}
\vft{2}{37}{$p(2247)-38$}{1}{0}{Ergebnisse \emph{LMF} mit Variante 4.3 für TEI4 2. Durchlauf}{lmf43_TEI4_2}
\end{multicols}
\begin{multicols}{2}
\vft{1}{$4984$}{0}{0}{0}{Ergebnisse \emph{LMF} mit Variante 4.3 für TEI5 1. Durchlauf}{lmf43_TEI5_1}
\vft{2}{4006}{$p(2775)-4007$}{1}{0}{Ergebnisse \emph{LMF} mit Variante 4.3 für TEI5 2. Durchlauf}{lmf43_TEI5_2}
\end{multicols}

\begin{multicols}{2}
\vft{1}{$1051$}{0}{0}{0}{Ergebnisse \emph{LMF} mit Variante 4.3 für TEI6 1. Durchlauf}{lmf43_TEI6_1}
\vft{2}{2}{$p(1323)-3$}{1}{0}{Ergebnisse \emph{LMF} mit Variante 4.3 für TEI6 2. Durchlauf}{lmf43_TEI6_2}
\end{multicols}

\begin{multicols}{2}
\vft{1}{$161294$}{0}{0}{0}{Ergebnisse \emph{LMF} mit Variante 4.3 für TEI7 1. Durchlauf}{lmf43_TEI7_1}
\vft{2}{5433499}{$p(52150)-5433500$}{1}{0}{Ergebnisse \emph{LMF} mit Variante 4.3 für TEI7 2. Durchlauf}{lmf43_TEI7_2}
\end{multicols}

\section*{Ergebnisse für Variante 4.4}
\begin{multicols}{3}
\vft{1}{25}{$p(44)-26$}{1}{0}{Ergebnisse \emph{LMF} mit Variante 4.4 für TEI1 1.~\mbox{Durchlauf}}{lmf44_TEI1_1}
\vft{1}{1286}{$p(55)-1287$}{1}{0}{Ergebnisse \emph{LMF} mit Variante 4.4 für TEI2 1.~\mbox{Durchlauf}}{lmf44_TEI2_1}
\vft{1}{981}{$p(50)-982$}{1}{0}{Ergebnisse \emph{LMF} mit Variante 4.4 für TEI3 1.~\mbox{Durchlauf}}{lmf44_TEI3_1}
\end{multicols}
\begin{multicols}{2}
\vft{1}{$1174$}{0}{0}{0}{Ergebnisse \emph{LMF} mit Variante 4.4 für TEI4 1. Durchlauf}{lmf44_TEI4_1}
\vft{2}{1}{$p(2247)-2$}{1}{0}{Ergebnisse \emph{LMF} mit Variante 4.4 für TEI4 2. Durchlauf}{lmf44_TEI4_2}
\end{multicols}
\begin{multicols}{2}
\vft{1}{$4984$}{0}{0}{0}{Ergebnisse \emph{LMF} mit Variante 4.4 für TEI5 1. Durchlauf}{lmf44_TEI5_1}
\vft{2}{31}{$p(2775)-32$}{1}{0}{Ergebnisse \emph{LMF} mit Variante 4.4 für TEI5 2. Durchlauf}{lmf44_TEI5_2}
\end{multicols}

\begin{multicols}{2}
\vft{1}{$1051$}{0}{0}{0}{Ergebnisse \emph{LMF} mit Variante 4.4 für TEI6 1. Durchlauf}{lmf44_TEI6_1}
\vft{2}{0}{$p(1323)-1$}{1}{0}{Ergebnisse \emph{LMF} mit Variante 4.4 für TEI6 2. Durchlauf}{lmf44_TEI6_2}
\end{multicols}

\begin{multicols}{2}
\vft{1}{$161294$}{0}{0}{0}{Ergebnisse \emph{LMF} mit Variante 4.4 für TEI7 1. Durchlauf}{lmf44_TEI7_1}
\vft{2}{500063}{$p(52150)-500064$}{1}{0}{Ergebnisse \emph{LMF} mit Variante 4.4 für TEI7 2. Durchlauf}{lmf44_TEI7_2}
\end{multicols}


\addcontentsline{toc}{chapter}{Literaturverzeichnis}
\bibliography{thesisbib}{}



%\section{Beispiel-Implementierungen f�r die Matcher}\label{matcherExamples}
Um die Beispiel-Implementierungen der Matcher nachvollziehen zu k�nnen, ist es notwendig die Implementierung der darin verwendeten Klassen aufzuzeigen. Daher sind diese in \lstsrefs{LST_superclass_impl}{LST_SuperWrapperReturnSubWrapperParamClass_impl} aufgef�hrt. Dabei handelt es sich zum einen um die Implementierungen der Klassen, die in den Szenarien der Abschnitte \ref{exactTypeMatcher} - \ref{structTypeMatcher} beschrieben wurden. Zum anderen handelt es sich um Implementierung weiterer Klassen, die in Szenarien verwendet werden, welche in den folgenden Abschnitten aufgef�hrt werden. Um einen �berblick zu gew�hrleisten, zeigt \abbref{cd_alltypes} alle Typen auf, die in den Szenarien verwendet werden.

\myBigFigure{cd_alltypes}{Alle Typen/Klassen, die in Matcher-Szenarien verwendet werden}{cd_alltypes}

\begin{lstlisting}[{caption = Implemetierung: SuperClass
},{label = LST_superclass_impl}]
public class SuperClass {

  private String string;

  public SuperClass( String string ) {
    this.string = string;
  }

  public String getString() {
    return string;
  }
}
\end{lstlisting}


\begin{lstlisting}[{caption = Implemetierung: SubClass
},{label = LST_subclass_impl}]
public class SubClass extends SuperClass {

  public SubClass( String string ) {
    super( "Sub" + string );
  }

  public String getStringWithoutPrefix() {
    return getString().substring( 3 );
  }
}
\end{lstlisting}




\begin{lstlisting}[{caption = Implemetierung: SubWrapper
},{label = LST_subwrapper_impl}]
public class SubWrapper {

  private SubClass wrapped;

  public SubWrapper( String string ) {
    this.wrapped = new SubClass( string );
  }

  @Override
  public String toString() {
    return "WRAPPED_" + this.wrapped.getStringWithoutPrefix();
  }

  public String toStringWithPrefix() {
    return "WRAPPED_" + this.wrapped.getString();
  }
}
\end{lstlisting}





\begin{lstlisting}[{caption = Implemetierung: SuperWrapperReturnSubWrapperParamClass
},{label = LST_SuperWrapperReturnSubWrapperParamClass_impl}]
public class SuperWrapperReturnSubWrapperParamClass {

  public SuperWrapper addHello( SubWrapper a ) {
    return new SuperWrapper( a.toString() + "hello" );
  }

  public SuperWrapper add( SubWrapper a, SubWrapper b ) {
    return new SuperWrapper( a.toString() + b.toString() );
  }
}
\end{lstlisting}

\subsection{Beispiel f�r den ExactTypeMatcher}\label{exactMatcherExample}
In \lstref{LST_exactTypeMatcher_matching} ist die Implementierung eines JUnit-Tests aufgef�hrt, in dem das Matching �ber den ExactTypeMatcher f�r unterschiedliche Source- und Target-Typen nachgewiesen werden soll. Die Test-Methode match enth�lt dabei die Aufrufe, bei denen das Matching festgestellt werden kann. Dementsprechend enth�lt die Test-Methode noMatch die Aufrufe, bei denen das Matching fehlschl�gt.\\\\
\lstref{LST_exactTypeMatcher_conversion} enth�lt die Implementierung f�r einen JUnit-Test, in dem die Konvertierung, die durch den ExactTypeMatcher beschrieben wird, nachgewiesen wird. Hierbei wird von dem Szenario aus \ref{exactTypeMatcher} ausgegangen.
\begin{lstlisting}[{caption = ExactTypeMatcher Matching Test
},{label = LST_exactTypeMatcher_matching}]
public class ExactTypeMatcher_MatcherTest {

  @Test
  public void match() {
    ExactTypeMatcher matcher = new ExactTypeMatcher();
    assertTrue( matcher.matchesType( String.class, String.class ) );
    assertTrue( matcher.matchesType( int.class, int.class ) );
    assertTrue( matcher.matchesType( Object.class, Object.class ) );
    assertTrue( matcher.matchesType( SuperClass.class, SuperClass.class ) );
  }

  @Test
  public void noMatch() {
    ExactTypeMatcher matcher = new ExactTypeMatcher();
    assertFalse( matcher.matchesType( String.class, int.class ) );
    assertFalse( matcher.matchesType( int.class, Object.class ) );
    assertFalse( matcher.matchesType( Object.class, String.class ) );
    assertFalse( matcher.matchesType( SuperClass.class, SubClass.class ) );
  }
}
\end{lstlisting}

\begin{lstlisting}[{caption = ExactTypeMatcher Konvertierung Test
},{label = LST_exactTypeMatcher_conversion}]
public class ExactTypeMatcher_ConversionTest {

  @Test
  public void convertString() {
    SuperClass target = new SuperClass( "A" );
    Collection<ModuleMatchingInfo> matchingInfos = new ExactTypeMatcher().calculateTypeMatchingInfos( SuperClass.class,
        SuperClass.class );
    
    ModuleMatchingInfo moduleMatchingInfo = matchingInfos.iterator().next();

    ProxyFactory<SuperClass> proxyFactory = moduleMatchingInfo.getConverterCreator()
        .createProxyFactory( SuperClass.class );
    Collection<MethodMatchingInfo> methodMatchingInfos = moduleMatchingInfo.getMethodMatchingInfos();

    SuperClass source = proxyFactory.createProxy( target, methodMatchingInfos );

    assertTrue( source.getString().equals( "A" ) );
  }
}
\end{lstlisting}

\subsection{Beispiel f�r den GenTypeMatcher}\label{genMatcherExample}
Der GenTypeMatcher und der SpecTypeMatcher wurden gemeinsam implementiert. Daher wird in den folgenden Beispielen jeweils ein Matcher aus der Klasse GenSpecTypeMatcher erzeugt. Die weitere Verwendung den Matchers bezieht sich in diesen Beispielen aber auf die Definition des GenTypeMatchers aus \ref{genTypeMatcher}.\\\\
In \lstref{LST_genTypeMatcher_matching} ist die Implementierung eines JUnit-Tests aufgef�hrt, in dem das Matching �ber den GenTypeMatcher f�r unterschiedliche Source- und Target-Typen nachgewiesen werden soll. Die Test-Methode match enth�lt dabei die Aufrufe, bei denen das Matching festgestellt werden kann. Dementsprechend enth�lt die Test-Methode noMatch die Aufrufe, bei denen das Matching fehlschl�gt.\\\\
\lstref{LST_genTypeMatcher_conversion} enth�lt die Implementierung f�r einen JUnit-Test, in dem die Konvertierung, die durch den GenTypeMatcher beschrieben wird, nachgewiesen wird. Hierbei wird von dem Szenario aus \ref{genTypeMatcher} ausgegangen.
\begin{lstlisting}[{caption = GenTypeMatcher Matching Test
},{label = LST_genTypeMatcher_matching}]
public class GenSpecTypeMatcher_Gen_MatcherTest {

  @Test
  public void match() {
    GenSpecTypeMatcher matcher = new GenSpecTypeMatcher();
    assertTrue( matcher.matchesType( Object.class, String.class ) );
    assertTrue( matcher.matchesType( SuperClass.class, SubClass.class ) );
    assertTrue( matcher.matchesType( Number.class, Integer.class ) );
  }

  @Test
  public void noMatch() {
    GenSpecTypeMatcher matcher = new GenSpecTypeMatcher();
    assertFalse( matcher.matchesType( int.class, String.class ) );
  }
}
\end{lstlisting}
\begin{lstlisting}[{caption = GenTypeMatcher Konvertierung Test
},{label = LST_genTypeMatcher_conversion}]
public class GenSpecTypeMatcher_Gen_ConversionTest {

  @Test
  public void convertSpec2Gen() {
   SubClass target = new SubClass( "A" );
    Collection<ModuleMatchingInfo> matchingInfos = new GenSpecTypeMatcher().calculateTypeMatchingInfos(
        SuperClass.class, SubClass.class );
    
    ModuleMatchingInfo moduleMatchingInfo = matchingInfos.iterator().next();

    ProxyFactory<SuperClass> proxyFactory = moduleMatchingInfo.getConverterCreator()
        .createProxyFactory( SuperClass.class );
    Collection<MethodMatchingInfo> methodMatchingInfos = moduleMatchingInfo.getMethodMatchingInfos();

    SuperClass source = proxyFactory.createProxy( target, methodMatchingInfos );

    assertTrue( source.getString().equals( "SubA" ) );
  }
}
\end{lstlisting}


\subsection{Beispiel f�r den SpecTypeMatcher}\label{specMatcherExample}
Wie in \ref{genTypeMatcher} bereits erw�hnt wurde der GenTypeMatcher gemeinsam mit dem SpecTypeMatcher implementiert. Daher wird in den folgenden Beispielen jeweils ein Matcher aus der Klasse GenSpecTypeMatcher erzeugt. Die weitere Verwendung den Matchers bezieht sich in diesen Beispielen aber auf die Definition des SpecTypeMatcher aus \ref{specTypeMatcher}.\\\\
In \lstref{LST_specTypeMatcher_matching} ist die Implementierung eines JUnit-Tests aufgef�hrt, in dem das Matching �ber den GenTypeMatcher f�r unterschiedliche Source- und Target-Typen nachgewiesen werden soll. Die Test-Methode match enth�lt dabei die Aufrufe, bei denen das Matching festgestellt werden kann. Dementsprechend enth�lt die Test-Methode noMatch die Aufrufe, bei denen das Matching fehlschl�gt.\\\\
\lstref{LST_specTypeMatcher_conversion} enth�lt die Implementierung f�r einen JUnit-Test, in dem die Konvertierung, die durch den SpecTypeMatcher beschrieben wird, nachgewiesen wird. Hierbei wird von dem Szenario aus \ref{specTypeMatcher} ausgegangen. Die Test-Methode convertGen2Spec\_positivCall enth�lt den Aufruf der ersten Methoden aus dem Szenario (getString). Die Test-Methoden convertGen2Spec\_negativeCall beinhaltet den fehlschlagenden Aufruf der Methoden getStringWithoutPrefix.
\begin{lstlisting}[{caption = SpecTypeMatcher Matching Test
},{label = LST_specTypeMatcher_matching}]
public class GenSpecTypeMatcher_Spec_MatcherTest {

  @Test
  public void match() {
    GenSpecTypeMatcher matcher = new GenSpecTypeMatcher();
    assertTrue( matcher.matchesType( String.class, Object.class ) );
    assertTrue( matcher.matchesType( SubClass.class, SuperClass.class ) );
    assertTrue( matcher.matchesType( Integer.class, Number.class ) );
  }

  @Test
  public void noMatch() {
    GenSpecTypeMatcher matcher = new GenSpecTypeMatcher();
    assertFalse( matcher.matchesType( int.class, String.class ) );
  }
}
\end{lstlisting}
\begin{lstlisting}[{caption = SpecTypeMatcher Konvertierung Test
},{label = LST_specTypeMatcher_conversion}]
public class GenSpecTypeMatcher_Spec_ConversionTest {

  @Test
  public void convertGen2Spec_positivCall() {
    SuperClass offeredComponent = new SuperClass( "A" );
    Collection<ModuleMatchingInfo> matchingInfos = new GenSpecTypeMatcher().calculateTypeMatchingInfos(
        SubClass.class, SuperClass.class );
    
    ModuleMatchingInfo moduleMatchingInfo = matchingInfos.iterator().next();

    ProxyFactory<SubClass> proxyFactory = moduleMatchingInfo.getConverterCreator().createProxyFactory( SubClass.class );
    Collection<MethodMatchingInfo> methodMatchingInfos = moduleMatchingInfo.getMethodMatchingInfos();

    SubClass proxy = proxyFactory.createProxy( offeredComponent, methodMatchingInfos );

    assertTrue( proxy.getString().equals( "A" ) );
  }

  @Test( expected = SigMaGlueException.class )
  public void convertGen2Spec_negativeCall() {
    SuperClass offeredComponent = new SuperClass( "A" );
    Collection<ModuleMatchingInfo> matchingInfos = new GenSpecTypeMatcher().calculateTypeMatchingInfos(
        SubClass.class, SuperClass.class );
    
    ModuleMatchingInfo moduleMatchingInfo = matchingInfos.iterator().next();

    ProxyFactory<SubClass> proxyFactory = moduleMatchingInfo.getConverterCreator().createProxyFactory( SubClass.class );
    Collection<MethodMatchingInfo> methodMatchingInfos = moduleMatchingInfo.getMethodMatchingInfos();

    SubClass proxy = proxyFactory.createProxy( offeredComponent, methodMatchingInfos );

    assertTrue( proxy.getString().equals( "A" ) );

    proxy.getStringWithoutPrefix();
  }
}
\end{lstlisting}



\subsection{Beispiel f�r den WrappedTypeMatcher}\label{wrappedMatcherExample}
Der WrappedTypeMatcher und der WrapperTypeMatcher wurden gemeinsam implementiert. Daher wird in den folgenden Beispielen jeweils ein Matcher aus der Klasse WrappedTypeMatcher erzeugt. Die weitere Verwendung den Matchers bezieht sich in diesen Beispielen aber auf die Definition des WrappedTypeMatcher aus \ref{wrappedTypeMatcher}.\\\\
In \lstref{LST_wrappedTypeMatcher_matching} ist die Implementierung eines JUnit-Tests aufgef�hrt, in dem das Matching �ber den WrappedTypeMatcher f�r unterschiedliche Source- und Target-Typen nachgewiesen werden soll. Die Test-Methoden mit dem Pr�fix match enthalten dabei die Aufrufe, bei denen das Matching festgestellt werden kann. Dementsprechend enth�lt die Test-Methode noMatch die Aufrufe, bei denen das Matching fehlschl�gt.\\\\
\lstref{LST_wrappedTypeMatcher_conversion} enth�lt die Implementierung f�r einen JUnit-Test, in dem die Konvertierung, die durch den WrappedTypeMatcher beschrieben wird, nachgewiesen wird. In der Test-Methode convertSubWrapper2SubClass wird von dem Szenario aus \ref{wrappedTypeMatcher} ausgegangen. Die anderen Test-Methoden stellen weitere Szenarien dar, die in den folgenden Unterabschnitten beschrieben werden.
\begin{lstlisting}[{caption = WrappedTypeMatcher Matching Test
},{label = LST_wrappedTypeMatcher_matching}]
public class WrappedTypeMatcher_Wrapped_MatcherTest {

  private WrappedTypeMatcher matcher = new WrappedTypeMatcher(
      MatcherCombiner.combine( new ExactTypeMatcher(), new GenSpecTypeMatcher() ) );

  @Test
  public void match() {
    assertTrue( matcher.matchesType( boolean.class, Boolean.class ) );
    assertTrue( matcher.matchesType( int.class, Integer.class ) );
  }

  @Test
  public void match_wrapped_exact() {
    assertTrue( matcher.matchesType( SubClass.class, SubWrapper.class ) );
  }

  @Test
  public void match_wrapped_spec() {
    assertTrue( matcher.matchesType( SubClass.class, SuperWrapper.class ) );
  }

  @Test
  public void match_wrapped_gen() {
    assertTrue( matcher.matchesType( SuperClass.class, SubWrapper.class ) );
  }

  @Test
  public void noMatch() {
    assertFalse( matcher.matchesType( String.class, String.class ) );
  }
}
\end{lstlisting}
\begin{lstlisting}[{caption = WrappedTypeMatcher Konvertierung Test
},{label = LST_wrappedTypeMatcher_conversion}]
public class WrappedTypeMatcher_Wrapped_ConversionTest {

  private WrappedTypeMatcher matcher = new WrappedTypeMatcher(
      MatcherCombiner.combine( new ExactTypeMatcher(), new GenSpecTypeMatcher() ) );

  @Test
  public void convertSubWrapper2SubClass() {
    SubWrapper offeredComponent = new SubWrapper( "A" );
    Collection<ModuleMatchingInfo> matchingInfos = matcher.calculateTypeMatchingInfos(
        SubClass.class, SubWrapper.class );
  
    ModuleMatchingInfo moduleMatchingInfo = matchingInfos.iterator().next();

    ProxyFactory<SubClass> proxyFactory = moduleMatchingInfo.getConverterCreator()
        .createProxyFactory( SubClass.class );
    Collection<MethodMatchingInfo> methodMatchingInfos = moduleMatchingInfo.getMethodMatchingInfos();

    SubClass proxy = proxyFactory.createProxy( offeredComponent, methodMatchingInfos );

    assertTrue( proxy.getString().equals( "SubA" ) );
    assertTrue( proxy.getStringWithoutPrefix().equals( "A" ) );
  }

  @Test
  public void convertSuperWrapper2SubClass_positiveCall() {
    SuperWrapper offeredComponent = new SuperWrapper( "A" );
    Collection<ModuleMatchingInfo> matchingInfos = matcher.calculateTypeMatchingInfos(
        SubClass.class, SuperWrapper.class );
  
    ModuleMatchingInfo moduleMatchingInfo = matchingInfos.iterator().next();

    ProxyFactory<SubClass> proxyFactory = moduleMatchingInfo.getConverterCreator()
        .createProxyFactory( SubClass.class );
    Collection<MethodMatchingInfo> methodMatchingInfos = moduleMatchingInfo.getMethodMatchingInfos();

    SubClass proxy = proxyFactory.createProxy( offeredComponent, methodMatchingInfos );

    assertTrue( proxy.getString().equals( "A" ) );
  }

  @Test( expected = SigMaGlueException.class )
  public void convertSuperWrapper2SubClass_negativeCall() {
    SuperWrapper offeredComponent = new SuperWrapper( "A" );
    Collection<ModuleMatchingInfo> matchingInfos = matcher.calculateTypeMatchingInfos(
        SubClass.class, SuperWrapper.class );
  
    ModuleMatchingInfo moduleMatchingInfo = matchingInfos.iterator().next();

    ProxyFactory<SubClass> proxyFactory = moduleMatchingInfo.getConverterCreator()
        .createProxyFactory( SubClass.class );
    Collection<MethodMatchingInfo> methodMatchingInfos = moduleMatchingInfo.getMethodMatchingInfos();

    SubClass proxy = proxyFactory.createProxy( offeredComponent, methodMatchingInfos );
    proxy.getStringWithoutPrefix();
  }

  @Test
  public void convertSubWrapper2SuperClass() {
    SubWrapper offeredComponent = new SubWrapper( "A" );
    Collection<ModuleMatchingInfo> matchingInfos = matcher.calculateTypeMatchingInfos(
        SuperClass.class, SubWrapper.class );
   
    ModuleMatchingInfo moduleMatchingInfo = matchingInfos.iterator().next();

    ProxyFactory<SuperClass> proxyFactory = moduleMatchingInfo.getConverterCreator()
        .createProxyFactory( SuperClass.class );
    Collection<MethodMatchingInfo> methodMatchingInfos = moduleMatchingInfo.getMethodMatchingInfos();

    SuperClass proxy = proxyFactory.createProxy( offeredComponent, methodMatchingInfos );

    assertTrue( proxy.getString().equals( "SubA" ) );
  }
}
\end{lstlisting}





\subsection{Beispiel f�r den WrapperTypeMatcher}\label{wrapperMatcherExample}
Wie bereits im vorherigen Abschnitt erw�hnt wurden der WrappedTypeMatcher und der WrapperTypeMatcher gemeinsam implementiert. Daher wird in den folgenden Beispielen jeweils ein Matcher aus der Klasse WrappedTypeMatcher erzeugt. Die weitere Verwendung den Matchers bezieht sich in diesen Beispielen aber auf die Definition des WrapperTypeMatcher aus \ref{wrapperTypeMatcher}.\\\\
In \lstref{LST_wrapperTypeMatcher_matching} ist die Implementierung eines JUnit-Tests aufgef�hrt, in dem das Matching �ber den WrapperTypeMatcher f�r unterschiedliche Source- und Target-Typen nachgewiesen werden soll. Die Test-Methoden mit dem Pr�fix match enthalten dabei die Aufrufe, bei denen das Matching festgestellt werden kann. Dementsprechend enth�lt die Test-Methode noMatch die Aufrufe, bei denen das Matching fehlschl�gt.\\\\
\lstref{LST_wrapperTypeMatcher_conversion} enth�lt die Implementierung f�r einen JUnit-Test, in dem die Konvertierung, die durch den WrapperTypeMatcher beschrieben wird, nachgewiesen wird. In der Test-Methode convertSubClass2SubWrapper wird von dem Szenario aus \ref{wrapperTypeMatcher} ausgegangen. Die anderen Test-Methoden stellen weitere Szenarien dar, die in den folgenden Unterabschnitten beschrieben werden.
\begin{lstlisting}[{caption = WrapperTypeMatcher Matching Test
},{label = LST_wrapperTypeMatcher_matching}]
public class WrappedTypeMatcher_Wrapper_MatcherTest {

  private WrappedTypeMatcher matcher = new WrappedTypeMatcher(
      MatcherCombiner.combine( new ExactTypeMatcher(), new GenSpecTypeMatcher() ) );

  @Test
  public void match() {
    assertTrue( matcher.matchesType( Boolean.class, boolean.class ) );
    assertTrue( matcher.matchesType( Integer.class, int.class ) );
  }

  @Test
  public void match_wrapped_exact() {
    assertTrue( matcher.matchesType( SubWrapper.class, SubClass.class ) );
  }

  @Test
  public void match_wrapped_spec() {
    assertTrue( matcher.matchesType( SuperWrapper.class, SubClass.class ) );
  }

  @Test
  public void match_wrapped_gen() {
    assertTrue( matcher.matchesType( SubWrapper.class, SuperClass.class ) );
  }

  @Test
  public void noMatch() {
    assertFalse( matcher.matchesType( String.class, String.class ) );
  }
}
\end{lstlisting}
\begin{lstlisting}[{caption = WrapperTypeMatcher Konvertierung Test
},{label = LST_wrapperTypeMatcher_conversion}]
public class WrappedTypeMatcher_Wrapper_ConversionTest {

  private WrappedTypeMatcher matcher = new WrappedTypeMatcher(
      MatcherCombiner.combine( new ExactTypeMatcher(), new GenSpecTypeMatcher() ) );

  @Test
  public void convertSubClass2SubWrapper() {
    SubClass offeredComponent = new SubClass( "A" );
    Collection<ModuleMatchingInfo> matchingInfos = matcher.calculateTypeMatchingInfos(
        SubWrapper.class, SubClass.class );

    ModuleMatchingInfo moduleMatchingInfo = matchingInfos.iterator().next();

    ProxyFactory<SubWrapper> proxyFactory = moduleMatchingInfo.getConverterCreator()
        .createProxyFactory( SubWrapper.class );
    Collection<MethodMatchingInfo> methodMatchingInfos = moduleMatchingInfo.getMethodMatchingInfos();

    SubWrapper proxy = proxyFactory.createProxy( offeredComponent, methodMatchingInfos );

    assertTrue( proxy.toString().equals( "WRAPPED_A" ) );
    assertTrue( proxy.toStringWithPrefix().equals( "WRAPPED_SubA" ) );
  }

  @Test
  public void convertSuperWrapper2SubClass() {
    SubClass offeredComponent = new SubClass( "A" );
    Collection<ModuleMatchingInfo> matchingInfos = matcher.calculateTypeMatchingInfos(
        SuperWrapper.class, SubClass.class );

    ModuleMatchingInfo moduleMatchingInfo = matchingInfos.iterator().next();

    ProxyFactory<SuperWrapper> proxyFactory = moduleMatchingInfo.getConverterCreator()
        .createProxyFactory( SuperWrapper.class );
    Collection<MethodMatchingInfo> methodMatchingInfos = moduleMatchingInfo.getMethodMatchingInfos();

    SuperWrapper proxy = proxyFactory.createProxy( offeredComponent, methodMatchingInfos );

    assertTrue( proxy.toString().equals( "WRAPPED_SubA" ) );
    assertFalse( proxy.hashCode() == offeredComponent.hashCode() );
  }

  @Test
  public void convertSubWrapper2SuperClass_positiveCall() {
    SuperClass offeredComponent = new SuperClass( "A" );
    Collection<ModuleMatchingInfo> matchingInfos = matcher.calculateTypeMatchingInfos(
        SubWrapper.class, SuperClass.class );

    ModuleMatchingInfo moduleMatchingInfo = matchingInfos.iterator().next();

    ProxyFactory<SubWrapper> proxyFactory = moduleMatchingInfo.getConverterCreator()
        .createProxyFactory( SubWrapper.class );
    Collection<MethodMatchingInfo> methodMatchingInfos = moduleMatchingInfo.getMethodMatchingInfos();

    SubWrapper proxy = proxyFactory.createProxy( offeredComponent, methodMatchingInfos );

    assertTrue( proxy.toStringWithPrefix().equals( "WRAPPED_A" ) );
    assertFalse( proxy.hashCode() == offeredComponent.hashCode() );
  }

  @Test( expected = SigMaGlueException.class )
  public void convertSubWrapper2SuperClass_negativeCall() {
    SuperClass offeredComponent = new SuperClass( "A" );
    Collection<ModuleMatchingInfo> matchingInfos = matcher.calculateTypeMatchingInfos(
        SubWrapper.class, SuperClass.class );
    // Der WrappedTypeMatcher erzeugt nur eine ModuleMatchingInfo (kein rekursives Matching)
    ModuleMatchingInfo moduleMatchingInfo = matchingInfos.iterator().next();

    ProxyFactory<SubWrapper> proxyFactory = moduleMatchingInfo.getConverterCreator()
        .createProxyFactory( SubWrapper.class );
    Collection<MethodMatchingInfo> methodMatchingInfos = moduleMatchingInfo.getMethodMatchingInfos();

    SubWrapper proxy = proxyFactory.createProxy( offeredComponent, methodMatchingInfos );

    proxy.toString().equals( "WRAPPED_A" );
  }
}

\end{lstlisting}



\subsection{Beispiel f�r den StructuralTypeMatcher}\label{structMatcherExample}
In \lstref{LST_structTypeMatcher_matching} ist die Implementierung eines JUnit-Tests aufgef�hrt, in dem das Matching �ber den StructuralTypeMatcher f�r unterschiedliche Source- und Target-Typen nachgewiesen werden soll. Alle Test-Methoden enthalten Aufrufe, bei denen das Matching festgestellt werden kann. Die Test-Methode match\_genReturn\_specParam bezieht sich auf das Szenario aus \ref{wrapperTypeMatcher}.   Da es f�r jedes Paar von Source- und Target-Typ mehrere M�glichkeiten zur Feststellung der strukturellen Gleichheit gibt, wird die Evaluation der Testf�lle in einer Schleife �ber diese M�glichkeiten durchgef�hrt.\\\\
\lstref{LST_structTypeMatcher_conversion} enth�lt die Implementierung f�r einen JUnit-Test, in dem die Konvertierung, die durch den StructTypeMatcher beschrieben wird, nachgewiesen wird. Die Test-Methode convert\_genReturn\_specParam bezieht sich dabei auf das Szenario aus \ref{structTypeMatcher}. Die anderen Test-Methoden stellen weitere Szenarien dar, die in den folgenden Unterabschnitten beschrieben werden.\\\\
In beiden F�llen wurde von einem StructuralTypeMatcher ausgegangen, der als internen Type-Matcher eine Kombination aus den zuvor genannten Matchern verwendet.
\begin{lstlisting}[{caption = StructTypeMatcher Matching Test
},{label = LST_structTypeMatcher_matching}]
public class StructuralTypeMatcher_MatcherTest {

  private StructuralTypeMatcher matcher = new StructuralTypeMatcher(
      MatcherCombiner.combine( new ExactTypeMatcher(), new GenSpecTypeMatcher(),
          new WrappedTypeMatcher( MatcherCombiner.combine( new ExactTypeMatcher(), new GenSpecTypeMatcher() ) ) ) );

  @Test
  public void match_exactReturn_exactParam() {
    assertTrue( matcher.matchesType( SubReturnSubParamClass1.class, SubReturnSubParamClass2.class ) );
  }

  @Test
  public void match_exactReturn_genParam() {
    assertTrue( matcher.matchesType( SubReturnSuperParamClass.class, SubReturnSubParamClass1.class ) );
  }

  @Test
  public void match_exactReturn_specParam() {
    assertTrue( matcher.matchesType( SubReturnSubParamClass1.class, SubReturnSuperParamClass.class ) );
  }

  @Test
  public void match_genReturn_specParam() {
    assertTrue( matcher.matchesType( SuperReturnSubParamClass.class, SubReturnSuperParamClass.class ) );
  }

  @Test
  public void match_specReturn_genParam() {
    assertTrue( matcher.matchesType( SubReturnSuperParamClass.class, SuperReturnSubParamClass.class ) );
  }

  @Test
  public void match_specReturn_wrapperGenParam() {
    assertTrue( matcher.matchesType( SubReturnSuperWrapperParamClass.class, SuperReturnSubParamClass.class ) );
  }

  @Test
  public void match_wrapperGenReturn_specParam() {
    assertTrue( matcher.matchesType( SuperWrapperReturnSubParamClass.class, SubReturnSuperParamClass.class ) );
  }

  @Test
  public void match_wrapperGenReturn_wrapperSpecParam() {
    assertTrue( matcher.matchesType( SuperWrapperReturnSubWrapperParamClass.class, SubReturnSuperParamClass.class ) );
  }

  @Test
  public void match_wrapperSpecReturn_wrapperExactParam() {
    assertTrue( matcher.matchesType( SubWrapperReturnSubParamClass.class, SuperReturnSubParamClass.class ) );
  }
}
\end{lstlisting}

\begin{lstlisting}[{caption = StructuralTypeMatcher Konvertierung Test
},{label = LST_structTypeMatcher_conversion}]
public class StructuralTypeMatcher_ConversionTest {

  private StructuralTypeMatcher matcher = new StructuralTypeMatcher(
      MatcherCombiner.combine( new ExactTypeMatcher(), new GenSpecTypeMatcher(),
          new WrappedTypeMatcher( MatcherCombiner.combine( new ExactTypeMatcher(), new GenSpecTypeMatcher() ) ) ) );

  @Test
  public void convert_exactReturn_exactParam() {
    SubReturnSubParamClass2 offeredComponent = new SubReturnSubParamClass2();
    Collection<ModuleMatchingInfo> matchingInfos = matcher.calculateTypeMatchingInfos( SubReturnSubParamClass1.class,
        SubReturnSubParamClass2.class );

    for ( ModuleMatchingInfo moduleMatchingInfo : matchingInfos ) {
      ProxyFactory<SubReturnSubParamClass1> proxyFactory = moduleMatchingInfo.getConverterCreator()
          .createProxyFactory( SubReturnSubParamClass1.class );
      Collection<MethodMatchingInfo> methodMatchingInfos = moduleMatchingInfo.getMethodMatchingInfos();

      SubReturnSubParamClass1 proxy = proxyFactory.createProxy( offeredComponent, methodMatchingInfos );

      SubClass param1 = new SubClass( "A" );
      SubClass param2 = new SubClass( "B" );
      assertTrue( proxy.addHello( param1 ).getString().equals( "SubSubAhello" ) );
      assertTrue( proxy.addHello( param1 ).getStringWithoutPrefix().equals( "SubAhello" ) );
      assertTrue( proxy.add( param1, param2 ).getString().equals( "SubSubASubB" ) );
      assertTrue( proxy.add( param1, param2 ).getStringWithoutPrefix().equals( "SubASubB" ) );
    }
  }

  @Test
  public void convert_exactReturn_genParam() {
    SubReturnSubParamClass1 offeredComponent = new SubReturnSubParamClass1();
    Collection<ModuleMatchingInfo> matchingInfos = matcher.calculateTypeMatchingInfos( SubReturnSuperParamClass.class,
        SubReturnSubParamClass1.class );
    for ( ModuleMatchingInfo moduleMatchingInfo : matchingInfos ) {
      ProxyFactory<SubReturnSuperParamClass> proxyFactory = moduleMatchingInfo.getConverterCreator()
          .createProxyFactory( SubReturnSuperParamClass.class );
      Collection<MethodMatchingInfo> methodMatchingInfos = moduleMatchingInfo.getMethodMatchingInfos();

      SubReturnSuperParamClass proxy = proxyFactory.createProxy( offeredComponent, methodMatchingInfos );

      SuperClass param1 = new SuperClass( "A" );
      SuperClass param2 = new SuperClass( "B" );
      assertTrue( proxy.addHello( param1 ).getString().equals( "SubAhello" ) );
      assertTrue( proxy.addHello( param1 ).getStringWithoutPrefix().equals( "Ahello" ) );
      assertTrue( proxy.add( param1, param2 ).getString().equals( "SubAB" ) );
      assertTrue( proxy.add( param1, param2 ).getStringWithoutPrefix().equals( "AB" ) );
    }
  }

  @Test
  public void convert_exactReturn_specParam() {
    SubReturnSuperParamClass offeredComponent = new SubReturnSuperParamClass();
    Collection<ModuleMatchingInfo> matchingInfos = matcher.calculateTypeMatchingInfos( SubReturnSubParamClass1.class,
        SubReturnSuperParamClass.class );
    for ( ModuleMatchingInfo moduleMatchingInfo : matchingInfos ) {
      ProxyFactory<SubReturnSubParamClass1> proxyFactory = moduleMatchingInfo.getConverterCreator()
          .createProxyFactory( SubReturnSubParamClass1.class );
      Collection<MethodMatchingInfo> methodMatchingInfos = moduleMatchingInfo.getMethodMatchingInfos();

      SubReturnSubParamClass1 proxy = proxyFactory.createProxy( offeredComponent, methodMatchingInfos );

      SubClass param1 = new SubClass( "A" );
      SubClass param2 = new SubClass( "B" );
      assertTrue( proxy.addHello( param1 ).getString().equals( "SubSubAhello" ) );
      assertTrue( proxy.addHello( param1 ).getStringWithoutPrefix().equals( "SubAhello" ) );
      assertTrue( proxy.add( param1, param2 ).getString().equals( "SubSubASubB" ) );
      assertTrue( proxy.add( param1, param2 ).getStringWithoutPrefix().equals( "SubASubB" ) );
    }
  }

  @Test
  public void convert_genReturn_specParam() {
    SubReturnSuperParamClass offeredComponent = new SubReturnSuperParamClass();
    Collection<ModuleMatchingInfo> matchingInfos = matcher.calculateTypeMatchingInfos( SuperReturnSubParamClass.class,
        SubReturnSuperParamClass.class );
    for ( ModuleMatchingInfo moduleMatchingInfo : matchingInfos ) {
      ProxyFactory<SuperReturnSubParamClass> proxyFactory = moduleMatchingInfo.getConverterCreator()
          .createProxyFactory( SuperReturnSubParamClass.class );
      Collection<MethodMatchingInfo> methodMatchingInfos = moduleMatchingInfo.getMethodMatchingInfos();

      SuperReturnSubParamClass proxy = proxyFactory.createProxy( offeredComponent, methodMatchingInfos );

      SubClass param1 = new SubClass( "A" );
      SubClass param2 = new SubClass( "B" );
      assertTrue( proxy.helloAdd( param1 ).getString().equals( "helloSubA" ) );
      assertTrue( proxy.addParams( param1, param2 ).getString().equals( "SubASubB" ) );
    }
  }

  @Test
  public void convert_specReturn_genParam() {
    SuperReturnSubParamClass offeredComponent = new SuperReturnSubParamClass();
    Collection<ModuleMatchingInfo> matchingInfos = matcher.calculateTypeMatchingInfos( SubReturnSuperParamClass.class,
        SuperReturnSubParamClass.class );
    for ( ModuleMatchingInfo moduleMatchingInfo : matchingInfos ) {
      ProxyFactory<SubReturnSuperParamClass> proxyFactory = moduleMatchingInfo.getConverterCreator()
          .createProxyFactory( SubReturnSuperParamClass.class );
      Collection<MethodMatchingInfo> methodMatchingInfos = moduleMatchingInfo.getMethodMatchingInfos();

      SubReturnSuperParamClass proxy = proxyFactory.createProxy( offeredComponent, methodMatchingInfos );

      SuperClass param1 = new SuperClass( "A" );
      SuperClass param2 = new SuperClass( "B" );
      assertTrue( proxy.addHello( param1 ).getString().equals( "SubAhello" ) );
      assertTrue( proxy.addHello( param1 ).getStringWithoutPrefix().equals( "Ahello" ) );
      assertTrue( proxy.add( param1, param2 ).getString().equals( "SubAB" ) );
      assertTrue( proxy.add( param1, param2 ).getStringWithoutPrefix().equals( "AB" ) );
    }

  }

  @Test
  public void convert_specReturn_wrapperGenParam() {
    SuperReturnSubParamClass offeredComponent = new SuperReturnSubParamClass();
    Collection<ModuleMatchingInfo> matchingInfos = matcher.calculateTypeMatchingInfos(
        SubReturnSuperWrapperParamClass.class,
        SuperReturnSubParamClass.class );
    for ( ModuleMatchingInfo moduleMatchingInfo : matchingInfos ) {
      ProxyFactory<SubReturnSuperWrapperParamClass> proxyFactory = moduleMatchingInfo.getConverterCreator()
          .createProxyFactory( SubReturnSuperWrapperParamClass.class );
      Collection<MethodMatchingInfo> methodMatchingInfos = moduleMatchingInfo.getMethodMatchingInfos();

      SubReturnSuperWrapperParamClass proxy = proxyFactory.createProxy( offeredComponent, methodMatchingInfos );

      SuperWrapper param1 = new SuperWrapper( "A" );
      SuperWrapper param2 = new SuperWrapper( "B" );
      assertTrue( proxy.addHello( param1 ).getString().equals( "SubWRAPPED_Ahello" ) );
      assertTrue( proxy.addHello( param1 ).getStringWithoutPrefix().equals( "WRAPPED_Ahello" ) );
      assertTrue( proxy.add( param1, param2 ).getString().equals( "SubWRAPPED_AWRAPPED_B" ) );
      assertTrue( proxy.add( param1, param2 ).getStringWithoutPrefix().equals( "WRAPPED_AWRAPPED_B" ) );
    }
  }

  @Test
  public void convert_wrapperGenReturn_specParam() {
    SubReturnSuperParamClass offeredComponent = new SubReturnSuperParamClass();
    Collection<ModuleMatchingInfo> matchingInfos = matcher.calculateTypeMatchingInfos(
        SuperWrapperReturnSubParamClass.class,
        SubReturnSuperParamClass.class );
    for ( ModuleMatchingInfo moduleMatchingInfo : matchingInfos ) {
      ProxyFactory<SuperWrapperReturnSubParamClass> proxyFactory = moduleMatchingInfo.getConverterCreator()
          .createProxyFactory( SuperWrapperReturnSubParamClass.class );
      Collection<MethodMatchingInfo> methodMatchingInfos = moduleMatchingInfo.getMethodMatchingInfos();

      SuperWrapperReturnSubParamClass proxy = proxyFactory.createProxy( offeredComponent, methodMatchingInfos );

      SubClass param1 = new SubClass( "A" );
      SubClass param2 = new SubClass( "B" );
      assertTrue( proxy.addHello( param1 ).toString().equals( "WRAPPED_SubAhello" ) );
      assertTrue( proxy.add( param1, param2 ).toString().equals( "WRAPPED_SubASubB" ) );
    }
  }

  @Test
  public void convert_wrapperGenReturn_wrapperSpecParam() {
    SubReturnSuperParamClass offeredComponent = new SubReturnSuperParamClass();
    Collection<ModuleMatchingInfo> matchingInfos = matcher.calculateTypeMatchingInfos(
        SuperWrapperReturnSubWrapperParamClass.class,
        SubReturnSuperParamClass.class );
    for ( ModuleMatchingInfo moduleMatchingInfo : matchingInfos ) {
      ProxyFactory<SuperWrapperReturnSubWrapperParamClass> proxyFactory = moduleMatchingInfo.getConverterCreator()
          .createProxyFactory( SuperWrapperReturnSubWrapperParamClass.class );
      Collection<MethodMatchingInfo> methodMatchingInfos = moduleMatchingInfo.getMethodMatchingInfos();

      SuperWrapperReturnSubWrapperParamClass proxy = proxyFactory.createProxy( offeredComponent, methodMatchingInfos );

      SubWrapper param1 = new SubWrapper( "A" );
      SubWrapper param2 = new SubWrapper( "B" );
      assertTrue( proxy.addHello( param1 ).toString().equals( "WRAPPED_WRAPPED_Ahello" ) );
      assertTrue( proxy.add( param1, param2 ).toString().equals( "WRAPPED_WRAPPED_AWRAPPED_B" ) );
    }
  }

  @Test
  public void convert_wrapperSpecReturn_wrapperExactParam() {

    SuperReturnSubParamClass offeredComponent = new SuperReturnSubParamClass();
    Collection<ModuleMatchingInfo> matchingInfos = matcher.calculateTypeMatchingInfos(
        SubWrapperReturnSubParamClass.class,
        SuperReturnSubParamClass.class );
    for ( ModuleMatchingInfo moduleMatchingInfo : matchingInfos ) {
      ProxyFactory<SubWrapperReturnSubParamClass> proxyFactory = moduleMatchingInfo.getConverterCreator()
          .createProxyFactory( SubWrapperReturnSubParamClass.class );
      Collection<MethodMatchingInfo> methodMatchingInfos = moduleMatchingInfo.getMethodMatchingInfos();

      SubWrapperReturnSubParamClass proxy = proxyFactory.createProxy( offeredComponent, methodMatchingInfos );

      SubClass param1 = new SubClass( "A" );
      SubClass param2 = new SubClass( "B" );
      assertTrue( proxy.addHello( param1 ).toString().equals( "WRAPPED_SubAhello" ) );
      assertTrue( proxy.addHello( param1 ).toStringWithPrefix().equals( "WRAPPED_SubSubAhello" ) );
      assertTrue( proxy.add( param1, param2 ).toString().equals( "WRAPPED_SubASubB" ) );
      assertTrue( proxy.add( param1, param2 ).toStringWithPrefix().equals( "WRAPPED_SubSubASubB" ) );
    }
  }
}
\end{lstlisting}

%\include{AnhangB}


%\include{cd}


%\phantomsection


\end{document}
