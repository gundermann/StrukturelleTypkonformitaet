

\newglossaryentry{komponente}
{
    name=Komponente,
    description={Eine Komponente beschreibt in der Softwarearchitektur im Allgemeinen ein Teil eines Softwaresystems. Die Definition dieses Begriffs wird in speziellen Frameworks weiter spezifiziert. Bezogen auf das in der Arbeit verwendete EJB-Framework, werden bspw. die Beans als Komponenten betrachtet (vgl. \cite{ejbspec}).
}
}
\newglossaryentry{artefakt}
{
    name=Artefakt,
    description={Ein Artefakt beschreibt in der Software-Entwicklung die Spezifikation einer physischen Informationseinheit als Ergebnis des Software-Entwicklungsprozesses oder dem Deployment bzw. der Ausführung eines Systems. In der UML Spezifikation 2.1.2 werden folgende konkrete Beispiele für Artefakte genannt (vgl. \cite{uml})
    \begin{itemize}
    \item Dateien in denen Source Code enthalten ist
    \item Skripte
    \item Datenbanktabellen    
    \end{itemize}
    \noindent
    Im Kontext dieser Arbeit sind insbesondere die Dateien, in denen Source Code enthalten ist, allgemein als Artefakt bezeichnet.
}
}

\newglossaryentry{jndi}
{
    name=JNDI,
    description={}
}


\newglossaryentry{injection}
{
    name=Injection,
    description={}
}


\newglossaryentry{proxy}
{
    name=Proxy,
    description={}
}