\chapter{Evaluierung}\label{chap_evaluation}
In dem System, welches für die Evaluation der Heuristiken verwendet wird, sind insgesamt 891 \emph{provided Typen} und 7 \emph{required Typen} enthalten. In \tabref{eIShort} sind die Namen der \emph{required Typen} zusammen mit jeweils einem Kürzel und den Namen der strukturell und semantisch matchenden Kombinationen von \emph{provided Typen} aufgeführt, die bei der Exploration ermittelt werden sollen. Die Kürzel dienen im weiteren Verlauf der Identifizierung der \emph{required Typen}.
\begin{table}[h!]
\centering
\small
\begin{tabular}{|p{6cm}|p{1.5cm}|p{6.5cm}|}
\hline
\hline
\centering\textbf{required Typ} & \textbf{Kürzel} & \textbf{Kombination von provided Typen}\\
\hline
\hline
ElerFTFoerderprogrammeProvider & TEI1 & ElerFTStammdatenAuskunftService\\
\hline
FoerderprogrammeProvider & TEI2 & StammdatenAuskunftService\\
\hline
MinimalFoerderprogrammeProvider & TEI3 & StammdatenAuskunftService\\
\hline
IntubatingFireFighter & TEI4 & Doctor, FireFigher\\
\hline
IntubatingFreeing & TEI5 & Doctor, FireFigher\\
\hline
IntubatingPatientFireFighter & TEI6 & Doctor, FireFigher\\
\hline
KOFGPCProvider & TEI7 & ElerFTStammdatenAuskunftService, StammdatenAuskunftService\\
\hline
\hline
\end{tabular}
\caption{Required Typen mit Kürzeln von matchenden Kombinationen von provided Typen für die Evaluation}
 \label{tab:eIShort}
\end{table}
\noindent
\\
Die Deklaration der \emph{required Typen} und der \emph{provided Typen} aus \tabref{eIShort} ist im Anhang \ref{app_evalTypes} zu finden. Aufgrund der Geheimhaltungspflicht bzgl. der Implementierungsdetails kann auf die Deklaration der Java-Interfaces, die sich aus dieser Deklaration der \emph{required} und \emph{provided Typen} ableiten lassen, und deren Implementierungen in dieser Arbeit nicht genauer eingegangen werden.
\\\\
Um die Ergebnisse nachstellen zu können, kann die Implementierung, welche im Abschnitt \ref{sec_impl_descos} beschrieben wurde, mit einer beliebigen Bibliothek, welche sich ebenfalls durch die in Abschnitt \ref{sec:strukturTypen} beschriebene Struktur von Typen abbilden lässt, verwendet werden.

\section{Darstellung der Evaluationsergebnisse}
Die Evaluationsergebnisse werden in der Form von Vier-Felder-Tafeln dargestellt (Beispiel siehe \tabref{vft:beispiel}). Für jedes desired Interface wird eine Vier-Felder-Tafel für jeden Durchlauf der Schleife innerhalb der Methode $\texttt{semanticEval}$ des Explorationsalgorithmus (siehe Abschnitt \ref{sec_semEval}) aufgezeigt. Aus der jeweiligen Tafel geht hervor, wie viele Proxies über die Funktion $\mathit{targetSets}$ (vgl. Abschnitt \ref{sec_semEval}) in dem aktuellen Iterationsschritt erzeugt werden. Der Wert, den die Iterationsvariable $\texttt{i}$ im betrachteten Durchlauf enthält, wird in der oberen rechten Ecke der Tafel abgebildet.
In der Spalte ``positiv'' ist die Anzahl der Proxies verzeichnet, die innerhalb des Durchlaufs im schlimmsten Fall evaluiert werden. Die Zahl in der Spalte ``negativ'' drückt hingegen aus, wie viele der Proxies aufgrund bestimmter Kriterien (bzw. Heuristiken) nicht evaluiert wurden.  Die Zeile ``falsch'' beschreibt die Anzahl der relevanten Proxies, welche die semantische Evaluation nicht bestehen. Dementsprechend stellt die Zeile ``richtig'' die Anzahl der Proxies dar, welche die semantischen Evaluation bestehen.
\\\\
Aus Abschnitt \ref{sec_proxyCount} geht hervor, dass die Anzahl der Proxies, die für ein desired Interface $R$ mit einer Menge von provided Interfaces $T$ über die Funktion $\mathit{proxyCount(R,T)}$ näherungsweise bestimmt werden kann. Für eine vereinfachte Darstellung der Evaluationsergebnisse bzgl. eines desired Interfaces $R$ aus einer Bibliothek $L$ und einem Iterationsschritt $i$ wird die Anzahl der Proxies für die Anzahl von Mengen von provided Interfaces $A$ wie folgt beschrieben:
\begin{gather*}
p(A) = \begin{array}{l|l}
\mathit{proxyCount(R,targetSets(T,i)} & |T| = A \wedge T \in \mathit{cover(R,L)}
\end{array}
\end{gather*}
\noindent
%Für die Anzahl der zu erzeugenden Proxies $P$ wird der höchste mögliche Wert angenommen. Dieser ist von der Anzahl der angebotenen Methoden $am$ sowie der Anzahl der erwarteten Methoden $em$ abhängig und wird wie folgt berechnet: 
%\begin{equation*}
%MK = \frac{am!}{(am-em)!*em!}
%\end{equation*}
%\noindent
%Die Anzahl der angebotenen Methoden $am$ ist wiederum abhängig von den angebotenen Interfaces deren Typ-Konvertierungsvarianten im jeweiligen Durchlauf miteinander kombiniert wurden. Die Anzahl der Kombinationen von Typ-Konvertierungsvarianten innerhalb des Durchlaufs sei mit $TK$ beschrieben. Der Wert für $TK$ berechnet sich in Abhängigkeit von der Nummer des Durchlaufs $d$ und der Anzahl der strukturell passenden angebotenen Interfaces $n$ (siehe auch Abschnitt Explorationskomponente, 2. Stufe, 2. Kombination von Typ-Konvertierungsvarianten).
%\begin{equation*}
%TK = \frac{n!}{(n-d)!*d!}
%\end{equation*}
%\noindent
%Da die Anzahl der angebotenen Methoden von System zu System schwanken kann, sei die Funktion $am(TK)$ eine näherungsweise Darstellung von $am$, in Abhängigkeit von der Anzahl der kombinierten Typ-Konvertierungsvarianten $TK$.\\\\
%Da durch die Heuristiken letztendlich Methoden-Konvertierungsvarianten aus der Suche herausfallen, wird die Anzahl der entsprechenden Methoden-Konvertierungsvarianten in dem jeweiligen Feld der Vier-Felder-Tafeln als Funktion $mk(TK)$ dargestellt, die wie folgt definiert wird:
%\begin{equation*}
%mk(TK) = \frac{am(TK)!}{(am(TK)-em)!*em!}
%\end{equation*}
%\noindent
\tabref{vft:beispiel} zeigt ein Beispiel für eine solche Vier-Felder-Tafel, in der die Ergebnisse des 1. Iterationsschritt dargestellt sind. Dabei wurden Proxies aus 11 Kombinationen von provided Interfaces evaluiert. 10 dieser Kombinationen bestanden die Evaluation nicht. Die Proxies, die aus 20 Kombinationen von provided Interfaces erzeugt werden konnten, wurden durch Heuristiken im Vorfeld aussortiert. Weiterhin zeigt das Beispiel, dass es einen Proxy gab, der die semantische Evaluation bestand.
\vft{1}{$p(10)$}{$p(20)$}{1}{0}{Beispiel: Vier-Felder-Tafel}{vft:beispiel}


\section{Ausgangspunkt}
Für ein \emph{reqiured Typ} können mehrere \emph{provided Typen} gefunden werden, die eine strukturelle Übereinstimmung aufwiesen. \tabref{amountMatchedInterfaces} zeigt die Anzahl der strukturell übereinstimmenden \emph{provided Typen} je \emph{reqiured Typ}. Diese kommen einzeln oder in Kombination für die semantische Evaluation in Frage.
\begin{table}[H]
\centering
\small
\singlespacing
			\begin{tabular}[c]{|>{\centering\arraybackslash}p{2cm}|>{\centering\arraybackslash}p{5cm}|}
			\hline
			\hline
				 \textbf{required Interface} & \textbf{Anzahl strukturell übereinstimmender provided Interfaces} \\
				\hline\hline
				TEI1 & 221 \\
				\hline
				TEI2 & 272\\
				\hline
				TEI3 & 268\\
				\hline
				TEI4 & 75\\
				\hline
				TEI5 & 75\\
				\hline
				TEI6 & 53\\
				\hline
				TEI7 & 346\\				
				\hline
				\hline
			\end{tabular} 
 \caption{Anzahl strukturell übereinstimmender provided Typen je required Typ}
 \label{tab:amountMatchedInterfaces}
\onehalfspacing
\end{table}
\noindent
Die \tabsrefs{tmr_start_tei1}{tmr_start_tei7_2} zeigen die Vier-Felder-Tafeln, in denen die Ergebnisse der benötigten Iterationen innerhalb des Explorationsalgorithmus für jeden der \emph{required Typen} aus \tabref{amountMatchedInterfaces}. Dabei wurden keine Heuristiken verwendet. Somit stellt dies den Ausgangspunkt für die weitere Evaluation dar.
\begin{multicols}{3}
\vft{1}{$p(44)-1$}{0}{1}{0}{Ausgangspunkt für TEI1}{tmr_start_tei1}\columnbreak
\vft{1}{$p(55)-1$}{0}{1}{0}{Ausgangspunkt für TEI2}{tmr_start_tei2}\columnbreak
\vft{1}{$p(50)-1$}{0}{1}{0}{Ausgangspunkt für TEI3}{tmr_start_tei3}
\end{multicols}
\begin{multicols}{2}
\vft{1}{$p(42)$}{0}{0}{0}{Ausgangspunkt für TEI4 \\1. Durchlauf}{tmr_start_tei4_1}\columnbreak
\vft{2}{$p(2247)-1$}{0}{1}{0}{Ausgangspunkt für TEI4 \\2. Durchlauf}{tmr_start_tei4_2}
\end{multicols}
\begin{multicols}{2}
\vft{1}{$p(75)$}{0}{0}{0}{Ausgangspunkt für TEI5 \\1. Durchlauf}{tmr_start_tei5_1}\columnbreak
\vft{2}{$p(2775)-1$}{0}{1}{0}{Ausgangspunkt für TEI5 \\2. Durchlauf}{tmr_start_tei5_2}
\end{multicols}
\begin{multicols}{2}
\vft{1}{$p(41)$}{0}{0}{0}{Ausgangspunkt für TEI6 \\1. Durchlauf}{tmr_start_tei6_1}
\columnbreak
\vft{2}{$p(1323)-1$}{0}{1}{0}{Ausgangspunkt für TEI6 \\2. Durchlauf}{tmr_start_tei6_2}
\end{multicols}
\begin{multicols}{2}
\vft{1}{$p(174)$}{0}{0}{0}{Ausgangspunkt für TEI7 \\1. Durchlauf}{tmr_start_tei7_1}
\columnbreak
\vft{2}{$p(52150)-1$}{0}{1}{0}{Ausgangspunkt für TEI7 \\2. Durchlauf}{tmr_start_tei7_2}
\end{multicols}
\noindent
Für die \emph{required Typen} \emph{TEI4}-\emph{TEI7} werden zwei Durchläufe benötigt, da die semantischen Test nur von einem Proxy bestanden werden, der aus einer Kombination zweier \emph{provided Typen} erzeugt wurde (siehe auch \tabref{eIShort}).

\section{Ergebnisse für die Heuristik LMF}\label{sec_evalLMF}
In Bezug auf die Heuristik \emph{LMF} gilt es nicht nur zu evaluieren, ob die Suche nach einem Proxy, der die vordefinierten Tests besteht, beschleunigt werden kann, sondern auch, mit welcher Variante zur Bestimmung des Matcherratings (vgl. Abschnitt \ref{sec_lmf}) die besten Ergebnisse erzielt werden können. 
\\\\
Hierzu wird die Exploration für alle der oben genannten \emph{required Typen} für jede Variante zur Bestimmung der Matcherratings durchgeführt (siehe Abschnitt \ref{sec_lmf} Tabelle \ref{tab_matcherratingvarianten}). Im folgenden Verlauf wird lediglich auf die Variante eingegangen, die die besten Ergebnisse hervorgebracht hat. Die Ergebnisse unter Verwendung der übrigen Varianten sind im Anhang \ref{app_matcherratingEval} zu finden.
\\\\
Die Variante \emph{1.1} (vgl. Tabelle \ref{tab_matcherratingvarianten}) erbrachte die besten Ergebnisse. Die folgenden Vier-Felder-Tafeln zeigen die Ergebnisse mit dieser Variante zur Bestimmung der Matcherratings für die \emph{required Typen} \emph{TEI1}-\emph{TEI3} auf.
\begin{multicols}{3}
\vft{1}{5}{$p(44)-6$}{1}{0}{Ergebnisse \emph{LMF} mit Variante 1.1 für TEI1 \\1. Durchlauf}{lmf11_TEI1_1}
\vft{1}{1889}{$p(55)-1890$}{1}{0}{Ergebnisse \emph{LMF} mit Variante 1.1 für TEI2 1.~\mbox{Durchlauf}}{lmf11_TEI2_1}
\vft{1}{1463}{$p(50)-1464$}{1}{0}{Ergebnisse \emph{LMF} mit Variante 1.1 für TEI3 1.~\mbox{Durchlauf}}{lmf11_TEI3_1}
\end{multicols}
\noindent
Die Ergebnisse für die \emph{required Typen} \emph{TEI4}-\emph{TEI7} zeigen die folgenden Vier-Felder-Tafeln. 
\begin{multicols}{2}
\vft{1}{$1174$}{0}{0}{0}{Ergebnisse \emph{LMF} mit Variante 1.1 für TEI4 1.~\mbox{Durchlauf}}{lmf11_TEI4_1}
\vft{2}{2}{$p(2247)-3$}{1}{0}{Ergebnisse \emph{LMF} mit Variante 1.1 für TEI4 2.~\mbox{Durchlauf}}{lmf11_TEI4_2}
\end{multicols}

\begin{multicols}{2}
\vft{1}{$4984$}{0}{0}{0}{Ergebnisse \emph{LMF} mit Variante 1.1 für TEI5 1.~\mbox{Durchlauf}}{lmf11_TEI5_1}
\vft{2}{32}{$p(2775)-33$}{1}{0}{Ergebnisse \emph{LMF} mit Variante 1.1 für TEI5 2.~\mbox{Durchlauf}}{lmf11_TEI5_2}
\end{multicols}

\begin{multicols}{2}
\vft{1}{$1051$}{0}{0}{0}{Ergebnisse \emph{LMF} mit Variante 1.1 für TEI6 1.~\mbox{Durchlauf}}{lmf11_TEI6_1}
\vft{2}{0}{$p(1323)-1$}{1}{0}{Ergebnisse \emph{LMF} mit Variante 1.1 für TEI6 2.~\mbox{Durchlauf}}{lmf11_TEI6_2}
\end{multicols}

\begin{multicols}{2}
\vft{1}{$161294$}{0}{0}{0}{Ergebnisse \emph{LMF} mit Variante 1.1 für TEI7 1.~\mbox{Durchlauf}}{lmf11_TEI7_1}
\vft{2}{7641}{$p(52150)-7642$}{1}{0}{Ergebnisse \emph{LMF} mit Variante 1.1 für TEI7 2.~\mbox{Durchlauf}}{lmf11_TEI7_2}
\end{multicols}
\noindent
Folgendes kann aus diesen Ergebnissen abgeleitet werden:
\begin{enumerate}
\item Die Heuristik \emph{LMF} erzielt eine Reduktion der zu erzeugenden Proxies. Dies wird durch einen Vergleich der Spalte ``positiv'' innerhalb der Vier-Felder-Tafeln zum jeweiligen \emph{required Typ} belegt.

\item Die Heuristik \emph{LMF} hat keine Auswirkung auf einen Durchlauf, in dem kein Proxy erzeugt wird, mit dem die semantischen Tests erfolgreich durchgeführt werden können. Dies kann durch einen Vergleich des ersten Durchlaufs für die \emph{required Typen} \emph{TEI4}-\emph{TEI7} im Ausgangspunkt (Tabellen \ref{tab:tmr_start_tei4_1}, \ref{tab:tmr_start_tei5_1}, \ref{tab:tmr_start_tei6_1} und \ref{tab:tmr_start_tei6_1}) mit dem ersten Durchlauf unter Anwendung der Heuristik (Tabellen \ref{tab:lmf11_TEI4_1}, \ref{tab:lmf11_TEI5_1}, \ref{tab:lmf11_TEI6_1} und \ref{tab:lmf11_TEI7_1}) festgestellt werden.
\end{enumerate}



%Aus diesen Ergebnissen lässt sich folgendes ableiten:
%\begin{enumerate}
%\item Das Akkumulationsverfahren Nummer 3. (Minimum) führt sowohl für die Typ- und Methoden-Konvertierungsvarianten zu schlechteren Ergebnissen als die anderen drei Akkumulationsverfahren. Es sollte daher für die Heuristik TMR\_Quant nicht verwendet werden.
%\item Die Ergebnisse von 1-2 und 3-2 unterscheiden sich nur geringfügig, obwohl bei 3-2 das Akkumulationsverfahren Nummer 3. zum Einsatz kam. Dies konnte auch bei anderen Kombinationen festgestellt werden, bei denen das 3. Akkumulationsverfahren für die Akkumulation des Type-Matcher Ratings der Typ-Konvertierungsvariante verwendet wurde. Das lässt vermuten, dass die Beachtung des Type-Matcher Ratings einer ganzen Typ-Konvertierungsvariante weitgehend unerheblich für die Heuristik TMR\_Quant ist.
%, wenn das Type-Matcher Rating je Methoden-Konvertierungsvarianten über ein entsprechend gutes Akkumulationsverfahren ermittelt wurde. 
%Dies ist jedoch darauf zurückzuführen, dass das Type-Matcher Rating je Methoden-Konvertierungsvariante die Parameter für die Ermittlung des Type-Matcher Ratings einer Typ-Konvertierungsvariante darstellen.
%\item An den Ergebnissen zu den erwarteten Interfaces TEI4-TEI6 ist zu erkennen, dass die Heuristik TMR\_Quant keinen Einfluss auf den 1. Durchlauf hat. Daraus kann geschlussfolgert werden, dass die Heuristik nur in dem Durchlauf einen Gewinn bringt, in dem auch eine passende benötigte Komponente gefunden werden kann. 
%\end{enumerate}
%Aufgrund der Ergebnisse stehen für die weitere Verwendung der Heuristik TMR\_Qual mehrere Kombinationen von Akkumulationsverfahren zur Auswahl. Die Entscheidung fällt aufgrund der etwas geringeren Komplexität auf die Kombination 1-2. 

\section{Ergebnisse für die Heuristik PTTF}\label{sec_evalPTTF}
Für die \Gls{Heuristik} \emph{PTTF} gilt es zu evaluieren, ob die Suche nach einem \emph{Proxy}, der die vordefinierten Tests besteht, beschleunigt werden kann. Hierzu wird der \emph{Explorationsprozess} für alle in Tabelle \ref{tab:eIShort} genannten \emph{required Typen} unter der Verwendung der in Abschnitt \ref{sec_pttf} beschriebenen \Gls{Heuristik} durchgeführt.
\\\\
Die folgenden Vier-Felder-Tafeln zeigen die Ergebnisse für die \emph{required Typen} \emph{TEI1}-\emph{TEI7} auf.
\begin{multicols}{3}
\vft{1}{29}{$p_1(44)-30$}{1}{0}{Ergebnisse \emph{PTTF} für TEI1 1.~\mbox{Durchlauf}}{pttf_TEI1_1}
\vft{1}{5544}{$p_1(55)-5545$}{1}{0}{Ergebnisse \emph{PTTF} für TEI2 1.~\mbox{Durchlauf}}{pttf_TEI2_1}
\vft{1}{4761}{$p_1(50)-4762$}{1}{0}{Ergebnisse \emph{PTTF} für TEI3 1.~\mbox{Durchlauf}}{pttf_TEI3_1}
\end{multicols}

\begin{multicols}{2}
\vft{1}{$1174$}{0}{0}{0}{Ergebnisse \emph{PTTF} für TEI4 1.~\mbox{Durchlauf}}{pttf_TEI4_1}
\vft{2}{466}{$p_2(2247)-467$}{1}{0}{Ergebnisse \emph{PTTF} für TEI4 2.~\mbox{Durchlauf}}{pttf_TEI4_2}
\end{multicols}
\pagebreak
\begin{multicols}{2}
\vft{1}{$4984$}{0}{0}{0}{Ergebnisse \emph{PTTF} für TEI5 1.~\mbox{Durchlauf}}{pttf_TEI5_1}
\vft{2}{2172}{$p_2(2775)-2173$}{1}{0}{Ergebnisse \emph{PTTF} für TEI5 2.~\mbox{Durchlauf}}{pttf_TEI5_2}
\end{multicols}

\begin{multicols}{2}
\vft{1}{$1051$}{0}{0}{0}{Ergebnisse \emph{PTTF} für TEI6 1.~\mbox{Durchlauf}}{pttf_TEI6_1}
\vft{2}{13122}{$p_2(1323)-13123$}{1}{0}{Ergebnisse \emph{PTTF} für TEI6 2.~\mbox{Durchlauf}}{pttf_TEI6_2}
\end{multicols}

\begin{multicols}{2}
\vft{1}{$161294$}{0}{0}{0}{Ergebnisse \emph{PTTF} für TEI7 1.~\mbox{Durchlauf}}{pttf_TEI7_1}
\vft{2}{149961}{$p_2(52150)-149962$}{1}{0}{Ergebnisse \emph{PTTF} für TEI7 2.~\mbox{Durchlauf}}{pttf_TEI7_2}
\end{multicols}
\newpage
\noindent
Folgendes kann aus diesen Ergebnissen abgeleitet werden:
\begin{enumerate}
\item Die \Gls{Heuristik} \emph{PTTF} erzielt im Vergleich zum Ausgangspunkt (Abschnitt \ref{sec_ausgangspunkt}) für jeden \emph{required Typ} eine weitere Reduktion der zu prüfenden \emph{Proxies}.

\item Die Heuristik \emph{PTTF} hat keine Auswirkung auf einen Durchlauf, in dem kein \emph{Proxy} erzeugt wird, mit dem die vordefinierten Tests erfolgreich durchgeführt werden können. Dies kann durch einen Vergleich des ersten Durchlaufs für den \emph{required Typ} \emph{TEI4}-\emph{TEI7} im Ausgangspunkt (Tabelle \ref{tab:tmr_start_tei4_1}, \ref{tab:tmr_start_tei5_1}, \ref{tab:tmr_start_tei6_1} und \ref{tab:tmr_start_tei6_1}) mit dem ersten Durchlauf unter Anwendung der Heuristik (Tabellen \ref{tab:pttf_TEI4_1}, \ref{tab:pttf_TEI5_1}, \ref{tab:pttf_TEI6_1} und \ref{tab:pttf_TEI7_1}) festgestellt werden. Aus diesem Grund kommt die in Punkt 1 beschriebene Reduktion erst im jeweils letzten Durchlauf zum Tragen.
\end{enumerate}
\section{Ergebnisse für die Heuristik BL\_NMC}\label{sec_evalBLNMC}
Für die \Gls{Heuristik} \emph{BL\_NMC} gilt es zu evaluieren, ob die Suche nach einem \emph{Proxy}, der die vordefinierten Tests besteht, beschleunigt werden kann. Hierzu wird der \emph{Explorationsprozess} für alle in Tabelle \ref{tab:eIShort}genannten \emph{required Typen} unter der Verwendung der in Abschnitt \ref{sec_bl_nmc} beschriebenen \gls{Heuristik} durchgeführt.
\\\\
Die folgenden Vier-Felder-Tafeln zeigen die Ergebnisse für die \emph{required Typen} \emph{TEI1}-\emph{TEI7} auf.
\begin{multicols}{3}
\vft{1}{105}{$p_1(44)-106$}{1}{0}{Ergebnisse \emph{BL\_NMC} für TEI1 1.~\mbox{Durchlauf}}{blnmc_TEI1_1}
\vft{1}{342}{$p_1(55)-343$}{1}{0}{Ergebnisse \emph{BL\_NMC} für TEI2 1.~\mbox{Durchlauf}}{blnmc_TEI2_1}
\vft{1}{357}{$p_1(50)-358$}{1}{0}{Ergebnisse \emph{BL\_NMC} für TEI3 1.~\mbox{Durchlauf}}{blnmc_TEI3_1}
\end{multicols}

\begin{multicols}{2}
\vft{1}{120}{$1054$}{0}{0}{Ergebnisse \emph{BL\_NMC} für TEI4 1.~\mbox{Durchlauf}}{blnmc_TEI4_1}
\vft{2}{442}{$p_2(2247)-443$}{1}{0}{Ergebnisse \emph{BL\_NMC} für TEI4 2.~\mbox{Durchlauf}}{blnmc_TEI4_2}
\end{multicols}

\begin{multicols}{2}
\vft{1}{550}{$4434$}{0}{0}{Ergebnisse \emph{BL\_NMC} für TEI5 1.~\mbox{Durchlauf}}{blnmc_TEI5_1}
\vft{2}{1304}{$p_2(2775)-1305$}{1}{0}{Ergebnisse \emph{BL\_NMC} für TEI5 2.~\mbox{Durchlauf}}{blnmc_TEI5_2}
\end{multicols}
\pagebreak
\begin{multicols}{2}
\vft{1}{366}{$685$}{0}{0}{Ergebnisse \emph{BL\_NMC} für TEI6 1.~\mbox{Durchlauf}}{blnmc_TEI6_1}
\vft{2}{204}{$p_2(1323)-205$}{1}{0}{Ergebnisse \emph{BL\_NMC} für TEI6 2.~\mbox{Durchlauf}}{blnmc_TEI6_2}
\end{multicols}

\begin{multicols}{2}
\vft{1}{1051}{$160243$}{0}{0}{Ergebnisse \emph{BL\_NMC} für TEI7 1.~\mbox{Durchlauf}}{blnmc_TEI7_1}
\vft{2}{135089}{$p_2(52150)-135090$}{1}{0}{Ergebnisse \emph{BL\_NMC} für TEI7 2.~\mbox{Durchlauf}}{blnmc_TEI7_2}
\end{multicols}

Folgendes kann aus diesen Ergebnissen abgeleitet werden:
\begin{enumerate}
\item Die \Gls{Heuristik} \emph{BL\_NMC} erzielt im Vergleich zum Ausgangspunkt (Abschnitt \ref{sec_ausgangspunkt}) für jeden \emph{required Typ} eine weitere Reduktion der zu prüfenden \emph{Proxies}.

\item Die Heuristik \emph{BL\_NMC} hat das Potential jeden Durchlauf innerhalb der \emph{semantischen Evaluation} zu beschleunigen. Für den jeweils ersten Durchlauf kann dies durch einen Vergleich der Tabellen \ref{tab:tmr_start_tei1}, \ref{tab:tmr_start_tei2}, \ref{tab:tmr_start_tei3}, \ref{tab:tmr_start_tei4_1}, \ref{tab:tmr_start_tei5_1}, \ref{tab:tmr_start_tei6_1} und \ref{tab:tmr_start_tei7_1} zum Ausgangspunkt mit den Tabellen \ref{tab:blnmc_TEI1_1}, \ref{tab:blnmc_TEI2_1}, \ref{tab:blnmc_TEI3_1}, \ref{tab:blnmc_TEI4_1}, \ref{tab:blnmc_TEI5_1}, \ref{tab:blnmc_TEI6_1} und \ref{tab:blnmc_TEI7_1} festgestellt werden. Ein Vergleich der Tabelle \ref{tab:tmr_start_tei4_2}, \ref{tab:tmr_start_tei5_2}, \ref{tab:tmr_start_tei6_2} und \ref{tab:tmr_start_tei7_2} im Ausgangspunkt mit den Tabellen \ref{tab:blnmc_TEI4_2}, \ref{tab:blnmc_TEI5_2}, \ref{tab:blnmc_TEI6_2} und \ref{tab:blnmc_TEI7_2} belegt dies für den zweiten Durchlauf auf.
\end{enumerate}


\section{Ergebnisse für die Kombination der Heuristiken}
Nachdem gezeigt wurde, dass die Exploration durch jede der beschriebenen Heuristiken beschleunigt werden kann. Dabei wurden Exploration mit jeweils einer der Heuristiken durchgeführt. In den folgenden Abschnitten soll evaluiert werden, ob die Verwendung einer Kombination der einzelnen Heuristiken bei der Exploration einen zusätzlichen Vorteil bringt.
\\\\
Hierzu werden die Ergebnisse aller Kombinationen der einzelnen Heuristiken aufgeführt und im Anschluss bewertet.
\subsection{Kombination: LMF + PTTF}\label{sec_evalLMFPTTF}
Die folgenden Vier-Felder Tafeln zeigen die Ergebnisse mit der Kombination der Heuristiken \emph{LMF} und \emph{PTTF}.
\begin{multicols}{3}
\vft{1}{5}{$p(44)-6$}{1}{0}{Ergebnisse \emph{LMF} + \emph{PTTF} für TEI1}{lmfpttf_TEI1_1}\columnbreak
\vft{1}{1877}{$p(55)-1878$}{1}{0}{Ergebnisse \emph{LMF} + \emph{PTTF} für TEI2 1. Durchlauf}{lmfpttf_TEI2_1}\columnbreak
\vft{1}{1473}{$p(50)-1474$}{1}{0}{Ergebnisse \emph{LMF} + \emph{PTTF} für TEI3 1. Durchlauf}{lmfpttf_TEI3_1}
\end{multicols}

\begin{multicols}{2}
\vft{1}{$p(42)$}{0}{0}{0}{Ergebnisse \emph{LMF} + \emph{PTTF} für TEI4 1. Durchlauf}{lmfpttf_TEI4_1}\columnbreak
\vft{2}{4}{$p(2247)-5$}{1}{0}{Ergebnisse \emph{LMF} + \emph{PTTF} für TEI4 2. Durchlauf}{lmfpttf_TEI4_2}
\end{multicols}

\begin{multicols}{2}
\vft{1}{$p(75)$}{0}{0}{0}{Ergebnisse \emph{LMF} + \emph{PTTF} für TEI5 1. Durchlauf}{lmfpttf_TEI5_1}\columnbreak
\vft{2}{34}{$p(2346)-35$}{1}{0}{Ergebnisse \emph{LMF} + \emph{PTTF} für TEI5 2. Durchlauf}{lmfpttf_TEI5_2}
\end{multicols}

\begin{multicols}{2}
\vft{1}{$p(41)$}{0}{0}{0}{Ergebnisse \emph{LMF} + \emph{PTTF} für TEI6 1. Durchlauf}{lmfpttf_TEI6_1}\columnbreak
\vft{2}{0}{$p(1323)-1$}{1}{0}{Ergebnisse \emph{LMF} + \emph{PTTF} für TEI6 2. Durchlauf}{lmfpttf_TEI6_2}
\end{multicols}

\begin{multicols}{2}
\vft{1}{$p(174)$}{0}{0}{0}{Ergebnisse \emph{LMF} + \emph{PTTF} für TEI7 1. Durchlauf}{lmfpttf_TEI7_1}\columnbreak
\vft{2}{1076}{$p(52150)-1077$}{1}{0}{Ergebnisse \emph{LMF} + \emph{PTTF} für TEI7 2. Durchlauf}{lmfpttf_TEI7_2}
\end{multicols}
\noindent
Aus diesen Ergebnisse lässt sich folgenden Ableiten:
\begin{enumerate}
\item Auf den ersten Durchlauf während der Exploration wirkt sich die Kombination der Heuristiken \emph{LMF} und \emph{PTTF} nicht nennenswert aus. Hierzu sind die Tabellen \ref{tab:lmfpttf_TEI1_1}, \ref{tab:lmfpttf_TEI2_1}, \ref{tab:lmfpttf_TEI3_1}, \ref{tab:lmfpttf_TEI4_1}, \ref{tab:lmfpttf_TEI5_1}, \ref{tab:lmfpttf_TEI6_1} und \ref{tab:lmfpttf_TEI7_1} mit den Tabellen der Heuristik mit den besseren Ergebnissen im ersten Durchlauf (\emph{LMF}) zu vergleichen (siehe Abschnitt \ref{sec_evalLMF} Tabellen \ref{tab:lmf11_TEI1_1}, \ref{tab:lmf11_TEI2_1}, \ref{tab:lmf11_TEI3_1}, \ref{tab:lmf11_TEI4_1}, \ref{tab:lmf11_TEI5_1}, \ref{tab:lmf11_TEI6_1} und \ref{tab:lmf11_TEI7_1}).
\item Für den zweiten Durchlauf während der Exploration ist eine Verbesserung zu erkennen. Diese bezieht sich jedoch nur auf die Exploration für \emph{TEI7} (vergleiche Tabelle \ref{tab:lmf11_TEI7_2} aus Abschnitt \ref{sec_evalLMF} mit Tabelle \ref{tab:lmfpttf_TEI7_2}).
\end{enumerate}

\subsection{Kombination: LMF + BL\_NMC}\label{sec_evalLMFBLNMC}
Die folgenden Vier-Felder Tafeln zeigen die Ergebnisse mit der Kombination der Heuristiken \emph{LMF} und \emph{BL\_NMC}.
\begin{multicols}{3}
\vft{1}{0}{$p(44)-1$}{1}{0}{Ergebnisse \emph{LMF} + \emph{BL\_NMC} für TEI1}{lmfbl_TEI1_1}\columnbreak
\vft{1}{83}{$p(55)-84$}{1}{0}{Ergebnisse \emph{LMF} + \emph{BL\_NMC} für TEI2 1. Durchlauf}{lmfbl_TEI2_1}\columnbreak
\vft{1}{89}{$p(50)-90$}{1}{0}{Ergebnisse \emph{LMF} + \emph{BL\_NMC} für TEI3 1. Durchlauf}{lmfbl_TEI3_1}
\end{multicols}

\begin{multicols}{2}
\vft{1}{120}{$p(42)-120$}{0}{0}{Ergebnisse \emph{LMF} + \emph{BL\_NMC} für TEI4 1. Durchlauf}{lmfbl_TEI4_1}\columnbreak
\vft{2}{4}{$p(2247)-5$}{1}{0}{Ergebnisse \emph{LMF} + \emph{BL\_NMC} für TEI4 2. Durchlauf}{lmfbl_TEI4_2}
\end{multicols}

\begin{multicols}{2}
\vft{1}{550}{$p(75)-550$}{0}{0}{Ergebnisse \emph{LMF} + \emph{BL\_NMC} für TEI5 1. Durchlauf}{lmfbl_TEI5_1}\columnbreak
\vft{2}{34}{$p(2346)-35$}{1}{0}{Ergebnisse \emph{LMF} + \emph{BL\_NMC} für TEI5 2. Durchlauf}{lmfbl_TEI5_2}
\end{multicols}

\begin{multicols}{2}
\vft{1}{115}{$p(41)-115$}{0}{0}{Ergebnisse \emph{LMF} + \emph{PTTF} für TEI6 1. Durchlauf}{lmfbl_TEI6_1}\columnbreak
\vft{2}{0}{$p(1323)-1$}{1}{0}{Ergebnisse \emph{LMF} + \emph{PTTF} für TEI6 2. Durchlauf}{lmfbl_TEI6_2}
\end{multicols}

\begin{multicols}{2}
\vft{1}{2448}{$p(174)-2448$}{0}{0}{Ergebnisse \emph{LMF} + \emph{BL\_NMC} für TEI7 1. Durchlauf}{lmfbl_TEI7_1}\columnbreak
\vft{2}{954}{$p(52150)-955$}{1}{0}{Ergebnisse \emph{LMF} + \emph{BL\_NMC} für TEI7 2. Durchlauf}{lmfbl_TEI7_2}
\end{multicols}
\noindent
Aus diesen Ergebnisse lässt sich folgenden Ableiten:
\begin{enumerate}
\item Auf den ersten Durchlauf während der Exploration wirkt sich die Kombination der Heuristiken \emph{LMF} und \emph{BL\_NMC} positiv aus. Hierzu sind ist die Tabelle \ref{tab:lmfbl_TEI1_1} mit der Tabelle \ref{tab:lmf11_TEI1_1} aus Abschnitt \ref{sec_evalLMF} sowie die Tabellen \ref{tab:lmfbl_TEI2_1}, \ref{tab:lmfbl_TEI3_1} und \ref{tab:lmfbl_TEI6_1} mit den Tabellen \ref{tab:blnmc_TEI2_1}, \ref{tab:blnmc_TEI3_1}, \ref{tab:blnmc_TEI3_1} und \ref{tab:blnmc_TEI6_1} aus Abschnitt \ref{sec_evalBLNMC} zu vergleichen.
\item Für den zweiten Durchlauf während der Exploration ist ebenfalls eine Verbesserung zu erkennen. Diese bezieht sich jedoch nur auf die Exploration für \emph{TEI7} (vergleiche Tabelle \ref{tab:lmf11_TEI7_2} aus Abschnitt \ref{sec_evalLMF} mit Tabelle \ref{tab:lmfbl_TEI7_2}).
\end{enumerate}

\subsection{Kombination: PTTF + BL\_NMC}\label{sec_evalPTTFBLNMC}
Die folgenden Vier-Felder Tafeln zeigen die Ergebnisse mit der Kombination der Heuristiken \emph{PTTF} und \emph{BL\_NMC}.
\begin{multicols}{3}
\vft{1}{104}{$p(44)-105$}{1}{0}{Ergebnisse \emph{PTTF} + \emph{BL\_NMC} für TEI1}{pttfbl_TEI1_1}\columnbreak
\vft{1}{337}{$p(55)-338$}{1}{0}{Ergebnisse \emph{PTTF} + \emph{BL\_NMC} für TEI2 1. Durchlauf}{pttfbl_TEI2_1}\columnbreak
\vft{1}{357}{$p(50)-358$}{1}{0}{Ergebnisse \emph{PTTF} + \emph{BL\_NMC} für TEI3 1. Durchlauf}{pttfbl_TEI3_1}
\end{multicols}

\begin{multicols}{2}
\vft{1}{120}{$p(42)-120$}{0}{0}{Ergebnisse \emph{PTTF} + \emph{BL\_NMC} für TEI4 1. Durchlauf}{pttfbl_TEI4_1}\columnbreak
\vft{2}{47}{$p(2247)-48$}{1}{0}{Ergebnisse \emph{PTTF} + \emph{BL\_NMC} für TEI4 2. Durchlauf}{pttfbl_TEI4_2}
\end{multicols}

\begin{multicols}{2}
\vft{1}{550}{$p(75)-550$}{0}{0}{Ergebnisse \emph{PTTF} + \emph{BL\_NMC} für TEI5 1. Durchlauf}{pttfbl_TEI5_1}\columnbreak
\vft{2}{219}{$p(2346)-220$}{1}{0}{Ergebnisse \emph{PTTF} + \emph{BL\_NMC} für TEI5 2. Durchlauf}{pttfbl_TEI5_2}
\end{multicols}

\begin{multicols}{2}
\vft{1}{366}{$p(41)-366$}{0}{0}{Ergebnisse \emph{PTTF} + \emph{PTTF} für TEI6 1. Durchlauf}{pttfbl_TEI6_1}\columnbreak
\vft{2}{204}{$p(1323)-205$}{1}{0}{Ergebnisse \emph{PTTF} + \emph{PTTF} für TEI6 2. Durchlauf}{pttfbl_TEI6_2}
\end{multicols}

\begin{multicols}{2}
\vft{1}{1036}{$p(174)-1036$}{0}{0}{Ergebnisse \emph{PTTF} + \emph{BL\_NMC} für TEI7 1. Durchlauf}{pttfbl_TEI7_1}\columnbreak
\vft{2}{6015}{$p(52150)-6016$}{1}{0}{Ergebnisse \emph{PTTF} + \emph{BL\_NMC} für TEI7 2. Durchlauf}{pttfbl_TEI7_2}
\end{multicols}
\noindent
Aus diesen Ergebnisse lässt sich folgenden Ableiten:
\begin{enumerate}
\item Auf den ersten Durchlauf während der Exploration hat die Kombination der der Heuristiken \emph{PTTF} und \emph{BL\_NMC} keine Auswirkung. Die Ergebnisse sind nahezu identisch mit denen der Exploration mit der Heuristik \emph{BL\_NMC} aus Abschnitt \ref{sec_evalBLNMC}. (Vergleiche Tabellen \ref{tab:pttfbl_TEI1_1}, \ref{tab:pttfbl_TEI2_1}, \ref{tab:pttfbl_TEI3_1}, \ref{tab:pttfbl_TEI4_1}, \ref{tab:pttfbl_TEI5_1}, \ref{tab:lmfbl_TEI6_1} und  \ref{tab:pttfbl_TEI7_1} mit den Tabellen \ref{tab:blnmc_TEI1_1}, \ref{tab:blnmc_TEI2_1}, \ref{tab:blnmc_TEI3_1}, \ref{tab:blnmc_TEI4_1}, \ref{tab:blnmc_TEI5_1}, \ref{tab:blnmc_TEI6_1} und \ref{tab:blnmc_TEI7_1}.
\item Für den zweiten Durchlauf während der Exploration ist eine Verbesserung  zu erkennen. Da mit der Heuristik \emph{BL\_NMC} bessere Ergebnisse erzielt wurden als mit der Heuristik \emph{PTTF} (vergleiche Ergebnisse aus Abschnitt \ref{sec_evalBLNMC} mit den Ergebnissen aus Abschnitt \ref{sec_evalPTTF}) kann dies durch den Vergleich der Tabellen \ref{tab:pttfbl_TEI4_2}, \ref{tab:pttfbl_TEI5_2}, \ref{tab:lmfbl_TEI6_2} und  \ref{tab:pttfbl_TEI7_2} mit den Tabellen \ref{tab:blnmc_TEI4_2}, \ref{tab:blnmc_TEI5_2}, \ref{tab:blnmc_TEI6_2} und \ref{tab:blnmc_TEI7_2} belegt werden.
\end{enumerate}


\subsection{Kombination: LMF + PTTF + BL\_NMC}
Die folgenden Vier-Felder Tafeln zeigen die Ergebnisse mit der Kombination der Heuristiken \emph{LMF}, \emph{PTTF} und \emph{BL\_NMC}.
\begin{multicols}{3}
\vft{1}{2}{$p(44)-3$}{1}{0}{Ergebnisse \emph{LMF} + \emph{PTTF} + \emph{BL\_NMC} für TEI1}{lmfpttfbl_TEI1_1}\columnbreak
\vft{1}{79}{$p(55)-80$}{1}{0}{Ergebnisse \emph{LMF} + \emph{PTTF} + \emph{BL\_NMC} für TEI2 1. Durchlauf}{lmfpttfbl_TEI2_1}\columnbreak
\vft{1}{86}{$p(50)-87$}{1}{0}{Ergebnisse \emph{LMF} + \emph{PTTF} + \emph{BL\_NMC} für TEI3 1. Durchlauf}{lmfpttfbl_TEI3_1}
\end{multicols}

\begin{multicols}{2}
\vft{1}{120}{$p(42)-120$}{0}{0}{Ergebnisse \emph{LMF} + \emph{PTTF} + \emph{BL\_NMC} für TEI4 1. Durchlauf}{lmfpttfbl_TEI4_1}\columnbreak
\vft{2}{4}{$p(2247)-5$}{1}{0}{Ergebnisse \emph{LMF} + \emph{PTTF} + \emph{BL\_NMC} für TEI4 2. Durchlauf}{lmfpttfbl_TEI4_2}
\end{multicols}

\begin{multicols}{2}
\vft{1}{550}{$p(75)-550$}{0}{0}{Ergebnisse \emph{LMF} + \emph{PTTF} + \emph{BL\_NMC} für TEI5 1. Durchlauf}{lmfpttfbl_TEI5_1}\columnbreak
\vft{2}{34}{$p(2346)-35$}{1}{0}{Ergebnisse \emph{LMF} + \emph{PTTF} + \emph{BL\_NMC} für TEI5 2. Durchlauf}{lmfpttfbl_TEI5_2}
\end{multicols}

\begin{multicols}{2}
\vft{1}{115}{$p(41)-115$}{0}{0}{Ergebnisse \emph{LMF} + \emph{PTTF} + \emph{PTTF} für TEI6 1. Durchlauf}{lmfpttfbl_TEI6_1}\columnbreak
\vft{2}{0}{$p(1323)-1$}{1}{0}{Ergebnisse \emph{LMF} + \emph{PTTF} + \emph{PTTF} für TEI6 2. Durchlauf}{lmfpttfbl_TEI6_2}
\end{multicols}

\begin{multicols}{2}
\vft{1}{2448}{$p(174)-2448$}{0}{0}{Ergebnisse \emph{LMF} + \emph{PTTF} + \emph{BL\_NMC} für TEI7 1. Durchlauf}{lmfpttfbl_TEI7_1}\columnbreak
\vft{2}{12}{$p(52150)-13$}{1}{0}{Ergebnisse \emph{LMF} + \emph{PTTF} + \emph{BL\_NMC} für TEI7 2. Durchlauf}{lmfpttfbl_TEI7_2}
\end{multicols}
\noindent
Aus diesen Ergebnisse lässt sich folgenden Ableiten:
\begin{enumerate}
\item Auf den ersten Durchlauf während der Exploration wirkt sich die Kombination der Heuristiken \emph{LMF}, \emph{PTTF} und \emph{BL\_NMC} nicht besser aus, als die Kombination der Heuristiken \emph{LMF} und \emph{BL\_NMC} (siehe Abschnitt \ref{sec_evalLMFBLNMC}). Die Ergebnisse sind nahezu identisch.
\item Für den zweiten Durchlauf während der Exploration gilt zumindest für die \emph{required Typen} \emph{TEI4}-\emph{TEI6} dasselbe, wie für den ersten Durchlauf. Für den \emph{required Typ} \emph{TEI7} ist hingegen nochmals eine Verbesserung im Vergleich zu den 2er-Kombinationen (siehe Abschnitte \ref{sec_evalLMFPTTF}-\ref{sec_evalPTTFBLNMC}) zu erkennen.
\end{enumerate}
\noindent
%Das ist schon Diskussion!!!!!
Zusammenfassend ist zu Ergebnissen bzgl. der Kombination der Heuristiken zu sagen, dass sich zwischen den Heuristiken Synergien ergeben, wodurch im Allgemeinen weniger Proxies evaluiert werden müssen, als wenn die Exploration mit lediglich einer der Heuristiken durchgeführt wurde. 
\\\\
Allerdings gibt es auch Fälle, in denen sich die Kombination negativ auswirkt.   Hierzu ist die Exploration für \emph{TEI7} anzumerken, die im ersten Durchlauf  bei den Kombinationen der Heuristiken 2448 Proxies erzeugte und evaluiert, sofern die Heuristik \emph{LMF} in Kombination mit einer oder mehrerer anderer Heuristiken verwendet wurde. Die Ergebnisse zur Exploration mit der Heuristik \emph{BL\_NMC} zeigen hingegen, dass im ersten Durchlauf für \emph{TEI7} lediglich 1051 Proxies erzeugt und evaluiert wurden. 