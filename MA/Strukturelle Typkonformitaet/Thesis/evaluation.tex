\chapter{Evaluierung}
Die Evaluierung erfolgt innerhalb von Systemen, in denen mindestens 889 angebotene Interfaces existieren. Es wird zwischen einem Test-System und einem Heiß-System unterschieden.\\\\
Das Test-System wurde vorrangig für die Evaluation der Type-Matcher Rating basierten Heuristiken verwendet, da für diese Heuristiken keine Implementierungen der angebotenen Interfaces vorliegen müssen.\\\\
Das Heiß-System wurde vorrangig für die Evaluation der testergebnis basierten Heuristiken verwendet, da hier zu jedem der 889 angebotenen Interfaces eine Implementierung existiert. Die angebotenen Komponenten wurden  im Heiß-System als Java Enterprise Beans umgesetzt.

\myparagraph{Darstellung der Evaluationsergebnisse}
Die Evaluationsergebnisse werden in der Form von Vier-Felder-Tafeln dargestellt (Beispiel siehe \tabref{vft:beispiel}). Für jedes erwartete Interface wird eine Vier-Felder-Tafel für jeden Durchlauf des Explorationsalgorithmus aufgezeigt. Aus der jeweiligen Tafel geht hervor, wie viele Kombinationen von Methoden-Konvertierungsvarianten aus den Kombinationen der ermittelten Typ-Konvertierungsvarianten innerhalb des Durchlaufs erzeugt werden könnten. Die Nummer des Durchlaufs wird in der oberen rechten Ecke der Tafel abgebildet. In der Spalte positiv ist die Anzahl der Kombinationen von Methoden-Konvertierungsvarianten verzeichnet, die innerhalb des Durchlaufs tatsächlich erzeugt wurden. Die Zahl in der Spalte ``negativ'' drückt hingegen aus, wie viele der Kombinationen aufgrund bestimmter Kriterien (bzw. Heuristiken) gar nicht erst erzeugt wurden.  Die Zeile ``falsch'' beschreibt die Anzahl der relevanten Kombinationen, aus denen benötigte Komponenten erzeugt werden, welche die semantischen Tests nicht bestehen. Dementsprechend stellt die Zeile ``richtig'' die Anzahl der Kombinationen dar, aus denen sich benötigte Komponenten erzeugen lassen, welche die semantischen Test bestehen. Der Fall, in dem eine Kombination nicht erzeugt wurde, aber dennoch für die Erstellung einer benötigten Komponente genutzt wurde und die semantischen Tests besteht, (negativ und richtig) kann nicht auftreten.\\\\
Für die Anzahl der zu kombinierenden Methoden-Konvertierungsvarianten $MK$ wird der höchste mögliche Wert angenommen. Dieser ist von der Anzahl der angebotenen Methoden $am$ sowie der Anzahl der erwarteten Methoden $em$ abhängig und wird wie folgt berechnet: 
\begin{equation*}
MK = \frac{am!}{(am-em)!*em!}
\end{equation*}
\noindent
Die Anzahl der angebotenen Methoden $am$ ist wiederum abhängig von den angebotenen Interfaces deren Typ-Konvertierungsvarianten im jeweiligen Durchlauf miteinander kombiniert wurden. Die Anzahl der Kombinationen von Typ-Konvertierungsvarianten innerhalb des Durchlaufs sei mit $TK$ beschrieben. Der Wert für $TK$ berechnet sich in Abhängigkeit von der Nummer des Durchlaufs $d$ und der Anzahl der strukturell passenden angebotenen Interfaces $n$ (siehe auch Abschnitt Explorationskomponente, 2. Stufe, 2. Kombination von Typ-Konvertierungsvarianten).
\begin{equation*}
TK = \frac{n!}{(n-d)!*d!}
\end{equation*}
\noindent
Da die Anzahl der angebotenen Methoden von System zu System schwanken kann, sei die Funktion $am(TK)$ eine näherungsweise Darstellung von $am$, in Abhängigkeit von der Anzahl der kombinierten Typ-Konvertierungsvarianten $TK$.\\\\
Da durch die Heuristiken letztendlich Methoden-Konvertierungsvarianten aus der Suche herausfallen, wird die Anzahl der entsprechenden Methoden-Konvertierungsvarianten in dem jeweiligen Feld der Vier-Felder-Tafeln als Funktion $mk(TK)$ dargestellt, die wie folgt definiert wird:
\begin{equation*}
mk(TK) = \frac{am(TK)!}{(am(TK)-em)!*em!}
\end{equation*}
\noindent
\tabref{vft:beispiel} zeigt ein Beispiel für eine solche Vier-Felder-Tafel, in der die Ergebnisse des 1. Durchlauf des Explorationsalgorithmus dargestellt sind. Dabei wurden Methoden-Konvertierungsvarianten aus 10 Kombinationen von Typ-Konvertierungsvarianten erzeugt. Den Methoden-Konvertierungsvarianten, die nicht beachtet wurden, lagen insgesamt 20 Typ-Konvertierungsvarianten  zugrunde. Weiterhin zeigt das Beispiel, dass es eine Kombination von Methoden-Konvertierungsvarianten gibt, aus der eine passende benötigte Komponente erzeugt werden konnte.
\vft{1}{$mk(10)$}{$mk(20)$}{1}{0}{Beispiel: Vier-Felder-Tafel}{vft:beispiel}


%\section{Test-System}
Wie bereits erw�hnt werden im Test-System die 889 angebotenen Interfaces verwendet, die auch im Hei�-System verwendet werden. Dar�ber hinaus wurden noch 6 weitere angebotene Interfaces dem Test-System hinzugef�gt, um bestimmte Konstellationen gezielter zu evaluieren. Damit stehen in diesem System f�r die Evaluation der insgesamt wurde 6 erwartete Interfaces f�r die Evaluation verwendet. Die 6 erwarteten Interfaces wurden wie folgt deklariert (siehe Abbildung 16 - 21).
\section{Ergebnisse für die Heuristik LMF}\label{sec_evalLMF}
In Bezug auf die Heuristik \emph{LMF} gilt es nicht nur zu evaluieren, ob die Suche nach einem Proxy, der die vordefinierten Tests besteht, beschleunigt werden kann, sondern auch, mit welcher Variante zur Bestimmung des Matcherratings (vgl. Abschnitt \ref{sec_lmf}) die besten Ergebnisse erzielt werden können. 
\\\\
Hierzu wird die Exploration für alle der oben genannten \emph{required Typen} für jede Variante zur Bestimmung der Matcherratings durchgeführt (siehe Abschnitt \ref{sec_lmf} Tabelle \ref{tab_matcherratingvarianten}). Im folgenden Verlauf wird lediglich auf die Variante eingegangen, die die besten Ergebnisse hervorgebracht hat. Die Ergebnisse unter Verwendung der übrigen Varianten sind im Anhang \ref{app_matcherratingEval} zu finden.
\\\\
Die Variante \emph{1.1} (vgl. Tabelle \ref{tab_matcherratingvarianten}) erbrachte die besten Ergebnisse. Die folgenden Vier-Felder-Tafeln zeigen die Ergebnisse mit dieser Variante zur Bestimmung der Matcherratings für die \emph{required Typen} \emph{TEI1}-\emph{TEI3} auf.
\begin{multicols}{3}
\vft{1}{5}{$p(44)-6$}{1}{0}{Ergebnisse \emph{LMF} mit Variante 1.1 für TEI1 \\1. Durchlauf}{lmf11_TEI1_1}
\vft{1}{1889}{$p(55)-1890$}{1}{0}{Ergebnisse \emph{LMF} mit Variante 1.1 für TEI2 1.~\mbox{Durchlauf}}{lmf11_TEI2_1}
\vft{1}{1463}{$p(50)-1464$}{1}{0}{Ergebnisse \emph{LMF} mit Variante 1.1 für TEI3 1.~\mbox{Durchlauf}}{lmf11_TEI3_1}
\end{multicols}
\noindent
Die Ergebnisse für die \emph{required Typen} \emph{TEI4}-\emph{TEI7} zeigen die folgenden Vier-Felder-Tafeln. 
\begin{multicols}{2}
\vft{1}{$1174$}{0}{0}{0}{Ergebnisse \emph{LMF} mit Variante 1.1 für TEI4 1.~\mbox{Durchlauf}}{lmf11_TEI4_1}
\vft{2}{2}{$p(2247)-3$}{1}{0}{Ergebnisse \emph{LMF} mit Variante 1.1 für TEI4 2.~\mbox{Durchlauf}}{lmf11_TEI4_2}
\end{multicols}

\begin{multicols}{2}
\vft{1}{$4984$}{0}{0}{0}{Ergebnisse \emph{LMF} mit Variante 1.1 für TEI5 1.~\mbox{Durchlauf}}{lmf11_TEI5_1}
\vft{2}{32}{$p(2775)-33$}{1}{0}{Ergebnisse \emph{LMF} mit Variante 1.1 für TEI5 2.~\mbox{Durchlauf}}{lmf11_TEI5_2}
\end{multicols}

\begin{multicols}{2}
\vft{1}{$1051$}{0}{0}{0}{Ergebnisse \emph{LMF} mit Variante 1.1 für TEI6 1.~\mbox{Durchlauf}}{lmf11_TEI6_1}
\vft{2}{0}{$p(1323)-1$}{1}{0}{Ergebnisse \emph{LMF} mit Variante 1.1 für TEI6 2.~\mbox{Durchlauf}}{lmf11_TEI6_2}
\end{multicols}

\begin{multicols}{2}
\vft{1}{$161294$}{0}{0}{0}{Ergebnisse \emph{LMF} mit Variante 1.1 für TEI7 1.~\mbox{Durchlauf}}{lmf11_TEI7_1}
\vft{2}{7641}{$p(52150)-7642$}{1}{0}{Ergebnisse \emph{LMF} mit Variante 1.1 für TEI7 2.~\mbox{Durchlauf}}{lmf11_TEI7_2}
\end{multicols}
\noindent
Folgendes kann aus diesen Ergebnissen abgeleitet werden:
\begin{enumerate}
\item Die Heuristik \emph{LMF} erzielt eine Reduktion der zu erzeugenden Proxies. Dies wird durch einen Vergleich der Spalte ``positiv'' innerhalb der Vier-Felder-Tafeln zum jeweiligen \emph{required Typ} belegt.

\item Die Heuristik \emph{LMF} hat keine Auswirkung auf einen Durchlauf, in dem kein Proxy erzeugt wird, mit dem die semantischen Tests erfolgreich durchgeführt werden können. Dies kann durch einen Vergleich des ersten Durchlaufs für die \emph{required Typen} \emph{TEI4}-\emph{TEI7} im Ausgangspunkt (Tabellen \ref{tab:tmr_start_tei4_1}, \ref{tab:tmr_start_tei5_1}, \ref{tab:tmr_start_tei6_1} und \ref{tab:tmr_start_tei6_1}) mit dem ersten Durchlauf unter Anwendung der Heuristik (Tabellen \ref{tab:lmf11_TEI4_1}, \ref{tab:lmf11_TEI5_1}, \ref{tab:lmf11_TEI6_1} und \ref{tab:lmf11_TEI7_1}) festgestellt werden.
\end{enumerate}



%Aus diesen Ergebnissen lässt sich folgendes ableiten:
%\begin{enumerate}
%\item Das Akkumulationsverfahren Nummer 3. (Minimum) führt sowohl für die Typ- und Methoden-Konvertierungsvarianten zu schlechteren Ergebnissen als die anderen drei Akkumulationsverfahren. Es sollte daher für die Heuristik TMR\_Quant nicht verwendet werden.
%\item Die Ergebnisse von 1-2 und 3-2 unterscheiden sich nur geringfügig, obwohl bei 3-2 das Akkumulationsverfahren Nummer 3. zum Einsatz kam. Dies konnte auch bei anderen Kombinationen festgestellt werden, bei denen das 3. Akkumulationsverfahren für die Akkumulation des Type-Matcher Ratings der Typ-Konvertierungsvariante verwendet wurde. Das lässt vermuten, dass die Beachtung des Type-Matcher Ratings einer ganzen Typ-Konvertierungsvariante weitgehend unerheblich für die Heuristik TMR\_Quant ist.
%, wenn das Type-Matcher Rating je Methoden-Konvertierungsvarianten über ein entsprechend gutes Akkumulationsverfahren ermittelt wurde. 
%Dies ist jedoch darauf zurückzuführen, dass das Type-Matcher Rating je Methoden-Konvertierungsvariante die Parameter für die Ermittlung des Type-Matcher Ratings einer Typ-Konvertierungsvariante darstellen.
%\item An den Ergebnissen zu den erwarteten Interfaces TEI4-TEI6 ist zu erkennen, dass die Heuristik TMR\_Quant keinen Einfluss auf den 1. Durchlauf hat. Daraus kann geschlussfolgert werden, dass die Heuristik nur in dem Durchlauf einen Gewinn bringt, in dem auch eine passende benötigte Komponente gefunden werden kann. 
%\end{enumerate}
%Aufgrund der Ergebnisse stehen für die weitere Verwendung der Heuristik TMR\_Qual mehrere Kombinationen von Akkumulationsverfahren zur Auswahl. Die Entscheidung fällt aufgrund der etwas geringeren Komplexität auf die Kombination 1-2. 

%\subsection{Hei�-System}