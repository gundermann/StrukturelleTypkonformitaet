\section{Ausblick}
Das Matcherrating hängt im Grunde genommen vom Basisrating der verwendeten Matchers abhängig. In dieser Arbeit wurde dieses Basisrating explizit angegeben. Allerdings sind die vorgestellten Matcher auf Typkonstellationen abgestimmt, die in vielen Programmiersprachen auftreten können. Insofern können diese Matcher als allgemeingültig betitelt werden. Vor diesem Hintergrund könnte das Basisrating eines Matchers implizit bestimmt werden. Dazu müssten die Verbindung, welche die Typen zueinander haben, quantifiziert werden. Darauf aufbauend könnten die von den Matchern adressierte Verbindung zwischen den Typen analysiert werden, um das Basisrating zu ermitteln.
\\\\
Weiterhin zeigen die Ergebnisse der Arbeit, dass die Exploration von EJBs zur Laufzeit grundsätzlich funktioniert. Dementsprechend wäre es interessant zu untersuchen, ob und wie dieser Ansatz in anderen Systemtypen wie bspw. Self-Contained-Systems funktioniert. Mitunter ergeben sich bei diesen Untersuchungen weitere Vorteile oder Probleme dieses Ansatzes.
\\\\
Darüber hinaus bieten die in Abschnitt \ref{sec_discApproach} aufgestellten Thesen bzgl. der höheren Verfügbarkeit (Abschnitt \ref{sec_sideeffects} und dem Nutzen des Ansatzes für den Entwickler im Verhältnis zu dessen Wissen über das System das Potential für weitere Untersuchungen. Der Nutzen des Ansatzes könnte dabei über eine Feldstudie in unterschiedlichen Unternehmen durchgeführt werden.
\\\\
Zuletzt sei noch erwähnt, dass die Heuristiken zwar im Rahmen der Exploration zur Laufzeit entworfen wurden. Sie können jedoch auch in bestehenden Source Engines wie Sourcerer oder CodeGenie integriert werden, um so den Nutzen der Heuristiken für diese Engines zu untersuchen.