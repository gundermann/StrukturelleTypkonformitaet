\usepackage[latin1]{inputenc}
\usepackage[pdftex]{color,graphicx}
\usepackage[hypertexnames=false]{hyperref}
\usepackage[german,ngerman]{babel}
\usepackage{fancyhdr}
\usepackage{amssymb}
\usepackage[pages=some]{background} % Draft Wasserzeichen mit Option pages=all sonst pages=some
\usepackage{amsmath}
\usepackage[rflt]{floatflt}
\usepackage{tabularx}
\usepackage{ausarbeitung}

%% Package-Erweiterungen
\usepackage{listings}
\usepackage{changepage}
\usepackage{amsmath}
\usepackage{multirow}
\usepackage{multicol}
\usepackage{setspace}
\usepackage{color}
\usepackage{colortbl}

%% Diese Farben werden f�r den Quelltext verwendet
\definecolor{srcblue}{rgb}{0,0,0.5}
\definecolor{srcgray}{rgb}{0.5,0.5,0.5}
\definecolor{srcred}{rgb}{0.5,0,0}

\definecolor{rot}{rgb}{1,0.3,0}
\definecolor{gelb}{rgb}{1,1,0}
\definecolor{gruen}{rgb}{0,1,0.4}

\graphicspath{{pics/}}

% --- Farbdefinitionen ----------------------------------------
\definecolor{rot}{rgb}{1,0.3,0}
\definecolor{gelb}{rgb}{1,1,0}
\definecolor{gruen}{rgb}{0,1,0.4}
\definecolor{darkblue}{rgb}{0.2,0.3,1}
\definecolor{lightblue}{rgb}{0.6,0.7,1}
\definecolor{white}{rgb}{1,1,1}
\definecolor{pblue}{rgb}{0.13,0.13,1}
\definecolor{pgreen}{rgb}{0,0.5,0}
\definecolor{pred}{rgb}{0.9,0,0}
\definecolor{pgrey}{rgb}{0.46,0.45,0.48}

\bibliographystyle{geralpha}

\usepackage[figuresright]{rotating}
\usepackage{geometry}
\geometry{a4paper,left=20mm,right=20mm} 

\newlength{\fullwidth} % Width of text plus margin notes
\setlength{\fullwidth}{\textwidth}


\lstset{language=Java,
  showspaces=false,
  showtabs=false,
  breaklines=true,
  showstringspaces=false,
  breakatwhitespace=true,
  commentstyle=\color{pgreen},
  keywordstyle=\color{pblue},
  stringstyle=\color{pred},
  basicstyle=\fontsize{9}{10}\selectfont\ttfamily,
  moredelim=[il][\textcolor{pgrey}]{$ $},
  moredelim=[is][\textcolor{pgrey}]{\%\%}{\%\%}
}

\usepackage{varwidth}
\newcommand\tabrotate[1]{\begin{turn}{90}\rlap{#1}\end{turn}}
\newcommand\tabvarwidth[2][3cm]{\begin{varwidth}[b]{#1}\centering #2\end{varwidth}}
\newcommand{\myparagraph}[1]{\paragraph{#1}\mbox{}\\}
\newcommand{\mysubparagraph}[1]{\underline{#1}\mbox{}\\}



%----------------------------------------------------------------------------------
% \backmatter	{ PAGE_COUNTER }
%
% Setzt die Umgebungen fuer den Teil nach dem Hauptteil
\newcommand{\backmatter}[1]{
%  \thispagestyle{empty}
  \fancyhf{}
  \lhead{\rightmark}
  \cfoot{\thepage}
  \pagenumbering{roman}
  \setcounter{page}{#1}
}

%----------------------------------------------------------------------------------
% \vft		{ NUMBER }
%			{ UPPER_LEFT }
%			{ UPPER_RIGHT }
%			{ LOWER_LEFT }
%			{ LOWER_RIGHT }
%			{ CAPTION }
%			{ LABEL }
%
% Gleichung, die ein Matching darstellt
\newcommand{\vft}[7]{
\begin{table}[H]
\centering
\small
\doublespacing
\begin{tabular}[c]{|c|c|c|}
\hline
#1 & \cellcolor{gruen}\textbf{positiv} & \cellcolor{rot}\textbf{negativ} \\
\hline
\cellcolor{rot}&\cellcolor{rot}&\cellcolor{rot}\\
\tabrotate{\cellcolor{rot}\textbf{falsch}} & \multirow{-2}{*}{\cellcolor{rot}#2}&\multirow{-2}{*}{\cellcolor{rot}#3}\\
\hline
\cellcolor{gruen}&\cellcolor{gruen}&\cellcolor{rot}\\
\tabrotate{\cellcolor{gruen}richtig} &\multirow{-2}{*}{\cellcolor{gruen}#4}&\multirow{-2}{*}{\cellcolor{rot}#5} \\
\hline
\end{tabular}
\singlespacing
\caption{#6}
 \label{tab:#7}
\end{table}
}
%----------------------------------------------------------------------------------
% \matchTyp		{ LEFT }
%				{ MATCHERVAR }
%				{ RIGHT }
%
% Gleichung, die ein Matching darstellt
\newcommand{\matchTyp}[3]
{
#1 \equiv_{#2} #3
}

%----------------------------------------------------------------------------------
% \inhTyp		{ CHILD }
%				{ PARENT }
%
% Gleichung, die eine Vererbung darstellt
\newcommand{\inhTyp}[2]
{
#1 < #2
}

%----------------------------------------------------------------------------------
% \selTyp		{ WRAPPER }
%				{ ATTR }
%
% Gleichung, die eine Selektion darstellt
\newcommand{\selTyp}[2]
{
#1 \# #2
}

%----------------------------------------------------------------------------------
% \delegate		{ SOURCE }
%				{ TARGET }
%
% Gleichung, die eine Delegation darstellt
\newcommand{\delegate}[2]
{
#1 \Rightarrow #2
}

%----------------------------------------------------------------------------------
% \applyMatcher		{ SOURCE }
%					{ TARGET }
%
% Gleichung, die Applikation eines Matchers auf einen Parameter 
\newcommand{\applyMatcher}[2]
{
(#1)#2
}

%----------------------------------------------------------------------------------
% matcherEquivDef		{ MATCHER }
%						
%
% Umgebung f�r die Definition der �bereinstimmung eines Matchers
\newenvironment{matcherEquivDef}[1]{
\begin{adjustwidth}{2cm}{0cm}
\underline{�bereinstimmung (#1)}
\end{adjustwidth}
\begin{eqnarray*}
}
{
\end{eqnarray*}
}

%----------------------------------------------------------------------------------
% matcherConvDef		{ MATCHER }
%						{ ANNAHME }
%
% Umgebung f�r die Definition der �bereinstimmung eines Matchers
\newenvironment{matcherConvDef}[2]{
\begin{adjustwidth}{2cm}{0cm}
\underline{Konvertierung (#1)}\newline
#2
\end{adjustwidth}
\begin{eqnarray*}
}
{
\end{eqnarray*}
}

%----------------------------------------------------------------------------------
% \myFigure	[ LABEL_PREFIX (optional) ]
%				{ FILENAME (without extension) }
%				{ CAPTION TEXT }
%				{ SHORT VERSION OF CAPTION TEXT }
%
%Bild wird in Originalgroesse gesetzt
%picture using full width of the page
\newcommand{\myFigure}[3]
{
\begin{figure}[H]
	\center{
	\begin{minipage}{\fullwidth}
	\center{
		\includegraphics{#1}
		\caption{#2}
		\label{abb:#3}
		}
	\end{minipage}
	}
\end{figure}
}


%----------------------------------------------------------------------------------
% \myScalableFigure	[ scale (optional) ]
%				{ FILENAME (without extension) }
%				{ CAPTION TEXT }
%				{ SHORT VERSION OF CAPTION TEXT }
%
%Bild wird in Originalgroesse gesetzt
%picture using full width of the page
\newcommand{\myScalableFigure}[4][0.5\linewidth]
{
\begin{figure}[H]
	\center{
	\begin{minipage}{\fullwidth}
	\center{
		\includegraphics[width=#1]{#2}
		\caption{#3}
		\label{abb:#4}
		}
	\end{minipage}
	}
\end{figure}
}

%----------------------------------------------------------------------------------
% \myBigFigure	[ WIDTH (optional) ]
%				{ FILENAME (without extension) }
%				{ CAPTION TEXT }
%				{ SHORT VERSION OF CAPTION TEXT }
%
%Bild wird in kompletter Breite gesetzt
%picture using full width of the page
\newcommand{\myBigFigure}[4][\fullwidth]
{
\begin{figure}[H]
	\center{
	\begin{minipage}{#1}
		\includegraphics[width=#1]{#2}
		\caption{#3}
		\label{abb:#4}
	\end{minipage}
	}
\end{figure}
}


%----------------------------------------------------------------------------------
% \myNeighbourFigures	NOT WORKING!!!!
%				{ PROPORTION OF 1  }
%				{ FILENAME OF 1 (without extension) }
%				{ CAPTION TEXT OF 1 }
%				{ REF OF 1}
%   			[ PROPORTION OF DISTANCE (optional)]
%				{ PROPORTION OF 1  }
%				{ FILENAME OF 1 (without extension) }
%				{ CAPTION TEXT OF 1 }
%				{ REF OF 1}
%
%Zwei Bilder nebeneinander
%\newcommand{\myNeighbourFigures}[9]
%{
%\begin{figure}[H]
%\begin{minipage}[b]{#1\linewidth}
%  \centering
%  \includegraphics[width=\linewidth]{#2}
%  \captionof{figure}{#3}
%  \label{abb:#4}%

%\end{minipage}%
%\hspace{#5\linewidth}% Abstand zwischen Bilder
%\begin{minipage}[b]{(#6\linewidth}


 % \centering
 % \includegraphics[width=\linewidth]{#7}
 % \captionof{figure}{#8}
 % \label{abb:#9}

%\end{minipage}
%\end{figure}

%}


%----------------------------------------------------------------------------------
% \myBigFigureGraphic	
%						[ PARAMS (optional) ]
%				{ FILENAME (without extension) }
%				{ CAPTION TEXT }
%				{ SHORT VERSION OF CAPTION TEXT }
%
%Bild wird in kompletter Breite gesetzt
%picture using full width of the page
\newcommand{\myBigFigureGraphic}[5][width=\fullwidth]
{
\begin{figure}[H]
	\center
	\includegraphics[#1]{#2}
	\caption{#3}
	\label{abb:#4}
\end{figure}
}



%----------------------------------------------------------------------------------
% \myBigFigure	[ LABEL_PREFIX (optional) ]
%				{ FILENAME (without extension) }
%				{ CAPTION TEXT }
%				{ SHORT VERSION OF CAPTION TEXT }
%
%Bild wird in kompletter Breite gesetzt
%picture using full width of the page
\newcommand{\myBigFigureCited}[5][abb:]
{
\begin{figure}[H]
	\center{
	\begin{minipage}{\fullwidth}
		\includegraphics[width= \fullwidth]{#2}
		\caption[#3]{#3#4}
		\label{#1_#5}
	\end{minipage}
	}
\end{figure}
}


%----------------------------------------------------------------------------------
% \dcite	{ Text }
%				{ source }
%				{ page }
%
%Direktes Zitat
\newcommand{\dcite}[3]
{
\emph{\glqq#1\grqq}\cite[S.#3]{#2}}

%----------------------------------------------------------------------------------
% \dcite	{ Text }
%				{ source }
%				{ page }
%
%Direktes Zitat
\newcommand{\simpledcite}[2]
{
\emph{\glqq#1\grqq}\cite{#2}}


%----------------------------------------------------------------------------------
% \vcite	
%				{ source }
%				{ page }
%
%Direktes Zitat
\newcommand{\vcite}[2]
{
(vgl. \cite[S.#2]{#1})}


%----------------------------------------------------------------------------------
% \vcite	
%				{ source }
%				{ page }
%
%Direktes Zitat
\newcommand{\simplevcite}[1]
{
(vgl. \cite{#1})}

%----------------------------------------------------------------------------------
% \myHUGEFigure	[ LABEL_PREFIX (optional) ]
%				{ FILENAME (without extension) }
%				{ CAPTION TEXT }
%				{ SHORT VERSION OF CAPTION TEXT }
%
%Bild wird rotiert und quer in kompletter Breite gesetzt
%landscape picture using the full width of the rotated page
\newcommand{\myHugeFigure}[4][abb:]
{
\begin{sidewaysfigure}[H]
	
		\includegraphics[width= \textheight]{#2}
		\caption{#3}
		\label{#1_#4}
	
\end{sidewaysfigure}
}


%----------------------------------------------------------------------------------
% \myHUGEFigure	[ LABEL_PREFIX (optional) ]
%				{ FILENAME (without extension) }
%				{ CAPTION TEXT }
%				{ SHORT VERSION OF CAPTION TEXT }
%
%Bild wird rotiert und quer in kompletter Breite gesetzt
%landscape picture using the full width of the rotated page
\newcommand{\myHugeFigureCited}[5][abb:]
{
\begin{sidewaysfigure}[t!bp]
	
		\includegraphics[width= \textheight]{#2}
		\caption[#3]{#3#4}
		\label{#1_#5}
	
\end{sidewaysfigure}
}


%-----------------------------------------------------------------------
% \abbref		{ PIC REFERENCE }
%
%Verweis auf eine Abbildung
\newcommand{\abbref}[1]{Abbildung \ref{abb:#1}}



%-----------------------------------------------------------------------
% \tabref		{ PIC REFERENCE }
%
%Verweis auf eine Tabelle
\newcommand{\tabref}[1]{Tabelle \ref{tab:#1}}

%-----------------------------------------------------------------------
% \tabsrefs		{ TAB REFERENCE_START }
%				{ TAB REFERENCE_END }
%
%Verweis auf eine Tabelle
\newcommand{\tabsrefs}[2]{Tabellen \ref{tab:#1}-\ref{tab:#2}}

%-----------------------------------------------------------------------
% \abbsrefs		{ PIC REFERENCE_START }
%				{ PIC REFERENCE_END }
%
%Verweis auf eine Tabelle
\newcommand{\abbsrefs}[2]{Abbildungen \ref{abb:#1}-\ref{abb:#2}}




%-----------------------------------------------------------------------
% \lstref		{ PIC REFERENCE }
%
%Verweis auf ein Listing
\newcommand{\lstref}[1]{Listing \ref{#1}}


%-----------------------------------------------------------------------
% \lstsrefs		{ LST REFERENCE Begin }
%				{ LST REFERENCE End }
%
%Verweis auf mehrere Listings
\newcommand{\lstsrefs}[2]{Listings \ref{#1}-\ref{#2}}

%-----------------------------------------------------------------------
% \appref		{ APPENDIX REFERENCE }
%
%Verweis auf ein Anhang
\newcommand{\appref}[1]{Anhang \ref{#1}}

\newcommand{\gloss}[1]{\emph{\gls{#1}}}

\newcommand{\glossLink}[2]{\emph{\glslink{#1}{#2}}}


