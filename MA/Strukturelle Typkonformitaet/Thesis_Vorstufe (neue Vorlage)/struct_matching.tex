\subsection{1. Stufe - Strukturelle �bereinstimmung}
Wie in \cite{hummel08} wird in der ersten Stufe der Suche versucht die angebotenen Interfaces herauszusuchen, die strukturell mit dem erwarteten Interface �bereinstimmen. Zu diesem Zweck wird ein Type-Matcher verwendet, der in Abschnitt \ref{structTypeMatcher} beschrieben wird. Dar�ber hinaus werden weitere Type-Matcher verwendet (siehe Abschnitte \refs{exactTypeMatcher}{wrapperTypeMatcher}), die das Matching zweier Typen auf der Basis der Beziehung, in der diese beiden Typen zueinander stehen, feststellen. Allgemein beschrieben, kann durch jeden dieser Type-Matcher festgestellt werden, ob sich ein Typ in einen anderen Typ konvertieren l�sst.\\\\
Die Konvertierung erfolgt zur Laufzeit �ber Proxies, die ihre Methodenaufrufe delegieren. So wird bspw. bei der Konvertierung eines Objektes von TypA in ein Objekt von TypB ein Proxy-Objekt f�r TypB erzeugt, welches die Methodenaufrufe auf dem Objekt von TypA delegiert (vgl. \abbref{combinated_components}).\\\\
Hummel hatte hierzu bereits auf einige Matcher von Zaremski und Wing \cite{moormann} zur�ckgegriffen, die in dieser Arbeit ebenfalls zum Einsatz kommen (siehe Abschnitte \refs{exactTypeMatcher}{specTypeMatcher}). Weiterhin wurde in \cite{hummel08} ein Anwendungsfall f�r einen Matcher skizziert, der in der Lage ist Wrapper-Typen zu Matcher. Aus diese Idee wird in den Abschnitten \ref{wrappedTypeMatcher} und \ref{wrapperTypeMatcher} zur�ckgegriffen. Die Definitionen der Matcher beziehen sich vorrangig auf die Programmiersprache Java, weshalb grundlegend von einer nominalen Typkonformit�t auszugehen ist.


\subsubsection{SpecTypeMatcher}\label{specTypeMatcher}
\myparagraph{Szenario}
Analog zum GenTypeMatcher stellt der SpecTypeMatcher ebenfalls das Matching zwischen Typen fest, die in einer Vererbungsbeziehung stehen. Allerdings ist der Source-Typ in diesem Matcher der Subtyp und der Target-Typ der Supertyp. In dem Szenario wird wiederum von den Klassen SuperClass und SubClass aus \abbref{cd_subclass_extends_superclass} ausgegangen. Der Methodenaufruf erfolgt hier aber auf dem Subtypen und wird an den Supertypen delegiert (siehe \abbref{sd_spec_sub2super}).
\myScalableFigure[0.7\linewidth]{sd_spec_sub2super}{Szenario SpecTypeMatcher}{sd_spec_sub2super}
\noindent
Dabei sind zwei Methodenaufrufe auf dem Subtyp beschrieben. W�hrend der Aufruf der Methode getString erfolgreich delegiert werden kann, f�hrt der Aufruf der Methode getStringWithoutPrefix zu einem Laufzeitfehler, da der Matcher keine passende Methode in dem Target-Typ ermitteln kann. Dieses Problem tritt bei allen Methoden auf, die nicht vom Supertyp an den Subtyp vererbt oder �berschrieben wurden.\footnote{Anders gesagt, erm�glicht dieser Matcher einen Downcast, bei dem ein Objekt eines allgemeinen Typen auf einen spezielleren Typen gecastet wird. Das Problem bzgl. des fehlschlagenden Methodenaufrufs in der beschriebene Form ist bei einem Downcast allgegenw�rtig.} Aus diesem Grund muss diese Bedingung in der Definition der Konvertierung dieses Matchers mit aufgenommen werden.
\myparagraph{Definition}
\begin{matcherEquivDef}{SpecTypeMatcher}
\matchTyp{A}{genspec}{B} \text{ wenn } \inhTyp{A}{B}
\end{matcherEquivDef}
\begin{matcherConvDef}{SpecTypeMatcher}{
Sei $ m $ eine Methode des Typs $ A $, die von $ B $ an $ A $ vererbt wurde.}
\delegate{A.m}{B.m}
\end{matcherConvDef}\\
Ein Beispiel f�r die Verwendung des Matchers ist in \appref{specMatcherExample} zu finden.
\subsubsection{WrappedTypeMatcher}\label{wrappedTypeMatcher}
\myparagraph{Szenario}
Die bisherigen Type-Matcher sind in der Lage das Matching f�r zwei Typen festzustellen, ohne daf�r R�cksicht auf deren innere Struktur nehmen zu m�ssen. Dies ist f�r identische oder hierarchisch organisierte Typen auch nicht notwendig. Es ist jedoch auch denkbar, dass sich beiden Typen auf anderem Wegen assoziieren lassen. Ein Beispiel daf�r w�re Boxed- bzw. - noch allgemeiner gefasst - Wrapper-Typen. In \abbref{cd_subclass_subwrapper} sind zwei Klassen dargestellt, die in einer solchen Beziehung zueinander stehen. Bez�glich des Matchings sind auch hier wiederum zwei F�lle zu unterscheiden. Der erste Fall, in dem das Matching des Source-Typen SubClass mit dem Typen eines Attributs wrapped des Traget-Typen SubWrapper festgestellt werden kann, ist in \abbref{sd_wrapped_sub2subwrapped} dargestellt.
\begin{figure}[H]
\begin{minipage}[b]{.33\linewidth}
  \centering
  \includegraphics[width=\linewidth]{cd_subclass_subwrapper}
  \caption{Beziehung zwischen SubClass und SubWrapper}
  \label{abb:cd_subclass_subwrapper}

\end{minipage}%
\hspace{.04\linewidth}% Abstand zwischen Bilder
\begin{minipage}[b]{.63\linewidth}


  \centering
  \includegraphics[width=\linewidth]{sd_wrapped_sub2subwrapped}
  \caption{Szenario WrappedTypeMatcher}
  \label{abb:sd_wrapped_sub2subwrapped}

\end{minipage}
\end{figure}
\noindent
Der WrappedTypeMatcher stellt das Matching f�r ein solches Szenario fest. Das Matching der beiden Typen beruht letztendlich auf einem Matching zwischen dem Source-Type und dem Typen eines Attributs des Target-Typs. Der Matcher, �ber den dieses Matching innerhalb des WrappedTypeMatchers festgestellt wird, wird als interner Matcher bezeichnet. In dem Szenario aus \abbref{sd_wrapped_sub2subwrapped} wird als interner Matcher der bereits beschriebene ExactTypeMatcher verwendet, weil der Source-Type und der Typ des Attributs wrapped identisch sind.
\myparagraph{Definition}
\begin{matcherEquivDef}{WrappedTypeMatcher}
\matchTyp{A}{wrapped}{B} \text{ wenn } \exists \selTyp{B}{attr} : \matchTyp{A}{M}{attr}
\end{matcherEquivDef}\\
Der zuvor genannte interne Matcher wird in der Definition mit $M$ beschrieben, was stellvertretend f�r eine Menge von Matchern steht. Als interne Matcher kommen hierbei der ExactTypeMatcher, der GenTypeMatcher und der SpecTypeMatcher in Frage.\\
\begin{matcherConvDef}{WrappedTypeMatcher}{
Sei $m$ eine Methode des Typs $A$. Sei weiterhin $B$ ein Typ, der ein Attribut vom Typ $attr$ enth�lt, f�r den gilt $\matchTyp{A}{M}{attr}$.
}
\delegate{A.m}{(\applyMatcher{A}{attr}).m}
\end{matcherConvDef}\\
Ein Beispiel f�r die Verwendung des Matchers in Bezug auf das o.g. Szenario ist in \appref{wrappedMatcherExample} zu finden. Au�erdem sind dort auch weitere Szenarien aufgef�ht, in denen der GenTypeMatcher oder der SpecTypeMatcher als interner Matcher zur Anwendung kommen.


\subsubsection{WrapperTypeMatcher}\label{wrapperTypeMatcher}
\myparagraph{Szenario}
Dieser Matcher stellt das Pendant zum WrappedTypeMatcher dar. Der Unterschied bzgl. des Szenarios besteht darin, dass nun der Source-Typ derjenige ist, der ein Attribut enth�lt, f�r dessen Typ ein Matching zum Target-Typen �ber den ExactTypeMatcher, den GenTypeMatcher oder den SpecTypeMatcher festgestellt werden kann. F�r das Szenario ist wiederum von den Typen aus \abbref{cd_subclass_subwrapper} auszugehen. Die Delegation der m�glichen Methodenaufrufe am Source-Typen, sind in \abbref{sd_wrapper_subwrapped2sub} abgebildet. Hierbei ist hervorzuheben, dass zur Laufzeit das Objekt vom Target-Typen in das Attribut des Objektes vom Source-Typen injiziert wird. Dies soll in \abbref{sd_wrapper_subwrapped2sub} durch die Bezeichnung des Targets mit wrapped (dem Namen des Attributs) und target dargestellt werden. Eine Methoden-Delegation findet nur dann statt, wenn sie auch im Wrapper-Typen (Source-Typen) implementiert wurde\footnote{Implementierung von SubWrapper: siehe \appref{matcherExamples} \lstref{LST_subwrapper_impl}}.
\myScalableFigure[0.8\linewidth]{sd_wrapper_subwrapped2sub}{Szenario WrapperTypeMatcher}{sd_wrapper_subwrapped2sub}
\myparagraph{Definition}
\begin{matcherEquivDef}{WrapperTypeMatcher}
\matchTyp{A}{wrapper}{B} \text{ wenn } \exists\selTyp{A}{attr} : \matchTyp{B}{M}{attr} 
\end{matcherEquivDef}\\
Wie an dieser Beschreibung zu erkennen ist, werden auch hier wieder ein interner Matcher $M$ verwendet. Analog zum WrappedTypeMatcher kommen auch hier der ExactTypeMatcher, der GenTypeMatcher und der SpecTypeMatcher in Frage.\\
\begin{matcherConvDef}{WrappedTypeMatcher}{
Sei $m$ eine Methode des Typs $A$.
}
\delegate{A.m}{A.m}
\end{matcherConvDef}\\
Ein Beispiel f�r die Verwendung des Matchers in Bezug auf das o.g. Szenario ist in \appref{wrapperMatcherExample} zu finden. Au�erdem sind dort auch weitere Szenarien aufgef�ht, in denen der GenTypeMatcher oder der SpecTypeMatcher als interner Matcher zur Anwendung kommen.

\subsubsection{StructuralTypeMatcher}\label{structTypeMatcher}
Die bisher beschriebene Type-Matcher erlauben lediglich ein Matching zwischen Typen, die syntaktisch miteinander in einer direkten Beziehung stehen. Ein Ziel dieser Arbeit ist es jedoch Typen, die voneinander syntaktisch unabh�ngig sind, miteinander zu matchen, um darauf aufbauend, deren Semantik zu �berpr�fen. Hierf�r soll wie auch in \cite{hummel08} die strukturelle �bereinstimmung der beiden Typen ermittelt und verwendet werden. Diesem Zweck dient der StructuralTypeMatcher.
\myparagraph{Szenario}
Um die grundlegenden Eigenschaften des StructuralTypeMatchers darzustellen, wird von einem Szenario ausgegangen, in dem der Target-Typ (angebotenes Interface) zu jeder Methode des Sources-Typs (ben�tigtes Interface) eine passende Methode anbietet. Eine Kombination von angebotenen Interfaces ist somit in diesem Szenario nicht notwendig.\\\\
\abbref{cd_superrsubp_subrsuperp} zeigt die Typen, von denen in dem folgenden Szenario ausgegangen wird. Hierbei sind zwei Klassen aufgef�hrt, die jeweils zwei Methoden anbieten. Die Parameter- und R�ckgabetypen der Methoden sind aus den Szenarien zu den anderen Matchern bekannt. Die Klasse SuperReturnSubParamClass wird in dem folgenden Szenario als Source-Typ und die Klasse SubReturnSuperWrapperParamClass wird als Target-Typ verwendet. Um die strukturelle �bereinstimmung der beiden Typen festzustellen, muss der StructuralTypeMatcher ein Matching zwischen den Parameter- und R�ckgabetypen der einzelnen Methoden herstellen. Dies erfolgt wiederum �ber interne Type-Matcher. An dieser Stelle k�nnen alle zuvor genannten Type-Matcher als interner Type-Matcher verwendet werden. Die Delegation der Methode-Aufrufe erfolgt dann an die Methode des Target-Objekts, die als �bereinstimmende bzw. passende Methode ermittelt wurde (siehe \abbref{sd_struct}). Da beide Methoden eine unterschiedliche Anzahl von Parametern haben, ist in diesem Beispiel leicht nachzuvollziehen, welche Methoden zusammenpassen. Als interner Type-Matcher wurde in diesem Szenario der GenTypeMatcher verwendet. 
\myBigFigure{cd_superrsubp_subrsuperp}{SuperReturnSubParamClass und SubReturnSuperParamClass}{cd_superrsubp_subrsuperp}
\myBigFigure{sd_struct}{Szenario StructTypeMatcher}{sd_struct}
\myparagraph{Definition}
\begin{matcherEquivDef}{StructuralTypeMatcher}
&\matchTyp{A}{struct}{B} \text{ wenn}\\
&\exists(A.m(MP) : MR) : \exists (B.n(NP):NR):\matchTyp{MP}{P}{NP} \wedge \matchTyp{NR}{R}{MR}
\end{matcherEquivDef}\\
Da die Notation es nicht hergibt, ist zus�tzlich zu erw�hnen, dass die Reihenfolge der Parameter in $m$ und $n$ irrelevant ist.\\
%fuer Formatierung
\\
\begin{matcherConvDef}{StructuralTypeMatcher}{
Sei $m$ eine Methode des Typs $A$.\\
Der R�ckgabetyp von $m$ sei $MR$ und $MP$ der Parametertyp von $m$.\\
Weiterhin sei $n$ eine Methode des Typs $B$.\\
Der R�ckgabetyp von $n$ sei $NR$ und $NP$ der Parametertyp von $n$.
}
\delegate{A.m(MP):MR}{B.n(\applyMatcher{NP}{MP}) : \applyMatcher{MR}{NR}}
\end{matcherConvDef}\\
Ein Beispiel f�r die Verwendung des Matchers in Bezug auf das o.g. Szenario ist in \appref{structMatcherExample} zu finden. Au�erdem sind dort auch weitere Szenarien aufgef�ht, in denen andere Matcher als interner Matcher zur Anwendung kommen.

\subsubsection*{Typ-Konvertierungsvariante}
Die Konvertierung der einzelnen Type-Matcher liefert eine Menge von so genannten Typ-Konvertierungsvarianten. Eine Typ-Konvertierungsvariante beschreibt eine M�glichkeit, wie ein Typ in einen anderen konvertiert werden kann. Zu diesem Zweck enth�lt eine Typ-Konvertierungsvariante zwei Arten von Information: 
\begin{enumerate}
\item Objekterzeugungsrelevante Informationen
\item Methodendelegationsrelevante Informationen
\end{enumerate}
Typ-Konvertierungsvarianten werden von einem konkreten Typ-Matcher f�r jede m�gliche Form der �bereinstimmung erzeugt. Im speziellen Fall des ExactTypeMatchers und des SpecGenTypeMatcher kann, wenn �berhaupt, nur eine Typ-Konvertierungsvariante erzeugt werden. Da die anderen Typ-Matcher mit internen Type-Matchern arbeiten, sind von diesen mehrere Typ-Konvertierungsvarianten zu erwarten.\\\\
Die objekterzeugungsrelevanten Informationen sorgen daf�r, dass das Proxy-Objekt zum Source-Typ korrekt erzeugt werden kann.\\\\
Die methodendelegationsrelevanten Informationen sorgen daf�r, dass das R�ckgabe-Objekt und die Parameter-Objekte beim Methodenaufruf korrekt konvertiert werden und dass der Aufruf an die richtige Methode des Target-Typs delegiert wird. Daher wird eine methodendelegationsrelevante Information auch als Methoden-Konvertierungsvariante bezeichnet.\\\\
In \abbref{konvar_voll} ist dieser Zusammenhang f�r ein angebotenes Interface AIv und einem erwarteten Interface EI, welche jeweils zwei Methoden enthalten (AM * bzw. EM *), skizziert. Hier wird angenommen, dass jeder der angebotenen Methoden strukturell zu jeder der erwarteten Methoden passen w�rde. Dementsprechend enth�lt die Typ-Konvertierungsvariante (TKV) insgesamt 4 Methoden-Konvertierungsvarianten, wovon jede eine Konvertierung entlang der eingezeichneten Pfeile erm�glicht.
\myFigure{konvar_voll}{Typ- und Methoden-Konvertierungsvarianten von AIv}{konvar_voll}
\noindent
Dabei gilt jedoch, aufgrund der �berlegungen zur Kombination von angebotenen Komponenten (siehe auch \abbref{combinated_components}), dass eine Typ-Konvertierungsvariante nicht zu jeder der erwarteten Methoden solche methodendelegationsrelevanten Informationen enth�lt. \abbref{konvar_unv} zeigt einen solchen Fall mit dem angebotenen Interface AIu.\myFigure{konvar_unv}{Typ- und Methoden-Konvertierungsvarianten von AIu}{konvar_unv}
