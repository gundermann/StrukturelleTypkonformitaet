\section{Einleitung}
\subsection{Motivation}
In gr��eren Software-Systemen ist es �blich, dass mehrere Komponenten miteinander �ber Schnittstellen kommunizieren. In der Regel werden diese Schnittstellen so konzipiert, dass sie Informationen oder Services anbieten, die von anderen Komponenten abgefragt und benutzt werden k�nnen. Dabei wird zwischen der Komponente, welche die Schnittstelle implementiert - als angebotene Komponente - und der Komponente, welche die Schnittstelle nutzen soll - als nachfragende Komponente - unterschieden (siehe \abbref{motiv}). 
\myScalableFigure[0.6\linewidth]{motiv}{Abh�ngigkeiten von nachfragenden und angebotenen Komponenten}{motiv}
\noindent
Wird von einer nachfragenden Komponente eine Information ben�tigt, die in dieser Form noch nicht angeboten wird, so wird h�ufig ein neues Interface f�r diese ben�tigte Information erstellt, welches dann passend dazu implementiert wird. Dabei muss neben der Anpassung der nachfragenden Komponente auch eine Anpassung oder Erzeugung der anbietenden Komponente erfolgen und zus�tzlich das neue Interface deklariert werden. Zudem bedingt eine nachtr�gliche �nderung der neuen Schnittstelle ebenfalls eine Anpassung der drei genannten Artefakte.\\\\
In einem gro�en Software-System mit einer Vielzahl von bestehenden Schnittstellen ist eine gewisse Wahrscheinlichkeit gegeben, dass die Informationen oder Services, die von einer neuen nachfragenden Komponente ben�tigt werden, in einer �hnlichen Form bereits existieren. Das Problem ist jedoch, dass die manuelle Evaluation der Schnittstellen mitunter sehr aufwendig bis, aufgrund von unzureichender Dokumentation und Kenntnis �ber die bestehenden Schnittstellen, unm�glich ist.\\\\
Weiterhin ist es denkbar, dass ein Software-System auf unterschiedlichen Maschinen verteilt wurde und dadurch Teile des Systems ausfallen k�nnen. Das hat zur Folge, dass die Implementierung bestimmter Schnittstellen nicht erreichbar ist. Dadurch, dass eine Schnittstelle durch eine nachfragende Komponente explizit referenziert wird, kann eine solche Komponente nicht korrekt arbeiten, wenn die Implementierung der Schnittstelle nicht erreichbar ist, obwohl die ben�tigten Informationen und Services vielleicht durch andere Schnittstellen, deren Implementierung durchaus zur Verf�gung stehen, bereitgestellt werden k�nnten.\\\\
Dies f�hrt zu der �berlegung, ob es nicht m�glich ist, dass eine nachfragende Komponente einfach selbst spezifizieren kann, welche Informationen oder Services sie erwartet, wodurch auf der Basis dieser Spezifikation eine passende anbietende Komponente gefunden werden kann.

\section{Verwandte Arbeiten}
Ein solcher Ansatz wurde bereits in \cite{sourcerer} von Bajaracharya et al.  verfolgt. Diese Gruppe entwickelte eine Search Engine namens Sourcerer, welche Suche von Open Source Code im Internet erm�glichte. Darauf aufbauend wurde von derselben Gruppe in \cite{Lemos} ein Tool namens CodeGenie entwickelt, welches einem Softwareentwickler die Code Suche �ber ein Eclipse-Plugin erm�glicht. In diesem Zusammenhang wurde erstmals der Begriff der Test-Driven Code Search (TDCS) etabliert. Parallel dazu wurde in Verbindung mit der Dissertation Oliver Hummel \cite{hummel08} ebenfalls eine Weiterentwicklung von Sourcerer ver�ffentlicht, welche unter dem Namen Merobase bekannt ist, welches ebenfalls das Konzept der TDCS verfolgt. TDCS beruht grundlegend darauf, dass der Entwickler Testf�lle spezifiziert, die im Anschluss verwendet werden, um relevanten Source Code aus einem Repository hinsichtlich dieser Testf�lle zu evaluieren. Damit kann das jeweilige Tool dem Entwickler Vorschl�ge f�r die Wiederverwendung bestehenden Codes unterbreiten.\\\\
Bezogen auf die am Ende des vorherigen Abschnitts formulierte �berlegung erm�glichen die genannten Search Engines, das Internet nach bestehendem Source Code zu durchsuchen und damit bereits bestehende Implementierungen f�r eine nachfragende Komponente zu ermitteln. 
\subsection{Gegenstand dieser Arbeit}
In dieser Arbeit soll jedoch nicht das gesamte Internet als Quelle oder Repository f�r die Codesuche dienen. Vielmehr wird der Suchbereich weiter eingeschr�nkt. \\\\
Es wird von einem System ausgegangen, in dem ein EJB-Container zur Verf�gung steht. Die Suche soll sich auf die Menge der angemeldeten Bean-Implementierungen beschr�nken. Die angemeldeten Bean-Implementierungen stellen damit die Menge der angebotenen Komponenten dar. Dabei wird eine angebotenen Komponente als Kombination eines Interfaces, welches die Schnittstelle f�r die Aufrufer definiert, und einer Implementierung dieses Interfaces beschrieben. Das Interfaces einer angebotenen Komponente wird im Folgenden auch als angebotenes Interfaces bezeichnet. Die Beans werden bspw. als Provider f�r Informationen oder im weitesten Sinne  auch als Services verwenden, die von unterschiedlichen Komponenten des Systems verwendet werden. Bei der Entwicklung bzw. Weiterentwicklung einer Komponente kann es zu folgendem Szenario kommen, welches durch die unten aufgef�hrten Annahmen charakterisiert wird:
\begin{itemize}
\item Es werden Informationen und Services ben�tigt, bei denen der Entwickler davon ausgehen kann, dass es innerhalb des Systems angebotene Komponenten gibt, die diese Informationen liefern k�nnen bzw. die Services erf�llen.
\item Der Entwickler wei� nicht, �ber welche konkreten angebotenen Komponenten er die Informationen abfragen bzw. die Services in Anspruch nehmen kann.
\end{itemize}
\subsubsection{Funktionale Anforderungen}
In dieser Arbeit soll ein Konzept entwickelt werden, welches dem Entwickler erm�glicht ,die Erwartungen an die angebotenen Komponenten zu spezifizieren. Darauf aufbauend soll ein Algorithmus vorgeschlagen werden, welcher die angebotenen Komponenten zur Laufzeit hinsichtlich der spezifizierten Erwartungen des Entwicklers evaluiert und eine Auswahl derer trifft, die diese Erwartungen erf�llen. Da die Evaluation zur Laufzeit durchgef�hrt wird, kann der Entwickler, anders als bei den oben genannten Arbeiten, nicht aus einer Liste von Vorschl�gen ausw�hlen, welche der evaluierten Komponenten letztendlich verwendet werden soll. Diese Entscheidung ist durch den Algorithmus zu treffen.
\subsubsection{Nichtfunktionale Anforderungen}
Aufgrund bestimmter Konfigurationen des Gesamtsystems gibt es folgende weitere nichtfunktionale Anforderungen:
\begin{itemize}
\item Die Suche muss innerhalb des Transaktionstimeouts zu einem Ergebnis f�hren. (Im verwendeten System ist dieses auf 5 Minuten festgesetzt.)
\item Die Suche soll hinsichtlich der Besonderheiten des System, in dem sie verwendet wird, angepasst werden k�nnen. (Bspw. bei der Verwendung bestimmter Typen, deren Fachlogik bei der Suche nicht untergraben werden darf.)
\item Bei einem Fehlschlag der Suche, sollen dem Entwickler Informationen zur Verf�gung gestellt werden, die eine zielgerichtete Anpassung seiner spezifizierten Erwartungen erlauben.
\end{itemize}