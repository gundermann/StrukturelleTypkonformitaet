\section{Beispiel-Implementierungen f�r die Matcher}\label{matcherExamples}
Um die Beispiel-Implementierungen der Matcher nachvollziehen zu k�nnen, ist es notwendig die Implementierung der darin verwendeten Klassen aufzuzeigen. Daher sind diese in \lstsrefs{LST_superclass_impl}{LST_SuperWrapperReturnSubWrapperParamClass_impl} aufgef�hrt. Dabei handelt es sich zum einen um die Implementierungen der Klassen, die in den Szenarien der Abschnitte \ref{exactTypeMatcher} - \ref{structTypeMatcher} beschrieben wurden. Zum anderen handelt es sich um Implementierung weiterer Klassen, die in Szenarien verwendet werden, welche in den folgenden Abschnitten aufgef�hrt werden. Um einen �berblick zu gew�hrleisten, zeigt \abbref{cd_alltypes} alle Typen auf, die in den Szenarien verwendet werden.

\myBigFigure{cd_alltypes}{Alle Typen/Klassen, die in Matcher-Szenarien verwendet werden}{cd_alltypes}

\begin{lstlisting}[{caption = Implemetierung: SuperClass
},{label = LST_superclass_impl}]
public class SuperClass {

  private String string;

  public SuperClass( String string ) {
    this.string = string;
  }

  public String getString() {
    return string;
  }
}
\end{lstlisting}


\begin{lstlisting}[{caption = Implemetierung: SubClass
},{label = LST_subclass_impl}]
public class SubClass extends SuperClass {

  public SubClass( String string ) {
    super( "Sub" + string );
  }

  public String getStringWithoutPrefix() {
    return getString().substring( 3 );
  }
}
\end{lstlisting}





\begin{lstlisting}[{caption = Implemetierung: SuperWrapperReturnSubWrapperParamClass
},{label = LST_SuperWrapperReturnSubWrapperParamClass_impl}]
public class SuperWrapperReturnSubWrapperParamClass {

  public SuperWrapper addHello( SubWrapper a ) {
    return new SuperWrapper( a.toString() + "hello" );
  }

  public SuperWrapper add( SubWrapper a, SubWrapper b ) {
    return new SuperWrapper( a.toString() + b.toString() );
  }
}
\end{lstlisting}

\subsection{Beispiel f�r den ExactTypeMatcher}\label{exactMatcherExample}
In \lstref{LST_exactTypeMatcher_matching} ist die Implementierung eines JUnit-Tests aufgef�hrt, in dem das Matching �ber den ExactTypeMatcher f�r unterschiedliche Source- und Target-Typen nachgewiesen werden soll. Die Test-Methode match enth�lt dabei die Aufrufe, bei denen das Matching festgestellt werden kann. Dementsprechend enth�lt die Test-Methode noMatch die Aufrufe, bei denen das Matching fehlschl�gt.\\\\
\lstref{LST_exactTypeMatcher_conversion} enth�lt die Implementierung f�r einen JUnit-Test, in dem die Konvertierung, die durch den ExactTypeMatcher beschrieben wird, nachgewiesen wird. Hierbei wird von dem Szenario aus \ref{exactTypeMatcher} ausgegangen.
\begin{lstlisting}[{caption = ExactTypeMatcher Matching Test
},{label = LST_exactTypeMatcher_matching}]
public class ExactTypeMatcher_MatcherTest {

  @Test
  public void match() {
    ExactTypeMatcher matcher = new ExactTypeMatcher();
    assertTrue( matcher.matchesType( String.class, String.class ) );
    assertTrue( matcher.matchesType( int.class, int.class ) );
    assertTrue( matcher.matchesType( Object.class, Object.class ) );
    assertTrue( matcher.matchesType( SuperClass.class, SuperClass.class ) );
  }

  @Test
  public void noMatch() {
    ExactTypeMatcher matcher = new ExactTypeMatcher();
    assertFalse( matcher.matchesType( String.class, int.class ) );
    assertFalse( matcher.matchesType( int.class, Object.class ) );
    assertFalse( matcher.matchesType( Object.class, String.class ) );
    assertFalse( matcher.matchesType( SuperClass.class, SubClass.class ) );
  }
}
\end{lstlisting}

\begin{lstlisting}[{caption = ExactTypeMatcher Konvertierung Test
},{label = LST_exactTypeMatcher_conversion}]
public class ExactTypeMatcher_ConversionTest {

  @Test
  public void convertString() {
    SuperClass target = new SuperClass( "A" );
    Collection<ModuleMatchingInfo> matchingInfos = new ExactTypeMatcher().calculateTypeMatchingInfos( SuperClass.class,
        SuperClass.class );
    
    ModuleMatchingInfo moduleMatchingInfo = matchingInfos.iterator().next();

    ProxyFactory<SuperClass> proxyFactory = moduleMatchingInfo.getConverterCreator()
        .createProxyFactory( SuperClass.class );
    Collection<MethodMatchingInfo> methodMatchingInfos = moduleMatchingInfo.getMethodMatchingInfos();

    SuperClass source = proxyFactory.createProxy( target, methodMatchingInfos );

    assertTrue( source.getString().equals( "A" ) );
  }
}
\end{lstlisting}

\subsection{Beispiel f�r den GenTypeMatcher}\label{genMatcherExample}
Der GenTypeMatcher und der SpecTypeMatcher wurden gemeinsam implementiert. Daher wird in den folgenden Beispielen jeweils ein Matcher aus der Klasse GenSpecTypeMatcher erzeugt. Die weitere Verwendung den Matchers bezieht sich in diesen Beispielen aber auf die Definition des GenTypeMatchers aus \ref{genTypeMatcher}.\\\\
In \lstref{LST_genTypeMatcher_matching} ist die Implementierung eines JUnit-Tests aufgef�hrt, in dem das Matching �ber den GenTypeMatcher f�r unterschiedliche Source- und Target-Typen nachgewiesen werden soll. Die Test-Methode match enth�lt dabei die Aufrufe, bei denen das Matching festgestellt werden kann. Dementsprechend enth�lt die Test-Methode noMatch die Aufrufe, bei denen das Matching fehlschl�gt.\\\\
\lstref{LST_genTypeMatcher_conversion} enth�lt die Implementierung f�r einen JUnit-Test, in dem die Konvertierung, die durch den GenTypeMatcher beschrieben wird, nachgewiesen wird. Hierbei wird von dem Szenario aus \ref{genTypeMatcher} ausgegangen.
\begin{lstlisting}[{caption = GenTypeMatcher Matching Test
},{label = LST_genTypeMatcher_matching}]
public class GenSpecTypeMatcher_Gen_MatcherTest {

  @Test
  public void match() {
    GenSpecTypeMatcher matcher = new GenSpecTypeMatcher();
    assertTrue( matcher.matchesType( Object.class, String.class ) );
    assertTrue( matcher.matchesType( SuperClass.class, SubClass.class ) );
    assertTrue( matcher.matchesType( Number.class, Integer.class ) );
  }

  @Test
  public void noMatch() {
    GenSpecTypeMatcher matcher = new GenSpecTypeMatcher();
    assertFalse( matcher.matchesType( int.class, String.class ) );
  }
}
\end{lstlisting}
\begin{lstlisting}[{caption = GenTypeMatcher Konvertierung Test
},{label = LST_genTypeMatcher_conversion}]
public class GenSpecTypeMatcher_Gen_ConversionTest {

  @Test
  public void convertSpec2Gen() {
   SubClass target = new SubClass( "A" );
    Collection<ModuleMatchingInfo> matchingInfos = new GenSpecTypeMatcher().calculateTypeMatchingInfos(
        SuperClass.class, SubClass.class );
    
    ModuleMatchingInfo moduleMatchingInfo = matchingInfos.iterator().next();

    ProxyFactory<SuperClass> proxyFactory = moduleMatchingInfo.getConverterCreator()
        .createProxyFactory( SuperClass.class );
    Collection<MethodMatchingInfo> methodMatchingInfos = moduleMatchingInfo.getMethodMatchingInfos();

    SuperClass source = proxyFactory.createProxy( target, methodMatchingInfos );

    assertTrue( source.getString().equals( "SubA" ) );
  }
}
\end{lstlisting}


\subsection{Beispiel f�r den SpecTypeMatcher}\label{specMatcherExample}
Wie in \ref{genTypeMatcher} bereits erw�hnt wurde der GenTypeMatcher gemeinsam mit dem SpecTypeMatcher implementiert. Daher wird in den folgenden Beispielen jeweils ein Matcher aus der Klasse GenSpecTypeMatcher erzeugt. Die weitere Verwendung den Matchers bezieht sich in diesen Beispielen aber auf die Definition des SpecTypeMatcher aus \ref{specTypeMatcher}.\\\\
In \lstref{LST_specTypeMatcher_matching} ist die Implementierung eines JUnit-Tests aufgef�hrt, in dem das Matching �ber den GenTypeMatcher f�r unterschiedliche Source- und Target-Typen nachgewiesen werden soll. Die Test-Methode match enth�lt dabei die Aufrufe, bei denen das Matching festgestellt werden kann. Dementsprechend enth�lt die Test-Methode noMatch die Aufrufe, bei denen das Matching fehlschl�gt.\\\\
\lstref{LST_specTypeMatcher_conversion} enth�lt die Implementierung f�r einen JUnit-Test, in dem die Konvertierung, die durch den SpecTypeMatcher beschrieben wird, nachgewiesen wird. Hierbei wird von dem Szenario aus \ref{specTypeMatcher} ausgegangen. Die Test-Methode convertGen2Spec\_positivCall enth�lt den Aufruf der ersten Methoden aus dem Szenario (getString). Die Test-Methoden convertGen2Spec\_negativeCall beinhaltet den fehlschlagenden Aufruf der Methoden getStringWithoutPrefix.
\begin{lstlisting}[{caption = SpecTypeMatcher Matching Test
},{label = LST_specTypeMatcher_matching}]
public class GenSpecTypeMatcher_Spec_MatcherTest {

  @Test
  public void match() {
    GenSpecTypeMatcher matcher = new GenSpecTypeMatcher();
    assertTrue( matcher.matchesType( String.class, Object.class ) );
    assertTrue( matcher.matchesType( SubClass.class, SuperClass.class ) );
    assertTrue( matcher.matchesType( Integer.class, Number.class ) );
  }

  @Test
  public void noMatch() {
    GenSpecTypeMatcher matcher = new GenSpecTypeMatcher();
    assertFalse( matcher.matchesType( int.class, String.class ) );
  }
}
\end{lstlisting}
\begin{lstlisting}[{caption = SpecTypeMatcher Konvertierung Test
},{label = LST_specTypeMatcher_conversion}]
public class GenSpecTypeMatcher_Spec_ConversionTest {

  @Test
  public void convertGen2Spec_positivCall() {
    SuperClass offeredComponent = new SuperClass( "A" );
    Collection<ModuleMatchingInfo> matchingInfos = new GenSpecTypeMatcher().calculateTypeMatchingInfos(
        SubClass.class, SuperClass.class );
    
    ModuleMatchingInfo moduleMatchingInfo = matchingInfos.iterator().next();

    ProxyFactory<SubClass> proxyFactory = moduleMatchingInfo.getConverterCreator().createProxyFactory( SubClass.class );
    Collection<MethodMatchingInfo> methodMatchingInfos = moduleMatchingInfo.getMethodMatchingInfos();

    SubClass proxy = proxyFactory.createProxy( offeredComponent, methodMatchingInfos );

    assertTrue( proxy.getString().equals( "A" ) );
  }

  @Test( expected = SigMaGlueException.class )
  public void convertGen2Spec_negativeCall() {
    SuperClass offeredComponent = new SuperClass( "A" );
    Collection<ModuleMatchingInfo> matchingInfos = new GenSpecTypeMatcher().calculateTypeMatchingInfos(
        SubClass.class, SuperClass.class );
    
    ModuleMatchingInfo moduleMatchingInfo = matchingInfos.iterator().next();

    ProxyFactory<SubClass> proxyFactory = moduleMatchingInfo.getConverterCreator().createProxyFactory( SubClass.class );
    Collection<MethodMatchingInfo> methodMatchingInfos = moduleMatchingInfo.getMethodMatchingInfos();

    SubClass proxy = proxyFactory.createProxy( offeredComponent, methodMatchingInfos );

    assertTrue( proxy.getString().equals( "A" ) );

    proxy.getStringWithoutPrefix();
  }
}
\end{lstlisting}



\subsection{Beispiel f�r den WrappedTypeMatcher}\label{wrappedMatcherExample}
Der WrappedTypeMatcher und der WrapperTypeMatcher wurden gemeinsam implementiert. Daher wird in den folgenden Beispielen jeweils ein Matcher aus der Klasse WrappedTypeMatcher erzeugt. Die weitere Verwendung den Matchers bezieht sich in diesen Beispielen aber auf die Definition des WrappedTypeMatcher aus \ref{wrappedTypeMatcher}.\\\\
In \lstref{LST_wrappedTypeMatcher_matching} ist die Implementierung eines JUnit-Tests aufgef�hrt, in dem das Matching �ber den WrappedTypeMatcher f�r unterschiedliche Source- und Target-Typen nachgewiesen werden soll. Die Test-Methoden mit dem Pr�fix match enthalten dabei die Aufrufe, bei denen das Matching festgestellt werden kann. Dementsprechend enth�lt die Test-Methode noMatch die Aufrufe, bei denen das Matching fehlschl�gt.\\\\
\lstref{LST_wrappedTypeMatcher_conversion} enth�lt die Implementierung f�r einen JUnit-Test, in dem die Konvertierung, die durch den WrappedTypeMatcher beschrieben wird, nachgewiesen wird. In der Test-Methode convertSubWrapper2SubClass wird von dem Szenario aus \ref{wrappedTypeMatcher} ausgegangen. Die anderen Test-Methoden stellen weitere Szenarien dar, die in den folgenden Abschnitten beschrieben werden.
\begin{lstlisting}[{caption = WrappedTypeMatcher Matching Test
},{label = LST_wrappedTypeMatcher_matching}]
public class WrappedTypeMatcher_Wrapped_MatcherTest {

  private WrappedTypeMatcher matcher = new WrappedTypeMatcher(
      MatcherCombiner.combine( new ExactTypeMatcher(), new GenSpecTypeMatcher() ) );

  @Test
  public void match() {
    assertTrue( matcher.matchesType( boolean.class, Boolean.class ) );
    assertTrue( matcher.matchesType( int.class, Integer.class ) );
  }

  @Test
  public void match_wrapped_exact() {
    assertTrue( matcher.matchesType( SubClass.class, SubWrapper.class ) );
  }

  @Test
  public void match_wrapped_spec() {
    assertTrue( matcher.matchesType( SubClass.class, SuperWrapper.class ) );
  }

  @Test
  public void match_wrapped_gen() {
    assertTrue( matcher.matchesType( SuperClass.class, SubWrapper.class ) );
  }

  @Test
  public void noMatch() {
    assertFalse( matcher.matchesType( String.class, String.class ) );
  }
}
\end{lstlisting}
\begin{lstlisting}[{caption = WrappedTypeMatcher Konvertierung Test
},{label = LST_wrappedTypeMatcher_conversion}]
public class WrappedTypeMatcher_Wrapped_ConversionTest {

  private WrappedTypeMatcher matcher = new WrappedTypeMatcher(
      MatcherCombiner.combine( new ExactTypeMatcher(), new GenSpecTypeMatcher() ) );

  @Test
  public void convertSubWrapper2SubClass() {
    SubWrapper offeredComponent = new SubWrapper( "A" );
    Collection<ModuleMatchingInfo> matchingInfos = matcher.calculateTypeMatchingInfos(
        SubClass.class, SubWrapper.class );
  
    ModuleMatchingInfo moduleMatchingInfo = matchingInfos.iterator().next();

    ProxyFactory<SubClass> proxyFactory = moduleMatchingInfo.getConverterCreator()
        .createProxyFactory( SubClass.class );
    Collection<MethodMatchingInfo> methodMatchingInfos = moduleMatchingInfo.getMethodMatchingInfos();

    SubClass proxy = proxyFactory.createProxy( offeredComponent, methodMatchingInfos );

    assertTrue( proxy.getString().equals( "SubA" ) );
    assertTrue( proxy.getStringWithoutPrefix().equals( "A" ) );
  }

  @Test
  public void convertSuperWrapper2SubClass_positiveCall() {
    SuperWrapper offeredComponent = new SuperWrapper( "A" );
    Collection<ModuleMatchingInfo> matchingInfos = matcher.calculateTypeMatchingInfos(
        SubClass.class, SuperWrapper.class );
  
    ModuleMatchingInfo moduleMatchingInfo = matchingInfos.iterator().next();

    ProxyFactory<SubClass> proxyFactory = moduleMatchingInfo.getConverterCreator()
        .createProxyFactory( SubClass.class );
    Collection<MethodMatchingInfo> methodMatchingInfos = moduleMatchingInfo.getMethodMatchingInfos();

    SubClass proxy = proxyFactory.createProxy( offeredComponent, methodMatchingInfos );

    assertTrue( proxy.getString().equals( "A" ) );
  }

  @Test( expected = SigMaGlueException.class )
  public void convertSuperWrapper2SubClass_negativeCall() {
    SuperWrapper offeredComponent = new SuperWrapper( "A" );
    Collection<ModuleMatchingInfo> matchingInfos = matcher.calculateTypeMatchingInfos(
        SubClass.class, SuperWrapper.class );
  
    ModuleMatchingInfo moduleMatchingInfo = matchingInfos.iterator().next();

    ProxyFactory<SubClass> proxyFactory = moduleMatchingInfo.getConverterCreator()
        .createProxyFactory( SubClass.class );
    Collection<MethodMatchingInfo> methodMatchingInfos = moduleMatchingInfo.getMethodMatchingInfos();

    SubClass proxy = proxyFactory.createProxy( offeredComponent, methodMatchingInfos );
    proxy.getStringWithoutPrefix();
  }

  @Test
  public void convertSubWrapper2SuperClass() {
    SubWrapper offeredComponent = new SubWrapper( "A" );
    Collection<ModuleMatchingInfo> matchingInfos = matcher.calculateTypeMatchingInfos(
        SuperClass.class, SubWrapper.class );
   
    ModuleMatchingInfo moduleMatchingInfo = matchingInfos.iterator().next();

    ProxyFactory<SuperClass> proxyFactory = moduleMatchingInfo.getConverterCreator()
        .createProxyFactory( SuperClass.class );
    Collection<MethodMatchingInfo> methodMatchingInfos = moduleMatchingInfo.getMethodMatchingInfos();

    SuperClass proxy = proxyFactory.createProxy( offeredComponent, methodMatchingInfos );

    assertTrue( proxy.getString().equals( "SubA" ) );
  }
}
\end{lstlisting}